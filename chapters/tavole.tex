\chapter{Tavole numeriche}

In questa appendice riportiamo i valori tabulati di alcune funzioni statistiche
rilevanti (e.g., l'integrale normale degli errori). Le tavole possono essere
utili quando non si ha a disposizione un calcolatore.


\clearpage


\section{Integrale normale degli errori---I}
\label{sec:tavola_erf3}

\autohstack[3pt]{%% Creator: Matplotlib, PGF backend
%%
%% To include the figure in your LaTeX document, write
%%   \input{<filename>.pgf}
%%
%% Make sure the required packages are loaded in your preamble
%%   \usepackage{pgf}
%%
%% and, on pdftex
%%   \usepackage[utf8]{inputenc}\DeclareUnicodeCharacter{2212}{-}
%%
%% or, on luatex and xetex
%%   \usepackage{unicode-math}
%%
%% Figures using additional raster images can only be included by \input if
%% they are in the same directory as the main LaTeX file. For loading figures
%% from other directories you can use the `import` package
%%   \usepackage{import}
%%
%% and then include the figures with
%%   \import{<path to file>}{<filename>.pgf}
%%
%% Matplotlib used the following preamble
%%   \usepackage[nice]{nicefrac}
%%   \usepackage{amsmath}
%%   \DeclareUnicodeCharacter{2212}{-}
%%
\begingroup%
\makeatletter%
\begin{pgfpicture}%
\pgfpathrectangle{\pgfpointorigin}{\pgfqpoint{2.250000in}{1.850000in}}%
\pgfusepath{use as bounding box, clip}%
\begin{pgfscope}%
\pgfsetbuttcap%
\pgfsetmiterjoin%
\definecolor{currentfill}{rgb}{1.000000,1.000000,1.000000}%
\pgfsetfillcolor{currentfill}%
\pgfsetlinewidth{0.000000pt}%
\definecolor{currentstroke}{rgb}{1.000000,1.000000,1.000000}%
\pgfsetstrokecolor{currentstroke}%
\pgfsetdash{}{0pt}%
\pgfpathmoveto{\pgfqpoint{0.000000in}{0.000000in}}%
\pgfpathlineto{\pgfqpoint{2.250000in}{0.000000in}}%
\pgfpathlineto{\pgfqpoint{2.250000in}{1.850000in}}%
\pgfpathlineto{\pgfqpoint{0.000000in}{1.850000in}}%
\pgfpathclose%
\pgfusepath{fill}%
\end{pgfscope}%
\begin{pgfscope}%
\pgfsetbuttcap%
\pgfsetmiterjoin%
\definecolor{currentfill}{rgb}{1.000000,1.000000,1.000000}%
\pgfsetfillcolor{currentfill}%
\pgfsetlinewidth{0.000000pt}%
\definecolor{currentstroke}{rgb}{0.000000,0.000000,0.000000}%
\pgfsetstrokecolor{currentstroke}%
\pgfsetstrokeopacity{0.000000}%
\pgfsetdash{}{0pt}%
\pgfpathmoveto{\pgfqpoint{0.405000in}{0.333000in}}%
\pgfpathlineto{\pgfqpoint{2.137500in}{0.333000in}}%
\pgfpathlineto{\pgfqpoint{2.137500in}{1.776000in}}%
\pgfpathlineto{\pgfqpoint{0.405000in}{1.776000in}}%
\pgfpathclose%
\pgfusepath{fill}%
\end{pgfscope}%
\begin{pgfscope}%
\pgfpathrectangle{\pgfqpoint{0.405000in}{0.333000in}}{\pgfqpoint{1.732500in}{1.443000in}}%
\pgfusepath{clip}%
\pgfsetbuttcap%
\pgfsetroundjoin%
\definecolor{currentfill}{rgb}{0.501961,0.501961,0.501961}%
\pgfsetfillcolor{currentfill}%
\pgfsetfillopacity{0.500000}%
\pgfsetlinewidth{0.000000pt}%
\definecolor{currentstroke}{rgb}{0.000000,0.000000,0.000000}%
\pgfsetstrokecolor{currentstroke}%
\pgfsetdash{}{0pt}%
\pgfpathmoveto{\pgfqpoint{1.280000in}{1.483408in}}%
\pgfpathlineto{\pgfqpoint{1.280000in}{0.333000in}}%
\pgfpathlineto{\pgfqpoint{1.297500in}{0.333000in}}%
\pgfpathlineto{\pgfqpoint{1.315000in}{0.333000in}}%
\pgfpathlineto{\pgfqpoint{1.332500in}{0.333000in}}%
\pgfpathlineto{\pgfqpoint{1.350000in}{0.333000in}}%
\pgfpathlineto{\pgfqpoint{1.367500in}{0.333000in}}%
\pgfpathlineto{\pgfqpoint{1.385000in}{0.333000in}}%
\pgfpathlineto{\pgfqpoint{1.402500in}{0.333000in}}%
\pgfpathlineto{\pgfqpoint{1.420000in}{0.333000in}}%
\pgfpathlineto{\pgfqpoint{1.437500in}{0.333000in}}%
\pgfpathlineto{\pgfqpoint{1.455000in}{0.333000in}}%
\pgfpathlineto{\pgfqpoint{1.472500in}{0.333000in}}%
\pgfpathlineto{\pgfqpoint{1.490000in}{0.333000in}}%
\pgfpathlineto{\pgfqpoint{1.507500in}{0.333000in}}%
\pgfpathlineto{\pgfqpoint{1.525000in}{0.333000in}}%
\pgfpathlineto{\pgfqpoint{1.542500in}{0.333000in}}%
\pgfpathlineto{\pgfqpoint{1.560000in}{0.333000in}}%
\pgfpathlineto{\pgfqpoint{1.577500in}{0.333000in}}%
\pgfpathlineto{\pgfqpoint{1.595000in}{0.333000in}}%
\pgfpathlineto{\pgfqpoint{1.595000in}{0.709625in}}%
\pgfpathlineto{\pgfqpoint{1.595000in}{0.709625in}}%
\pgfpathlineto{\pgfqpoint{1.577500in}{0.756600in}}%
\pgfpathlineto{\pgfqpoint{1.560000in}{0.806333in}}%
\pgfpathlineto{\pgfqpoint{1.542500in}{0.858462in}}%
\pgfpathlineto{\pgfqpoint{1.525000in}{0.912536in}}%
\pgfpathlineto{\pgfqpoint{1.507500in}{0.968014in}}%
\pgfpathlineto{\pgfqpoint{1.490000in}{1.024274in}}%
\pgfpathlineto{\pgfqpoint{1.472500in}{1.080621in}}%
\pgfpathlineto{\pgfqpoint{1.455000in}{1.136297in}}%
\pgfpathlineto{\pgfqpoint{1.437500in}{1.190503in}}%
\pgfpathlineto{\pgfqpoint{1.420000in}{1.242408in}}%
\pgfpathlineto{\pgfqpoint{1.402500in}{1.291178in}}%
\pgfpathlineto{\pgfqpoint{1.385000in}{1.335992in}}%
\pgfpathlineto{\pgfqpoint{1.367500in}{1.376069in}}%
\pgfpathlineto{\pgfqpoint{1.350000in}{1.410687in}}%
\pgfpathlineto{\pgfqpoint{1.332500in}{1.439207in}}%
\pgfpathlineto{\pgfqpoint{1.315000in}{1.461091in}}%
\pgfpathlineto{\pgfqpoint{1.297500in}{1.475920in}}%
\pgfpathlineto{\pgfqpoint{1.280000in}{1.483408in}}%
\pgfpathclose%
\pgfusepath{fill}%
\end{pgfscope}%
\begin{pgfscope}%
\pgfsetbuttcap%
\pgfsetroundjoin%
\definecolor{currentfill}{rgb}{0.000000,0.000000,0.000000}%
\pgfsetfillcolor{currentfill}%
\pgfsetlinewidth{0.803000pt}%
\definecolor{currentstroke}{rgb}{0.000000,0.000000,0.000000}%
\pgfsetstrokecolor{currentstroke}%
\pgfsetdash{}{0pt}%
\pgfsys@defobject{currentmarker}{\pgfqpoint{0.000000in}{-0.048611in}}{\pgfqpoint{0.000000in}{0.000000in}}{%
\pgfpathmoveto{\pgfqpoint{0.000000in}{0.000000in}}%
\pgfpathlineto{\pgfqpoint{0.000000in}{-0.048611in}}%
\pgfusepath{stroke,fill}%
}%
\begin{pgfscope}%
\pgfsys@transformshift{0.405000in}{0.333000in}%
\pgfsys@useobject{currentmarker}{}%
\end{pgfscope}%
\end{pgfscope}%
\begin{pgfscope}%
\definecolor{textcolor}{rgb}{0.000000,0.000000,0.000000}%
\pgfsetstrokecolor{textcolor}%
\pgfsetfillcolor{textcolor}%
\pgftext[x=0.405000in,y=0.235778in,,top]{\color{textcolor}\rmfamily\fontsize{9.000000}{10.800000}\selectfont −4}%
\end{pgfscope}%
\begin{pgfscope}%
\pgfsetbuttcap%
\pgfsetroundjoin%
\definecolor{currentfill}{rgb}{0.000000,0.000000,0.000000}%
\pgfsetfillcolor{currentfill}%
\pgfsetlinewidth{0.803000pt}%
\definecolor{currentstroke}{rgb}{0.000000,0.000000,0.000000}%
\pgfsetstrokecolor{currentstroke}%
\pgfsetdash{}{0pt}%
\pgfsys@defobject{currentmarker}{\pgfqpoint{0.000000in}{-0.048611in}}{\pgfqpoint{0.000000in}{0.000000in}}{%
\pgfpathmoveto{\pgfqpoint{0.000000in}{0.000000in}}%
\pgfpathlineto{\pgfqpoint{0.000000in}{-0.048611in}}%
\pgfusepath{stroke,fill}%
}%
\begin{pgfscope}%
\pgfsys@transformshift{0.838125in}{0.333000in}%
\pgfsys@useobject{currentmarker}{}%
\end{pgfscope}%
\end{pgfscope}%
\begin{pgfscope}%
\definecolor{textcolor}{rgb}{0.000000,0.000000,0.000000}%
\pgfsetstrokecolor{textcolor}%
\pgfsetfillcolor{textcolor}%
\pgftext[x=0.838125in,y=0.235778in,,top]{\color{textcolor}\rmfamily\fontsize{9.000000}{10.800000}\selectfont −2}%
\end{pgfscope}%
\begin{pgfscope}%
\pgfsetbuttcap%
\pgfsetroundjoin%
\definecolor{currentfill}{rgb}{0.000000,0.000000,0.000000}%
\pgfsetfillcolor{currentfill}%
\pgfsetlinewidth{0.803000pt}%
\definecolor{currentstroke}{rgb}{0.000000,0.000000,0.000000}%
\pgfsetstrokecolor{currentstroke}%
\pgfsetdash{}{0pt}%
\pgfsys@defobject{currentmarker}{\pgfqpoint{0.000000in}{-0.048611in}}{\pgfqpoint{0.000000in}{0.000000in}}{%
\pgfpathmoveto{\pgfqpoint{0.000000in}{0.000000in}}%
\pgfpathlineto{\pgfqpoint{0.000000in}{-0.048611in}}%
\pgfusepath{stroke,fill}%
}%
\begin{pgfscope}%
\pgfsys@transformshift{1.271250in}{0.333000in}%
\pgfsys@useobject{currentmarker}{}%
\end{pgfscope}%
\end{pgfscope}%
\begin{pgfscope}%
\definecolor{textcolor}{rgb}{0.000000,0.000000,0.000000}%
\pgfsetstrokecolor{textcolor}%
\pgfsetfillcolor{textcolor}%
\pgftext[x=1.271250in,y=0.235778in,,top]{\color{textcolor}\rmfamily\fontsize{9.000000}{10.800000}\selectfont 0}%
\end{pgfscope}%
\begin{pgfscope}%
\pgfsetbuttcap%
\pgfsetroundjoin%
\definecolor{currentfill}{rgb}{0.000000,0.000000,0.000000}%
\pgfsetfillcolor{currentfill}%
\pgfsetlinewidth{0.803000pt}%
\definecolor{currentstroke}{rgb}{0.000000,0.000000,0.000000}%
\pgfsetstrokecolor{currentstroke}%
\pgfsetdash{}{0pt}%
\pgfsys@defobject{currentmarker}{\pgfqpoint{0.000000in}{-0.048611in}}{\pgfqpoint{0.000000in}{0.000000in}}{%
\pgfpathmoveto{\pgfqpoint{0.000000in}{0.000000in}}%
\pgfpathlineto{\pgfqpoint{0.000000in}{-0.048611in}}%
\pgfusepath{stroke,fill}%
}%
\begin{pgfscope}%
\pgfsys@transformshift{1.704375in}{0.333000in}%
\pgfsys@useobject{currentmarker}{}%
\end{pgfscope}%
\end{pgfscope}%
\begin{pgfscope}%
\definecolor{textcolor}{rgb}{0.000000,0.000000,0.000000}%
\pgfsetstrokecolor{textcolor}%
\pgfsetfillcolor{textcolor}%
\pgftext[x=1.704375in,y=0.235778in,,top]{\color{textcolor}\rmfamily\fontsize{9.000000}{10.800000}\selectfont 2}%
\end{pgfscope}%
\begin{pgfscope}%
\pgfsetbuttcap%
\pgfsetroundjoin%
\definecolor{currentfill}{rgb}{0.000000,0.000000,0.000000}%
\pgfsetfillcolor{currentfill}%
\pgfsetlinewidth{0.803000pt}%
\definecolor{currentstroke}{rgb}{0.000000,0.000000,0.000000}%
\pgfsetstrokecolor{currentstroke}%
\pgfsetdash{}{0pt}%
\pgfsys@defobject{currentmarker}{\pgfqpoint{0.000000in}{-0.048611in}}{\pgfqpoint{0.000000in}{0.000000in}}{%
\pgfpathmoveto{\pgfqpoint{0.000000in}{0.000000in}}%
\pgfpathlineto{\pgfqpoint{0.000000in}{-0.048611in}}%
\pgfusepath{stroke,fill}%
}%
\begin{pgfscope}%
\pgfsys@transformshift{2.137500in}{0.333000in}%
\pgfsys@useobject{currentmarker}{}%
\end{pgfscope}%
\end{pgfscope}%
\begin{pgfscope}%
\definecolor{textcolor}{rgb}{0.000000,0.000000,0.000000}%
\pgfsetstrokecolor{textcolor}%
\pgfsetfillcolor{textcolor}%
\pgftext[x=2.137500in,y=0.235778in,,top]{\color{textcolor}\rmfamily\fontsize{9.000000}{10.800000}\selectfont 4}%
\end{pgfscope}%
\begin{pgfscope}%
\definecolor{textcolor}{rgb}{0.000000,0.000000,0.000000}%
\pgfsetstrokecolor{textcolor}%
\pgfsetfillcolor{textcolor}%
\pgftext[x=1.271250in,y=0.069111in,,top]{\color{textcolor}\rmfamily\fontsize{9.000000}{10.800000}\selectfont \(\displaystyle z\)}%
\end{pgfscope}%
\begin{pgfscope}%
\pgfsetbuttcap%
\pgfsetroundjoin%
\definecolor{currentfill}{rgb}{0.000000,0.000000,0.000000}%
\pgfsetfillcolor{currentfill}%
\pgfsetlinewidth{0.803000pt}%
\definecolor{currentstroke}{rgb}{0.000000,0.000000,0.000000}%
\pgfsetstrokecolor{currentstroke}%
\pgfsetdash{}{0pt}%
\pgfsys@defobject{currentmarker}{\pgfqpoint{-0.048611in}{0.000000in}}{\pgfqpoint{0.000000in}{0.000000in}}{%
\pgfpathmoveto{\pgfqpoint{0.000000in}{0.000000in}}%
\pgfpathlineto{\pgfqpoint{-0.048611in}{0.000000in}}%
\pgfusepath{stroke,fill}%
}%
\begin{pgfscope}%
\pgfsys@transformshift{0.405000in}{0.333000in}%
\pgfsys@useobject{currentmarker}{}%
\end{pgfscope}%
\end{pgfscope}%
\begin{pgfscope}%
\definecolor{textcolor}{rgb}{0.000000,0.000000,0.000000}%
\pgfsetstrokecolor{textcolor}%
\pgfsetfillcolor{textcolor}%
\pgftext[x=0.143620in, y=0.289597in, left, base]{\color{textcolor}\rmfamily\fontsize{9.000000}{10.800000}\selectfont 0.0}%
\end{pgfscope}%
\begin{pgfscope}%
\pgfsetbuttcap%
\pgfsetroundjoin%
\definecolor{currentfill}{rgb}{0.000000,0.000000,0.000000}%
\pgfsetfillcolor{currentfill}%
\pgfsetlinewidth{0.803000pt}%
\definecolor{currentstroke}{rgb}{0.000000,0.000000,0.000000}%
\pgfsetstrokecolor{currentstroke}%
\pgfsetdash{}{0pt}%
\pgfsys@defobject{currentmarker}{\pgfqpoint{-0.048611in}{0.000000in}}{\pgfqpoint{0.000000in}{0.000000in}}{%
\pgfpathmoveto{\pgfqpoint{0.000000in}{0.000000in}}%
\pgfpathlineto{\pgfqpoint{-0.048611in}{0.000000in}}%
\pgfusepath{stroke,fill}%
}%
\begin{pgfscope}%
\pgfsys@transformshift{0.405000in}{0.621600in}%
\pgfsys@useobject{currentmarker}{}%
\end{pgfscope}%
\end{pgfscope}%
\begin{pgfscope}%
\definecolor{textcolor}{rgb}{0.000000,0.000000,0.000000}%
\pgfsetstrokecolor{textcolor}%
\pgfsetfillcolor{textcolor}%
\pgftext[x=0.143620in, y=0.578197in, left, base]{\color{textcolor}\rmfamily\fontsize{9.000000}{10.800000}\selectfont 0.1}%
\end{pgfscope}%
\begin{pgfscope}%
\pgfsetbuttcap%
\pgfsetroundjoin%
\definecolor{currentfill}{rgb}{0.000000,0.000000,0.000000}%
\pgfsetfillcolor{currentfill}%
\pgfsetlinewidth{0.803000pt}%
\definecolor{currentstroke}{rgb}{0.000000,0.000000,0.000000}%
\pgfsetstrokecolor{currentstroke}%
\pgfsetdash{}{0pt}%
\pgfsys@defobject{currentmarker}{\pgfqpoint{-0.048611in}{0.000000in}}{\pgfqpoint{0.000000in}{0.000000in}}{%
\pgfpathmoveto{\pgfqpoint{0.000000in}{0.000000in}}%
\pgfpathlineto{\pgfqpoint{-0.048611in}{0.000000in}}%
\pgfusepath{stroke,fill}%
}%
\begin{pgfscope}%
\pgfsys@transformshift{0.405000in}{0.910200in}%
\pgfsys@useobject{currentmarker}{}%
\end{pgfscope}%
\end{pgfscope}%
\begin{pgfscope}%
\definecolor{textcolor}{rgb}{0.000000,0.000000,0.000000}%
\pgfsetstrokecolor{textcolor}%
\pgfsetfillcolor{textcolor}%
\pgftext[x=0.143620in, y=0.866797in, left, base]{\color{textcolor}\rmfamily\fontsize{9.000000}{10.800000}\selectfont 0.2}%
\end{pgfscope}%
\begin{pgfscope}%
\pgfsetbuttcap%
\pgfsetroundjoin%
\definecolor{currentfill}{rgb}{0.000000,0.000000,0.000000}%
\pgfsetfillcolor{currentfill}%
\pgfsetlinewidth{0.803000pt}%
\definecolor{currentstroke}{rgb}{0.000000,0.000000,0.000000}%
\pgfsetstrokecolor{currentstroke}%
\pgfsetdash{}{0pt}%
\pgfsys@defobject{currentmarker}{\pgfqpoint{-0.048611in}{0.000000in}}{\pgfqpoint{0.000000in}{0.000000in}}{%
\pgfpathmoveto{\pgfqpoint{0.000000in}{0.000000in}}%
\pgfpathlineto{\pgfqpoint{-0.048611in}{0.000000in}}%
\pgfusepath{stroke,fill}%
}%
\begin{pgfscope}%
\pgfsys@transformshift{0.405000in}{1.198800in}%
\pgfsys@useobject{currentmarker}{}%
\end{pgfscope}%
\end{pgfscope}%
\begin{pgfscope}%
\definecolor{textcolor}{rgb}{0.000000,0.000000,0.000000}%
\pgfsetstrokecolor{textcolor}%
\pgfsetfillcolor{textcolor}%
\pgftext[x=0.143620in, y=1.155397in, left, base]{\color{textcolor}\rmfamily\fontsize{9.000000}{10.800000}\selectfont 0.3}%
\end{pgfscope}%
\begin{pgfscope}%
\pgfsetbuttcap%
\pgfsetroundjoin%
\definecolor{currentfill}{rgb}{0.000000,0.000000,0.000000}%
\pgfsetfillcolor{currentfill}%
\pgfsetlinewidth{0.803000pt}%
\definecolor{currentstroke}{rgb}{0.000000,0.000000,0.000000}%
\pgfsetstrokecolor{currentstroke}%
\pgfsetdash{}{0pt}%
\pgfsys@defobject{currentmarker}{\pgfqpoint{-0.048611in}{0.000000in}}{\pgfqpoint{0.000000in}{0.000000in}}{%
\pgfpathmoveto{\pgfqpoint{0.000000in}{0.000000in}}%
\pgfpathlineto{\pgfqpoint{-0.048611in}{0.000000in}}%
\pgfusepath{stroke,fill}%
}%
\begin{pgfscope}%
\pgfsys@transformshift{0.405000in}{1.487400in}%
\pgfsys@useobject{currentmarker}{}%
\end{pgfscope}%
\end{pgfscope}%
\begin{pgfscope}%
\definecolor{textcolor}{rgb}{0.000000,0.000000,0.000000}%
\pgfsetstrokecolor{textcolor}%
\pgfsetfillcolor{textcolor}%
\pgftext[x=0.143620in, y=1.443997in, left, base]{\color{textcolor}\rmfamily\fontsize{9.000000}{10.800000}\selectfont 0.4}%
\end{pgfscope}%
\begin{pgfscope}%
\pgfsetbuttcap%
\pgfsetroundjoin%
\definecolor{currentfill}{rgb}{0.000000,0.000000,0.000000}%
\pgfsetfillcolor{currentfill}%
\pgfsetlinewidth{0.803000pt}%
\definecolor{currentstroke}{rgb}{0.000000,0.000000,0.000000}%
\pgfsetstrokecolor{currentstroke}%
\pgfsetdash{}{0pt}%
\pgfsys@defobject{currentmarker}{\pgfqpoint{-0.048611in}{0.000000in}}{\pgfqpoint{0.000000in}{0.000000in}}{%
\pgfpathmoveto{\pgfqpoint{0.000000in}{0.000000in}}%
\pgfpathlineto{\pgfqpoint{-0.048611in}{0.000000in}}%
\pgfusepath{stroke,fill}%
}%
\begin{pgfscope}%
\pgfsys@transformshift{0.405000in}{1.776000in}%
\pgfsys@useobject{currentmarker}{}%
\end{pgfscope}%
\end{pgfscope}%
\begin{pgfscope}%
\definecolor{textcolor}{rgb}{0.000000,0.000000,0.000000}%
\pgfsetstrokecolor{textcolor}%
\pgfsetfillcolor{textcolor}%
\pgftext[x=0.143620in, y=1.732597in, left, base]{\color{textcolor}\rmfamily\fontsize{9.000000}{10.800000}\selectfont 0.5}%
\end{pgfscope}%
\begin{pgfscope}%
\definecolor{textcolor}{rgb}{0.000000,0.000000,0.000000}%
\pgfsetstrokecolor{textcolor}%
\pgfsetfillcolor{textcolor}%
\pgftext[x=0.088064in,y=1.054500in,,bottom,rotate=90.000000]{\color{textcolor}\rmfamily\fontsize{9.000000}{10.800000}\selectfont \(\displaystyle N(z)\)}%
\end{pgfscope}%
\begin{pgfscope}%
\pgfpathrectangle{\pgfqpoint{0.405000in}{0.333000in}}{\pgfqpoint{1.732500in}{1.443000in}}%
\pgfusepath{clip}%
\pgfsetrectcap%
\pgfsetroundjoin%
\pgfsetlinewidth{1.254687pt}%
\definecolor{currentstroke}{rgb}{0.000000,0.000000,0.000000}%
\pgfsetstrokecolor{currentstroke}%
\pgfsetdash{}{0pt}%
\pgfpathmoveto{\pgfqpoint{0.405000in}{0.333386in}}%
\pgfpathlineto{\pgfqpoint{0.422500in}{0.333532in}}%
\pgfpathlineto{\pgfqpoint{0.440000in}{0.333728in}}%
\pgfpathlineto{\pgfqpoint{0.457500in}{0.333989in}}%
\pgfpathlineto{\pgfqpoint{0.475000in}{0.334336in}}%
\pgfpathlineto{\pgfqpoint{0.492500in}{0.334792in}}%
\pgfpathlineto{\pgfqpoint{0.510000in}{0.335388in}}%
\pgfpathlineto{\pgfqpoint{0.527500in}{0.336162in}}%
\pgfpathlineto{\pgfqpoint{0.545000in}{0.337160in}}%
\pgfpathlineto{\pgfqpoint{0.562500in}{0.338437in}}%
\pgfpathlineto{\pgfqpoint{0.580000in}{0.340060in}}%
\pgfpathlineto{\pgfqpoint{0.597500in}{0.342108in}}%
\pgfpathlineto{\pgfqpoint{0.615000in}{0.344673in}}%
\pgfpathlineto{\pgfqpoint{0.632500in}{0.347864in}}%
\pgfpathlineto{\pgfqpoint{0.650000in}{0.351803in}}%
\pgfpathlineto{\pgfqpoint{0.667500in}{0.356631in}}%
\pgfpathlineto{\pgfqpoint{0.685000in}{0.362505in}}%
\pgfpathlineto{\pgfqpoint{0.702500in}{0.369600in}}%
\pgfpathlineto{\pgfqpoint{0.720000in}{0.378106in}}%
\pgfpathlineto{\pgfqpoint{0.737500in}{0.388227in}}%
\pgfpathlineto{\pgfqpoint{0.755000in}{0.400178in}}%
\pgfpathlineto{\pgfqpoint{0.772500in}{0.414184in}}%
\pgfpathlineto{\pgfqpoint{0.790000in}{0.430471in}}%
\pgfpathlineto{\pgfqpoint{0.807500in}{0.449264in}}%
\pgfpathlineto{\pgfqpoint{0.825000in}{0.470777in}}%
\pgfpathlineto{\pgfqpoint{0.842500in}{0.495209in}}%
\pgfpathlineto{\pgfqpoint{0.860000in}{0.522731in}}%
\pgfpathlineto{\pgfqpoint{0.877500in}{0.553477in}}%
\pgfpathlineto{\pgfqpoint{0.895000in}{0.587539in}}%
\pgfpathlineto{\pgfqpoint{0.912500in}{0.624950in}}%
\pgfpathlineto{\pgfqpoint{0.930000in}{0.665680in}}%
\pgfpathlineto{\pgfqpoint{0.947500in}{0.709625in}}%
\pgfpathlineto{\pgfqpoint{0.965000in}{0.756600in}}%
\pgfpathlineto{\pgfqpoint{0.982500in}{0.806333in}}%
\pgfpathlineto{\pgfqpoint{1.000000in}{0.858462in}}%
\pgfpathlineto{\pgfqpoint{1.017500in}{0.912536in}}%
\pgfpathlineto{\pgfqpoint{1.035000in}{0.968014in}}%
\pgfpathlineto{\pgfqpoint{1.052500in}{1.024274in}}%
\pgfpathlineto{\pgfqpoint{1.070000in}{1.080621in}}%
\pgfpathlineto{\pgfqpoint{1.087500in}{1.136297in}}%
\pgfpathlineto{\pgfqpoint{1.105000in}{1.190503in}}%
\pgfpathlineto{\pgfqpoint{1.122500in}{1.242408in}}%
\pgfpathlineto{\pgfqpoint{1.140000in}{1.291178in}}%
\pgfpathlineto{\pgfqpoint{1.157500in}{1.335992in}}%
\pgfpathlineto{\pgfqpoint{1.175000in}{1.376069in}}%
\pgfpathlineto{\pgfqpoint{1.192500in}{1.410687in}}%
\pgfpathlineto{\pgfqpoint{1.210000in}{1.439207in}}%
\pgfpathlineto{\pgfqpoint{1.227500in}{1.461091in}}%
\pgfpathlineto{\pgfqpoint{1.245000in}{1.475920in}}%
\pgfpathlineto{\pgfqpoint{1.262500in}{1.483408in}}%
\pgfpathlineto{\pgfqpoint{1.280000in}{1.483408in}}%
\pgfpathlineto{\pgfqpoint{1.297500in}{1.475920in}}%
\pgfpathlineto{\pgfqpoint{1.315000in}{1.461091in}}%
\pgfpathlineto{\pgfqpoint{1.332500in}{1.439207in}}%
\pgfpathlineto{\pgfqpoint{1.350000in}{1.410687in}}%
\pgfpathlineto{\pgfqpoint{1.367500in}{1.376069in}}%
\pgfpathlineto{\pgfqpoint{1.385000in}{1.335992in}}%
\pgfpathlineto{\pgfqpoint{1.402500in}{1.291178in}}%
\pgfpathlineto{\pgfqpoint{1.420000in}{1.242408in}}%
\pgfpathlineto{\pgfqpoint{1.437500in}{1.190503in}}%
\pgfpathlineto{\pgfqpoint{1.455000in}{1.136297in}}%
\pgfpathlineto{\pgfqpoint{1.472500in}{1.080621in}}%
\pgfpathlineto{\pgfqpoint{1.490000in}{1.024274in}}%
\pgfpathlineto{\pgfqpoint{1.507500in}{0.968014in}}%
\pgfpathlineto{\pgfqpoint{1.525000in}{0.912536in}}%
\pgfpathlineto{\pgfqpoint{1.542500in}{0.858462in}}%
\pgfpathlineto{\pgfqpoint{1.560000in}{0.806333in}}%
\pgfpathlineto{\pgfqpoint{1.577500in}{0.756600in}}%
\pgfpathlineto{\pgfqpoint{1.595000in}{0.709625in}}%
\pgfpathlineto{\pgfqpoint{1.612500in}{0.665680in}}%
\pgfpathlineto{\pgfqpoint{1.630000in}{0.624950in}}%
\pgfpathlineto{\pgfqpoint{1.647500in}{0.587539in}}%
\pgfpathlineto{\pgfqpoint{1.665000in}{0.553477in}}%
\pgfpathlineto{\pgfqpoint{1.682500in}{0.522731in}}%
\pgfpathlineto{\pgfqpoint{1.700000in}{0.495209in}}%
\pgfpathlineto{\pgfqpoint{1.717500in}{0.470777in}}%
\pgfpathlineto{\pgfqpoint{1.735000in}{0.449264in}}%
\pgfpathlineto{\pgfqpoint{1.752500in}{0.430471in}}%
\pgfpathlineto{\pgfqpoint{1.770000in}{0.414184in}}%
\pgfpathlineto{\pgfqpoint{1.787500in}{0.400178in}}%
\pgfpathlineto{\pgfqpoint{1.805000in}{0.388227in}}%
\pgfpathlineto{\pgfqpoint{1.822500in}{0.378106in}}%
\pgfpathlineto{\pgfqpoint{1.840000in}{0.369600in}}%
\pgfpathlineto{\pgfqpoint{1.857500in}{0.362505in}}%
\pgfpathlineto{\pgfqpoint{1.875000in}{0.356631in}}%
\pgfpathlineto{\pgfqpoint{1.892500in}{0.351803in}}%
\pgfpathlineto{\pgfqpoint{1.910000in}{0.347864in}}%
\pgfpathlineto{\pgfqpoint{1.927500in}{0.344673in}}%
\pgfpathlineto{\pgfqpoint{1.945000in}{0.342108in}}%
\pgfpathlineto{\pgfqpoint{1.962500in}{0.340060in}}%
\pgfpathlineto{\pgfqpoint{1.980000in}{0.338437in}}%
\pgfpathlineto{\pgfqpoint{1.997500in}{0.337160in}}%
\pgfpathlineto{\pgfqpoint{2.015000in}{0.336162in}}%
\pgfpathlineto{\pgfqpoint{2.032500in}{0.335388in}}%
\pgfpathlineto{\pgfqpoint{2.050000in}{0.334792in}}%
\pgfpathlineto{\pgfqpoint{2.067500in}{0.334336in}}%
\pgfpathlineto{\pgfqpoint{2.085000in}{0.333989in}}%
\pgfpathlineto{\pgfqpoint{2.102500in}{0.333728in}}%
\pgfpathlineto{\pgfqpoint{2.120000in}{0.333532in}}%
\pgfpathlineto{\pgfqpoint{2.137500in}{0.333386in}}%
\pgfusepath{stroke}%
\end{pgfscope}%
\begin{pgfscope}%
\pgfsetrectcap%
\pgfsetmiterjoin%
\pgfsetlinewidth{0.803000pt}%
\definecolor{currentstroke}{rgb}{0.000000,0.000000,0.000000}%
\pgfsetstrokecolor{currentstroke}%
\pgfsetdash{}{0pt}%
\pgfpathmoveto{\pgfqpoint{0.405000in}{0.333000in}}%
\pgfpathlineto{\pgfqpoint{0.405000in}{1.776000in}}%
\pgfusepath{stroke}%
\end{pgfscope}%
\begin{pgfscope}%
\pgfsetrectcap%
\pgfsetmiterjoin%
\pgfsetlinewidth{0.803000pt}%
\definecolor{currentstroke}{rgb}{0.000000,0.000000,0.000000}%
\pgfsetstrokecolor{currentstroke}%
\pgfsetdash{}{0pt}%
\pgfpathmoveto{\pgfqpoint{2.137500in}{0.333000in}}%
\pgfpathlineto{\pgfqpoint{2.137500in}{1.776000in}}%
\pgfusepath{stroke}%
\end{pgfscope}%
\begin{pgfscope}%
\pgfsetrectcap%
\pgfsetmiterjoin%
\pgfsetlinewidth{0.803000pt}%
\definecolor{currentstroke}{rgb}{0.000000,0.000000,0.000000}%
\pgfsetstrokecolor{currentstroke}%
\pgfsetdash{}{0pt}%
\pgfpathmoveto{\pgfqpoint{0.405000in}{0.333000in}}%
\pgfpathlineto{\pgfqpoint{2.137500in}{0.333000in}}%
\pgfusepath{stroke}%
\end{pgfscope}%
\begin{pgfscope}%
\pgfsetrectcap%
\pgfsetmiterjoin%
\pgfsetlinewidth{0.803000pt}%
\definecolor{currentstroke}{rgb}{0.000000,0.000000,0.000000}%
\pgfsetstrokecolor{currentstroke}%
\pgfsetdash{}{0pt}%
\pgfpathmoveto{\pgfqpoint{0.405000in}{1.776000in}}%
\pgfpathlineto{\pgfqpoint{2.137500in}{1.776000in}}%
\pgfusepath{stroke}%
\end{pgfscope}%
\end{pgfpicture}%
\makeatother%
\endgroup%
}{
  Valori tabulati per l'integrale di una distribuzione Gaussiana in forma
  standard corrispondente all'ombreggiatura in figura
  \begin{align*}
    P = \Phi(z) - \frac{1}{2} =
    \frac{1}{2}\,\erf{\frac{z}{\sqrt{2}}}.
  \end{align*}
  Nella tabella la prima colonna identifica le prime due cifre di $z$, mentre
  la riga di intestazione identifica la seconda cifra decimale.
}

\vspace*{\fill}

\begin{table}[!hb]
  \begin{center}
    {\small
      \begin{tabular*}{\textwidth}{@{ \extracolsep{\fill}}ccccccccccc}
        \hline
        $z$ & $0$ & $1$ & $2$ & $3$ & $4$ & $5$ & $6$ & $7$ & $8$ & $9$ \\
\hline
\hline
$0.0$ & $0.00000$ & $0.00399$ & $0.00798$ & $0.01197$ & $0.01595$ & $0.01994$ & $0.02392$ & $0.02790$ & $0.03188$ & $0.03586$ \\
$0.1$ & $0.03983$ & $0.04380$ & $0.04776$ & $0.05172$ & $0.05567$ & $0.05962$ & $0.06356$ & $0.06749$ & $0.07142$ & $0.07535$ \\
$0.2$ & $0.07926$ & $0.08317$ & $0.08706$ & $0.09095$ & $0.09483$ & $0.09871$ & $0.10257$ & $0.10642$ & $0.11026$ & $0.11409$ \\
$0.3$ & $0.11791$ & $0.12172$ & $0.12552$ & $0.12930$ & $0.13307$ & $0.13683$ & $0.14058$ & $0.14431$ & $0.14803$ & $0.15173$ \\
$0.4$ & $0.15542$ & $0.15910$ & $0.16276$ & $0.16640$ & $0.17003$ & $0.17364$ & $0.17724$ & $0.18082$ & $0.18439$ & $0.18793$ \\
$0.5$ & $0.19146$ & $0.19497$ & $0.19847$ & $0.20194$ & $0.20540$ & $0.20884$ & $0.21226$ & $0.21566$ & $0.21904$ & $0.22240$ \\
$0.6$ & $0.22575$ & $0.22907$ & $0.23237$ & $0.23565$ & $0.23891$ & $0.24215$ & $0.24537$ & $0.24857$ & $0.25175$ & $0.25490$ \\
$0.7$ & $0.25804$ & $0.26115$ & $0.26424$ & $0.26730$ & $0.27035$ & $0.27337$ & $0.27637$ & $0.27935$ & $0.28230$ & $0.28524$ \\
$0.8$ & $0.28814$ & $0.29103$ & $0.29389$ & $0.29673$ & $0.29955$ & $0.30234$ & $0.30511$ & $0.30785$ & $0.31057$ & $0.31327$ \\
$0.9$ & $0.31594$ & $0.31859$ & $0.32121$ & $0.32381$ & $0.32639$ & $0.32894$ & $0.33147$ & $0.33398$ & $0.33646$ & $0.33891$ \\
$1.0$ & $0.34134$ & $0.34375$ & $0.34614$ & $0.34849$ & $0.35083$ & $0.35314$ & $0.35543$ & $0.35769$ & $0.35993$ & $0.36214$ \\
$1.1$ & $0.36433$ & $0.36650$ & $0.36864$ & $0.37076$ & $0.37286$ & $0.37493$ & $0.37698$ & $0.37900$ & $0.38100$ & $0.38298$ \\
$1.2$ & $0.38493$ & $0.38686$ & $0.38877$ & $0.39065$ & $0.39251$ & $0.39435$ & $0.39617$ & $0.39796$ & $0.39973$ & $0.40147$ \\
$1.3$ & $0.40320$ & $0.40490$ & $0.40658$ & $0.40824$ & $0.40988$ & $0.41149$ & $0.41309$ & $0.41466$ & $0.41621$ & $0.41774$ \\
$1.4$ & $0.41924$ & $0.42073$ & $0.42220$ & $0.42364$ & $0.42507$ & $0.42647$ & $0.42785$ & $0.42922$ & $0.43056$ & $0.43189$ \\
$1.5$ & $0.43319$ & $0.43448$ & $0.43574$ & $0.43699$ & $0.43822$ & $0.43943$ & $0.44062$ & $0.44179$ & $0.44295$ & $0.44408$ \\
$1.6$ & $0.44520$ & $0.44630$ & $0.44738$ & $0.44845$ & $0.44950$ & $0.45053$ & $0.45154$ & $0.45254$ & $0.45352$ & $0.45449$ \\
$1.7$ & $0.45543$ & $0.45637$ & $0.45728$ & $0.45818$ & $0.45907$ & $0.45994$ & $0.46080$ & $0.46164$ & $0.46246$ & $0.46327$ \\
$1.8$ & $0.46407$ & $0.46485$ & $0.46562$ & $0.46638$ & $0.46712$ & $0.46784$ & $0.46856$ & $0.46926$ & $0.46995$ & $0.47062$ \\
$1.9$ & $0.47128$ & $0.47193$ & $0.47257$ & $0.47320$ & $0.47381$ & $0.47441$ & $0.47500$ & $0.47558$ & $0.47615$ & $0.47670$ \\
$2.0$ & $0.47725$ & $0.47778$ & $0.47831$ & $0.47882$ & $0.47932$ & $0.47982$ & $0.48030$ & $0.48077$ & $0.48124$ & $0.48169$ \\
$2.1$ & $0.48214$ & $0.48257$ & $0.48300$ & $0.48341$ & $0.48382$ & $0.48422$ & $0.48461$ & $0.48500$ & $0.48537$ & $0.48574$ \\
$2.2$ & $0.48610$ & $0.48645$ & $0.48679$ & $0.48713$ & $0.48745$ & $0.48778$ & $0.48809$ & $0.48840$ & $0.48870$ & $0.48899$ \\
$2.3$ & $0.48928$ & $0.48956$ & $0.48983$ & $0.49010$ & $0.49036$ & $0.49061$ & $0.49086$ & $0.49111$ & $0.49134$ & $0.49158$ \\
$2.4$ & $0.49180$ & $0.49202$ & $0.49224$ & $0.49245$ & $0.49266$ & $0.49286$ & $0.49305$ & $0.49324$ & $0.49343$ & $0.49361$ \\
$2.5$ & $0.49379$ & $0.49396$ & $0.49413$ & $0.49430$ & $0.49446$ & $0.49461$ & $0.49477$ & $0.49492$ & $0.49506$ & $0.49520$ \\
$2.6$ & $0.49534$ & $0.49547$ & $0.49560$ & $0.49573$ & $0.49585$ & $0.49598$ & $0.49609$ & $0.49621$ & $0.49632$ & $0.49643$ \\
$2.7$ & $0.49653$ & $0.49664$ & $0.49674$ & $0.49683$ & $0.49693$ & $0.49702$ & $0.49711$ & $0.49720$ & $0.49728$ & $0.49736$ \\
$2.8$ & $0.49744$ & $0.49752$ & $0.49760$ & $0.49767$ & $0.49774$ & $0.49781$ & $0.49788$ & $0.49795$ & $0.49801$ & $0.49807$ \\
$2.9$ & $0.49813$ & $0.49819$ & $0.49825$ & $0.49831$ & $0.49836$ & $0.49841$ & $0.49846$ & $0.49851$ & $0.49856$ & $0.49861$ \\
$3.0$ & $0.49865$ & $0.49869$ & $0.49874$ & $0.49878$ & $0.49882$ & $0.49886$ & $0.49889$ & $0.49893$ & $0.49896$ & $0.49900$ \\
$3.1$ & $0.49903$ & $0.49906$ & $0.49910$ & $0.49913$ & $0.49916$ & $0.49918$ & $0.49921$ & $0.49924$ & $0.49926$ & $0.49929$ \\
$3.2$ & $0.49931$ & $0.49934$ & $0.49936$ & $0.49938$ & $0.49940$ & $0.49942$ & $0.49944$ & $0.49946$ & $0.49948$ & $0.49950$ \\
$3.3$ & $0.49952$ & $0.49953$ & $0.49955$ & $0.49957$ & $0.49958$ & $0.49960$ & $0.49961$ & $0.49962$ & $0.49964$ & $0.49965$ \\
$3.4$ & $0.49966$ & $0.49968$ & $0.49969$ & $0.49970$ & $0.49971$ & $0.49972$ & $0.49973$ & $0.49974$ & $0.49975$ & $0.49976$ \\
$3.5$ & $0.49977$ & $0.49978$ & $0.49978$ & $0.49979$ & $0.49980$ & $0.49981$ & $0.49981$ & $0.49982$ & $0.49983$ & $0.49983$ \\
$3.6$ & $0.49984$ & $0.49985$ & $0.49985$ & $0.49986$ & $0.49986$ & $0.49987$ & $0.49987$ & $0.49988$ & $0.49988$ & $0.49989$ \\
$3.7$ & $0.49989$ & $0.49990$ & $0.49990$ & $0.49990$ & $0.49991$ & $0.49991$ & $0.49992$ & $0.49992$ & $0.49992$ & $0.49992$ \\
$3.8$ & $0.49993$ & $0.49993$ & $0.49993$ & $0.49994$ & $0.49994$ & $0.49994$ & $0.49994$ & $0.49995$ & $0.49995$ & $0.49995$ \\
$3.9$ & $0.49995$ & $0.49995$ & $0.49996$ & $0.49996$ & $0.49996$ & $0.49996$ & $0.49996$ & $0.49996$ & $0.49997$ & $0.49997$\\
        \hline
      \end{tabular*}
    }
  \end{center}
\end{table}

\clearpage

\section{Integrale normale degli errori---II}
\label{sec:tavola_erf5}

\autohstack[3pt]{%% Creator: Matplotlib, PGF backend
%%
%% To include the figure in your LaTeX document, write
%%   \input{<filename>.pgf}
%%
%% Make sure the required packages are loaded in your preamble
%%   \usepackage{pgf}
%%
%% and, on pdftex
%%   \usepackage[utf8]{inputenc}\DeclareUnicodeCharacter{2212}{-}
%%
%% or, on luatex and xetex
%%   \usepackage{unicode-math}
%%
%% Figures using additional raster images can only be included by \input if
%% they are in the same directory as the main LaTeX file. For loading figures
%% from other directories you can use the `import` package
%%   \usepackage{import}
%%
%% and then include the figures with
%%   \import{<path to file>}{<filename>.pgf}
%%
%% Matplotlib used the following preamble
%%   \usepackage[nice]{nicefrac}
%%   \usepackage{amsmath}
%%   \DeclareUnicodeCharacter{2212}{-}
%%
\begingroup%
\makeatletter%
\begin{pgfpicture}%
\pgfpathrectangle{\pgfpointorigin}{\pgfqpoint{2.250000in}{1.850000in}}%
\pgfusepath{use as bounding box, clip}%
\begin{pgfscope}%
\pgfsetbuttcap%
\pgfsetmiterjoin%
\definecolor{currentfill}{rgb}{1.000000,1.000000,1.000000}%
\pgfsetfillcolor{currentfill}%
\pgfsetlinewidth{0.000000pt}%
\definecolor{currentstroke}{rgb}{1.000000,1.000000,1.000000}%
\pgfsetstrokecolor{currentstroke}%
\pgfsetdash{}{0pt}%
\pgfpathmoveto{\pgfqpoint{0.000000in}{0.000000in}}%
\pgfpathlineto{\pgfqpoint{2.250000in}{0.000000in}}%
\pgfpathlineto{\pgfqpoint{2.250000in}{1.850000in}}%
\pgfpathlineto{\pgfqpoint{0.000000in}{1.850000in}}%
\pgfpathclose%
\pgfusepath{fill}%
\end{pgfscope}%
\begin{pgfscope}%
\pgfsetbuttcap%
\pgfsetmiterjoin%
\definecolor{currentfill}{rgb}{1.000000,1.000000,1.000000}%
\pgfsetfillcolor{currentfill}%
\pgfsetlinewidth{0.000000pt}%
\definecolor{currentstroke}{rgb}{0.000000,0.000000,0.000000}%
\pgfsetstrokecolor{currentstroke}%
\pgfsetstrokeopacity{0.000000}%
\pgfsetdash{}{0pt}%
\pgfpathmoveto{\pgfqpoint{0.405000in}{0.333000in}}%
\pgfpathlineto{\pgfqpoint{2.137500in}{0.333000in}}%
\pgfpathlineto{\pgfqpoint{2.137500in}{1.776000in}}%
\pgfpathlineto{\pgfqpoint{0.405000in}{1.776000in}}%
\pgfpathclose%
\pgfusepath{fill}%
\end{pgfscope}%
\begin{pgfscope}%
\pgfpathrectangle{\pgfqpoint{0.405000in}{0.333000in}}{\pgfqpoint{1.732500in}{1.443000in}}%
\pgfusepath{clip}%
\pgfsetbuttcap%
\pgfsetroundjoin%
\definecolor{currentfill}{rgb}{0.501961,0.501961,0.501961}%
\pgfsetfillcolor{currentfill}%
\pgfsetfillopacity{0.500000}%
\pgfsetlinewidth{0.000000pt}%
\definecolor{currentstroke}{rgb}{0.000000,0.000000,0.000000}%
\pgfsetstrokecolor{currentstroke}%
\pgfsetdash{}{0pt}%
\pgfpathmoveto{\pgfqpoint{1.612500in}{0.665680in}}%
\pgfpathlineto{\pgfqpoint{1.612500in}{0.333000in}}%
\pgfpathlineto{\pgfqpoint{1.630000in}{0.333000in}}%
\pgfpathlineto{\pgfqpoint{1.647500in}{0.333000in}}%
\pgfpathlineto{\pgfqpoint{1.665000in}{0.333000in}}%
\pgfpathlineto{\pgfqpoint{1.682500in}{0.333000in}}%
\pgfpathlineto{\pgfqpoint{1.700000in}{0.333000in}}%
\pgfpathlineto{\pgfqpoint{1.717500in}{0.333000in}}%
\pgfpathlineto{\pgfqpoint{1.735000in}{0.333000in}}%
\pgfpathlineto{\pgfqpoint{1.752500in}{0.333000in}}%
\pgfpathlineto{\pgfqpoint{1.770000in}{0.333000in}}%
\pgfpathlineto{\pgfqpoint{1.787500in}{0.333000in}}%
\pgfpathlineto{\pgfqpoint{1.805000in}{0.333000in}}%
\pgfpathlineto{\pgfqpoint{1.822500in}{0.333000in}}%
\pgfpathlineto{\pgfqpoint{1.840000in}{0.333000in}}%
\pgfpathlineto{\pgfqpoint{1.857500in}{0.333000in}}%
\pgfpathlineto{\pgfqpoint{1.875000in}{0.333000in}}%
\pgfpathlineto{\pgfqpoint{1.892500in}{0.333000in}}%
\pgfpathlineto{\pgfqpoint{1.910000in}{0.333000in}}%
\pgfpathlineto{\pgfqpoint{1.927500in}{0.333000in}}%
\pgfpathlineto{\pgfqpoint{1.945000in}{0.333000in}}%
\pgfpathlineto{\pgfqpoint{1.962500in}{0.333000in}}%
\pgfpathlineto{\pgfqpoint{1.980000in}{0.333000in}}%
\pgfpathlineto{\pgfqpoint{1.997500in}{0.333000in}}%
\pgfpathlineto{\pgfqpoint{2.015000in}{0.333000in}}%
\pgfpathlineto{\pgfqpoint{2.032500in}{0.333000in}}%
\pgfpathlineto{\pgfqpoint{2.050000in}{0.333000in}}%
\pgfpathlineto{\pgfqpoint{2.067500in}{0.333000in}}%
\pgfpathlineto{\pgfqpoint{2.085000in}{0.333000in}}%
\pgfpathlineto{\pgfqpoint{2.102500in}{0.333000in}}%
\pgfpathlineto{\pgfqpoint{2.120000in}{0.333000in}}%
\pgfpathlineto{\pgfqpoint{2.137500in}{0.333000in}}%
\pgfpathlineto{\pgfqpoint{2.137500in}{0.333386in}}%
\pgfpathlineto{\pgfqpoint{2.137500in}{0.333386in}}%
\pgfpathlineto{\pgfqpoint{2.120000in}{0.333532in}}%
\pgfpathlineto{\pgfqpoint{2.102500in}{0.333728in}}%
\pgfpathlineto{\pgfqpoint{2.085000in}{0.333989in}}%
\pgfpathlineto{\pgfqpoint{2.067500in}{0.334336in}}%
\pgfpathlineto{\pgfqpoint{2.050000in}{0.334792in}}%
\pgfpathlineto{\pgfqpoint{2.032500in}{0.335388in}}%
\pgfpathlineto{\pgfqpoint{2.015000in}{0.336162in}}%
\pgfpathlineto{\pgfqpoint{1.997500in}{0.337160in}}%
\pgfpathlineto{\pgfqpoint{1.980000in}{0.338437in}}%
\pgfpathlineto{\pgfqpoint{1.962500in}{0.340060in}}%
\pgfpathlineto{\pgfqpoint{1.945000in}{0.342108in}}%
\pgfpathlineto{\pgfqpoint{1.927500in}{0.344673in}}%
\pgfpathlineto{\pgfqpoint{1.910000in}{0.347864in}}%
\pgfpathlineto{\pgfqpoint{1.892500in}{0.351803in}}%
\pgfpathlineto{\pgfqpoint{1.875000in}{0.356631in}}%
\pgfpathlineto{\pgfqpoint{1.857500in}{0.362505in}}%
\pgfpathlineto{\pgfqpoint{1.840000in}{0.369600in}}%
\pgfpathlineto{\pgfqpoint{1.822500in}{0.378106in}}%
\pgfpathlineto{\pgfqpoint{1.805000in}{0.388227in}}%
\pgfpathlineto{\pgfqpoint{1.787500in}{0.400178in}}%
\pgfpathlineto{\pgfqpoint{1.770000in}{0.414184in}}%
\pgfpathlineto{\pgfqpoint{1.752500in}{0.430471in}}%
\pgfpathlineto{\pgfqpoint{1.735000in}{0.449264in}}%
\pgfpathlineto{\pgfqpoint{1.717500in}{0.470777in}}%
\pgfpathlineto{\pgfqpoint{1.700000in}{0.495209in}}%
\pgfpathlineto{\pgfqpoint{1.682500in}{0.522731in}}%
\pgfpathlineto{\pgfqpoint{1.665000in}{0.553477in}}%
\pgfpathlineto{\pgfqpoint{1.647500in}{0.587539in}}%
\pgfpathlineto{\pgfqpoint{1.630000in}{0.624950in}}%
\pgfpathlineto{\pgfqpoint{1.612500in}{0.665680in}}%
\pgfpathclose%
\pgfusepath{fill}%
\end{pgfscope}%
\begin{pgfscope}%
\pgfsetbuttcap%
\pgfsetroundjoin%
\definecolor{currentfill}{rgb}{0.000000,0.000000,0.000000}%
\pgfsetfillcolor{currentfill}%
\pgfsetlinewidth{0.803000pt}%
\definecolor{currentstroke}{rgb}{0.000000,0.000000,0.000000}%
\pgfsetstrokecolor{currentstroke}%
\pgfsetdash{}{0pt}%
\pgfsys@defobject{currentmarker}{\pgfqpoint{0.000000in}{-0.048611in}}{\pgfqpoint{0.000000in}{0.000000in}}{%
\pgfpathmoveto{\pgfqpoint{0.000000in}{0.000000in}}%
\pgfpathlineto{\pgfqpoint{0.000000in}{-0.048611in}}%
\pgfusepath{stroke,fill}%
}%
\begin{pgfscope}%
\pgfsys@transformshift{0.405000in}{0.333000in}%
\pgfsys@useobject{currentmarker}{}%
\end{pgfscope}%
\end{pgfscope}%
\begin{pgfscope}%
\definecolor{textcolor}{rgb}{0.000000,0.000000,0.000000}%
\pgfsetstrokecolor{textcolor}%
\pgfsetfillcolor{textcolor}%
\pgftext[x=0.405000in,y=0.235778in,,top]{\color{textcolor}\rmfamily\fontsize{9.000000}{10.800000}\selectfont −4}%
\end{pgfscope}%
\begin{pgfscope}%
\pgfsetbuttcap%
\pgfsetroundjoin%
\definecolor{currentfill}{rgb}{0.000000,0.000000,0.000000}%
\pgfsetfillcolor{currentfill}%
\pgfsetlinewidth{0.803000pt}%
\definecolor{currentstroke}{rgb}{0.000000,0.000000,0.000000}%
\pgfsetstrokecolor{currentstroke}%
\pgfsetdash{}{0pt}%
\pgfsys@defobject{currentmarker}{\pgfqpoint{0.000000in}{-0.048611in}}{\pgfqpoint{0.000000in}{0.000000in}}{%
\pgfpathmoveto{\pgfqpoint{0.000000in}{0.000000in}}%
\pgfpathlineto{\pgfqpoint{0.000000in}{-0.048611in}}%
\pgfusepath{stroke,fill}%
}%
\begin{pgfscope}%
\pgfsys@transformshift{0.838125in}{0.333000in}%
\pgfsys@useobject{currentmarker}{}%
\end{pgfscope}%
\end{pgfscope}%
\begin{pgfscope}%
\definecolor{textcolor}{rgb}{0.000000,0.000000,0.000000}%
\pgfsetstrokecolor{textcolor}%
\pgfsetfillcolor{textcolor}%
\pgftext[x=0.838125in,y=0.235778in,,top]{\color{textcolor}\rmfamily\fontsize{9.000000}{10.800000}\selectfont −2}%
\end{pgfscope}%
\begin{pgfscope}%
\pgfsetbuttcap%
\pgfsetroundjoin%
\definecolor{currentfill}{rgb}{0.000000,0.000000,0.000000}%
\pgfsetfillcolor{currentfill}%
\pgfsetlinewidth{0.803000pt}%
\definecolor{currentstroke}{rgb}{0.000000,0.000000,0.000000}%
\pgfsetstrokecolor{currentstroke}%
\pgfsetdash{}{0pt}%
\pgfsys@defobject{currentmarker}{\pgfqpoint{0.000000in}{-0.048611in}}{\pgfqpoint{0.000000in}{0.000000in}}{%
\pgfpathmoveto{\pgfqpoint{0.000000in}{0.000000in}}%
\pgfpathlineto{\pgfqpoint{0.000000in}{-0.048611in}}%
\pgfusepath{stroke,fill}%
}%
\begin{pgfscope}%
\pgfsys@transformshift{1.271250in}{0.333000in}%
\pgfsys@useobject{currentmarker}{}%
\end{pgfscope}%
\end{pgfscope}%
\begin{pgfscope}%
\definecolor{textcolor}{rgb}{0.000000,0.000000,0.000000}%
\pgfsetstrokecolor{textcolor}%
\pgfsetfillcolor{textcolor}%
\pgftext[x=1.271250in,y=0.235778in,,top]{\color{textcolor}\rmfamily\fontsize{9.000000}{10.800000}\selectfont 0}%
\end{pgfscope}%
\begin{pgfscope}%
\pgfsetbuttcap%
\pgfsetroundjoin%
\definecolor{currentfill}{rgb}{0.000000,0.000000,0.000000}%
\pgfsetfillcolor{currentfill}%
\pgfsetlinewidth{0.803000pt}%
\definecolor{currentstroke}{rgb}{0.000000,0.000000,0.000000}%
\pgfsetstrokecolor{currentstroke}%
\pgfsetdash{}{0pt}%
\pgfsys@defobject{currentmarker}{\pgfqpoint{0.000000in}{-0.048611in}}{\pgfqpoint{0.000000in}{0.000000in}}{%
\pgfpathmoveto{\pgfqpoint{0.000000in}{0.000000in}}%
\pgfpathlineto{\pgfqpoint{0.000000in}{-0.048611in}}%
\pgfusepath{stroke,fill}%
}%
\begin{pgfscope}%
\pgfsys@transformshift{1.704375in}{0.333000in}%
\pgfsys@useobject{currentmarker}{}%
\end{pgfscope}%
\end{pgfscope}%
\begin{pgfscope}%
\definecolor{textcolor}{rgb}{0.000000,0.000000,0.000000}%
\pgfsetstrokecolor{textcolor}%
\pgfsetfillcolor{textcolor}%
\pgftext[x=1.704375in,y=0.235778in,,top]{\color{textcolor}\rmfamily\fontsize{9.000000}{10.800000}\selectfont 2}%
\end{pgfscope}%
\begin{pgfscope}%
\pgfsetbuttcap%
\pgfsetroundjoin%
\definecolor{currentfill}{rgb}{0.000000,0.000000,0.000000}%
\pgfsetfillcolor{currentfill}%
\pgfsetlinewidth{0.803000pt}%
\definecolor{currentstroke}{rgb}{0.000000,0.000000,0.000000}%
\pgfsetstrokecolor{currentstroke}%
\pgfsetdash{}{0pt}%
\pgfsys@defobject{currentmarker}{\pgfqpoint{0.000000in}{-0.048611in}}{\pgfqpoint{0.000000in}{0.000000in}}{%
\pgfpathmoveto{\pgfqpoint{0.000000in}{0.000000in}}%
\pgfpathlineto{\pgfqpoint{0.000000in}{-0.048611in}}%
\pgfusepath{stroke,fill}%
}%
\begin{pgfscope}%
\pgfsys@transformshift{2.137500in}{0.333000in}%
\pgfsys@useobject{currentmarker}{}%
\end{pgfscope}%
\end{pgfscope}%
\begin{pgfscope}%
\definecolor{textcolor}{rgb}{0.000000,0.000000,0.000000}%
\pgfsetstrokecolor{textcolor}%
\pgfsetfillcolor{textcolor}%
\pgftext[x=2.137500in,y=0.235778in,,top]{\color{textcolor}\rmfamily\fontsize{9.000000}{10.800000}\selectfont 4}%
\end{pgfscope}%
\begin{pgfscope}%
\definecolor{textcolor}{rgb}{0.000000,0.000000,0.000000}%
\pgfsetstrokecolor{textcolor}%
\pgfsetfillcolor{textcolor}%
\pgftext[x=1.271250in,y=0.069111in,,top]{\color{textcolor}\rmfamily\fontsize{9.000000}{10.800000}\selectfont \(\displaystyle z\)}%
\end{pgfscope}%
\begin{pgfscope}%
\pgfsetbuttcap%
\pgfsetroundjoin%
\definecolor{currentfill}{rgb}{0.000000,0.000000,0.000000}%
\pgfsetfillcolor{currentfill}%
\pgfsetlinewidth{0.803000pt}%
\definecolor{currentstroke}{rgb}{0.000000,0.000000,0.000000}%
\pgfsetstrokecolor{currentstroke}%
\pgfsetdash{}{0pt}%
\pgfsys@defobject{currentmarker}{\pgfqpoint{-0.048611in}{0.000000in}}{\pgfqpoint{0.000000in}{0.000000in}}{%
\pgfpathmoveto{\pgfqpoint{0.000000in}{0.000000in}}%
\pgfpathlineto{\pgfqpoint{-0.048611in}{0.000000in}}%
\pgfusepath{stroke,fill}%
}%
\begin{pgfscope}%
\pgfsys@transformshift{0.405000in}{0.333000in}%
\pgfsys@useobject{currentmarker}{}%
\end{pgfscope}%
\end{pgfscope}%
\begin{pgfscope}%
\definecolor{textcolor}{rgb}{0.000000,0.000000,0.000000}%
\pgfsetstrokecolor{textcolor}%
\pgfsetfillcolor{textcolor}%
\pgftext[x=0.143620in, y=0.289597in, left, base]{\color{textcolor}\rmfamily\fontsize{9.000000}{10.800000}\selectfont 0.0}%
\end{pgfscope}%
\begin{pgfscope}%
\pgfsetbuttcap%
\pgfsetroundjoin%
\definecolor{currentfill}{rgb}{0.000000,0.000000,0.000000}%
\pgfsetfillcolor{currentfill}%
\pgfsetlinewidth{0.803000pt}%
\definecolor{currentstroke}{rgb}{0.000000,0.000000,0.000000}%
\pgfsetstrokecolor{currentstroke}%
\pgfsetdash{}{0pt}%
\pgfsys@defobject{currentmarker}{\pgfqpoint{-0.048611in}{0.000000in}}{\pgfqpoint{0.000000in}{0.000000in}}{%
\pgfpathmoveto{\pgfqpoint{0.000000in}{0.000000in}}%
\pgfpathlineto{\pgfqpoint{-0.048611in}{0.000000in}}%
\pgfusepath{stroke,fill}%
}%
\begin{pgfscope}%
\pgfsys@transformshift{0.405000in}{0.621600in}%
\pgfsys@useobject{currentmarker}{}%
\end{pgfscope}%
\end{pgfscope}%
\begin{pgfscope}%
\definecolor{textcolor}{rgb}{0.000000,0.000000,0.000000}%
\pgfsetstrokecolor{textcolor}%
\pgfsetfillcolor{textcolor}%
\pgftext[x=0.143620in, y=0.578197in, left, base]{\color{textcolor}\rmfamily\fontsize{9.000000}{10.800000}\selectfont 0.1}%
\end{pgfscope}%
\begin{pgfscope}%
\pgfsetbuttcap%
\pgfsetroundjoin%
\definecolor{currentfill}{rgb}{0.000000,0.000000,0.000000}%
\pgfsetfillcolor{currentfill}%
\pgfsetlinewidth{0.803000pt}%
\definecolor{currentstroke}{rgb}{0.000000,0.000000,0.000000}%
\pgfsetstrokecolor{currentstroke}%
\pgfsetdash{}{0pt}%
\pgfsys@defobject{currentmarker}{\pgfqpoint{-0.048611in}{0.000000in}}{\pgfqpoint{0.000000in}{0.000000in}}{%
\pgfpathmoveto{\pgfqpoint{0.000000in}{0.000000in}}%
\pgfpathlineto{\pgfqpoint{-0.048611in}{0.000000in}}%
\pgfusepath{stroke,fill}%
}%
\begin{pgfscope}%
\pgfsys@transformshift{0.405000in}{0.910200in}%
\pgfsys@useobject{currentmarker}{}%
\end{pgfscope}%
\end{pgfscope}%
\begin{pgfscope}%
\definecolor{textcolor}{rgb}{0.000000,0.000000,0.000000}%
\pgfsetstrokecolor{textcolor}%
\pgfsetfillcolor{textcolor}%
\pgftext[x=0.143620in, y=0.866797in, left, base]{\color{textcolor}\rmfamily\fontsize{9.000000}{10.800000}\selectfont 0.2}%
\end{pgfscope}%
\begin{pgfscope}%
\pgfsetbuttcap%
\pgfsetroundjoin%
\definecolor{currentfill}{rgb}{0.000000,0.000000,0.000000}%
\pgfsetfillcolor{currentfill}%
\pgfsetlinewidth{0.803000pt}%
\definecolor{currentstroke}{rgb}{0.000000,0.000000,0.000000}%
\pgfsetstrokecolor{currentstroke}%
\pgfsetdash{}{0pt}%
\pgfsys@defobject{currentmarker}{\pgfqpoint{-0.048611in}{0.000000in}}{\pgfqpoint{0.000000in}{0.000000in}}{%
\pgfpathmoveto{\pgfqpoint{0.000000in}{0.000000in}}%
\pgfpathlineto{\pgfqpoint{-0.048611in}{0.000000in}}%
\pgfusepath{stroke,fill}%
}%
\begin{pgfscope}%
\pgfsys@transformshift{0.405000in}{1.198800in}%
\pgfsys@useobject{currentmarker}{}%
\end{pgfscope}%
\end{pgfscope}%
\begin{pgfscope}%
\definecolor{textcolor}{rgb}{0.000000,0.000000,0.000000}%
\pgfsetstrokecolor{textcolor}%
\pgfsetfillcolor{textcolor}%
\pgftext[x=0.143620in, y=1.155397in, left, base]{\color{textcolor}\rmfamily\fontsize{9.000000}{10.800000}\selectfont 0.3}%
\end{pgfscope}%
\begin{pgfscope}%
\pgfsetbuttcap%
\pgfsetroundjoin%
\definecolor{currentfill}{rgb}{0.000000,0.000000,0.000000}%
\pgfsetfillcolor{currentfill}%
\pgfsetlinewidth{0.803000pt}%
\definecolor{currentstroke}{rgb}{0.000000,0.000000,0.000000}%
\pgfsetstrokecolor{currentstroke}%
\pgfsetdash{}{0pt}%
\pgfsys@defobject{currentmarker}{\pgfqpoint{-0.048611in}{0.000000in}}{\pgfqpoint{0.000000in}{0.000000in}}{%
\pgfpathmoveto{\pgfqpoint{0.000000in}{0.000000in}}%
\pgfpathlineto{\pgfqpoint{-0.048611in}{0.000000in}}%
\pgfusepath{stroke,fill}%
}%
\begin{pgfscope}%
\pgfsys@transformshift{0.405000in}{1.487400in}%
\pgfsys@useobject{currentmarker}{}%
\end{pgfscope}%
\end{pgfscope}%
\begin{pgfscope}%
\definecolor{textcolor}{rgb}{0.000000,0.000000,0.000000}%
\pgfsetstrokecolor{textcolor}%
\pgfsetfillcolor{textcolor}%
\pgftext[x=0.143620in, y=1.443997in, left, base]{\color{textcolor}\rmfamily\fontsize{9.000000}{10.800000}\selectfont 0.4}%
\end{pgfscope}%
\begin{pgfscope}%
\pgfsetbuttcap%
\pgfsetroundjoin%
\definecolor{currentfill}{rgb}{0.000000,0.000000,0.000000}%
\pgfsetfillcolor{currentfill}%
\pgfsetlinewidth{0.803000pt}%
\definecolor{currentstroke}{rgb}{0.000000,0.000000,0.000000}%
\pgfsetstrokecolor{currentstroke}%
\pgfsetdash{}{0pt}%
\pgfsys@defobject{currentmarker}{\pgfqpoint{-0.048611in}{0.000000in}}{\pgfqpoint{0.000000in}{0.000000in}}{%
\pgfpathmoveto{\pgfqpoint{0.000000in}{0.000000in}}%
\pgfpathlineto{\pgfqpoint{-0.048611in}{0.000000in}}%
\pgfusepath{stroke,fill}%
}%
\begin{pgfscope}%
\pgfsys@transformshift{0.405000in}{1.776000in}%
\pgfsys@useobject{currentmarker}{}%
\end{pgfscope}%
\end{pgfscope}%
\begin{pgfscope}%
\definecolor{textcolor}{rgb}{0.000000,0.000000,0.000000}%
\pgfsetstrokecolor{textcolor}%
\pgfsetfillcolor{textcolor}%
\pgftext[x=0.143620in, y=1.732597in, left, base]{\color{textcolor}\rmfamily\fontsize{9.000000}{10.800000}\selectfont 0.5}%
\end{pgfscope}%
\begin{pgfscope}%
\definecolor{textcolor}{rgb}{0.000000,0.000000,0.000000}%
\pgfsetstrokecolor{textcolor}%
\pgfsetfillcolor{textcolor}%
\pgftext[x=0.088064in,y=1.054500in,,bottom,rotate=90.000000]{\color{textcolor}\rmfamily\fontsize{9.000000}{10.800000}\selectfont \(\displaystyle N(z)\)}%
\end{pgfscope}%
\begin{pgfscope}%
\pgfpathrectangle{\pgfqpoint{0.405000in}{0.333000in}}{\pgfqpoint{1.732500in}{1.443000in}}%
\pgfusepath{clip}%
\pgfsetrectcap%
\pgfsetroundjoin%
\pgfsetlinewidth{1.254687pt}%
\definecolor{currentstroke}{rgb}{0.000000,0.000000,0.000000}%
\pgfsetstrokecolor{currentstroke}%
\pgfsetdash{}{0pt}%
\pgfpathmoveto{\pgfqpoint{0.405000in}{0.333386in}}%
\pgfpathlineto{\pgfqpoint{0.422500in}{0.333532in}}%
\pgfpathlineto{\pgfqpoint{0.440000in}{0.333728in}}%
\pgfpathlineto{\pgfqpoint{0.457500in}{0.333989in}}%
\pgfpathlineto{\pgfqpoint{0.475000in}{0.334336in}}%
\pgfpathlineto{\pgfqpoint{0.492500in}{0.334792in}}%
\pgfpathlineto{\pgfqpoint{0.510000in}{0.335388in}}%
\pgfpathlineto{\pgfqpoint{0.527500in}{0.336162in}}%
\pgfpathlineto{\pgfqpoint{0.545000in}{0.337160in}}%
\pgfpathlineto{\pgfqpoint{0.562500in}{0.338437in}}%
\pgfpathlineto{\pgfqpoint{0.580000in}{0.340060in}}%
\pgfpathlineto{\pgfqpoint{0.597500in}{0.342108in}}%
\pgfpathlineto{\pgfqpoint{0.615000in}{0.344673in}}%
\pgfpathlineto{\pgfqpoint{0.632500in}{0.347864in}}%
\pgfpathlineto{\pgfqpoint{0.650000in}{0.351803in}}%
\pgfpathlineto{\pgfqpoint{0.667500in}{0.356631in}}%
\pgfpathlineto{\pgfqpoint{0.685000in}{0.362505in}}%
\pgfpathlineto{\pgfqpoint{0.702500in}{0.369600in}}%
\pgfpathlineto{\pgfqpoint{0.720000in}{0.378106in}}%
\pgfpathlineto{\pgfqpoint{0.737500in}{0.388227in}}%
\pgfpathlineto{\pgfqpoint{0.755000in}{0.400178in}}%
\pgfpathlineto{\pgfqpoint{0.772500in}{0.414184in}}%
\pgfpathlineto{\pgfqpoint{0.790000in}{0.430471in}}%
\pgfpathlineto{\pgfqpoint{0.807500in}{0.449264in}}%
\pgfpathlineto{\pgfqpoint{0.825000in}{0.470777in}}%
\pgfpathlineto{\pgfqpoint{0.842500in}{0.495209in}}%
\pgfpathlineto{\pgfqpoint{0.860000in}{0.522731in}}%
\pgfpathlineto{\pgfqpoint{0.877500in}{0.553477in}}%
\pgfpathlineto{\pgfqpoint{0.895000in}{0.587539in}}%
\pgfpathlineto{\pgfqpoint{0.912500in}{0.624950in}}%
\pgfpathlineto{\pgfqpoint{0.930000in}{0.665680in}}%
\pgfpathlineto{\pgfqpoint{0.947500in}{0.709625in}}%
\pgfpathlineto{\pgfqpoint{0.965000in}{0.756600in}}%
\pgfpathlineto{\pgfqpoint{0.982500in}{0.806333in}}%
\pgfpathlineto{\pgfqpoint{1.000000in}{0.858462in}}%
\pgfpathlineto{\pgfqpoint{1.017500in}{0.912536in}}%
\pgfpathlineto{\pgfqpoint{1.035000in}{0.968014in}}%
\pgfpathlineto{\pgfqpoint{1.052500in}{1.024274in}}%
\pgfpathlineto{\pgfqpoint{1.070000in}{1.080621in}}%
\pgfpathlineto{\pgfqpoint{1.087500in}{1.136297in}}%
\pgfpathlineto{\pgfqpoint{1.105000in}{1.190503in}}%
\pgfpathlineto{\pgfqpoint{1.122500in}{1.242408in}}%
\pgfpathlineto{\pgfqpoint{1.140000in}{1.291178in}}%
\pgfpathlineto{\pgfqpoint{1.157500in}{1.335992in}}%
\pgfpathlineto{\pgfqpoint{1.175000in}{1.376069in}}%
\pgfpathlineto{\pgfqpoint{1.192500in}{1.410687in}}%
\pgfpathlineto{\pgfqpoint{1.210000in}{1.439207in}}%
\pgfpathlineto{\pgfqpoint{1.227500in}{1.461091in}}%
\pgfpathlineto{\pgfqpoint{1.245000in}{1.475920in}}%
\pgfpathlineto{\pgfqpoint{1.262500in}{1.483408in}}%
\pgfpathlineto{\pgfqpoint{1.280000in}{1.483408in}}%
\pgfpathlineto{\pgfqpoint{1.297500in}{1.475920in}}%
\pgfpathlineto{\pgfqpoint{1.315000in}{1.461091in}}%
\pgfpathlineto{\pgfqpoint{1.332500in}{1.439207in}}%
\pgfpathlineto{\pgfqpoint{1.350000in}{1.410687in}}%
\pgfpathlineto{\pgfqpoint{1.367500in}{1.376069in}}%
\pgfpathlineto{\pgfqpoint{1.385000in}{1.335992in}}%
\pgfpathlineto{\pgfqpoint{1.402500in}{1.291178in}}%
\pgfpathlineto{\pgfqpoint{1.420000in}{1.242408in}}%
\pgfpathlineto{\pgfqpoint{1.437500in}{1.190503in}}%
\pgfpathlineto{\pgfqpoint{1.455000in}{1.136297in}}%
\pgfpathlineto{\pgfqpoint{1.472500in}{1.080621in}}%
\pgfpathlineto{\pgfqpoint{1.490000in}{1.024274in}}%
\pgfpathlineto{\pgfqpoint{1.507500in}{0.968014in}}%
\pgfpathlineto{\pgfqpoint{1.525000in}{0.912536in}}%
\pgfpathlineto{\pgfqpoint{1.542500in}{0.858462in}}%
\pgfpathlineto{\pgfqpoint{1.560000in}{0.806333in}}%
\pgfpathlineto{\pgfqpoint{1.577500in}{0.756600in}}%
\pgfpathlineto{\pgfqpoint{1.595000in}{0.709625in}}%
\pgfpathlineto{\pgfqpoint{1.612500in}{0.665680in}}%
\pgfpathlineto{\pgfqpoint{1.630000in}{0.624950in}}%
\pgfpathlineto{\pgfqpoint{1.647500in}{0.587539in}}%
\pgfpathlineto{\pgfqpoint{1.665000in}{0.553477in}}%
\pgfpathlineto{\pgfqpoint{1.682500in}{0.522731in}}%
\pgfpathlineto{\pgfqpoint{1.700000in}{0.495209in}}%
\pgfpathlineto{\pgfqpoint{1.717500in}{0.470777in}}%
\pgfpathlineto{\pgfqpoint{1.735000in}{0.449264in}}%
\pgfpathlineto{\pgfqpoint{1.752500in}{0.430471in}}%
\pgfpathlineto{\pgfqpoint{1.770000in}{0.414184in}}%
\pgfpathlineto{\pgfqpoint{1.787500in}{0.400178in}}%
\pgfpathlineto{\pgfqpoint{1.805000in}{0.388227in}}%
\pgfpathlineto{\pgfqpoint{1.822500in}{0.378106in}}%
\pgfpathlineto{\pgfqpoint{1.840000in}{0.369600in}}%
\pgfpathlineto{\pgfqpoint{1.857500in}{0.362505in}}%
\pgfpathlineto{\pgfqpoint{1.875000in}{0.356631in}}%
\pgfpathlineto{\pgfqpoint{1.892500in}{0.351803in}}%
\pgfpathlineto{\pgfqpoint{1.910000in}{0.347864in}}%
\pgfpathlineto{\pgfqpoint{1.927500in}{0.344673in}}%
\pgfpathlineto{\pgfqpoint{1.945000in}{0.342108in}}%
\pgfpathlineto{\pgfqpoint{1.962500in}{0.340060in}}%
\pgfpathlineto{\pgfqpoint{1.980000in}{0.338437in}}%
\pgfpathlineto{\pgfqpoint{1.997500in}{0.337160in}}%
\pgfpathlineto{\pgfqpoint{2.015000in}{0.336162in}}%
\pgfpathlineto{\pgfqpoint{2.032500in}{0.335388in}}%
\pgfpathlineto{\pgfqpoint{2.050000in}{0.334792in}}%
\pgfpathlineto{\pgfqpoint{2.067500in}{0.334336in}}%
\pgfpathlineto{\pgfqpoint{2.085000in}{0.333989in}}%
\pgfpathlineto{\pgfqpoint{2.102500in}{0.333728in}}%
\pgfpathlineto{\pgfqpoint{2.120000in}{0.333532in}}%
\pgfpathlineto{\pgfqpoint{2.137500in}{0.333386in}}%
\pgfusepath{stroke}%
\end{pgfscope}%
\begin{pgfscope}%
\pgfsetrectcap%
\pgfsetmiterjoin%
\pgfsetlinewidth{0.803000pt}%
\definecolor{currentstroke}{rgb}{0.000000,0.000000,0.000000}%
\pgfsetstrokecolor{currentstroke}%
\pgfsetdash{}{0pt}%
\pgfpathmoveto{\pgfqpoint{0.405000in}{0.333000in}}%
\pgfpathlineto{\pgfqpoint{0.405000in}{1.776000in}}%
\pgfusepath{stroke}%
\end{pgfscope}%
\begin{pgfscope}%
\pgfsetrectcap%
\pgfsetmiterjoin%
\pgfsetlinewidth{0.803000pt}%
\definecolor{currentstroke}{rgb}{0.000000,0.000000,0.000000}%
\pgfsetstrokecolor{currentstroke}%
\pgfsetdash{}{0pt}%
\pgfpathmoveto{\pgfqpoint{2.137500in}{0.333000in}}%
\pgfpathlineto{\pgfqpoint{2.137500in}{1.776000in}}%
\pgfusepath{stroke}%
\end{pgfscope}%
\begin{pgfscope}%
\pgfsetrectcap%
\pgfsetmiterjoin%
\pgfsetlinewidth{0.803000pt}%
\definecolor{currentstroke}{rgb}{0.000000,0.000000,0.000000}%
\pgfsetstrokecolor{currentstroke}%
\pgfsetdash{}{0pt}%
\pgfpathmoveto{\pgfqpoint{0.405000in}{0.333000in}}%
\pgfpathlineto{\pgfqpoint{2.137500in}{0.333000in}}%
\pgfusepath{stroke}%
\end{pgfscope}%
\begin{pgfscope}%
\pgfsetrectcap%
\pgfsetmiterjoin%
\pgfsetlinewidth{0.803000pt}%
\definecolor{currentstroke}{rgb}{0.000000,0.000000,0.000000}%
\pgfsetstrokecolor{currentstroke}%
\pgfsetdash{}{0pt}%
\pgfpathmoveto{\pgfqpoint{0.405000in}{1.776000in}}%
\pgfpathlineto{\pgfqpoint{2.137500in}{1.776000in}}%
\pgfusepath{stroke}%
\end{pgfscope}%
\end{pgfpicture}%
\makeatother%
\endgroup%
}{
  Valori tabulati per l'integrale di una distribuzione Gaussiana in forma
  standard corrispondente all'ombreggiatura in figura
  \begin{align*}
    P = 1 - \Phi(z) =
    \frac{1}{2} - \frac{1}{2}\,\erf{\frac{z}{\sqrt{2}}}.
  \end{align*}
  Nella tabella la prima colonna identifica le prime due cifre di $z$, mentre
  la riga di intestazione identifica la seconda cifra decimale.
}

\vspace*{\fill}

\begin{table}[!hb]
  \begin{center}
    {\small
      \begin{tabular*}{\textwidth}{@{ \extracolsep{\fill}}ccccccccccc}
        \hline
        $z$ & $0$ & $1$ & $2$ & $3$ & $4$ & $5$ & $6$ & $7$ & $8$ & $9$ \\
\hline
\hline
$4.0$ & $3.17$e-$05$ & $3.04$e-$05$ & $2.91$e-$05$ & $2.79$e-$05$ & $2.67$e-$05$ & $2.56$e-$05$ & $2.45$e-$05$ & $2.35$e-$05$ & $2.25$e-$05$ & $2.16$e-$05$ \\
$4.1$ & $2.07$e-$05$ & $1.98$e-$05$ & $1.89$e-$05$ & $1.81$e-$05$ & $1.74$e-$05$ & $1.66$e-$05$ & $1.59$e-$05$ & $1.52$e-$05$ & $1.46$e-$05$ & $1.39$e-$05$ \\
$4.2$ & $1.33$e-$05$ & $1.28$e-$05$ & $1.22$e-$05$ & $1.17$e-$05$ & $1.12$e-$05$ & $1.07$e-$05$ & $1.02$e-$05$ & $9.77$e-$06$ & $9.34$e-$06$ & $8.93$e-$06$ \\
$4.3$ & $8.54$e-$06$ & $8.16$e-$06$ & $7.80$e-$06$ & $7.46$e-$06$ & $7.12$e-$06$ & $6.81$e-$06$ & $6.50$e-$06$ & $6.21$e-$06$ & $5.93$e-$06$ & $5.67$e-$06$ \\
$4.4$ & $5.41$e-$06$ & $5.17$e-$06$ & $4.94$e-$06$ & $4.71$e-$06$ & $4.50$e-$06$ & $4.29$e-$06$ & $4.10$e-$06$ & $3.91$e-$06$ & $3.73$e-$06$ & $3.56$e-$06$ \\
$4.5$ & $3.40$e-$06$ & $3.24$e-$06$ & $3.09$e-$06$ & $2.95$e-$06$ & $2.81$e-$06$ & $2.68$e-$06$ & $2.56$e-$06$ & $2.44$e-$06$ & $2.32$e-$06$ & $2.22$e-$06$ \\
$4.6$ & $2.11$e-$06$ & $2.01$e-$06$ & $1.92$e-$06$ & $1.83$e-$06$ & $1.74$e-$06$ & $1.66$e-$06$ & $1.58$e-$06$ & $1.51$e-$06$ & $1.43$e-$06$ & $1.37$e-$06$ \\
$4.7$ & $1.30$e-$06$ & $1.24$e-$06$ & $1.18$e-$06$ & $1.12$e-$06$ & $1.07$e-$06$ & $1.02$e-$06$ & $9.68$e-$07$ & $9.21$e-$07$ & $8.76$e-$07$ & $8.34$e-$07$ \\
$4.8$ & $7.93$e-$07$ & $7.55$e-$07$ & $7.18$e-$07$ & $6.83$e-$07$ & $6.49$e-$07$ & $6.17$e-$07$ & $5.87$e-$07$ & $5.58$e-$07$ & $5.30$e-$07$ & $5.04$e-$07$ \\
$4.9$ & $4.79$e-$07$ & $4.55$e-$07$ & $4.33$e-$07$ & $4.11$e-$07$ & $3.91$e-$07$ & $3.71$e-$07$ & $3.52$e-$07$ & $3.35$e-$07$ & $3.18$e-$07$ & $3.02$e-$07$ \\
$5.0$ & $2.87$e-$07$ & $2.72$e-$07$ & $2.58$e-$07$ & $2.45$e-$07$ & $2.33$e-$07$ & $2.21$e-$07$ & $2.10$e-$07$ & $1.99$e-$07$ & $1.89$e-$07$ & $1.79$e-$07$ \\
$5.1$ & $1.70$e-$07$ & $1.61$e-$07$ & $1.53$e-$07$ & $1.45$e-$07$ & $1.37$e-$07$ & $1.30$e-$07$ & $1.23$e-$07$ & $1.17$e-$07$ & $1.11$e-$07$ & $1.05$e-$07$ \\
$5.2$ & $9.96$e-$08$ & $9.44$e-$08$ & $8.95$e-$08$ & $8.48$e-$08$ & $8.03$e-$08$ & $7.60$e-$08$ & $7.20$e-$08$ & $6.82$e-$08$ & $6.46$e-$08$ & $6.12$e-$08$ \\
$5.3$ & $5.79$e-$08$ & $5.48$e-$08$ & $5.19$e-$08$ & $4.91$e-$08$ & $4.65$e-$08$ & $4.40$e-$08$ & $4.16$e-$08$ & $3.94$e-$08$ & $3.72$e-$08$ & $3.52$e-$08$ \\
$5.4$ & $3.33$e-$08$ & $3.15$e-$08$ & $2.98$e-$08$ & $2.82$e-$08$ & $2.66$e-$08$ & $2.52$e-$08$ & $2.38$e-$08$ & $2.25$e-$08$ & $2.13$e-$08$ & $2.01$e-$08$ \\
$5.5$ & $1.90$e-$08$ & $1.79$e-$08$ & $1.69$e-$08$ & $1.60$e-$08$ & $1.51$e-$08$ & $1.43$e-$08$ & $1.35$e-$08$ & $1.27$e-$08$ & $1.20$e-$08$ & $1.14$e-$08$ \\
$5.6$ & $1.07$e-$08$ & $1.01$e-$08$ & $9.55$e-$09$ & $9.01$e-$09$ & $8.50$e-$09$ & $8.02$e-$09$ & $7.57$e-$09$ & $7.14$e-$09$ & $6.73$e-$09$ & $6.35$e-$09$ \\
$5.7$ & $5.99$e-$09$ & $5.65$e-$09$ & $5.33$e-$09$ & $5.02$e-$09$ & $4.73$e-$09$ & $4.46$e-$09$ & $4.21$e-$09$ & $3.96$e-$09$ & $3.74$e-$09$ & $3.52$e-$09$ \\
$5.8$ & $3.32$e-$09$ & $3.12$e-$09$ & $2.94$e-$09$ & $2.77$e-$09$ & $2.61$e-$09$ & $2.46$e-$09$ & $2.31$e-$09$ & $2.18$e-$09$ & $2.05$e-$09$ & $1.93$e-$09$ \\
$5.9$ & $1.82$e-$09$ & $1.71$e-$09$ & $1.61$e-$09$ & $1.51$e-$09$ & $1.43$e-$09$ & $1.34$e-$09$ & $1.26$e-$09$ & $1.19$e-$09$ & $1.12$e-$09$ & $1.05$e-$09$ \\
$6.0$ & $9.87$e-$10$ & $9.28$e-$10$ & $8.72$e-$10$ & $8.20$e-$10$ & $7.71$e-$10$ & $7.24$e-$10$ & $6.81$e-$10$ & $6.40$e-$10$ & $6.01$e-$10$ & $5.65$e-$10$ \\
$6.1$ & $5.30$e-$10$ & $4.98$e-$10$ & $4.68$e-$10$ & $4.39$e-$10$ & $4.13$e-$10$ & $3.87$e-$10$ & $3.64$e-$10$ & $3.41$e-$10$ & $3.21$e-$10$ & $3.01$e-$10$ \\
$6.2$ & $2.82$e-$10$ & $2.65$e-$10$ & $2.49$e-$10$ & $2.33$e-$10$ & $2.19$e-$10$ & $2.05$e-$10$ & $1.92$e-$10$ & $1.81$e-$10$ & $1.69$e-$10$ & $1.59$e-$10$ \\
$6.3$ & $1.49$e-$10$ & $1.40$e-$10$ & $1.31$e-$10$ & $1.23$e-$10$ & $1.15$e-$10$ & $1.08$e-$10$ & $1.01$e-$10$ & $9.45$e-$11$ & $8.85$e-$11$ & $8.29$e-$11$ \\
$6.4$ & $7.77$e-$11$ & $7.28$e-$11$ & $6.81$e-$11$ & $6.38$e-$11$ & $5.97$e-$11$ & $5.59$e-$11$ & $5.24$e-$11$ & $4.90$e-$11$ & $4.59$e-$11$ & $4.29$e-$11$ \\
$6.5$ & $4.02$e-$11$ & $3.76$e-$11$ & $3.52$e-$11$ & $3.29$e-$11$ & $3.08$e-$11$ & $2.88$e-$11$ & $2.69$e-$11$ & $2.52$e-$11$ & $2.35$e-$11$ & $2.20$e-$11$ \\
$6.6$ & $2.06$e-$11$ & $1.92$e-$11$ & $1.80$e-$11$ & $1.68$e-$11$ & $1.57$e-$11$ & $1.47$e-$11$ & $1.37$e-$11$ & $1.28$e-$11$ & $1.19$e-$11$ & $1.12$e-$11$ \\
$6.7$ & $1.04$e-$11$ & $9.73$e-$12$ & $9.09$e-$12$ & $8.48$e-$12$ & $7.92$e-$12$ & $7.39$e-$12$ & $6.90$e-$12$ & $6.44$e-$12$ & $6.01$e-$12$ & $5.61$e-$12$ \\
$6.8$ & $5.23$e-$12$ & $4.88$e-$12$ & $4.55$e-$12$ & $4.25$e-$12$ & $3.96$e-$12$ & $3.69$e-$12$ & $3.44$e-$12$ & $3.21$e-$12$ & $2.99$e-$12$ & $2.79$e-$12$ \\
$6.9$ & $2.60$e-$12$ & $2.42$e-$12$ & $2.26$e-$12$ & $2.10$e-$12$ & $1.96$e-$12$ & $1.83$e-$12$ & $1.70$e-$12$ & $1.58$e-$12$ & $1.48$e-$12$ & $1.37$e-$12$ \\
$7.0$ & $1.28$e-$12$ & $1.19$e-$12$ & $1.11$e-$12$ & $1.03$e-$12$ & $9.61$e-$13$ & $8.95$e-$13$ & $8.33$e-$13$ & $7.75$e-$13$ & $7.21$e-$13$ & $6.71$e-$13$ \\
$7.1$ & $6.24$e-$13$ & $5.80$e-$13$ & $5.40$e-$13$ & $5.02$e-$13$ & $4.67$e-$13$ & $4.34$e-$13$ & $4.03$e-$13$ & $3.75$e-$13$ & $3.49$e-$13$ & $3.24$e-$13$ \\
$7.2$ & $3.01$e-$13$ & $2.80$e-$13$ & $2.60$e-$13$ & $2.41$e-$13$ & $2.24$e-$13$ & $2.08$e-$13$ & $1.94$e-$13$ & $1.80$e-$13$ & $1.67$e-$13$ & $1.55$e-$13$ \\
$7.3$ & $1.44$e-$13$ & $1.34$e-$13$ & $1.24$e-$13$ & $1.15$e-$13$ & $1.07$e-$13$ & $9.91$e-$14$ & $9.20$e-$14$ & $8.53$e-$14$ & $7.92$e-$14$ & $7.34$e-$14$ \\
$7.4$ & $6.81$e-$14$ & $6.32$e-$14$ & $5.86$e-$14$ & $5.43$e-$14$ & $5.03$e-$14$ & $4.67$e-$14$ & $4.32$e-$14$ & $4.01$e-$14$ & $3.71$e-$14$ & $3.44$e-$14$ \\
$7.5$ & $3.19$e-$14$ & $2.96$e-$14$ & $2.74$e-$14$ & $2.54$e-$14$ & $2.35$e-$14$ & $2.18$e-$14$ & $2.02$e-$14$ & $1.87$e-$14$ & $1.73$e-$14$ & $1.60$e-$14$ \\
$7.6$ & $1.48$e-$14$ & $1.37$e-$14$ & $1.27$e-$14$ & $1.17$e-$14$ & $1.09$e-$14$ & $1.00$e-$14$ & $9.27$e-$15$ & $8.60$e-$15$ & $7.94$e-$15$ & $7.38$e-$15$ \\
$7.7$ & $6.83$e-$15$ & $6.27$e-$15$ & $5.83$e-$15$ & $5.38$e-$15$ & $5.00$e-$15$ & $4.61$e-$15$ & $4.22$e-$15$ & $3.94$e-$15$ & $3.61$e-$15$ & $3.33$e-$15$ \\
$7.8$ & $3.11$e-$15$ & $2.89$e-$15$ & $2.66$e-$15$ & $2.44$e-$15$ & $2.28$e-$15$ & $2.05$e-$15$ & $1.94$e-$15$ & $1.78$e-$15$ & $1.61$e-$15$ & $1.50$e-$15$ \\
$7.9$ & $1.39$e-$15$ & $1.28$e-$15$ & $1.17$e-$15$ & $1.11$e-$15$ & $9.99$e-$16$ & $9.44$e-$16$ & $8.33$e-$16$ & $7.77$e-$16$ & $7.22$e-$16$ & $6.66$e-$16$\\
        \hline
      \end{tabular*}
    }
  \end{center}
\end{table}


\begin{figure}
  %% Creator: Matplotlib, PGF backend
%%
%% To include the figure in your LaTeX document, write
%%   \input{<filename>.pgf}
%%
%% Make sure the required packages are loaded in your preamble
%%   \usepackage{pgf}
%%
%% Also ensure that all the required font packages are loaded; for instance,
%% the lmodern package is sometimes necessary when using math font.
%%   \usepackage{lmodern}
%%
%% Figures using additional raster images can only be included by \input if
%% they are in the same directory as the main LaTeX file. For loading figures
%% from other directories you can use the `import` package
%%   \usepackage{import}
%%
%% and then include the figures with
%%   \import{<path to file>}{<filename>.pgf}
%%
%% Matplotlib used the following preamble
%%   \usepackage[nice]{nicefrac}
%%   \usepackage{amsmath}
%%   \usepackage[utf8]{inputenc}
%%   \DeclareUnicodeCharacter{2212}{\ensuremath{-}}
%%
\begingroup%
\makeatletter%
\begin{pgfpicture}%
\pgfpathrectangle{\pgfpointorigin}{\pgfqpoint{6.380000in}{9.000000in}}%
\pgfusepath{use as bounding box, clip}%
\begin{pgfscope}%
\pgfsetbuttcap%
\pgfsetmiterjoin%
\definecolor{currentfill}{rgb}{1.000000,1.000000,1.000000}%
\pgfsetfillcolor{currentfill}%
\pgfsetlinewidth{0.000000pt}%
\definecolor{currentstroke}{rgb}{1.000000,1.000000,1.000000}%
\pgfsetstrokecolor{currentstroke}%
\pgfsetdash{}{0pt}%
\pgfpathmoveto{\pgfqpoint{0.000000in}{0.000000in}}%
\pgfpathlineto{\pgfqpoint{6.380000in}{0.000000in}}%
\pgfpathlineto{\pgfqpoint{6.380000in}{9.000000in}}%
\pgfpathlineto{\pgfqpoint{0.000000in}{9.000000in}}%
\pgfpathlineto{\pgfqpoint{0.000000in}{0.000000in}}%
\pgfpathclose%
\pgfusepath{fill}%
\end{pgfscope}%
\begin{pgfscope}%
\pgfsetbuttcap%
\pgfsetmiterjoin%
\definecolor{currentfill}{rgb}{1.000000,1.000000,1.000000}%
\pgfsetfillcolor{currentfill}%
\pgfsetlinewidth{0.000000pt}%
\definecolor{currentstroke}{rgb}{0.000000,0.000000,0.000000}%
\pgfsetstrokecolor{currentstroke}%
\pgfsetstrokeopacity{0.000000}%
\pgfsetdash{}{0pt}%
\pgfpathmoveto{\pgfqpoint{0.721718in}{0.509170in}}%
\pgfpathlineto{\pgfqpoint{6.245000in}{0.509170in}}%
\pgfpathlineto{\pgfqpoint{6.245000in}{8.821759in}}%
\pgfpathlineto{\pgfqpoint{0.721718in}{8.821759in}}%
\pgfpathlineto{\pgfqpoint{0.721718in}{0.509170in}}%
\pgfpathclose%
\pgfusepath{fill}%
\end{pgfscope}%
\begin{pgfscope}%
\pgfpathrectangle{\pgfqpoint{0.721718in}{0.509170in}}{\pgfqpoint{5.523282in}{8.312590in}}%
\pgfusepath{clip}%
\pgfsetbuttcap%
\pgfsetroundjoin%
\pgfsetlinewidth{0.250937pt}%
\definecolor{currentstroke}{rgb}{0.680000,0.680000,0.680000}%
\pgfsetstrokecolor{currentstroke}%
\pgfsetdash{{1.000000pt}{1.000000pt}}{0.000000pt}%
\pgfpathmoveto{\pgfqpoint{0.721718in}{0.509170in}}%
\pgfpathlineto{\pgfqpoint{0.721718in}{8.821759in}}%
\pgfusepath{stroke}%
\end{pgfscope}%
\begin{pgfscope}%
\pgfsetbuttcap%
\pgfsetroundjoin%
\definecolor{currentfill}{rgb}{0.000000,0.000000,0.000000}%
\pgfsetfillcolor{currentfill}%
\pgfsetlinewidth{0.803000pt}%
\definecolor{currentstroke}{rgb}{0.000000,0.000000,0.000000}%
\pgfsetstrokecolor{currentstroke}%
\pgfsetdash{}{0pt}%
\pgfsys@defobject{currentmarker}{\pgfqpoint{0.000000in}{-0.048611in}}{\pgfqpoint{0.000000in}{0.000000in}}{%
\pgfpathmoveto{\pgfqpoint{0.000000in}{0.000000in}}%
\pgfpathlineto{\pgfqpoint{0.000000in}{-0.048611in}}%
\pgfusepath{stroke,fill}%
}%
\begin{pgfscope}%
\pgfsys@transformshift{0.721718in}{0.509170in}%
\pgfsys@useobject{currentmarker}{}%
\end{pgfscope}%
\end{pgfscope}%
\begin{pgfscope}%
\definecolor{textcolor}{rgb}{0.000000,0.000000,0.000000}%
\pgfsetstrokecolor{textcolor}%
\pgfsetfillcolor{textcolor}%
\pgftext[x=0.721718in,y=0.411948in,,top]{\color{textcolor}\rmfamily\fontsize{9.000000}{10.800000}\selectfont \(\displaystyle {0}\)}%
\end{pgfscope}%
\begin{pgfscope}%
\pgfpathrectangle{\pgfqpoint{0.721718in}{0.509170in}}{\pgfqpoint{5.523282in}{8.312590in}}%
\pgfusepath{clip}%
\pgfsetbuttcap%
\pgfsetroundjoin%
\pgfsetlinewidth{0.250937pt}%
\definecolor{currentstroke}{rgb}{0.680000,0.680000,0.680000}%
\pgfsetstrokecolor{currentstroke}%
\pgfsetdash{{1.000000pt}{1.000000pt}}{0.000000pt}%
\pgfpathmoveto{\pgfqpoint{1.403605in}{0.509170in}}%
\pgfpathlineto{\pgfqpoint{1.403605in}{8.821759in}}%
\pgfusepath{stroke}%
\end{pgfscope}%
\begin{pgfscope}%
\pgfsetbuttcap%
\pgfsetroundjoin%
\definecolor{currentfill}{rgb}{0.000000,0.000000,0.000000}%
\pgfsetfillcolor{currentfill}%
\pgfsetlinewidth{0.803000pt}%
\definecolor{currentstroke}{rgb}{0.000000,0.000000,0.000000}%
\pgfsetstrokecolor{currentstroke}%
\pgfsetdash{}{0pt}%
\pgfsys@defobject{currentmarker}{\pgfqpoint{0.000000in}{-0.048611in}}{\pgfqpoint{0.000000in}{0.000000in}}{%
\pgfpathmoveto{\pgfqpoint{0.000000in}{0.000000in}}%
\pgfpathlineto{\pgfqpoint{0.000000in}{-0.048611in}}%
\pgfusepath{stroke,fill}%
}%
\begin{pgfscope}%
\pgfsys@transformshift{1.403605in}{0.509170in}%
\pgfsys@useobject{currentmarker}{}%
\end{pgfscope}%
\end{pgfscope}%
\begin{pgfscope}%
\definecolor{textcolor}{rgb}{0.000000,0.000000,0.000000}%
\pgfsetstrokecolor{textcolor}%
\pgfsetfillcolor{textcolor}%
\pgftext[x=1.403605in,y=0.411948in,,top]{\color{textcolor}\rmfamily\fontsize{9.000000}{10.800000}\selectfont \(\displaystyle {1}\)}%
\end{pgfscope}%
\begin{pgfscope}%
\pgfpathrectangle{\pgfqpoint{0.721718in}{0.509170in}}{\pgfqpoint{5.523282in}{8.312590in}}%
\pgfusepath{clip}%
\pgfsetbuttcap%
\pgfsetroundjoin%
\pgfsetlinewidth{0.250937pt}%
\definecolor{currentstroke}{rgb}{0.680000,0.680000,0.680000}%
\pgfsetstrokecolor{currentstroke}%
\pgfsetdash{{1.000000pt}{1.000000pt}}{0.000000pt}%
\pgfpathmoveto{\pgfqpoint{2.085492in}{0.509170in}}%
\pgfpathlineto{\pgfqpoint{2.085492in}{8.821759in}}%
\pgfusepath{stroke}%
\end{pgfscope}%
\begin{pgfscope}%
\pgfsetbuttcap%
\pgfsetroundjoin%
\definecolor{currentfill}{rgb}{0.000000,0.000000,0.000000}%
\pgfsetfillcolor{currentfill}%
\pgfsetlinewidth{0.803000pt}%
\definecolor{currentstroke}{rgb}{0.000000,0.000000,0.000000}%
\pgfsetstrokecolor{currentstroke}%
\pgfsetdash{}{0pt}%
\pgfsys@defobject{currentmarker}{\pgfqpoint{0.000000in}{-0.048611in}}{\pgfqpoint{0.000000in}{0.000000in}}{%
\pgfpathmoveto{\pgfqpoint{0.000000in}{0.000000in}}%
\pgfpathlineto{\pgfqpoint{0.000000in}{-0.048611in}}%
\pgfusepath{stroke,fill}%
}%
\begin{pgfscope}%
\pgfsys@transformshift{2.085492in}{0.509170in}%
\pgfsys@useobject{currentmarker}{}%
\end{pgfscope}%
\end{pgfscope}%
\begin{pgfscope}%
\definecolor{textcolor}{rgb}{0.000000,0.000000,0.000000}%
\pgfsetstrokecolor{textcolor}%
\pgfsetfillcolor{textcolor}%
\pgftext[x=2.085492in,y=0.411948in,,top]{\color{textcolor}\rmfamily\fontsize{9.000000}{10.800000}\selectfont \(\displaystyle {2}\)}%
\end{pgfscope}%
\begin{pgfscope}%
\pgfpathrectangle{\pgfqpoint{0.721718in}{0.509170in}}{\pgfqpoint{5.523282in}{8.312590in}}%
\pgfusepath{clip}%
\pgfsetbuttcap%
\pgfsetroundjoin%
\pgfsetlinewidth{0.250937pt}%
\definecolor{currentstroke}{rgb}{0.680000,0.680000,0.680000}%
\pgfsetstrokecolor{currentstroke}%
\pgfsetdash{{1.000000pt}{1.000000pt}}{0.000000pt}%
\pgfpathmoveto{\pgfqpoint{2.767378in}{0.509170in}}%
\pgfpathlineto{\pgfqpoint{2.767378in}{8.821759in}}%
\pgfusepath{stroke}%
\end{pgfscope}%
\begin{pgfscope}%
\pgfsetbuttcap%
\pgfsetroundjoin%
\definecolor{currentfill}{rgb}{0.000000,0.000000,0.000000}%
\pgfsetfillcolor{currentfill}%
\pgfsetlinewidth{0.803000pt}%
\definecolor{currentstroke}{rgb}{0.000000,0.000000,0.000000}%
\pgfsetstrokecolor{currentstroke}%
\pgfsetdash{}{0pt}%
\pgfsys@defobject{currentmarker}{\pgfqpoint{0.000000in}{-0.048611in}}{\pgfqpoint{0.000000in}{0.000000in}}{%
\pgfpathmoveto{\pgfqpoint{0.000000in}{0.000000in}}%
\pgfpathlineto{\pgfqpoint{0.000000in}{-0.048611in}}%
\pgfusepath{stroke,fill}%
}%
\begin{pgfscope}%
\pgfsys@transformshift{2.767378in}{0.509170in}%
\pgfsys@useobject{currentmarker}{}%
\end{pgfscope}%
\end{pgfscope}%
\begin{pgfscope}%
\definecolor{textcolor}{rgb}{0.000000,0.000000,0.000000}%
\pgfsetstrokecolor{textcolor}%
\pgfsetfillcolor{textcolor}%
\pgftext[x=2.767378in,y=0.411948in,,top]{\color{textcolor}\rmfamily\fontsize{9.000000}{10.800000}\selectfont \(\displaystyle {3}\)}%
\end{pgfscope}%
\begin{pgfscope}%
\pgfpathrectangle{\pgfqpoint{0.721718in}{0.509170in}}{\pgfqpoint{5.523282in}{8.312590in}}%
\pgfusepath{clip}%
\pgfsetbuttcap%
\pgfsetroundjoin%
\pgfsetlinewidth{0.250937pt}%
\definecolor{currentstroke}{rgb}{0.680000,0.680000,0.680000}%
\pgfsetstrokecolor{currentstroke}%
\pgfsetdash{{1.000000pt}{1.000000pt}}{0.000000pt}%
\pgfpathmoveto{\pgfqpoint{3.449265in}{0.509170in}}%
\pgfpathlineto{\pgfqpoint{3.449265in}{8.821759in}}%
\pgfusepath{stroke}%
\end{pgfscope}%
\begin{pgfscope}%
\pgfsetbuttcap%
\pgfsetroundjoin%
\definecolor{currentfill}{rgb}{0.000000,0.000000,0.000000}%
\pgfsetfillcolor{currentfill}%
\pgfsetlinewidth{0.803000pt}%
\definecolor{currentstroke}{rgb}{0.000000,0.000000,0.000000}%
\pgfsetstrokecolor{currentstroke}%
\pgfsetdash{}{0pt}%
\pgfsys@defobject{currentmarker}{\pgfqpoint{0.000000in}{-0.048611in}}{\pgfqpoint{0.000000in}{0.000000in}}{%
\pgfpathmoveto{\pgfqpoint{0.000000in}{0.000000in}}%
\pgfpathlineto{\pgfqpoint{0.000000in}{-0.048611in}}%
\pgfusepath{stroke,fill}%
}%
\begin{pgfscope}%
\pgfsys@transformshift{3.449265in}{0.509170in}%
\pgfsys@useobject{currentmarker}{}%
\end{pgfscope}%
\end{pgfscope}%
\begin{pgfscope}%
\definecolor{textcolor}{rgb}{0.000000,0.000000,0.000000}%
\pgfsetstrokecolor{textcolor}%
\pgfsetfillcolor{textcolor}%
\pgftext[x=3.449265in,y=0.411948in,,top]{\color{textcolor}\rmfamily\fontsize{9.000000}{10.800000}\selectfont \(\displaystyle {4}\)}%
\end{pgfscope}%
\begin{pgfscope}%
\pgfpathrectangle{\pgfqpoint{0.721718in}{0.509170in}}{\pgfqpoint{5.523282in}{8.312590in}}%
\pgfusepath{clip}%
\pgfsetbuttcap%
\pgfsetroundjoin%
\pgfsetlinewidth{0.250937pt}%
\definecolor{currentstroke}{rgb}{0.680000,0.680000,0.680000}%
\pgfsetstrokecolor{currentstroke}%
\pgfsetdash{{1.000000pt}{1.000000pt}}{0.000000pt}%
\pgfpathmoveto{\pgfqpoint{4.131151in}{0.509170in}}%
\pgfpathlineto{\pgfqpoint{4.131151in}{8.821759in}}%
\pgfusepath{stroke}%
\end{pgfscope}%
\begin{pgfscope}%
\pgfsetbuttcap%
\pgfsetroundjoin%
\definecolor{currentfill}{rgb}{0.000000,0.000000,0.000000}%
\pgfsetfillcolor{currentfill}%
\pgfsetlinewidth{0.803000pt}%
\definecolor{currentstroke}{rgb}{0.000000,0.000000,0.000000}%
\pgfsetstrokecolor{currentstroke}%
\pgfsetdash{}{0pt}%
\pgfsys@defobject{currentmarker}{\pgfqpoint{0.000000in}{-0.048611in}}{\pgfqpoint{0.000000in}{0.000000in}}{%
\pgfpathmoveto{\pgfqpoint{0.000000in}{0.000000in}}%
\pgfpathlineto{\pgfqpoint{0.000000in}{-0.048611in}}%
\pgfusepath{stroke,fill}%
}%
\begin{pgfscope}%
\pgfsys@transformshift{4.131151in}{0.509170in}%
\pgfsys@useobject{currentmarker}{}%
\end{pgfscope}%
\end{pgfscope}%
\begin{pgfscope}%
\definecolor{textcolor}{rgb}{0.000000,0.000000,0.000000}%
\pgfsetstrokecolor{textcolor}%
\pgfsetfillcolor{textcolor}%
\pgftext[x=4.131151in,y=0.411948in,,top]{\color{textcolor}\rmfamily\fontsize{9.000000}{10.800000}\selectfont \(\displaystyle {5}\)}%
\end{pgfscope}%
\begin{pgfscope}%
\pgfpathrectangle{\pgfqpoint{0.721718in}{0.509170in}}{\pgfqpoint{5.523282in}{8.312590in}}%
\pgfusepath{clip}%
\pgfsetbuttcap%
\pgfsetroundjoin%
\pgfsetlinewidth{0.250937pt}%
\definecolor{currentstroke}{rgb}{0.680000,0.680000,0.680000}%
\pgfsetstrokecolor{currentstroke}%
\pgfsetdash{{1.000000pt}{1.000000pt}}{0.000000pt}%
\pgfpathmoveto{\pgfqpoint{4.813038in}{0.509170in}}%
\pgfpathlineto{\pgfqpoint{4.813038in}{8.821759in}}%
\pgfusepath{stroke}%
\end{pgfscope}%
\begin{pgfscope}%
\pgfsetbuttcap%
\pgfsetroundjoin%
\definecolor{currentfill}{rgb}{0.000000,0.000000,0.000000}%
\pgfsetfillcolor{currentfill}%
\pgfsetlinewidth{0.803000pt}%
\definecolor{currentstroke}{rgb}{0.000000,0.000000,0.000000}%
\pgfsetstrokecolor{currentstroke}%
\pgfsetdash{}{0pt}%
\pgfsys@defobject{currentmarker}{\pgfqpoint{0.000000in}{-0.048611in}}{\pgfqpoint{0.000000in}{0.000000in}}{%
\pgfpathmoveto{\pgfqpoint{0.000000in}{0.000000in}}%
\pgfpathlineto{\pgfqpoint{0.000000in}{-0.048611in}}%
\pgfusepath{stroke,fill}%
}%
\begin{pgfscope}%
\pgfsys@transformshift{4.813038in}{0.509170in}%
\pgfsys@useobject{currentmarker}{}%
\end{pgfscope}%
\end{pgfscope}%
\begin{pgfscope}%
\definecolor{textcolor}{rgb}{0.000000,0.000000,0.000000}%
\pgfsetstrokecolor{textcolor}%
\pgfsetfillcolor{textcolor}%
\pgftext[x=4.813038in,y=0.411948in,,top]{\color{textcolor}\rmfamily\fontsize{9.000000}{10.800000}\selectfont \(\displaystyle {6}\)}%
\end{pgfscope}%
\begin{pgfscope}%
\pgfpathrectangle{\pgfqpoint{0.721718in}{0.509170in}}{\pgfqpoint{5.523282in}{8.312590in}}%
\pgfusepath{clip}%
\pgfsetbuttcap%
\pgfsetroundjoin%
\pgfsetlinewidth{0.250937pt}%
\definecolor{currentstroke}{rgb}{0.680000,0.680000,0.680000}%
\pgfsetstrokecolor{currentstroke}%
\pgfsetdash{{1.000000pt}{1.000000pt}}{0.000000pt}%
\pgfpathmoveto{\pgfqpoint{5.494925in}{0.509170in}}%
\pgfpathlineto{\pgfqpoint{5.494925in}{8.821759in}}%
\pgfusepath{stroke}%
\end{pgfscope}%
\begin{pgfscope}%
\pgfsetbuttcap%
\pgfsetroundjoin%
\definecolor{currentfill}{rgb}{0.000000,0.000000,0.000000}%
\pgfsetfillcolor{currentfill}%
\pgfsetlinewidth{0.803000pt}%
\definecolor{currentstroke}{rgb}{0.000000,0.000000,0.000000}%
\pgfsetstrokecolor{currentstroke}%
\pgfsetdash{}{0pt}%
\pgfsys@defobject{currentmarker}{\pgfqpoint{0.000000in}{-0.048611in}}{\pgfqpoint{0.000000in}{0.000000in}}{%
\pgfpathmoveto{\pgfqpoint{0.000000in}{0.000000in}}%
\pgfpathlineto{\pgfqpoint{0.000000in}{-0.048611in}}%
\pgfusepath{stroke,fill}%
}%
\begin{pgfscope}%
\pgfsys@transformshift{5.494925in}{0.509170in}%
\pgfsys@useobject{currentmarker}{}%
\end{pgfscope}%
\end{pgfscope}%
\begin{pgfscope}%
\definecolor{textcolor}{rgb}{0.000000,0.000000,0.000000}%
\pgfsetstrokecolor{textcolor}%
\pgfsetfillcolor{textcolor}%
\pgftext[x=5.494925in,y=0.411948in,,top]{\color{textcolor}\rmfamily\fontsize{9.000000}{10.800000}\selectfont \(\displaystyle {7}\)}%
\end{pgfscope}%
\begin{pgfscope}%
\pgfpathrectangle{\pgfqpoint{0.721718in}{0.509170in}}{\pgfqpoint{5.523282in}{8.312590in}}%
\pgfusepath{clip}%
\pgfsetbuttcap%
\pgfsetroundjoin%
\pgfsetlinewidth{0.250937pt}%
\definecolor{currentstroke}{rgb}{0.680000,0.680000,0.680000}%
\pgfsetstrokecolor{currentstroke}%
\pgfsetdash{{1.000000pt}{1.000000pt}}{0.000000pt}%
\pgfpathmoveto{\pgfqpoint{6.176811in}{0.509170in}}%
\pgfpathlineto{\pgfqpoint{6.176811in}{8.821759in}}%
\pgfusepath{stroke}%
\end{pgfscope}%
\begin{pgfscope}%
\pgfsetbuttcap%
\pgfsetroundjoin%
\definecolor{currentfill}{rgb}{0.000000,0.000000,0.000000}%
\pgfsetfillcolor{currentfill}%
\pgfsetlinewidth{0.803000pt}%
\definecolor{currentstroke}{rgb}{0.000000,0.000000,0.000000}%
\pgfsetstrokecolor{currentstroke}%
\pgfsetdash{}{0pt}%
\pgfsys@defobject{currentmarker}{\pgfqpoint{0.000000in}{-0.048611in}}{\pgfqpoint{0.000000in}{0.000000in}}{%
\pgfpathmoveto{\pgfqpoint{0.000000in}{0.000000in}}%
\pgfpathlineto{\pgfqpoint{0.000000in}{-0.048611in}}%
\pgfusepath{stroke,fill}%
}%
\begin{pgfscope}%
\pgfsys@transformshift{6.176811in}{0.509170in}%
\pgfsys@useobject{currentmarker}{}%
\end{pgfscope}%
\end{pgfscope}%
\begin{pgfscope}%
\definecolor{textcolor}{rgb}{0.000000,0.000000,0.000000}%
\pgfsetstrokecolor{textcolor}%
\pgfsetfillcolor{textcolor}%
\pgftext[x=6.176811in,y=0.411948in,,top]{\color{textcolor}\rmfamily\fontsize{9.000000}{10.800000}\selectfont \(\displaystyle {8}\)}%
\end{pgfscope}%
\begin{pgfscope}%
\definecolor{textcolor}{rgb}{0.000000,0.000000,0.000000}%
\pgfsetstrokecolor{textcolor}%
\pgfsetfillcolor{textcolor}%
\pgftext[x=3.483359in,y=0.245281in,,top]{\color{textcolor}\rmfamily\fontsize{9.000000}{10.800000}\selectfont \(\displaystyle z\)}%
\end{pgfscope}%
\begin{pgfscope}%
\pgfpathrectangle{\pgfqpoint{0.721718in}{0.509170in}}{\pgfqpoint{5.523282in}{8.312590in}}%
\pgfusepath{clip}%
\pgfsetbuttcap%
\pgfsetroundjoin%
\pgfsetlinewidth{0.250937pt}%
\definecolor{currentstroke}{rgb}{0.680000,0.680000,0.680000}%
\pgfsetstrokecolor{currentstroke}%
\pgfsetdash{{1.000000pt}{1.000000pt}}{0.000000pt}%
\pgfpathmoveto{\pgfqpoint{0.721718in}{0.509170in}}%
\pgfpathlineto{\pgfqpoint{6.245000in}{0.509170in}}%
\pgfusepath{stroke}%
\end{pgfscope}%
\begin{pgfscope}%
\pgfsetbuttcap%
\pgfsetroundjoin%
\definecolor{currentfill}{rgb}{0.000000,0.000000,0.000000}%
\pgfsetfillcolor{currentfill}%
\pgfsetlinewidth{0.803000pt}%
\definecolor{currentstroke}{rgb}{0.000000,0.000000,0.000000}%
\pgfsetstrokecolor{currentstroke}%
\pgfsetdash{}{0pt}%
\pgfsys@defobject{currentmarker}{\pgfqpoint{-0.048611in}{0.000000in}}{\pgfqpoint{-0.000000in}{0.000000in}}{%
\pgfpathmoveto{\pgfqpoint{-0.000000in}{0.000000in}}%
\pgfpathlineto{\pgfqpoint{-0.048611in}{0.000000in}}%
\pgfusepath{stroke,fill}%
}%
\begin{pgfscope}%
\pgfsys@transformshift{0.721718in}{0.509170in}%
\pgfsys@useobject{currentmarker}{}%
\end{pgfscope}%
\end{pgfscope}%
\begin{pgfscope}%
\definecolor{textcolor}{rgb}{0.000000,0.000000,0.000000}%
\pgfsetstrokecolor{textcolor}%
\pgfsetfillcolor{textcolor}%
\pgftext[x=0.306984in, y=0.464448in, left, base]{\color{textcolor}\rmfamily\fontsize{9.000000}{10.800000}\selectfont \(\displaystyle {10^{-16}}\)}%
\end{pgfscope}%
\begin{pgfscope}%
\pgfpathrectangle{\pgfqpoint{0.721718in}{0.509170in}}{\pgfqpoint{5.523282in}{8.312590in}}%
\pgfusepath{clip}%
\pgfsetbuttcap%
\pgfsetroundjoin%
\pgfsetlinewidth{0.250937pt}%
\definecolor{currentstroke}{rgb}{0.680000,0.680000,0.680000}%
\pgfsetstrokecolor{currentstroke}%
\pgfsetdash{{1.000000pt}{1.000000pt}}{0.000000pt}%
\pgfpathmoveto{\pgfqpoint{0.721718in}{1.028707in}}%
\pgfpathlineto{\pgfqpoint{6.245000in}{1.028707in}}%
\pgfusepath{stroke}%
\end{pgfscope}%
\begin{pgfscope}%
\pgfsetbuttcap%
\pgfsetroundjoin%
\definecolor{currentfill}{rgb}{0.000000,0.000000,0.000000}%
\pgfsetfillcolor{currentfill}%
\pgfsetlinewidth{0.803000pt}%
\definecolor{currentstroke}{rgb}{0.000000,0.000000,0.000000}%
\pgfsetstrokecolor{currentstroke}%
\pgfsetdash{}{0pt}%
\pgfsys@defobject{currentmarker}{\pgfqpoint{-0.048611in}{0.000000in}}{\pgfqpoint{-0.000000in}{0.000000in}}{%
\pgfpathmoveto{\pgfqpoint{-0.000000in}{0.000000in}}%
\pgfpathlineto{\pgfqpoint{-0.048611in}{0.000000in}}%
\pgfusepath{stroke,fill}%
}%
\begin{pgfscope}%
\pgfsys@transformshift{0.721718in}{1.028707in}%
\pgfsys@useobject{currentmarker}{}%
\end{pgfscope}%
\end{pgfscope}%
\begin{pgfscope}%
\definecolor{textcolor}{rgb}{0.000000,0.000000,0.000000}%
\pgfsetstrokecolor{textcolor}%
\pgfsetfillcolor{textcolor}%
\pgftext[x=0.306984in, y=0.983985in, left, base]{\color{textcolor}\rmfamily\fontsize{9.000000}{10.800000}\selectfont \(\displaystyle {10^{-15}}\)}%
\end{pgfscope}%
\begin{pgfscope}%
\pgfpathrectangle{\pgfqpoint{0.721718in}{0.509170in}}{\pgfqpoint{5.523282in}{8.312590in}}%
\pgfusepath{clip}%
\pgfsetbuttcap%
\pgfsetroundjoin%
\pgfsetlinewidth{0.250937pt}%
\definecolor{currentstroke}{rgb}{0.680000,0.680000,0.680000}%
\pgfsetstrokecolor{currentstroke}%
\pgfsetdash{{1.000000pt}{1.000000pt}}{0.000000pt}%
\pgfpathmoveto{\pgfqpoint{0.721718in}{1.548243in}}%
\pgfpathlineto{\pgfqpoint{6.245000in}{1.548243in}}%
\pgfusepath{stroke}%
\end{pgfscope}%
\begin{pgfscope}%
\pgfsetbuttcap%
\pgfsetroundjoin%
\definecolor{currentfill}{rgb}{0.000000,0.000000,0.000000}%
\pgfsetfillcolor{currentfill}%
\pgfsetlinewidth{0.803000pt}%
\definecolor{currentstroke}{rgb}{0.000000,0.000000,0.000000}%
\pgfsetstrokecolor{currentstroke}%
\pgfsetdash{}{0pt}%
\pgfsys@defobject{currentmarker}{\pgfqpoint{-0.048611in}{0.000000in}}{\pgfqpoint{-0.000000in}{0.000000in}}{%
\pgfpathmoveto{\pgfqpoint{-0.000000in}{0.000000in}}%
\pgfpathlineto{\pgfqpoint{-0.048611in}{0.000000in}}%
\pgfusepath{stroke,fill}%
}%
\begin{pgfscope}%
\pgfsys@transformshift{0.721718in}{1.548243in}%
\pgfsys@useobject{currentmarker}{}%
\end{pgfscope}%
\end{pgfscope}%
\begin{pgfscope}%
\definecolor{textcolor}{rgb}{0.000000,0.000000,0.000000}%
\pgfsetstrokecolor{textcolor}%
\pgfsetfillcolor{textcolor}%
\pgftext[x=0.306984in, y=1.503522in, left, base]{\color{textcolor}\rmfamily\fontsize{9.000000}{10.800000}\selectfont \(\displaystyle {10^{-14}}\)}%
\end{pgfscope}%
\begin{pgfscope}%
\pgfpathrectangle{\pgfqpoint{0.721718in}{0.509170in}}{\pgfqpoint{5.523282in}{8.312590in}}%
\pgfusepath{clip}%
\pgfsetbuttcap%
\pgfsetroundjoin%
\pgfsetlinewidth{0.250937pt}%
\definecolor{currentstroke}{rgb}{0.680000,0.680000,0.680000}%
\pgfsetstrokecolor{currentstroke}%
\pgfsetdash{{1.000000pt}{1.000000pt}}{0.000000pt}%
\pgfpathmoveto{\pgfqpoint{0.721718in}{2.067780in}}%
\pgfpathlineto{\pgfqpoint{6.245000in}{2.067780in}}%
\pgfusepath{stroke}%
\end{pgfscope}%
\begin{pgfscope}%
\pgfsetbuttcap%
\pgfsetroundjoin%
\definecolor{currentfill}{rgb}{0.000000,0.000000,0.000000}%
\pgfsetfillcolor{currentfill}%
\pgfsetlinewidth{0.803000pt}%
\definecolor{currentstroke}{rgb}{0.000000,0.000000,0.000000}%
\pgfsetstrokecolor{currentstroke}%
\pgfsetdash{}{0pt}%
\pgfsys@defobject{currentmarker}{\pgfqpoint{-0.048611in}{0.000000in}}{\pgfqpoint{-0.000000in}{0.000000in}}{%
\pgfpathmoveto{\pgfqpoint{-0.000000in}{0.000000in}}%
\pgfpathlineto{\pgfqpoint{-0.048611in}{0.000000in}}%
\pgfusepath{stroke,fill}%
}%
\begin{pgfscope}%
\pgfsys@transformshift{0.721718in}{2.067780in}%
\pgfsys@useobject{currentmarker}{}%
\end{pgfscope}%
\end{pgfscope}%
\begin{pgfscope}%
\definecolor{textcolor}{rgb}{0.000000,0.000000,0.000000}%
\pgfsetstrokecolor{textcolor}%
\pgfsetfillcolor{textcolor}%
\pgftext[x=0.306984in, y=2.023059in, left, base]{\color{textcolor}\rmfamily\fontsize{9.000000}{10.800000}\selectfont \(\displaystyle {10^{-13}}\)}%
\end{pgfscope}%
\begin{pgfscope}%
\pgfpathrectangle{\pgfqpoint{0.721718in}{0.509170in}}{\pgfqpoint{5.523282in}{8.312590in}}%
\pgfusepath{clip}%
\pgfsetbuttcap%
\pgfsetroundjoin%
\pgfsetlinewidth{0.250937pt}%
\definecolor{currentstroke}{rgb}{0.680000,0.680000,0.680000}%
\pgfsetstrokecolor{currentstroke}%
\pgfsetdash{{1.000000pt}{1.000000pt}}{0.000000pt}%
\pgfpathmoveto{\pgfqpoint{0.721718in}{2.587317in}}%
\pgfpathlineto{\pgfqpoint{6.245000in}{2.587317in}}%
\pgfusepath{stroke}%
\end{pgfscope}%
\begin{pgfscope}%
\pgfsetbuttcap%
\pgfsetroundjoin%
\definecolor{currentfill}{rgb}{0.000000,0.000000,0.000000}%
\pgfsetfillcolor{currentfill}%
\pgfsetlinewidth{0.803000pt}%
\definecolor{currentstroke}{rgb}{0.000000,0.000000,0.000000}%
\pgfsetstrokecolor{currentstroke}%
\pgfsetdash{}{0pt}%
\pgfsys@defobject{currentmarker}{\pgfqpoint{-0.048611in}{0.000000in}}{\pgfqpoint{-0.000000in}{0.000000in}}{%
\pgfpathmoveto{\pgfqpoint{-0.000000in}{0.000000in}}%
\pgfpathlineto{\pgfqpoint{-0.048611in}{0.000000in}}%
\pgfusepath{stroke,fill}%
}%
\begin{pgfscope}%
\pgfsys@transformshift{0.721718in}{2.587317in}%
\pgfsys@useobject{currentmarker}{}%
\end{pgfscope}%
\end{pgfscope}%
\begin{pgfscope}%
\definecolor{textcolor}{rgb}{0.000000,0.000000,0.000000}%
\pgfsetstrokecolor{textcolor}%
\pgfsetfillcolor{textcolor}%
\pgftext[x=0.306984in, y=2.542595in, left, base]{\color{textcolor}\rmfamily\fontsize{9.000000}{10.800000}\selectfont \(\displaystyle {10^{-12}}\)}%
\end{pgfscope}%
\begin{pgfscope}%
\pgfpathrectangle{\pgfqpoint{0.721718in}{0.509170in}}{\pgfqpoint{5.523282in}{8.312590in}}%
\pgfusepath{clip}%
\pgfsetbuttcap%
\pgfsetroundjoin%
\pgfsetlinewidth{0.250937pt}%
\definecolor{currentstroke}{rgb}{0.680000,0.680000,0.680000}%
\pgfsetstrokecolor{currentstroke}%
\pgfsetdash{{1.000000pt}{1.000000pt}}{0.000000pt}%
\pgfpathmoveto{\pgfqpoint{0.721718in}{3.106854in}}%
\pgfpathlineto{\pgfqpoint{6.245000in}{3.106854in}}%
\pgfusepath{stroke}%
\end{pgfscope}%
\begin{pgfscope}%
\pgfsetbuttcap%
\pgfsetroundjoin%
\definecolor{currentfill}{rgb}{0.000000,0.000000,0.000000}%
\pgfsetfillcolor{currentfill}%
\pgfsetlinewidth{0.803000pt}%
\definecolor{currentstroke}{rgb}{0.000000,0.000000,0.000000}%
\pgfsetstrokecolor{currentstroke}%
\pgfsetdash{}{0pt}%
\pgfsys@defobject{currentmarker}{\pgfqpoint{-0.048611in}{0.000000in}}{\pgfqpoint{-0.000000in}{0.000000in}}{%
\pgfpathmoveto{\pgfqpoint{-0.000000in}{0.000000in}}%
\pgfpathlineto{\pgfqpoint{-0.048611in}{0.000000in}}%
\pgfusepath{stroke,fill}%
}%
\begin{pgfscope}%
\pgfsys@transformshift{0.721718in}{3.106854in}%
\pgfsys@useobject{currentmarker}{}%
\end{pgfscope}%
\end{pgfscope}%
\begin{pgfscope}%
\definecolor{textcolor}{rgb}{0.000000,0.000000,0.000000}%
\pgfsetstrokecolor{textcolor}%
\pgfsetfillcolor{textcolor}%
\pgftext[x=0.306984in, y=3.062132in, left, base]{\color{textcolor}\rmfamily\fontsize{9.000000}{10.800000}\selectfont \(\displaystyle {10^{-11}}\)}%
\end{pgfscope}%
\begin{pgfscope}%
\pgfpathrectangle{\pgfqpoint{0.721718in}{0.509170in}}{\pgfqpoint{5.523282in}{8.312590in}}%
\pgfusepath{clip}%
\pgfsetbuttcap%
\pgfsetroundjoin%
\pgfsetlinewidth{0.250937pt}%
\definecolor{currentstroke}{rgb}{0.680000,0.680000,0.680000}%
\pgfsetstrokecolor{currentstroke}%
\pgfsetdash{{1.000000pt}{1.000000pt}}{0.000000pt}%
\pgfpathmoveto{\pgfqpoint{0.721718in}{3.626391in}}%
\pgfpathlineto{\pgfqpoint{6.245000in}{3.626391in}}%
\pgfusepath{stroke}%
\end{pgfscope}%
\begin{pgfscope}%
\pgfsetbuttcap%
\pgfsetroundjoin%
\definecolor{currentfill}{rgb}{0.000000,0.000000,0.000000}%
\pgfsetfillcolor{currentfill}%
\pgfsetlinewidth{0.803000pt}%
\definecolor{currentstroke}{rgb}{0.000000,0.000000,0.000000}%
\pgfsetstrokecolor{currentstroke}%
\pgfsetdash{}{0pt}%
\pgfsys@defobject{currentmarker}{\pgfqpoint{-0.048611in}{0.000000in}}{\pgfqpoint{-0.000000in}{0.000000in}}{%
\pgfpathmoveto{\pgfqpoint{-0.000000in}{0.000000in}}%
\pgfpathlineto{\pgfqpoint{-0.048611in}{0.000000in}}%
\pgfusepath{stroke,fill}%
}%
\begin{pgfscope}%
\pgfsys@transformshift{0.721718in}{3.626391in}%
\pgfsys@useobject{currentmarker}{}%
\end{pgfscope}%
\end{pgfscope}%
\begin{pgfscope}%
\definecolor{textcolor}{rgb}{0.000000,0.000000,0.000000}%
\pgfsetstrokecolor{textcolor}%
\pgfsetfillcolor{textcolor}%
\pgftext[x=0.306984in, y=3.581669in, left, base]{\color{textcolor}\rmfamily\fontsize{9.000000}{10.800000}\selectfont \(\displaystyle {10^{-10}}\)}%
\end{pgfscope}%
\begin{pgfscope}%
\pgfpathrectangle{\pgfqpoint{0.721718in}{0.509170in}}{\pgfqpoint{5.523282in}{8.312590in}}%
\pgfusepath{clip}%
\pgfsetbuttcap%
\pgfsetroundjoin%
\pgfsetlinewidth{0.250937pt}%
\definecolor{currentstroke}{rgb}{0.680000,0.680000,0.680000}%
\pgfsetstrokecolor{currentstroke}%
\pgfsetdash{{1.000000pt}{1.000000pt}}{0.000000pt}%
\pgfpathmoveto{\pgfqpoint{0.721718in}{4.145928in}}%
\pgfpathlineto{\pgfqpoint{6.245000in}{4.145928in}}%
\pgfusepath{stroke}%
\end{pgfscope}%
\begin{pgfscope}%
\pgfsetbuttcap%
\pgfsetroundjoin%
\definecolor{currentfill}{rgb}{0.000000,0.000000,0.000000}%
\pgfsetfillcolor{currentfill}%
\pgfsetlinewidth{0.803000pt}%
\definecolor{currentstroke}{rgb}{0.000000,0.000000,0.000000}%
\pgfsetstrokecolor{currentstroke}%
\pgfsetdash{}{0pt}%
\pgfsys@defobject{currentmarker}{\pgfqpoint{-0.048611in}{0.000000in}}{\pgfqpoint{-0.000000in}{0.000000in}}{%
\pgfpathmoveto{\pgfqpoint{-0.000000in}{0.000000in}}%
\pgfpathlineto{\pgfqpoint{-0.048611in}{0.000000in}}%
\pgfusepath{stroke,fill}%
}%
\begin{pgfscope}%
\pgfsys@transformshift{0.721718in}{4.145928in}%
\pgfsys@useobject{currentmarker}{}%
\end{pgfscope}%
\end{pgfscope}%
\begin{pgfscope}%
\definecolor{textcolor}{rgb}{0.000000,0.000000,0.000000}%
\pgfsetstrokecolor{textcolor}%
\pgfsetfillcolor{textcolor}%
\pgftext[x=0.357909in, y=4.101206in, left, base]{\color{textcolor}\rmfamily\fontsize{9.000000}{10.800000}\selectfont \(\displaystyle {10^{-9}}\)}%
\end{pgfscope}%
\begin{pgfscope}%
\pgfpathrectangle{\pgfqpoint{0.721718in}{0.509170in}}{\pgfqpoint{5.523282in}{8.312590in}}%
\pgfusepath{clip}%
\pgfsetbuttcap%
\pgfsetroundjoin%
\pgfsetlinewidth{0.250937pt}%
\definecolor{currentstroke}{rgb}{0.680000,0.680000,0.680000}%
\pgfsetstrokecolor{currentstroke}%
\pgfsetdash{{1.000000pt}{1.000000pt}}{0.000000pt}%
\pgfpathmoveto{\pgfqpoint{0.721718in}{4.665465in}}%
\pgfpathlineto{\pgfqpoint{6.245000in}{4.665465in}}%
\pgfusepath{stroke}%
\end{pgfscope}%
\begin{pgfscope}%
\pgfsetbuttcap%
\pgfsetroundjoin%
\definecolor{currentfill}{rgb}{0.000000,0.000000,0.000000}%
\pgfsetfillcolor{currentfill}%
\pgfsetlinewidth{0.803000pt}%
\definecolor{currentstroke}{rgb}{0.000000,0.000000,0.000000}%
\pgfsetstrokecolor{currentstroke}%
\pgfsetdash{}{0pt}%
\pgfsys@defobject{currentmarker}{\pgfqpoint{-0.048611in}{0.000000in}}{\pgfqpoint{-0.000000in}{0.000000in}}{%
\pgfpathmoveto{\pgfqpoint{-0.000000in}{0.000000in}}%
\pgfpathlineto{\pgfqpoint{-0.048611in}{0.000000in}}%
\pgfusepath{stroke,fill}%
}%
\begin{pgfscope}%
\pgfsys@transformshift{0.721718in}{4.665465in}%
\pgfsys@useobject{currentmarker}{}%
\end{pgfscope}%
\end{pgfscope}%
\begin{pgfscope}%
\definecolor{textcolor}{rgb}{0.000000,0.000000,0.000000}%
\pgfsetstrokecolor{textcolor}%
\pgfsetfillcolor{textcolor}%
\pgftext[x=0.357909in, y=4.620743in, left, base]{\color{textcolor}\rmfamily\fontsize{9.000000}{10.800000}\selectfont \(\displaystyle {10^{-8}}\)}%
\end{pgfscope}%
\begin{pgfscope}%
\pgfpathrectangle{\pgfqpoint{0.721718in}{0.509170in}}{\pgfqpoint{5.523282in}{8.312590in}}%
\pgfusepath{clip}%
\pgfsetbuttcap%
\pgfsetroundjoin%
\pgfsetlinewidth{0.250937pt}%
\definecolor{currentstroke}{rgb}{0.680000,0.680000,0.680000}%
\pgfsetstrokecolor{currentstroke}%
\pgfsetdash{{1.000000pt}{1.000000pt}}{0.000000pt}%
\pgfpathmoveto{\pgfqpoint{0.721718in}{5.185001in}}%
\pgfpathlineto{\pgfqpoint{6.245000in}{5.185001in}}%
\pgfusepath{stroke}%
\end{pgfscope}%
\begin{pgfscope}%
\pgfsetbuttcap%
\pgfsetroundjoin%
\definecolor{currentfill}{rgb}{0.000000,0.000000,0.000000}%
\pgfsetfillcolor{currentfill}%
\pgfsetlinewidth{0.803000pt}%
\definecolor{currentstroke}{rgb}{0.000000,0.000000,0.000000}%
\pgfsetstrokecolor{currentstroke}%
\pgfsetdash{}{0pt}%
\pgfsys@defobject{currentmarker}{\pgfqpoint{-0.048611in}{0.000000in}}{\pgfqpoint{-0.000000in}{0.000000in}}{%
\pgfpathmoveto{\pgfqpoint{-0.000000in}{0.000000in}}%
\pgfpathlineto{\pgfqpoint{-0.048611in}{0.000000in}}%
\pgfusepath{stroke,fill}%
}%
\begin{pgfscope}%
\pgfsys@transformshift{0.721718in}{5.185001in}%
\pgfsys@useobject{currentmarker}{}%
\end{pgfscope}%
\end{pgfscope}%
\begin{pgfscope}%
\definecolor{textcolor}{rgb}{0.000000,0.000000,0.000000}%
\pgfsetstrokecolor{textcolor}%
\pgfsetfillcolor{textcolor}%
\pgftext[x=0.357909in, y=5.140280in, left, base]{\color{textcolor}\rmfamily\fontsize{9.000000}{10.800000}\selectfont \(\displaystyle {10^{-7}}\)}%
\end{pgfscope}%
\begin{pgfscope}%
\pgfpathrectangle{\pgfqpoint{0.721718in}{0.509170in}}{\pgfqpoint{5.523282in}{8.312590in}}%
\pgfusepath{clip}%
\pgfsetbuttcap%
\pgfsetroundjoin%
\pgfsetlinewidth{0.250937pt}%
\definecolor{currentstroke}{rgb}{0.680000,0.680000,0.680000}%
\pgfsetstrokecolor{currentstroke}%
\pgfsetdash{{1.000000pt}{1.000000pt}}{0.000000pt}%
\pgfpathmoveto{\pgfqpoint{0.721718in}{5.704538in}}%
\pgfpathlineto{\pgfqpoint{6.245000in}{5.704538in}}%
\pgfusepath{stroke}%
\end{pgfscope}%
\begin{pgfscope}%
\pgfsetbuttcap%
\pgfsetroundjoin%
\definecolor{currentfill}{rgb}{0.000000,0.000000,0.000000}%
\pgfsetfillcolor{currentfill}%
\pgfsetlinewidth{0.803000pt}%
\definecolor{currentstroke}{rgb}{0.000000,0.000000,0.000000}%
\pgfsetstrokecolor{currentstroke}%
\pgfsetdash{}{0pt}%
\pgfsys@defobject{currentmarker}{\pgfqpoint{-0.048611in}{0.000000in}}{\pgfqpoint{-0.000000in}{0.000000in}}{%
\pgfpathmoveto{\pgfqpoint{-0.000000in}{0.000000in}}%
\pgfpathlineto{\pgfqpoint{-0.048611in}{0.000000in}}%
\pgfusepath{stroke,fill}%
}%
\begin{pgfscope}%
\pgfsys@transformshift{0.721718in}{5.704538in}%
\pgfsys@useobject{currentmarker}{}%
\end{pgfscope}%
\end{pgfscope}%
\begin{pgfscope}%
\definecolor{textcolor}{rgb}{0.000000,0.000000,0.000000}%
\pgfsetstrokecolor{textcolor}%
\pgfsetfillcolor{textcolor}%
\pgftext[x=0.357909in, y=5.659816in, left, base]{\color{textcolor}\rmfamily\fontsize{9.000000}{10.800000}\selectfont \(\displaystyle {10^{-6}}\)}%
\end{pgfscope}%
\begin{pgfscope}%
\pgfpathrectangle{\pgfqpoint{0.721718in}{0.509170in}}{\pgfqpoint{5.523282in}{8.312590in}}%
\pgfusepath{clip}%
\pgfsetbuttcap%
\pgfsetroundjoin%
\pgfsetlinewidth{0.250937pt}%
\definecolor{currentstroke}{rgb}{0.680000,0.680000,0.680000}%
\pgfsetstrokecolor{currentstroke}%
\pgfsetdash{{1.000000pt}{1.000000pt}}{0.000000pt}%
\pgfpathmoveto{\pgfqpoint{0.721718in}{6.224075in}}%
\pgfpathlineto{\pgfqpoint{6.245000in}{6.224075in}}%
\pgfusepath{stroke}%
\end{pgfscope}%
\begin{pgfscope}%
\pgfsetbuttcap%
\pgfsetroundjoin%
\definecolor{currentfill}{rgb}{0.000000,0.000000,0.000000}%
\pgfsetfillcolor{currentfill}%
\pgfsetlinewidth{0.803000pt}%
\definecolor{currentstroke}{rgb}{0.000000,0.000000,0.000000}%
\pgfsetstrokecolor{currentstroke}%
\pgfsetdash{}{0pt}%
\pgfsys@defobject{currentmarker}{\pgfqpoint{-0.048611in}{0.000000in}}{\pgfqpoint{-0.000000in}{0.000000in}}{%
\pgfpathmoveto{\pgfqpoint{-0.000000in}{0.000000in}}%
\pgfpathlineto{\pgfqpoint{-0.048611in}{0.000000in}}%
\pgfusepath{stroke,fill}%
}%
\begin{pgfscope}%
\pgfsys@transformshift{0.721718in}{6.224075in}%
\pgfsys@useobject{currentmarker}{}%
\end{pgfscope}%
\end{pgfscope}%
\begin{pgfscope}%
\definecolor{textcolor}{rgb}{0.000000,0.000000,0.000000}%
\pgfsetstrokecolor{textcolor}%
\pgfsetfillcolor{textcolor}%
\pgftext[x=0.357909in, y=6.179353in, left, base]{\color{textcolor}\rmfamily\fontsize{9.000000}{10.800000}\selectfont \(\displaystyle {10^{-5}}\)}%
\end{pgfscope}%
\begin{pgfscope}%
\pgfpathrectangle{\pgfqpoint{0.721718in}{0.509170in}}{\pgfqpoint{5.523282in}{8.312590in}}%
\pgfusepath{clip}%
\pgfsetbuttcap%
\pgfsetroundjoin%
\pgfsetlinewidth{0.250937pt}%
\definecolor{currentstroke}{rgb}{0.680000,0.680000,0.680000}%
\pgfsetstrokecolor{currentstroke}%
\pgfsetdash{{1.000000pt}{1.000000pt}}{0.000000pt}%
\pgfpathmoveto{\pgfqpoint{0.721718in}{6.743612in}}%
\pgfpathlineto{\pgfqpoint{6.245000in}{6.743612in}}%
\pgfusepath{stroke}%
\end{pgfscope}%
\begin{pgfscope}%
\pgfsetbuttcap%
\pgfsetroundjoin%
\definecolor{currentfill}{rgb}{0.000000,0.000000,0.000000}%
\pgfsetfillcolor{currentfill}%
\pgfsetlinewidth{0.803000pt}%
\definecolor{currentstroke}{rgb}{0.000000,0.000000,0.000000}%
\pgfsetstrokecolor{currentstroke}%
\pgfsetdash{}{0pt}%
\pgfsys@defobject{currentmarker}{\pgfqpoint{-0.048611in}{0.000000in}}{\pgfqpoint{-0.000000in}{0.000000in}}{%
\pgfpathmoveto{\pgfqpoint{-0.000000in}{0.000000in}}%
\pgfpathlineto{\pgfqpoint{-0.048611in}{0.000000in}}%
\pgfusepath{stroke,fill}%
}%
\begin{pgfscope}%
\pgfsys@transformshift{0.721718in}{6.743612in}%
\pgfsys@useobject{currentmarker}{}%
\end{pgfscope}%
\end{pgfscope}%
\begin{pgfscope}%
\definecolor{textcolor}{rgb}{0.000000,0.000000,0.000000}%
\pgfsetstrokecolor{textcolor}%
\pgfsetfillcolor{textcolor}%
\pgftext[x=0.357909in, y=6.698890in, left, base]{\color{textcolor}\rmfamily\fontsize{9.000000}{10.800000}\selectfont \(\displaystyle {10^{-4}}\)}%
\end{pgfscope}%
\begin{pgfscope}%
\pgfpathrectangle{\pgfqpoint{0.721718in}{0.509170in}}{\pgfqpoint{5.523282in}{8.312590in}}%
\pgfusepath{clip}%
\pgfsetbuttcap%
\pgfsetroundjoin%
\pgfsetlinewidth{0.250937pt}%
\definecolor{currentstroke}{rgb}{0.680000,0.680000,0.680000}%
\pgfsetstrokecolor{currentstroke}%
\pgfsetdash{{1.000000pt}{1.000000pt}}{0.000000pt}%
\pgfpathmoveto{\pgfqpoint{0.721718in}{7.263149in}}%
\pgfpathlineto{\pgfqpoint{6.245000in}{7.263149in}}%
\pgfusepath{stroke}%
\end{pgfscope}%
\begin{pgfscope}%
\pgfsetbuttcap%
\pgfsetroundjoin%
\definecolor{currentfill}{rgb}{0.000000,0.000000,0.000000}%
\pgfsetfillcolor{currentfill}%
\pgfsetlinewidth{0.803000pt}%
\definecolor{currentstroke}{rgb}{0.000000,0.000000,0.000000}%
\pgfsetstrokecolor{currentstroke}%
\pgfsetdash{}{0pt}%
\pgfsys@defobject{currentmarker}{\pgfqpoint{-0.048611in}{0.000000in}}{\pgfqpoint{-0.000000in}{0.000000in}}{%
\pgfpathmoveto{\pgfqpoint{-0.000000in}{0.000000in}}%
\pgfpathlineto{\pgfqpoint{-0.048611in}{0.000000in}}%
\pgfusepath{stroke,fill}%
}%
\begin{pgfscope}%
\pgfsys@transformshift{0.721718in}{7.263149in}%
\pgfsys@useobject{currentmarker}{}%
\end{pgfscope}%
\end{pgfscope}%
\begin{pgfscope}%
\definecolor{textcolor}{rgb}{0.000000,0.000000,0.000000}%
\pgfsetstrokecolor{textcolor}%
\pgfsetfillcolor{textcolor}%
\pgftext[x=0.357909in, y=7.218427in, left, base]{\color{textcolor}\rmfamily\fontsize{9.000000}{10.800000}\selectfont \(\displaystyle {10^{-3}}\)}%
\end{pgfscope}%
\begin{pgfscope}%
\pgfpathrectangle{\pgfqpoint{0.721718in}{0.509170in}}{\pgfqpoint{5.523282in}{8.312590in}}%
\pgfusepath{clip}%
\pgfsetbuttcap%
\pgfsetroundjoin%
\pgfsetlinewidth{0.250937pt}%
\definecolor{currentstroke}{rgb}{0.680000,0.680000,0.680000}%
\pgfsetstrokecolor{currentstroke}%
\pgfsetdash{{1.000000pt}{1.000000pt}}{0.000000pt}%
\pgfpathmoveto{\pgfqpoint{0.721718in}{7.782686in}}%
\pgfpathlineto{\pgfqpoint{6.245000in}{7.782686in}}%
\pgfusepath{stroke}%
\end{pgfscope}%
\begin{pgfscope}%
\pgfsetbuttcap%
\pgfsetroundjoin%
\definecolor{currentfill}{rgb}{0.000000,0.000000,0.000000}%
\pgfsetfillcolor{currentfill}%
\pgfsetlinewidth{0.803000pt}%
\definecolor{currentstroke}{rgb}{0.000000,0.000000,0.000000}%
\pgfsetstrokecolor{currentstroke}%
\pgfsetdash{}{0pt}%
\pgfsys@defobject{currentmarker}{\pgfqpoint{-0.048611in}{0.000000in}}{\pgfqpoint{-0.000000in}{0.000000in}}{%
\pgfpathmoveto{\pgfqpoint{-0.000000in}{0.000000in}}%
\pgfpathlineto{\pgfqpoint{-0.048611in}{0.000000in}}%
\pgfusepath{stroke,fill}%
}%
\begin{pgfscope}%
\pgfsys@transformshift{0.721718in}{7.782686in}%
\pgfsys@useobject{currentmarker}{}%
\end{pgfscope}%
\end{pgfscope}%
\begin{pgfscope}%
\definecolor{textcolor}{rgb}{0.000000,0.000000,0.000000}%
\pgfsetstrokecolor{textcolor}%
\pgfsetfillcolor{textcolor}%
\pgftext[x=0.357909in, y=7.737964in, left, base]{\color{textcolor}\rmfamily\fontsize{9.000000}{10.800000}\selectfont \(\displaystyle {10^{-2}}\)}%
\end{pgfscope}%
\begin{pgfscope}%
\pgfpathrectangle{\pgfqpoint{0.721718in}{0.509170in}}{\pgfqpoint{5.523282in}{8.312590in}}%
\pgfusepath{clip}%
\pgfsetbuttcap%
\pgfsetroundjoin%
\pgfsetlinewidth{0.250937pt}%
\definecolor{currentstroke}{rgb}{0.680000,0.680000,0.680000}%
\pgfsetstrokecolor{currentstroke}%
\pgfsetdash{{1.000000pt}{1.000000pt}}{0.000000pt}%
\pgfpathmoveto{\pgfqpoint{0.721718in}{8.302223in}}%
\pgfpathlineto{\pgfqpoint{6.245000in}{8.302223in}}%
\pgfusepath{stroke}%
\end{pgfscope}%
\begin{pgfscope}%
\pgfsetbuttcap%
\pgfsetroundjoin%
\definecolor{currentfill}{rgb}{0.000000,0.000000,0.000000}%
\pgfsetfillcolor{currentfill}%
\pgfsetlinewidth{0.803000pt}%
\definecolor{currentstroke}{rgb}{0.000000,0.000000,0.000000}%
\pgfsetstrokecolor{currentstroke}%
\pgfsetdash{}{0pt}%
\pgfsys@defobject{currentmarker}{\pgfqpoint{-0.048611in}{0.000000in}}{\pgfqpoint{-0.000000in}{0.000000in}}{%
\pgfpathmoveto{\pgfqpoint{-0.000000in}{0.000000in}}%
\pgfpathlineto{\pgfqpoint{-0.048611in}{0.000000in}}%
\pgfusepath{stroke,fill}%
}%
\begin{pgfscope}%
\pgfsys@transformshift{0.721718in}{8.302223in}%
\pgfsys@useobject{currentmarker}{}%
\end{pgfscope}%
\end{pgfscope}%
\begin{pgfscope}%
\definecolor{textcolor}{rgb}{0.000000,0.000000,0.000000}%
\pgfsetstrokecolor{textcolor}%
\pgfsetfillcolor{textcolor}%
\pgftext[x=0.357909in, y=8.257501in, left, base]{\color{textcolor}\rmfamily\fontsize{9.000000}{10.800000}\selectfont \(\displaystyle {10^{-1}}\)}%
\end{pgfscope}%
\begin{pgfscope}%
\pgfpathrectangle{\pgfqpoint{0.721718in}{0.509170in}}{\pgfqpoint{5.523282in}{8.312590in}}%
\pgfusepath{clip}%
\pgfsetbuttcap%
\pgfsetroundjoin%
\pgfsetlinewidth{0.250937pt}%
\definecolor{currentstroke}{rgb}{0.680000,0.680000,0.680000}%
\pgfsetstrokecolor{currentstroke}%
\pgfsetdash{{1.000000pt}{1.000000pt}}{0.000000pt}%
\pgfpathmoveto{\pgfqpoint{0.721718in}{8.821759in}}%
\pgfpathlineto{\pgfqpoint{6.245000in}{8.821759in}}%
\pgfusepath{stroke}%
\end{pgfscope}%
\begin{pgfscope}%
\pgfsetbuttcap%
\pgfsetroundjoin%
\definecolor{currentfill}{rgb}{0.000000,0.000000,0.000000}%
\pgfsetfillcolor{currentfill}%
\pgfsetlinewidth{0.803000pt}%
\definecolor{currentstroke}{rgb}{0.000000,0.000000,0.000000}%
\pgfsetstrokecolor{currentstroke}%
\pgfsetdash{}{0pt}%
\pgfsys@defobject{currentmarker}{\pgfqpoint{-0.048611in}{0.000000in}}{\pgfqpoint{-0.000000in}{0.000000in}}{%
\pgfpathmoveto{\pgfqpoint{-0.000000in}{0.000000in}}%
\pgfpathlineto{\pgfqpoint{-0.048611in}{0.000000in}}%
\pgfusepath{stroke,fill}%
}%
\begin{pgfscope}%
\pgfsys@transformshift{0.721718in}{8.821759in}%
\pgfsys@useobject{currentmarker}{}%
\end{pgfscope}%
\end{pgfscope}%
\begin{pgfscope}%
\definecolor{textcolor}{rgb}{0.000000,0.000000,0.000000}%
\pgfsetstrokecolor{textcolor}%
\pgfsetfillcolor{textcolor}%
\pgftext[x=0.438155in, y=8.777038in, left, base]{\color{textcolor}\rmfamily\fontsize{9.000000}{10.800000}\selectfont \(\displaystyle {10^{0}}\)}%
\end{pgfscope}%
\begin{pgfscope}%
\definecolor{textcolor}{rgb}{0.000000,0.000000,0.000000}%
\pgfsetstrokecolor{textcolor}%
\pgfsetfillcolor{textcolor}%
\pgftext[x=0.251428in,y=4.665465in,,bottom,rotate=90.000000]{\color{textcolor}\rmfamily\fontsize{9.000000}{10.800000}\selectfont \(\displaystyle 1 - \Phi(z)\)}%
\end{pgfscope}%
\begin{pgfscope}%
\pgfpathrectangle{\pgfqpoint{0.721718in}{0.509170in}}{\pgfqpoint{5.523282in}{8.312590in}}%
\pgfusepath{clip}%
\pgfsetrectcap%
\pgfsetroundjoin%
\pgfsetlinewidth{1.254687pt}%
\definecolor{currentstroke}{rgb}{0.000000,0.000000,0.000000}%
\pgfsetstrokecolor{currentstroke}%
\pgfsetdash{}{0pt}%
\pgfpathmoveto{\pgfqpoint{0.721718in}{8.665363in}}%
\pgfpathlineto{\pgfqpoint{0.804984in}{8.642294in}}%
\pgfpathlineto{\pgfqpoint{0.888249in}{8.616997in}}%
\pgfpathlineto{\pgfqpoint{0.971515in}{8.589391in}}%
\pgfpathlineto{\pgfqpoint{1.054781in}{8.559401in}}%
\pgfpathlineto{\pgfqpoint{1.138046in}{8.526956in}}%
\pgfpathlineto{\pgfqpoint{1.221312in}{8.491992in}}%
\pgfpathlineto{\pgfqpoint{1.304577in}{8.454449in}}%
\pgfpathlineto{\pgfqpoint{1.387843in}{8.414272in}}%
\pgfpathlineto{\pgfqpoint{1.471108in}{8.371410in}}%
\pgfpathlineto{\pgfqpoint{1.554374in}{8.325816in}}%
\pgfpathlineto{\pgfqpoint{1.637639in}{8.277447in}}%
\pgfpathlineto{\pgfqpoint{1.720905in}{8.226264in}}%
\pgfpathlineto{\pgfqpoint{1.804171in}{8.172229in}}%
\pgfpathlineto{\pgfqpoint{1.887436in}{8.115309in}}%
\pgfpathlineto{\pgfqpoint{1.970702in}{8.055474in}}%
\pgfpathlineto{\pgfqpoint{2.053967in}{7.992694in}}%
\pgfpathlineto{\pgfqpoint{2.137233in}{7.926944in}}%
\pgfpathlineto{\pgfqpoint{2.220498in}{7.858200in}}%
\pgfpathlineto{\pgfqpoint{2.303764in}{7.786439in}}%
\pgfpathlineto{\pgfqpoint{2.387029in}{7.711640in}}%
\pgfpathlineto{\pgfqpoint{2.470295in}{7.633785in}}%
\pgfpathlineto{\pgfqpoint{2.553561in}{7.552856in}}%
\pgfpathlineto{\pgfqpoint{2.636826in}{7.468836in}}%
\pgfpathlineto{\pgfqpoint{2.720092in}{7.381711in}}%
\pgfpathlineto{\pgfqpoint{2.803357in}{7.291467in}}%
\pgfpathlineto{\pgfqpoint{2.886623in}{7.198090in}}%
\pgfpathlineto{\pgfqpoint{2.969888in}{7.101569in}}%
\pgfpathlineto{\pgfqpoint{3.053154in}{7.001892in}}%
\pgfpathlineto{\pgfqpoint{3.136419in}{6.899049in}}%
\pgfpathlineto{\pgfqpoint{3.219685in}{6.793029in}}%
\pgfpathlineto{\pgfqpoint{3.302950in}{6.683825in}}%
\pgfpathlineto{\pgfqpoint{3.386216in}{6.571427in}}%
\pgfpathlineto{\pgfqpoint{3.469482in}{6.455827in}}%
\pgfpathlineto{\pgfqpoint{3.552747in}{6.337018in}}%
\pgfpathlineto{\pgfqpoint{3.636013in}{6.214993in}}%
\pgfpathlineto{\pgfqpoint{3.719278in}{6.089746in}}%
\pgfpathlineto{\pgfqpoint{3.802544in}{5.961269in}}%
\pgfpathlineto{\pgfqpoint{3.885809in}{5.829558in}}%
\pgfpathlineto{\pgfqpoint{3.969075in}{5.694607in}}%
\pgfpathlineto{\pgfqpoint{4.052340in}{5.556411in}}%
\pgfpathlineto{\pgfqpoint{4.163361in}{5.367093in}}%
\pgfpathlineto{\pgfqpoint{4.274382in}{5.171986in}}%
\pgfpathlineto{\pgfqpoint{4.385403in}{4.971081in}}%
\pgfpathlineto{\pgfqpoint{4.496423in}{4.764368in}}%
\pgfpathlineto{\pgfqpoint{4.607444in}{4.551839in}}%
\pgfpathlineto{\pgfqpoint{4.718465in}{4.333485in}}%
\pgfpathlineto{\pgfqpoint{4.829486in}{4.109300in}}%
\pgfpathlineto{\pgfqpoint{4.940506in}{3.879276in}}%
\pgfpathlineto{\pgfqpoint{5.051527in}{3.643407in}}%
\pgfpathlineto{\pgfqpoint{5.162548in}{3.401688in}}%
\pgfpathlineto{\pgfqpoint{5.273569in}{3.154111in}}%
\pgfpathlineto{\pgfqpoint{5.384589in}{2.900674in}}%
\pgfpathlineto{\pgfqpoint{5.495610in}{2.641363in}}%
\pgfpathlineto{\pgfqpoint{5.606631in}{2.376217in}}%
\pgfpathlineto{\pgfqpoint{5.717652in}{2.105158in}}%
\pgfpathlineto{\pgfqpoint{5.828672in}{1.827843in}}%
\pgfpathlineto{\pgfqpoint{5.967448in}{1.474637in}}%
\pgfpathlineto{\pgfqpoint{6.022959in}{1.328422in}}%
\pgfpathlineto{\pgfqpoint{6.050714in}{1.259043in}}%
\pgfpathlineto{\pgfqpoint{6.106224in}{1.111497in}}%
\pgfpathlineto{\pgfqpoint{6.133979in}{1.028526in}}%
\pgfpathlineto{\pgfqpoint{6.161734in}{0.971822in}}%
\pgfpathlineto{\pgfqpoint{6.189490in}{0.895903in}}%
\pgfpathlineto{\pgfqpoint{6.217245in}{0.780644in}}%
\pgfpathlineto{\pgfqpoint{6.245000in}{0.689158in}}%
\pgfpathlineto{\pgfqpoint{6.245000in}{0.689158in}}%
\pgfusepath{stroke}%
\end{pgfscope}%
\begin{pgfscope}%
\pgfpathrectangle{\pgfqpoint{0.721718in}{0.509170in}}{\pgfqpoint{5.523282in}{8.312590in}}%
\pgfusepath{clip}%
\pgfsetbuttcap%
\pgfsetroundjoin%
\definecolor{currentfill}{rgb}{0.000000,0.000000,0.000000}%
\pgfsetfillcolor{currentfill}%
\pgfsetlinewidth{1.003750pt}%
\definecolor{currentstroke}{rgb}{0.000000,0.000000,0.000000}%
\pgfsetstrokecolor{currentstroke}%
\pgfsetdash{}{0pt}%
\pgfsys@defobject{currentmarker}{\pgfqpoint{-0.027778in}{-0.027778in}}{\pgfqpoint{0.027778in}{0.027778in}}{%
\pgfpathmoveto{\pgfqpoint{0.000000in}{-0.027778in}}%
\pgfpathcurveto{\pgfqpoint{0.007367in}{-0.027778in}}{\pgfqpoint{0.014433in}{-0.024851in}}{\pgfqpoint{0.019642in}{-0.019642in}}%
\pgfpathcurveto{\pgfqpoint{0.024851in}{-0.014433in}}{\pgfqpoint{0.027778in}{-0.007367in}}{\pgfqpoint{0.027778in}{0.000000in}}%
\pgfpathcurveto{\pgfqpoint{0.027778in}{0.007367in}}{\pgfqpoint{0.024851in}{0.014433in}}{\pgfqpoint{0.019642in}{0.019642in}}%
\pgfpathcurveto{\pgfqpoint{0.014433in}{0.024851in}}{\pgfqpoint{0.007367in}{0.027778in}}{\pgfqpoint{0.000000in}{0.027778in}}%
\pgfpathcurveto{\pgfqpoint{-0.007367in}{0.027778in}}{\pgfqpoint{-0.014433in}{0.024851in}}{\pgfqpoint{-0.019642in}{0.019642in}}%
\pgfpathcurveto{\pgfqpoint{-0.024851in}{0.014433in}}{\pgfqpoint{-0.027778in}{0.007367in}}{\pgfqpoint{-0.027778in}{0.000000in}}%
\pgfpathcurveto{\pgfqpoint{-0.027778in}{-0.007367in}}{\pgfqpoint{-0.024851in}{-0.014433in}}{\pgfqpoint{-0.019642in}{-0.019642in}}%
\pgfpathcurveto{\pgfqpoint{-0.014433in}{-0.024851in}}{\pgfqpoint{-0.007367in}{-0.027778in}}{\pgfqpoint{0.000000in}{-0.027778in}}%
\pgfpathlineto{\pgfqpoint{0.000000in}{-0.027778in}}%
\pgfpathclose%
\pgfusepath{stroke,fill}%
}%
\begin{pgfscope}%
\pgfsys@transformshift{1.403605in}{8.406366in}%
\pgfsys@useobject{currentmarker}{}%
\end{pgfscope}%
\end{pgfscope}%
\begin{pgfscope}%
\pgfpathrectangle{\pgfqpoint{0.721718in}{0.509170in}}{\pgfqpoint{5.523282in}{8.312590in}}%
\pgfusepath{clip}%
\pgfsetbuttcap%
\pgfsetroundjoin%
\definecolor{currentfill}{rgb}{0.000000,0.000000,0.000000}%
\pgfsetfillcolor{currentfill}%
\pgfsetlinewidth{1.003750pt}%
\definecolor{currentstroke}{rgb}{0.000000,0.000000,0.000000}%
\pgfsetstrokecolor{currentstroke}%
\pgfsetdash{}{0pt}%
\pgfsys@defobject{currentmarker}{\pgfqpoint{-0.027778in}{-0.027778in}}{\pgfqpoint{0.027778in}{0.027778in}}{%
\pgfpathmoveto{\pgfqpoint{0.000000in}{-0.027778in}}%
\pgfpathcurveto{\pgfqpoint{0.007367in}{-0.027778in}}{\pgfqpoint{0.014433in}{-0.024851in}}{\pgfqpoint{0.019642in}{-0.019642in}}%
\pgfpathcurveto{\pgfqpoint{0.024851in}{-0.014433in}}{\pgfqpoint{0.027778in}{-0.007367in}}{\pgfqpoint{0.027778in}{0.000000in}}%
\pgfpathcurveto{\pgfqpoint{0.027778in}{0.007367in}}{\pgfqpoint{0.024851in}{0.014433in}}{\pgfqpoint{0.019642in}{0.019642in}}%
\pgfpathcurveto{\pgfqpoint{0.014433in}{0.024851in}}{\pgfqpoint{0.007367in}{0.027778in}}{\pgfqpoint{0.000000in}{0.027778in}}%
\pgfpathcurveto{\pgfqpoint{-0.007367in}{0.027778in}}{\pgfqpoint{-0.014433in}{0.024851in}}{\pgfqpoint{-0.019642in}{0.019642in}}%
\pgfpathcurveto{\pgfqpoint{-0.024851in}{0.014433in}}{\pgfqpoint{-0.027778in}{0.007367in}}{\pgfqpoint{-0.027778in}{0.000000in}}%
\pgfpathcurveto{\pgfqpoint{-0.027778in}{-0.007367in}}{\pgfqpoint{-0.024851in}{-0.014433in}}{\pgfqpoint{-0.019642in}{-0.019642in}}%
\pgfpathcurveto{\pgfqpoint{-0.014433in}{-0.024851in}}{\pgfqpoint{-0.007367in}{-0.027778in}}{\pgfqpoint{0.000000in}{-0.027778in}}%
\pgfpathlineto{\pgfqpoint{0.000000in}{-0.027778in}}%
\pgfpathclose%
\pgfusepath{stroke,fill}%
}%
\begin{pgfscope}%
\pgfsys@transformshift{2.085492in}{7.968152in}%
\pgfsys@useobject{currentmarker}{}%
\end{pgfscope}%
\end{pgfscope}%
\begin{pgfscope}%
\pgfpathrectangle{\pgfqpoint{0.721718in}{0.509170in}}{\pgfqpoint{5.523282in}{8.312590in}}%
\pgfusepath{clip}%
\pgfsetbuttcap%
\pgfsetroundjoin%
\definecolor{currentfill}{rgb}{0.000000,0.000000,0.000000}%
\pgfsetfillcolor{currentfill}%
\pgfsetlinewidth{1.003750pt}%
\definecolor{currentstroke}{rgb}{0.000000,0.000000,0.000000}%
\pgfsetstrokecolor{currentstroke}%
\pgfsetdash{}{0pt}%
\pgfsys@defobject{currentmarker}{\pgfqpoint{-0.027778in}{-0.027778in}}{\pgfqpoint{0.027778in}{0.027778in}}{%
\pgfpathmoveto{\pgfqpoint{0.000000in}{-0.027778in}}%
\pgfpathcurveto{\pgfqpoint{0.007367in}{-0.027778in}}{\pgfqpoint{0.014433in}{-0.024851in}}{\pgfqpoint{0.019642in}{-0.019642in}}%
\pgfpathcurveto{\pgfqpoint{0.024851in}{-0.014433in}}{\pgfqpoint{0.027778in}{-0.007367in}}{\pgfqpoint{0.027778in}{0.000000in}}%
\pgfpathcurveto{\pgfqpoint{0.027778in}{0.007367in}}{\pgfqpoint{0.024851in}{0.014433in}}{\pgfqpoint{0.019642in}{0.019642in}}%
\pgfpathcurveto{\pgfqpoint{0.014433in}{0.024851in}}{\pgfqpoint{0.007367in}{0.027778in}}{\pgfqpoint{0.000000in}{0.027778in}}%
\pgfpathcurveto{\pgfqpoint{-0.007367in}{0.027778in}}{\pgfqpoint{-0.014433in}{0.024851in}}{\pgfqpoint{-0.019642in}{0.019642in}}%
\pgfpathcurveto{\pgfqpoint{-0.024851in}{0.014433in}}{\pgfqpoint{-0.027778in}{0.007367in}}{\pgfqpoint{-0.027778in}{0.000000in}}%
\pgfpathcurveto{\pgfqpoint{-0.027778in}{-0.007367in}}{\pgfqpoint{-0.024851in}{-0.014433in}}{\pgfqpoint{-0.019642in}{-0.019642in}}%
\pgfpathcurveto{\pgfqpoint{-0.014433in}{-0.024851in}}{\pgfqpoint{-0.007367in}{-0.027778in}}{\pgfqpoint{0.000000in}{-0.027778in}}%
\pgfpathlineto{\pgfqpoint{0.000000in}{-0.027778in}}%
\pgfpathclose%
\pgfusepath{stroke,fill}%
}%
\begin{pgfscope}%
\pgfsys@transformshift{2.767378in}{7.330845in}%
\pgfsys@useobject{currentmarker}{}%
\end{pgfscope}%
\end{pgfscope}%
\begin{pgfscope}%
\pgfpathrectangle{\pgfqpoint{0.721718in}{0.509170in}}{\pgfqpoint{5.523282in}{8.312590in}}%
\pgfusepath{clip}%
\pgfsetbuttcap%
\pgfsetroundjoin%
\definecolor{currentfill}{rgb}{0.000000,0.000000,0.000000}%
\pgfsetfillcolor{currentfill}%
\pgfsetlinewidth{1.003750pt}%
\definecolor{currentstroke}{rgb}{0.000000,0.000000,0.000000}%
\pgfsetstrokecolor{currentstroke}%
\pgfsetdash{}{0pt}%
\pgfsys@defobject{currentmarker}{\pgfqpoint{-0.027778in}{-0.027778in}}{\pgfqpoint{0.027778in}{0.027778in}}{%
\pgfpathmoveto{\pgfqpoint{0.000000in}{-0.027778in}}%
\pgfpathcurveto{\pgfqpoint{0.007367in}{-0.027778in}}{\pgfqpoint{0.014433in}{-0.024851in}}{\pgfqpoint{0.019642in}{-0.019642in}}%
\pgfpathcurveto{\pgfqpoint{0.024851in}{-0.014433in}}{\pgfqpoint{0.027778in}{-0.007367in}}{\pgfqpoint{0.027778in}{0.000000in}}%
\pgfpathcurveto{\pgfqpoint{0.027778in}{0.007367in}}{\pgfqpoint{0.024851in}{0.014433in}}{\pgfqpoint{0.019642in}{0.019642in}}%
\pgfpathcurveto{\pgfqpoint{0.014433in}{0.024851in}}{\pgfqpoint{0.007367in}{0.027778in}}{\pgfqpoint{0.000000in}{0.027778in}}%
\pgfpathcurveto{\pgfqpoint{-0.007367in}{0.027778in}}{\pgfqpoint{-0.014433in}{0.024851in}}{\pgfqpoint{-0.019642in}{0.019642in}}%
\pgfpathcurveto{\pgfqpoint{-0.024851in}{0.014433in}}{\pgfqpoint{-0.027778in}{0.007367in}}{\pgfqpoint{-0.027778in}{0.000000in}}%
\pgfpathcurveto{\pgfqpoint{-0.027778in}{-0.007367in}}{\pgfqpoint{-0.024851in}{-0.014433in}}{\pgfqpoint{-0.019642in}{-0.019642in}}%
\pgfpathcurveto{\pgfqpoint{-0.014433in}{-0.024851in}}{\pgfqpoint{-0.007367in}{-0.027778in}}{\pgfqpoint{0.000000in}{-0.027778in}}%
\pgfpathlineto{\pgfqpoint{0.000000in}{-0.027778in}}%
\pgfpathclose%
\pgfusepath{stroke,fill}%
}%
\begin{pgfscope}%
\pgfsys@transformshift{3.449265in}{6.484189in}%
\pgfsys@useobject{currentmarker}{}%
\end{pgfscope}%
\end{pgfscope}%
\begin{pgfscope}%
\pgfpathrectangle{\pgfqpoint{0.721718in}{0.509170in}}{\pgfqpoint{5.523282in}{8.312590in}}%
\pgfusepath{clip}%
\pgfsetbuttcap%
\pgfsetroundjoin%
\definecolor{currentfill}{rgb}{0.000000,0.000000,0.000000}%
\pgfsetfillcolor{currentfill}%
\pgfsetlinewidth{1.003750pt}%
\definecolor{currentstroke}{rgb}{0.000000,0.000000,0.000000}%
\pgfsetstrokecolor{currentstroke}%
\pgfsetdash{}{0pt}%
\pgfsys@defobject{currentmarker}{\pgfqpoint{-0.027778in}{-0.027778in}}{\pgfqpoint{0.027778in}{0.027778in}}{%
\pgfpathmoveto{\pgfqpoint{0.000000in}{-0.027778in}}%
\pgfpathcurveto{\pgfqpoint{0.007367in}{-0.027778in}}{\pgfqpoint{0.014433in}{-0.024851in}}{\pgfqpoint{0.019642in}{-0.019642in}}%
\pgfpathcurveto{\pgfqpoint{0.024851in}{-0.014433in}}{\pgfqpoint{0.027778in}{-0.007367in}}{\pgfqpoint{0.027778in}{0.000000in}}%
\pgfpathcurveto{\pgfqpoint{0.027778in}{0.007367in}}{\pgfqpoint{0.024851in}{0.014433in}}{\pgfqpoint{0.019642in}{0.019642in}}%
\pgfpathcurveto{\pgfqpoint{0.014433in}{0.024851in}}{\pgfqpoint{0.007367in}{0.027778in}}{\pgfqpoint{0.000000in}{0.027778in}}%
\pgfpathcurveto{\pgfqpoint{-0.007367in}{0.027778in}}{\pgfqpoint{-0.014433in}{0.024851in}}{\pgfqpoint{-0.019642in}{0.019642in}}%
\pgfpathcurveto{\pgfqpoint{-0.024851in}{0.014433in}}{\pgfqpoint{-0.027778in}{0.007367in}}{\pgfqpoint{-0.027778in}{0.000000in}}%
\pgfpathcurveto{\pgfqpoint{-0.027778in}{-0.007367in}}{\pgfqpoint{-0.024851in}{-0.014433in}}{\pgfqpoint{-0.019642in}{-0.019642in}}%
\pgfpathcurveto{\pgfqpoint{-0.014433in}{-0.024851in}}{\pgfqpoint{-0.007367in}{-0.027778in}}{\pgfqpoint{0.000000in}{-0.027778in}}%
\pgfpathlineto{\pgfqpoint{0.000000in}{-0.027778in}}%
\pgfpathclose%
\pgfusepath{stroke,fill}%
}%
\begin{pgfscope}%
\pgfsys@transformshift{4.131151in}{5.422614in}%
\pgfsys@useobject{currentmarker}{}%
\end{pgfscope}%
\end{pgfscope}%
\begin{pgfscope}%
\pgfpathrectangle{\pgfqpoint{0.721718in}{0.509170in}}{\pgfqpoint{5.523282in}{8.312590in}}%
\pgfusepath{clip}%
\pgfsetbuttcap%
\pgfsetroundjoin%
\definecolor{currentfill}{rgb}{0.000000,0.000000,0.000000}%
\pgfsetfillcolor{currentfill}%
\pgfsetlinewidth{1.003750pt}%
\definecolor{currentstroke}{rgb}{0.000000,0.000000,0.000000}%
\pgfsetstrokecolor{currentstroke}%
\pgfsetdash{}{0pt}%
\pgfsys@defobject{currentmarker}{\pgfqpoint{-0.027778in}{-0.027778in}}{\pgfqpoint{0.027778in}{0.027778in}}{%
\pgfpathmoveto{\pgfqpoint{0.000000in}{-0.027778in}}%
\pgfpathcurveto{\pgfqpoint{0.007367in}{-0.027778in}}{\pgfqpoint{0.014433in}{-0.024851in}}{\pgfqpoint{0.019642in}{-0.019642in}}%
\pgfpathcurveto{\pgfqpoint{0.024851in}{-0.014433in}}{\pgfqpoint{0.027778in}{-0.007367in}}{\pgfqpoint{0.027778in}{0.000000in}}%
\pgfpathcurveto{\pgfqpoint{0.027778in}{0.007367in}}{\pgfqpoint{0.024851in}{0.014433in}}{\pgfqpoint{0.019642in}{0.019642in}}%
\pgfpathcurveto{\pgfqpoint{0.014433in}{0.024851in}}{\pgfqpoint{0.007367in}{0.027778in}}{\pgfqpoint{0.000000in}{0.027778in}}%
\pgfpathcurveto{\pgfqpoint{-0.007367in}{0.027778in}}{\pgfqpoint{-0.014433in}{0.024851in}}{\pgfqpoint{-0.019642in}{0.019642in}}%
\pgfpathcurveto{\pgfqpoint{-0.024851in}{0.014433in}}{\pgfqpoint{-0.027778in}{0.007367in}}{\pgfqpoint{-0.027778in}{0.000000in}}%
\pgfpathcurveto{\pgfqpoint{-0.027778in}{-0.007367in}}{\pgfqpoint{-0.024851in}{-0.014433in}}{\pgfqpoint{-0.019642in}{-0.019642in}}%
\pgfpathcurveto{\pgfqpoint{-0.014433in}{-0.024851in}}{\pgfqpoint{-0.007367in}{-0.027778in}}{\pgfqpoint{0.000000in}{-0.027778in}}%
\pgfpathlineto{\pgfqpoint{0.000000in}{-0.027778in}}%
\pgfpathclose%
\pgfusepath{stroke,fill}%
}%
\begin{pgfscope}%
\pgfsys@transformshift{4.813038in}{4.142881in}%
\pgfsys@useobject{currentmarker}{}%
\end{pgfscope}%
\end{pgfscope}%
\begin{pgfscope}%
\pgfpathrectangle{\pgfqpoint{0.721718in}{0.509170in}}{\pgfqpoint{5.523282in}{8.312590in}}%
\pgfusepath{clip}%
\pgfsetbuttcap%
\pgfsetroundjoin%
\definecolor{currentfill}{rgb}{0.000000,0.000000,0.000000}%
\pgfsetfillcolor{currentfill}%
\pgfsetlinewidth{1.003750pt}%
\definecolor{currentstroke}{rgb}{0.000000,0.000000,0.000000}%
\pgfsetstrokecolor{currentstroke}%
\pgfsetdash{}{0pt}%
\pgfsys@defobject{currentmarker}{\pgfqpoint{-0.027778in}{-0.027778in}}{\pgfqpoint{0.027778in}{0.027778in}}{%
\pgfpathmoveto{\pgfqpoint{0.000000in}{-0.027778in}}%
\pgfpathcurveto{\pgfqpoint{0.007367in}{-0.027778in}}{\pgfqpoint{0.014433in}{-0.024851in}}{\pgfqpoint{0.019642in}{-0.019642in}}%
\pgfpathcurveto{\pgfqpoint{0.024851in}{-0.014433in}}{\pgfqpoint{0.027778in}{-0.007367in}}{\pgfqpoint{0.027778in}{0.000000in}}%
\pgfpathcurveto{\pgfqpoint{0.027778in}{0.007367in}}{\pgfqpoint{0.024851in}{0.014433in}}{\pgfqpoint{0.019642in}{0.019642in}}%
\pgfpathcurveto{\pgfqpoint{0.014433in}{0.024851in}}{\pgfqpoint{0.007367in}{0.027778in}}{\pgfqpoint{0.000000in}{0.027778in}}%
\pgfpathcurveto{\pgfqpoint{-0.007367in}{0.027778in}}{\pgfqpoint{-0.014433in}{0.024851in}}{\pgfqpoint{-0.019642in}{0.019642in}}%
\pgfpathcurveto{\pgfqpoint{-0.024851in}{0.014433in}}{\pgfqpoint{-0.027778in}{0.007367in}}{\pgfqpoint{-0.027778in}{0.000000in}}%
\pgfpathcurveto{\pgfqpoint{-0.027778in}{-0.007367in}}{\pgfqpoint{-0.024851in}{-0.014433in}}{\pgfqpoint{-0.019642in}{-0.019642in}}%
\pgfpathcurveto{\pgfqpoint{-0.014433in}{-0.024851in}}{\pgfqpoint{-0.007367in}{-0.027778in}}{\pgfqpoint{0.000000in}{-0.027778in}}%
\pgfpathlineto{\pgfqpoint{0.000000in}{-0.027778in}}%
\pgfpathclose%
\pgfusepath{stroke,fill}%
}%
\begin{pgfscope}%
\pgfsys@transformshift{5.494925in}{2.642993in}%
\pgfsys@useobject{currentmarker}{}%
\end{pgfscope}%
\end{pgfscope}%
\begin{pgfscope}%
\pgfpathrectangle{\pgfqpoint{0.721718in}{0.509170in}}{\pgfqpoint{5.523282in}{8.312590in}}%
\pgfusepath{clip}%
\pgfsetbuttcap%
\pgfsetroundjoin%
\definecolor{currentfill}{rgb}{0.000000,0.000000,0.000000}%
\pgfsetfillcolor{currentfill}%
\pgfsetlinewidth{1.003750pt}%
\definecolor{currentstroke}{rgb}{0.000000,0.000000,0.000000}%
\pgfsetstrokecolor{currentstroke}%
\pgfsetdash{}{0pt}%
\pgfsys@defobject{currentmarker}{\pgfqpoint{-0.027778in}{-0.027778in}}{\pgfqpoint{0.027778in}{0.027778in}}{%
\pgfpathmoveto{\pgfqpoint{0.000000in}{-0.027778in}}%
\pgfpathcurveto{\pgfqpoint{0.007367in}{-0.027778in}}{\pgfqpoint{0.014433in}{-0.024851in}}{\pgfqpoint{0.019642in}{-0.019642in}}%
\pgfpathcurveto{\pgfqpoint{0.024851in}{-0.014433in}}{\pgfqpoint{0.027778in}{-0.007367in}}{\pgfqpoint{0.027778in}{0.000000in}}%
\pgfpathcurveto{\pgfqpoint{0.027778in}{0.007367in}}{\pgfqpoint{0.024851in}{0.014433in}}{\pgfqpoint{0.019642in}{0.019642in}}%
\pgfpathcurveto{\pgfqpoint{0.014433in}{0.024851in}}{\pgfqpoint{0.007367in}{0.027778in}}{\pgfqpoint{0.000000in}{0.027778in}}%
\pgfpathcurveto{\pgfqpoint{-0.007367in}{0.027778in}}{\pgfqpoint{-0.014433in}{0.024851in}}{\pgfqpoint{-0.019642in}{0.019642in}}%
\pgfpathcurveto{\pgfqpoint{-0.024851in}{0.014433in}}{\pgfqpoint{-0.027778in}{0.007367in}}{\pgfqpoint{-0.027778in}{0.000000in}}%
\pgfpathcurveto{\pgfqpoint{-0.027778in}{-0.007367in}}{\pgfqpoint{-0.024851in}{-0.014433in}}{\pgfqpoint{-0.019642in}{-0.019642in}}%
\pgfpathcurveto{\pgfqpoint{-0.014433in}{-0.024851in}}{\pgfqpoint{-0.007367in}{-0.027778in}}{\pgfqpoint{0.000000in}{-0.027778in}}%
\pgfpathlineto{\pgfqpoint{0.000000in}{-0.027778in}}%
\pgfpathclose%
\pgfusepath{stroke,fill}%
}%
\begin{pgfscope}%
\pgfsys@transformshift{6.176811in}{0.937040in}%
\pgfsys@useobject{currentmarker}{}%
\end{pgfscope}%
\end{pgfscope}%
\begin{pgfscope}%
\pgfsetrectcap%
\pgfsetmiterjoin%
\pgfsetlinewidth{0.803000pt}%
\definecolor{currentstroke}{rgb}{0.000000,0.000000,0.000000}%
\pgfsetstrokecolor{currentstroke}%
\pgfsetdash{}{0pt}%
\pgfpathmoveto{\pgfqpoint{0.721718in}{0.509170in}}%
\pgfpathlineto{\pgfqpoint{0.721718in}{8.821759in}}%
\pgfusepath{stroke}%
\end{pgfscope}%
\begin{pgfscope}%
\pgfsetrectcap%
\pgfsetmiterjoin%
\pgfsetlinewidth{0.803000pt}%
\definecolor{currentstroke}{rgb}{0.000000,0.000000,0.000000}%
\pgfsetstrokecolor{currentstroke}%
\pgfsetdash{}{0pt}%
\pgfpathmoveto{\pgfqpoint{6.245000in}{0.509170in}}%
\pgfpathlineto{\pgfqpoint{6.245000in}{8.821759in}}%
\pgfusepath{stroke}%
\end{pgfscope}%
\begin{pgfscope}%
\pgfsetrectcap%
\pgfsetmiterjoin%
\pgfsetlinewidth{0.803000pt}%
\definecolor{currentstroke}{rgb}{0.000000,0.000000,0.000000}%
\pgfsetstrokecolor{currentstroke}%
\pgfsetdash{}{0pt}%
\pgfpathmoveto{\pgfqpoint{0.721718in}{0.509170in}}%
\pgfpathlineto{\pgfqpoint{6.245000in}{0.509170in}}%
\pgfusepath{stroke}%
\end{pgfscope}%
\begin{pgfscope}%
\pgfsetrectcap%
\pgfsetmiterjoin%
\pgfsetlinewidth{0.803000pt}%
\definecolor{currentstroke}{rgb}{0.000000,0.000000,0.000000}%
\pgfsetstrokecolor{currentstroke}%
\pgfsetdash{}{0pt}%
\pgfpathmoveto{\pgfqpoint{0.721718in}{8.821759in}}%
\pgfpathlineto{\pgfqpoint{6.245000in}{8.821759in}}%
\pgfusepath{stroke}%
\end{pgfscope}%
\begin{pgfscope}%
\definecolor{textcolor}{rgb}{0.000000,0.000000,0.000000}%
\pgfsetstrokecolor{textcolor}%
\pgfsetfillcolor{textcolor}%
\pgftext[x=1.471794in,y=8.406366in,left,]{\color{textcolor}\rmfamily\fontsize{9.000000}{10.800000}\selectfont \(\displaystyle {1.59} \times 10^{-1}\)}%
\end{pgfscope}%
\begin{pgfscope}%
\definecolor{textcolor}{rgb}{0.000000,0.000000,0.000000}%
\pgfsetstrokecolor{textcolor}%
\pgfsetfillcolor{textcolor}%
\pgftext[x=2.153680in,y=7.968152in,left,]{\color{textcolor}\rmfamily\fontsize{9.000000}{10.800000}\selectfont \(\displaystyle {2.28} \times 10^{-2}\)}%
\end{pgfscope}%
\begin{pgfscope}%
\definecolor{textcolor}{rgb}{0.000000,0.000000,0.000000}%
\pgfsetstrokecolor{textcolor}%
\pgfsetfillcolor{textcolor}%
\pgftext[x=2.835567in,y=7.330845in,left,]{\color{textcolor}\rmfamily\fontsize{9.000000}{10.800000}\selectfont \(\displaystyle {1.35} \times 10^{-3}\)}%
\end{pgfscope}%
\begin{pgfscope}%
\definecolor{textcolor}{rgb}{0.000000,0.000000,0.000000}%
\pgfsetstrokecolor{textcolor}%
\pgfsetfillcolor{textcolor}%
\pgftext[x=3.517454in,y=6.484189in,left,]{\color{textcolor}\rmfamily\fontsize{9.000000}{10.800000}\selectfont \(\displaystyle {3.17} \times 10^{-5}\)}%
\end{pgfscope}%
\begin{pgfscope}%
\definecolor{textcolor}{rgb}{0.000000,0.000000,0.000000}%
\pgfsetstrokecolor{textcolor}%
\pgfsetfillcolor{textcolor}%
\pgftext[x=4.199340in,y=5.422614in,left,]{\color{textcolor}\rmfamily\fontsize{9.000000}{10.800000}\selectfont \(\displaystyle {2.87} \times 10^{-7}\)}%
\end{pgfscope}%
\begin{pgfscope}%
\definecolor{textcolor}{rgb}{0.000000,0.000000,0.000000}%
\pgfsetstrokecolor{textcolor}%
\pgfsetfillcolor{textcolor}%
\pgftext[x=4.881227in,y=4.142881in,left,]{\color{textcolor}\rmfamily\fontsize{9.000000}{10.800000}\selectfont \(\displaystyle {9.87} \times 10^{-10}\)}%
\end{pgfscope}%
\begin{pgfscope}%
\definecolor{textcolor}{rgb}{0.000000,0.000000,0.000000}%
\pgfsetstrokecolor{textcolor}%
\pgfsetfillcolor{textcolor}%
\pgftext[x=5.426736in,y=2.642993in,right,]{\color{textcolor}\rmfamily\fontsize{9.000000}{10.800000}\selectfont \(\displaystyle {1.28} \times 10^{-12}\)}%
\end{pgfscope}%
\begin{pgfscope}%
\definecolor{textcolor}{rgb}{0.000000,0.000000,0.000000}%
\pgfsetstrokecolor{textcolor}%
\pgfsetfillcolor{textcolor}%
\pgftext[x=6.108623in,y=0.937040in,right,]{\color{textcolor}\rmfamily\fontsize{9.000000}{10.800000}\selectfont \(\displaystyle {6.66} \times 10^{-16}\)}%
\end{pgfscope}%
\end{pgfpicture}%
\makeatother%
\endgroup%

  \caption{Grafico della funzione
    $1 - \Phi(z) = \frac{1}{2} - \frac{1}{2}\,\erf{\frac{z}{\sqrt{2}}}$.}
  \label{fig:erf_tail}
\end{figure}


\clearpage

\section{Funzione cumulativa del \texorpdfstring{$\chisq$}{chi quadro}}
\label{sec:tavola_chisq1}

\autohstack[3pt]{%% Creator: Matplotlib, PGF backend
%%
%% To include the figure in your LaTeX document, write
%%   \input{<filename>.pgf}
%%
%% Make sure the required packages are loaded in your preamble
%%   \usepackage{pgf}
%%
%% and, on pdftex
%%   \usepackage[utf8]{inputenc}\DeclareUnicodeCharacter{2212}{-}
%%
%% or, on luatex and xetex
%%   \usepackage{unicode-math}
%%
%% Figures using additional raster images can only be included by \input if
%% they are in the same directory as the main LaTeX file. For loading figures
%% from other directories you can use the `import` package
%%   \usepackage{import}
%%
%% and then include the figures with
%%   \import{<path to file>}{<filename>.pgf}
%%
%% Matplotlib used the following preamble
%%   \usepackage[nice]{nicefrac}
%%   \usepackage{amsmath}
%%   \DeclareUnicodeCharacter{2212}{-}
%%
\begingroup%
\makeatletter%
\begin{pgfpicture}%
\pgfpathrectangle{\pgfpointorigin}{\pgfqpoint{2.250000in}{1.850000in}}%
\pgfusepath{use as bounding box, clip}%
\begin{pgfscope}%
\pgfsetbuttcap%
\pgfsetmiterjoin%
\definecolor{currentfill}{rgb}{1.000000,1.000000,1.000000}%
\pgfsetfillcolor{currentfill}%
\pgfsetlinewidth{0.000000pt}%
\definecolor{currentstroke}{rgb}{1.000000,1.000000,1.000000}%
\pgfsetstrokecolor{currentstroke}%
\pgfsetdash{}{0pt}%
\pgfpathmoveto{\pgfqpoint{0.000000in}{0.000000in}}%
\pgfpathlineto{\pgfqpoint{2.250000in}{0.000000in}}%
\pgfpathlineto{\pgfqpoint{2.250000in}{1.850000in}}%
\pgfpathlineto{\pgfqpoint{0.000000in}{1.850000in}}%
\pgfpathclose%
\pgfusepath{fill}%
\end{pgfscope}%
\begin{pgfscope}%
\pgfsetbuttcap%
\pgfsetmiterjoin%
\definecolor{currentfill}{rgb}{1.000000,1.000000,1.000000}%
\pgfsetfillcolor{currentfill}%
\pgfsetlinewidth{0.000000pt}%
\definecolor{currentstroke}{rgb}{0.000000,0.000000,0.000000}%
\pgfsetstrokecolor{currentstroke}%
\pgfsetstrokeopacity{0.000000}%
\pgfsetdash{}{0pt}%
\pgfpathmoveto{\pgfqpoint{0.405000in}{0.333000in}}%
\pgfpathlineto{\pgfqpoint{2.137500in}{0.333000in}}%
\pgfpathlineto{\pgfqpoint{2.137500in}{1.776000in}}%
\pgfpathlineto{\pgfqpoint{0.405000in}{1.776000in}}%
\pgfpathclose%
\pgfusepath{fill}%
\end{pgfscope}%
\begin{pgfscope}%
\pgfpathrectangle{\pgfqpoint{0.405000in}{0.333000in}}{\pgfqpoint{1.732500in}{1.443000in}}%
\pgfusepath{clip}%
\pgfsetbuttcap%
\pgfsetroundjoin%
\definecolor{currentfill}{rgb}{0.501961,0.501961,0.501961}%
\pgfsetfillcolor{currentfill}%
\pgfsetfillopacity{0.500000}%
\pgfsetlinewidth{0.000000pt}%
\definecolor{currentstroke}{rgb}{0.000000,0.000000,0.000000}%
\pgfsetstrokecolor{currentstroke}%
\pgfsetdash{}{0pt}%
\pgfpathmoveto{\pgfqpoint{0.405000in}{0.333000in}}%
\pgfpathlineto{\pgfqpoint{0.405000in}{0.333000in}}%
\pgfpathlineto{\pgfqpoint{0.422500in}{0.333000in}}%
\pgfpathlineto{\pgfqpoint{0.440000in}{0.333000in}}%
\pgfpathlineto{\pgfqpoint{0.457500in}{0.333000in}}%
\pgfpathlineto{\pgfqpoint{0.475000in}{0.333000in}}%
\pgfpathlineto{\pgfqpoint{0.492500in}{0.333000in}}%
\pgfpathlineto{\pgfqpoint{0.510000in}{0.333000in}}%
\pgfpathlineto{\pgfqpoint{0.527500in}{0.333000in}}%
\pgfpathlineto{\pgfqpoint{0.545000in}{0.333000in}}%
\pgfpathlineto{\pgfqpoint{0.562500in}{0.333000in}}%
\pgfpathlineto{\pgfqpoint{0.580000in}{0.333000in}}%
\pgfpathlineto{\pgfqpoint{0.597500in}{0.333000in}}%
\pgfpathlineto{\pgfqpoint{0.615000in}{0.333000in}}%
\pgfpathlineto{\pgfqpoint{0.632500in}{0.333000in}}%
\pgfpathlineto{\pgfqpoint{0.650000in}{0.333000in}}%
\pgfpathlineto{\pgfqpoint{0.667500in}{0.333000in}}%
\pgfpathlineto{\pgfqpoint{0.685000in}{0.333000in}}%
\pgfpathlineto{\pgfqpoint{0.702500in}{0.333000in}}%
\pgfpathlineto{\pgfqpoint{0.720000in}{0.333000in}}%
\pgfpathlineto{\pgfqpoint{0.737500in}{0.333000in}}%
\pgfpathlineto{\pgfqpoint{0.755000in}{0.333000in}}%
\pgfpathlineto{\pgfqpoint{0.772500in}{0.333000in}}%
\pgfpathlineto{\pgfqpoint{0.790000in}{0.333000in}}%
\pgfpathlineto{\pgfqpoint{0.807500in}{0.333000in}}%
\pgfpathlineto{\pgfqpoint{0.825000in}{0.333000in}}%
\pgfpathlineto{\pgfqpoint{0.842500in}{0.333000in}}%
\pgfpathlineto{\pgfqpoint{0.860000in}{0.333000in}}%
\pgfpathlineto{\pgfqpoint{0.877500in}{0.333000in}}%
\pgfpathlineto{\pgfqpoint{0.895000in}{0.333000in}}%
\pgfpathlineto{\pgfqpoint{0.912500in}{0.333000in}}%
\pgfpathlineto{\pgfqpoint{0.930000in}{0.333000in}}%
\pgfpathlineto{\pgfqpoint{0.947500in}{0.333000in}}%
\pgfpathlineto{\pgfqpoint{0.965000in}{0.333000in}}%
\pgfpathlineto{\pgfqpoint{0.982500in}{0.333000in}}%
\pgfpathlineto{\pgfqpoint{1.000000in}{0.333000in}}%
\pgfpathlineto{\pgfqpoint{1.000000in}{1.429560in}}%
\pgfpathlineto{\pgfqpoint{1.000000in}{1.429560in}}%
\pgfpathlineto{\pgfqpoint{0.982500in}{1.435857in}}%
\pgfpathlineto{\pgfqpoint{0.965000in}{1.440662in}}%
\pgfpathlineto{\pgfqpoint{0.947500in}{1.443859in}}%
\pgfpathlineto{\pgfqpoint{0.930000in}{1.445327in}}%
\pgfpathlineto{\pgfqpoint{0.912500in}{1.444940in}}%
\pgfpathlineto{\pgfqpoint{0.895000in}{1.442572in}}%
\pgfpathlineto{\pgfqpoint{0.877500in}{1.438092in}}%
\pgfpathlineto{\pgfqpoint{0.860000in}{1.431365in}}%
\pgfpathlineto{\pgfqpoint{0.842500in}{1.422258in}}%
\pgfpathlineto{\pgfqpoint{0.825000in}{1.410635in}}%
\pgfpathlineto{\pgfqpoint{0.807500in}{1.396361in}}%
\pgfpathlineto{\pgfqpoint{0.790000in}{1.379302in}}%
\pgfpathlineto{\pgfqpoint{0.772500in}{1.359327in}}%
\pgfpathlineto{\pgfqpoint{0.755000in}{1.336312in}}%
\pgfpathlineto{\pgfqpoint{0.737500in}{1.310137in}}%
\pgfpathlineto{\pgfqpoint{0.720000in}{1.280694in}}%
\pgfpathlineto{\pgfqpoint{0.702500in}{1.247885in}}%
\pgfpathlineto{\pgfqpoint{0.685000in}{1.211633in}}%
\pgfpathlineto{\pgfqpoint{0.667500in}{1.171877in}}%
\pgfpathlineto{\pgfqpoint{0.650000in}{1.128587in}}%
\pgfpathlineto{\pgfqpoint{0.632500in}{1.081763in}}%
\pgfpathlineto{\pgfqpoint{0.615000in}{1.031450in}}%
\pgfpathlineto{\pgfqpoint{0.597500in}{0.977743in}}%
\pgfpathlineto{\pgfqpoint{0.580000in}{0.920803in}}%
\pgfpathlineto{\pgfqpoint{0.562500in}{0.860873in}}%
\pgfpathlineto{\pgfqpoint{0.545000in}{0.798302in}}%
\pgfpathlineto{\pgfqpoint{0.527500in}{0.733572in}}%
\pgfpathlineto{\pgfqpoint{0.510000in}{0.667345in}}%
\pgfpathlineto{\pgfqpoint{0.492500in}{0.600520in}}%
\pgfpathlineto{\pgfqpoint{0.475000in}{0.534338in}}%
\pgfpathlineto{\pgfqpoint{0.457500in}{0.470547in}}%
\pgfpathlineto{\pgfqpoint{0.440000in}{0.411750in}}%
\pgfpathlineto{\pgfqpoint{0.422500in}{0.362285in}}%
\pgfpathlineto{\pgfqpoint{0.405000in}{0.333000in}}%
\pgfpathclose%
\pgfusepath{fill}%
\end{pgfscope}%
\begin{pgfscope}%
\pgfsetbuttcap%
\pgfsetroundjoin%
\definecolor{currentfill}{rgb}{0.000000,0.000000,0.000000}%
\pgfsetfillcolor{currentfill}%
\pgfsetlinewidth{0.803000pt}%
\definecolor{currentstroke}{rgb}{0.000000,0.000000,0.000000}%
\pgfsetstrokecolor{currentstroke}%
\pgfsetdash{}{0pt}%
\pgfsys@defobject{currentmarker}{\pgfqpoint{0.000000in}{-0.048611in}}{\pgfqpoint{0.000000in}{0.000000in}}{%
\pgfpathmoveto{\pgfqpoint{0.000000in}{0.000000in}}%
\pgfpathlineto{\pgfqpoint{0.000000in}{-0.048611in}}%
\pgfusepath{stroke,fill}%
}%
\begin{pgfscope}%
\pgfsys@transformshift{0.405000in}{0.333000in}%
\pgfsys@useobject{currentmarker}{}%
\end{pgfscope}%
\end{pgfscope}%
\begin{pgfscope}%
\definecolor{textcolor}{rgb}{0.000000,0.000000,0.000000}%
\pgfsetstrokecolor{textcolor}%
\pgfsetfillcolor{textcolor}%
\pgftext[x=0.405000in,y=0.235778in,,top]{\color{textcolor}\rmfamily\fontsize{9.000000}{10.800000}\selectfont 0.0}%
\end{pgfscope}%
\begin{pgfscope}%
\pgfsetbuttcap%
\pgfsetroundjoin%
\definecolor{currentfill}{rgb}{0.000000,0.000000,0.000000}%
\pgfsetfillcolor{currentfill}%
\pgfsetlinewidth{0.803000pt}%
\definecolor{currentstroke}{rgb}{0.000000,0.000000,0.000000}%
\pgfsetstrokecolor{currentstroke}%
\pgfsetdash{}{0pt}%
\pgfsys@defobject{currentmarker}{\pgfqpoint{0.000000in}{-0.048611in}}{\pgfqpoint{0.000000in}{0.000000in}}{%
\pgfpathmoveto{\pgfqpoint{0.000000in}{0.000000in}}%
\pgfpathlineto{\pgfqpoint{0.000000in}{-0.048611in}}%
\pgfusepath{stroke,fill}%
}%
\begin{pgfscope}%
\pgfsys@transformshift{0.838125in}{0.333000in}%
\pgfsys@useobject{currentmarker}{}%
\end{pgfscope}%
\end{pgfscope}%
\begin{pgfscope}%
\definecolor{textcolor}{rgb}{0.000000,0.000000,0.000000}%
\pgfsetstrokecolor{textcolor}%
\pgfsetfillcolor{textcolor}%
\pgftext[x=0.838125in,y=0.235778in,,top]{\color{textcolor}\rmfamily\fontsize{9.000000}{10.800000}\selectfont 2.5}%
\end{pgfscope}%
\begin{pgfscope}%
\pgfsetbuttcap%
\pgfsetroundjoin%
\definecolor{currentfill}{rgb}{0.000000,0.000000,0.000000}%
\pgfsetfillcolor{currentfill}%
\pgfsetlinewidth{0.803000pt}%
\definecolor{currentstroke}{rgb}{0.000000,0.000000,0.000000}%
\pgfsetstrokecolor{currentstroke}%
\pgfsetdash{}{0pt}%
\pgfsys@defobject{currentmarker}{\pgfqpoint{0.000000in}{-0.048611in}}{\pgfqpoint{0.000000in}{0.000000in}}{%
\pgfpathmoveto{\pgfqpoint{0.000000in}{0.000000in}}%
\pgfpathlineto{\pgfqpoint{0.000000in}{-0.048611in}}%
\pgfusepath{stroke,fill}%
}%
\begin{pgfscope}%
\pgfsys@transformshift{1.271250in}{0.333000in}%
\pgfsys@useobject{currentmarker}{}%
\end{pgfscope}%
\end{pgfscope}%
\begin{pgfscope}%
\definecolor{textcolor}{rgb}{0.000000,0.000000,0.000000}%
\pgfsetstrokecolor{textcolor}%
\pgfsetfillcolor{textcolor}%
\pgftext[x=1.271250in,y=0.235778in,,top]{\color{textcolor}\rmfamily\fontsize{9.000000}{10.800000}\selectfont 5.0}%
\end{pgfscope}%
\begin{pgfscope}%
\pgfsetbuttcap%
\pgfsetroundjoin%
\definecolor{currentfill}{rgb}{0.000000,0.000000,0.000000}%
\pgfsetfillcolor{currentfill}%
\pgfsetlinewidth{0.803000pt}%
\definecolor{currentstroke}{rgb}{0.000000,0.000000,0.000000}%
\pgfsetstrokecolor{currentstroke}%
\pgfsetdash{}{0pt}%
\pgfsys@defobject{currentmarker}{\pgfqpoint{0.000000in}{-0.048611in}}{\pgfqpoint{0.000000in}{0.000000in}}{%
\pgfpathmoveto{\pgfqpoint{0.000000in}{0.000000in}}%
\pgfpathlineto{\pgfqpoint{0.000000in}{-0.048611in}}%
\pgfusepath{stroke,fill}%
}%
\begin{pgfscope}%
\pgfsys@transformshift{1.704375in}{0.333000in}%
\pgfsys@useobject{currentmarker}{}%
\end{pgfscope}%
\end{pgfscope}%
\begin{pgfscope}%
\definecolor{textcolor}{rgb}{0.000000,0.000000,0.000000}%
\pgfsetstrokecolor{textcolor}%
\pgfsetfillcolor{textcolor}%
\pgftext[x=1.704375in,y=0.235778in,,top]{\color{textcolor}\rmfamily\fontsize{9.000000}{10.800000}\selectfont 7.5}%
\end{pgfscope}%
\begin{pgfscope}%
\pgfsetbuttcap%
\pgfsetroundjoin%
\definecolor{currentfill}{rgb}{0.000000,0.000000,0.000000}%
\pgfsetfillcolor{currentfill}%
\pgfsetlinewidth{0.803000pt}%
\definecolor{currentstroke}{rgb}{0.000000,0.000000,0.000000}%
\pgfsetstrokecolor{currentstroke}%
\pgfsetdash{}{0pt}%
\pgfsys@defobject{currentmarker}{\pgfqpoint{0.000000in}{-0.048611in}}{\pgfqpoint{0.000000in}{0.000000in}}{%
\pgfpathmoveto{\pgfqpoint{0.000000in}{0.000000in}}%
\pgfpathlineto{\pgfqpoint{0.000000in}{-0.048611in}}%
\pgfusepath{stroke,fill}%
}%
\begin{pgfscope}%
\pgfsys@transformshift{2.137500in}{0.333000in}%
\pgfsys@useobject{currentmarker}{}%
\end{pgfscope}%
\end{pgfscope}%
\begin{pgfscope}%
\definecolor{textcolor}{rgb}{0.000000,0.000000,0.000000}%
\pgfsetstrokecolor{textcolor}%
\pgfsetfillcolor{textcolor}%
\pgftext[x=2.137500in,y=0.235778in,,top]{\color{textcolor}\rmfamily\fontsize{9.000000}{10.800000}\selectfont 10.0}%
\end{pgfscope}%
\begin{pgfscope}%
\pgfsetbuttcap%
\pgfsetroundjoin%
\definecolor{currentfill}{rgb}{0.000000,0.000000,0.000000}%
\pgfsetfillcolor{currentfill}%
\pgfsetlinewidth{0.803000pt}%
\definecolor{currentstroke}{rgb}{0.000000,0.000000,0.000000}%
\pgfsetstrokecolor{currentstroke}%
\pgfsetdash{}{0pt}%
\pgfsys@defobject{currentmarker}{\pgfqpoint{-0.048611in}{0.000000in}}{\pgfqpoint{0.000000in}{0.000000in}}{%
\pgfpathmoveto{\pgfqpoint{0.000000in}{0.000000in}}%
\pgfpathlineto{\pgfqpoint{-0.048611in}{0.000000in}}%
\pgfusepath{stroke,fill}%
}%
\begin{pgfscope}%
\pgfsys@transformshift{0.405000in}{0.333000in}%
\pgfsys@useobject{currentmarker}{}%
\end{pgfscope}%
\end{pgfscope}%
\begin{pgfscope}%
\definecolor{textcolor}{rgb}{0.000000,0.000000,0.000000}%
\pgfsetstrokecolor{textcolor}%
\pgfsetfillcolor{textcolor}%
\pgftext[x=0.143620in, y=0.289597in, left, base]{\color{textcolor}\rmfamily\fontsize{9.000000}{10.800000}\selectfont 0.0}%
\end{pgfscope}%
\begin{pgfscope}%
\pgfsetbuttcap%
\pgfsetroundjoin%
\definecolor{currentfill}{rgb}{0.000000,0.000000,0.000000}%
\pgfsetfillcolor{currentfill}%
\pgfsetlinewidth{0.803000pt}%
\definecolor{currentstroke}{rgb}{0.000000,0.000000,0.000000}%
\pgfsetstrokecolor{currentstroke}%
\pgfsetdash{}{0pt}%
\pgfsys@defobject{currentmarker}{\pgfqpoint{-0.048611in}{0.000000in}}{\pgfqpoint{0.000000in}{0.000000in}}{%
\pgfpathmoveto{\pgfqpoint{0.000000in}{0.000000in}}%
\pgfpathlineto{\pgfqpoint{-0.048611in}{0.000000in}}%
\pgfusepath{stroke,fill}%
}%
\begin{pgfscope}%
\pgfsys@transformshift{0.405000in}{1.054500in}%
\pgfsys@useobject{currentmarker}{}%
\end{pgfscope}%
\end{pgfscope}%
\begin{pgfscope}%
\definecolor{textcolor}{rgb}{0.000000,0.000000,0.000000}%
\pgfsetstrokecolor{textcolor}%
\pgfsetfillcolor{textcolor}%
\pgftext[x=0.143620in, y=1.011097in, left, base]{\color{textcolor}\rmfamily\fontsize{9.000000}{10.800000}\selectfont 0.1}%
\end{pgfscope}%
\begin{pgfscope}%
\pgfsetbuttcap%
\pgfsetroundjoin%
\definecolor{currentfill}{rgb}{0.000000,0.000000,0.000000}%
\pgfsetfillcolor{currentfill}%
\pgfsetlinewidth{0.803000pt}%
\definecolor{currentstroke}{rgb}{0.000000,0.000000,0.000000}%
\pgfsetstrokecolor{currentstroke}%
\pgfsetdash{}{0pt}%
\pgfsys@defobject{currentmarker}{\pgfqpoint{-0.048611in}{0.000000in}}{\pgfqpoint{0.000000in}{0.000000in}}{%
\pgfpathmoveto{\pgfqpoint{0.000000in}{0.000000in}}%
\pgfpathlineto{\pgfqpoint{-0.048611in}{0.000000in}}%
\pgfusepath{stroke,fill}%
}%
\begin{pgfscope}%
\pgfsys@transformshift{0.405000in}{1.776000in}%
\pgfsys@useobject{currentmarker}{}%
\end{pgfscope}%
\end{pgfscope}%
\begin{pgfscope}%
\definecolor{textcolor}{rgb}{0.000000,0.000000,0.000000}%
\pgfsetstrokecolor{textcolor}%
\pgfsetfillcolor{textcolor}%
\pgftext[x=0.143620in, y=1.732597in, left, base]{\color{textcolor}\rmfamily\fontsize{9.000000}{10.800000}\selectfont 0.2}%
\end{pgfscope}%
\begin{pgfscope}%
\definecolor{textcolor}{rgb}{0.000000,0.000000,0.000000}%
\pgfsetstrokecolor{textcolor}%
\pgfsetfillcolor{textcolor}%
\pgftext[x=0.088064in,y=1.054500in,,bottom,rotate=90.000000]{\color{textcolor}\rmfamily\fontsize{9.000000}{10.800000}\selectfont \(\displaystyle \wp(x)\)}%
\end{pgfscope}%
\begin{pgfscope}%
\pgfpathrectangle{\pgfqpoint{0.405000in}{0.333000in}}{\pgfqpoint{1.732500in}{1.443000in}}%
\pgfusepath{clip}%
\pgfsetrectcap%
\pgfsetroundjoin%
\pgfsetlinewidth{1.254687pt}%
\definecolor{currentstroke}{rgb}{0.000000,0.000000,0.000000}%
\pgfsetstrokecolor{currentstroke}%
\pgfsetdash{}{0pt}%
\pgfpathmoveto{\pgfqpoint{0.405000in}{0.333000in}}%
\pgfpathlineto{\pgfqpoint{0.422500in}{0.362285in}}%
\pgfpathlineto{\pgfqpoint{0.440000in}{0.411750in}}%
\pgfpathlineto{\pgfqpoint{0.457500in}{0.470547in}}%
\pgfpathlineto{\pgfqpoint{0.475000in}{0.534338in}}%
\pgfpathlineto{\pgfqpoint{0.492500in}{0.600520in}}%
\pgfpathlineto{\pgfqpoint{0.510000in}{0.667345in}}%
\pgfpathlineto{\pgfqpoint{0.527500in}{0.733572in}}%
\pgfpathlineto{\pgfqpoint{0.545000in}{0.798302in}}%
\pgfpathlineto{\pgfqpoint{0.562500in}{0.860873in}}%
\pgfpathlineto{\pgfqpoint{0.580000in}{0.920803in}}%
\pgfpathlineto{\pgfqpoint{0.597500in}{0.977743in}}%
\pgfpathlineto{\pgfqpoint{0.615000in}{1.031450in}}%
\pgfpathlineto{\pgfqpoint{0.632500in}{1.081763in}}%
\pgfpathlineto{\pgfqpoint{0.650000in}{1.128587in}}%
\pgfpathlineto{\pgfqpoint{0.667500in}{1.171877in}}%
\pgfpathlineto{\pgfqpoint{0.685000in}{1.211633in}}%
\pgfpathlineto{\pgfqpoint{0.702500in}{1.247885in}}%
\pgfpathlineto{\pgfqpoint{0.720000in}{1.280694in}}%
\pgfpathlineto{\pgfqpoint{0.737500in}{1.310137in}}%
\pgfpathlineto{\pgfqpoint{0.755000in}{1.336312in}}%
\pgfpathlineto{\pgfqpoint{0.772500in}{1.359327in}}%
\pgfpathlineto{\pgfqpoint{0.790000in}{1.379302in}}%
\pgfpathlineto{\pgfqpoint{0.807500in}{1.396361in}}%
\pgfpathlineto{\pgfqpoint{0.825000in}{1.410635in}}%
\pgfpathlineto{\pgfqpoint{0.842500in}{1.422258in}}%
\pgfpathlineto{\pgfqpoint{0.860000in}{1.431365in}}%
\pgfpathlineto{\pgfqpoint{0.877500in}{1.438092in}}%
\pgfpathlineto{\pgfqpoint{0.895000in}{1.442572in}}%
\pgfpathlineto{\pgfqpoint{0.912500in}{1.444940in}}%
\pgfpathlineto{\pgfqpoint{0.930000in}{1.445327in}}%
\pgfpathlineto{\pgfqpoint{0.947500in}{1.443859in}}%
\pgfpathlineto{\pgfqpoint{0.965000in}{1.440662in}}%
\pgfpathlineto{\pgfqpoint{0.982500in}{1.435857in}}%
\pgfpathlineto{\pgfqpoint{1.000000in}{1.429560in}}%
\pgfpathlineto{\pgfqpoint{1.017500in}{1.421885in}}%
\pgfpathlineto{\pgfqpoint{1.035000in}{1.412940in}}%
\pgfpathlineto{\pgfqpoint{1.052500in}{1.402829in}}%
\pgfpathlineto{\pgfqpoint{1.070000in}{1.391652in}}%
\pgfpathlineto{\pgfqpoint{1.087500in}{1.379504in}}%
\pgfpathlineto{\pgfqpoint{1.105000in}{1.366474in}}%
\pgfpathlineto{\pgfqpoint{1.122500in}{1.352651in}}%
\pgfpathlineto{\pgfqpoint{1.140000in}{1.338115in}}%
\pgfpathlineto{\pgfqpoint{1.157500in}{1.322943in}}%
\pgfpathlineto{\pgfqpoint{1.175000in}{1.307210in}}%
\pgfpathlineto{\pgfqpoint{1.192500in}{1.290984in}}%
\pgfpathlineto{\pgfqpoint{1.210000in}{1.274330in}}%
\pgfpathlineto{\pgfqpoint{1.227500in}{1.257311in}}%
\pgfpathlineto{\pgfqpoint{1.245000in}{1.239983in}}%
\pgfpathlineto{\pgfqpoint{1.262500in}{1.222400in}}%
\pgfpathlineto{\pgfqpoint{1.280000in}{1.204614in}}%
\pgfpathlineto{\pgfqpoint{1.297500in}{1.186671in}}%
\pgfpathlineto{\pgfqpoint{1.315000in}{1.168615in}}%
\pgfpathlineto{\pgfqpoint{1.332500in}{1.150487in}}%
\pgfpathlineto{\pgfqpoint{1.350000in}{1.132325in}}%
\pgfpathlineto{\pgfqpoint{1.367500in}{1.114165in}}%
\pgfpathlineto{\pgfqpoint{1.385000in}{1.096039in}}%
\pgfpathlineto{\pgfqpoint{1.402500in}{1.077977in}}%
\pgfpathlineto{\pgfqpoint{1.420000in}{1.060007in}}%
\pgfpathlineto{\pgfqpoint{1.437500in}{1.042154in}}%
\pgfpathlineto{\pgfqpoint{1.455000in}{1.024441in}}%
\pgfpathlineto{\pgfqpoint{1.472500in}{1.006890in}}%
\pgfpathlineto{\pgfqpoint{1.490000in}{0.989520in}}%
\pgfpathlineto{\pgfqpoint{1.507500in}{0.972348in}}%
\pgfpathlineto{\pgfqpoint{1.525000in}{0.955389in}}%
\pgfpathlineto{\pgfqpoint{1.542500in}{0.938659in}}%
\pgfpathlineto{\pgfqpoint{1.560000in}{0.922169in}}%
\pgfpathlineto{\pgfqpoint{1.577500in}{0.905931in}}%
\pgfpathlineto{\pgfqpoint{1.595000in}{0.889954in}}%
\pgfpathlineto{\pgfqpoint{1.612500in}{0.874247in}}%
\pgfpathlineto{\pgfqpoint{1.630000in}{0.858817in}}%
\pgfpathlineto{\pgfqpoint{1.647500in}{0.843671in}}%
\pgfpathlineto{\pgfqpoint{1.665000in}{0.828814in}}%
\pgfpathlineto{\pgfqpoint{1.682500in}{0.814249in}}%
\pgfpathlineto{\pgfqpoint{1.700000in}{0.799981in}}%
\pgfpathlineto{\pgfqpoint{1.717500in}{0.786012in}}%
\pgfpathlineto{\pgfqpoint{1.735000in}{0.772343in}}%
\pgfpathlineto{\pgfqpoint{1.752500in}{0.758976in}}%
\pgfpathlineto{\pgfqpoint{1.770000in}{0.745912in}}%
\pgfpathlineto{\pgfqpoint{1.787500in}{0.733149in}}%
\pgfpathlineto{\pgfqpoint{1.805000in}{0.720688in}}%
\pgfpathlineto{\pgfqpoint{1.822500in}{0.708527in}}%
\pgfpathlineto{\pgfqpoint{1.840000in}{0.696664in}}%
\pgfpathlineto{\pgfqpoint{1.857500in}{0.685097in}}%
\pgfpathlineto{\pgfqpoint{1.875000in}{0.673824in}}%
\pgfpathlineto{\pgfqpoint{1.892500in}{0.662841in}}%
\pgfpathlineto{\pgfqpoint{1.910000in}{0.652147in}}%
\pgfpathlineto{\pgfqpoint{1.927500in}{0.641736in}}%
\pgfpathlineto{\pgfqpoint{1.945000in}{0.631606in}}%
\pgfpathlineto{\pgfqpoint{1.962500in}{0.621752in}}%
\pgfpathlineto{\pgfqpoint{1.980000in}{0.612171in}}%
\pgfpathlineto{\pgfqpoint{1.997500in}{0.602858in}}%
\pgfpathlineto{\pgfqpoint{2.015000in}{0.593808in}}%
\pgfpathlineto{\pgfqpoint{2.032500in}{0.585016in}}%
\pgfpathlineto{\pgfqpoint{2.050000in}{0.576479in}}%
\pgfpathlineto{\pgfqpoint{2.067500in}{0.568191in}}%
\pgfpathlineto{\pgfqpoint{2.085000in}{0.560148in}}%
\pgfpathlineto{\pgfqpoint{2.102500in}{0.552344in}}%
\pgfpathlineto{\pgfqpoint{2.120000in}{0.544774in}}%
\pgfpathlineto{\pgfqpoint{2.137500in}{0.537434in}}%
\pgfusepath{stroke}%
\end{pgfscope}%
\begin{pgfscope}%
\pgfsetrectcap%
\pgfsetmiterjoin%
\pgfsetlinewidth{0.803000pt}%
\definecolor{currentstroke}{rgb}{0.000000,0.000000,0.000000}%
\pgfsetstrokecolor{currentstroke}%
\pgfsetdash{}{0pt}%
\pgfpathmoveto{\pgfqpoint{0.405000in}{0.333000in}}%
\pgfpathlineto{\pgfqpoint{0.405000in}{1.776000in}}%
\pgfusepath{stroke}%
\end{pgfscope}%
\begin{pgfscope}%
\pgfsetrectcap%
\pgfsetmiterjoin%
\pgfsetlinewidth{0.803000pt}%
\definecolor{currentstroke}{rgb}{0.000000,0.000000,0.000000}%
\pgfsetstrokecolor{currentstroke}%
\pgfsetdash{}{0pt}%
\pgfpathmoveto{\pgfqpoint{2.137500in}{0.333000in}}%
\pgfpathlineto{\pgfqpoint{2.137500in}{1.776000in}}%
\pgfusepath{stroke}%
\end{pgfscope}%
\begin{pgfscope}%
\pgfsetrectcap%
\pgfsetmiterjoin%
\pgfsetlinewidth{0.803000pt}%
\definecolor{currentstroke}{rgb}{0.000000,0.000000,0.000000}%
\pgfsetstrokecolor{currentstroke}%
\pgfsetdash{}{0pt}%
\pgfpathmoveto{\pgfqpoint{0.405000in}{0.333000in}}%
\pgfpathlineto{\pgfqpoint{2.137500in}{0.333000in}}%
\pgfusepath{stroke}%
\end{pgfscope}%
\begin{pgfscope}%
\pgfsetrectcap%
\pgfsetmiterjoin%
\pgfsetlinewidth{0.803000pt}%
\definecolor{currentstroke}{rgb}{0.000000,0.000000,0.000000}%
\pgfsetstrokecolor{currentstroke}%
\pgfsetdash{}{0pt}%
\pgfpathmoveto{\pgfqpoint{0.405000in}{1.776000in}}%
\pgfpathlineto{\pgfqpoint{2.137500in}{1.776000in}}%
\pgfusepath{stroke}%
\end{pgfscope}%
\end{pgfpicture}%
\makeatother%
\endgroup%
}{
  Valori tabulati del valore $\chisq_p$ per cui la funzione cumulativa (cioè
  l'area ombreggiata in figura) per una variabile $\chisq$ a $\nu$ gradi di
  libertà vale esattamente p.

  Per un numero di gradi di libertà $\nu > 40$ si può utilizzare un semplice
  programma al calcolatore oppure si può sfruttare l'approssimazione
  Gaussiana.
}

\vspace*{\fill}

\begin{table}[!hb]
  \begin{center}
    {\small
      \begin{tabular*}{\textwidth}{@{ \extracolsep{\fill}}cccccccccccccc}
        \hline
        $\nu / p$ & $0.005$ & $0.01$ & $0.05$ & $0.1$ & $0.25$ & $0.5$ & $0.75$ & $0.9$ & $0.95$ & $0.99$ & $0.995$ & $0.999$ & $0.9999$ \\
\hline
\hline
$1$ & $4$e-$5$ & $2$e-$4$ & $4$e-$3$ & $2$e-$2$ & $0.10$ & $0.45$ & $1.32$ & $2.71$ & $3.84$ & $6.63$ & $7.88$ & $10.8$ & $15.1$ \\
$2$ & $1$e-$2$ & $2$e-$2$ & $0.10$ & $0.21$ & $0.58$ & $1.39$ & $2.77$ & $4.61$ & $5.99$ & $9.21$ & $10.6$ & $13.8$ & $18.4$ \\
$3$ & $7$e-$2$ & $0.11$ & $0.35$ & $0.58$ & $1.21$ & $2.37$ & $4.11$ & $6.25$ & $7.81$ & $11.3$ & $12.8$ & $16.3$ & $21.1$ \\
$4$ & $0.21$ & $0.30$ & $0.71$ & $1.06$ & $1.92$ & $3.36$ & $5.39$ & $7.78$ & $9.49$ & $13.3$ & $14.9$ & $18.5$ & $23.5$ \\
$5$ & $0.41$ & $0.55$ & $1.15$ & $1.61$ & $2.67$ & $4.35$ & $6.63$ & $9.24$ & $11.1$ & $15.1$ & $16.7$ & $20.5$ & $25.7$ \\
$6$ & $0.68$ & $0.87$ & $1.64$ & $2.20$ & $3.45$ & $5.35$ & $7.84$ & $10.6$ & $12.6$ & $16.8$ & $18.5$ & $22.5$ & $27.9$ \\
$7$ & $0.99$ & $1.24$ & $2.17$ & $2.83$ & $4.25$ & $6.35$ & $9.04$ & $12.0$ & $14.1$ & $18.5$ & $20.3$ & $24.3$ & $29.9$ \\
$8$ & $1.34$ & $1.65$ & $2.73$ & $3.49$ & $5.07$ & $7.34$ & $10.2$ & $13.4$ & $15.5$ & $20.1$ & $22.0$ & $26.1$ & $31.8$ \\
$9$ & $1.73$ & $2.09$ & $3.33$ & $4.17$ & $5.90$ & $8.34$ & $11.4$ & $14.7$ & $16.9$ & $21.7$ & $23.6$ & $27.9$ & $33.7$ \\
$10$ & $2.16$ & $2.56$ & $3.94$ & $4.87$ & $6.74$ & $9.34$ & $12.5$ & $16.0$ & $18.3$ & $23.2$ & $25.2$ & $29.6$ & $35.6$ \\
$11$ & $2.60$ & $3.05$ & $4.57$ & $5.58$ & $7.58$ & $10.3$ & $13.7$ & $17.3$ & $19.7$ & $24.7$ & $26.8$ & $31.3$ & $37.4$ \\
$12$ & $3.07$ & $3.57$ & $5.23$ & $6.30$ & $8.44$ & $11.3$ & $14.8$ & $18.5$ & $21.0$ & $26.2$ & $28.3$ & $32.9$ & $39.1$ \\
$13$ & $3.57$ & $4.11$ & $5.89$ & $7.04$ & $9.30$ & $12.3$ & $16.0$ & $19.8$ & $22.4$ & $27.7$ & $29.8$ & $34.5$ & $40.9$ \\
$14$ & $4.07$ & $4.66$ & $6.57$ & $7.79$ & $10.2$ & $13.3$ & $17.1$ & $21.1$ & $23.7$ & $29.1$ & $31.3$ & $36.1$ & $42.6$ \\
$15$ & $4.60$ & $5.23$ & $7.26$ & $8.55$ & $11.0$ & $14.3$ & $18.2$ & $22.3$ & $25.0$ & $30.6$ & $32.8$ & $37.7$ & $44.3$ \\
$16$ & $5.14$ & $5.81$ & $7.96$ & $9.31$ & $11.9$ & $15.3$ & $19.4$ & $23.5$ & $26.3$ & $32.0$ & $34.3$ & $39.3$ & $45.9$ \\
$17$ & $5.70$ & $6.41$ & $8.67$ & $10.1$ & $12.8$ & $16.3$ & $20.5$ & $24.8$ & $27.6$ & $33.4$ & $35.7$ & $40.8$ & $47.6$ \\
$18$ & $6.26$ & $7.01$ & $9.39$ & $10.9$ & $13.7$ & $17.3$ & $21.6$ & $26.0$ & $28.9$ & $34.8$ & $37.2$ & $42.3$ & $49.2$ \\
$19$ & $6.84$ & $7.63$ & $10.1$ & $11.7$ & $14.6$ & $18.3$ & $22.7$ & $27.2$ & $30.1$ & $36.2$ & $38.6$ & $43.8$ & $50.8$ \\
$20$ & $7.43$ & $8.26$ & $10.9$ & $12.4$ & $15.5$ & $19.3$ & $23.8$ & $28.4$ & $31.4$ & $37.6$ & $40.0$ & $45.3$ & $52.4$ \\
$21$ & $8.03$ & $8.90$ & $11.6$ & $13.2$ & $16.3$ & $20.3$ & $24.9$ & $29.6$ & $32.7$ & $38.9$ & $41.4$ & $46.8$ & $54.0$ \\
$22$ & $8.64$ & $9.54$ & $12.3$ & $14.0$ & $17.2$ & $21.3$ & $26.0$ & $30.8$ & $33.9$ & $40.3$ & $42.8$ & $48.3$ & $55.5$ \\
$23$ & $9.26$ & $10.2$ & $13.1$ & $14.8$ & $18.1$ & $22.3$ & $27.1$ & $32.0$ & $35.2$ & $41.6$ & $44.2$ & $49.7$ & $57.1$ \\
$24$ & $9.89$ & $10.9$ & $13.8$ & $15.7$ & $19.0$ & $23.3$ & $28.2$ & $33.2$ & $36.4$ & $43.0$ & $45.6$ & $51.2$ & $58.6$ \\
$25$ & $10.5$ & $11.5$ & $14.6$ & $16.5$ & $19.9$ & $24.3$ & $29.3$ & $34.4$ & $37.7$ & $44.3$ & $46.9$ & $52.6$ & $60.1$ \\
$26$ & $11.2$ & $12.2$ & $15.4$ & $17.3$ & $20.8$ & $25.3$ & $30.4$ & $35.6$ & $38.9$ & $45.6$ & $48.3$ & $54.1$ & $61.7$ \\
$27$ & $11.8$ & $12.9$ & $16.2$ & $18.1$ & $21.7$ & $26.3$ & $31.5$ & $36.7$ & $40.1$ & $47.0$ & $49.6$ & $55.5$ & $63.2$ \\
$28$ & $12.5$ & $13.6$ & $16.9$ & $18.9$ & $22.7$ & $27.3$ & $32.6$ & $37.9$ & $41.3$ & $48.3$ & $51.0$ & $56.9$ & $64.7$ \\
$29$ & $13.1$ & $14.3$ & $17.7$ & $19.8$ & $23.6$ & $28.3$ & $33.7$ & $39.1$ & $42.6$ & $49.6$ & $52.3$ & $58.3$ & $66.2$ \\
$30$ & $13.8$ & $15.0$ & $18.5$ & $20.6$ & $24.5$ & $29.3$ & $34.8$ & $40.3$ & $43.8$ & $50.9$ & $53.7$ & $59.7$ & $67.6$ \\
$31$ & $14.5$ & $15.7$ & $19.3$ & $21.4$ & $25.4$ & $30.3$ & $35.9$ & $41.4$ & $45.0$ & $52.2$ & $55.0$ & $61.1$ & $69.1$ \\
$32$ & $15.1$ & $16.4$ & $20.1$ & $22.3$ & $26.3$ & $31.3$ & $37.0$ & $42.6$ & $46.2$ & $53.5$ & $56.3$ & $62.5$ & $70.6$ \\
$33$ & $15.8$ & $17.1$ & $20.9$ & $23.1$ & $27.2$ & $32.3$ & $38.1$ & $43.7$ & $47.4$ & $54.8$ & $57.6$ & $63.9$ & $72.0$ \\
$34$ & $16.5$ & $17.8$ & $21.7$ & $24.0$ & $28.1$ & $33.3$ & $39.1$ & $44.9$ & $48.6$ & $56.1$ & $59.0$ & $65.2$ & $73.5$ \\
$35$ & $17.2$ & $18.5$ & $22.5$ & $24.8$ & $29.1$ & $34.3$ & $40.2$ & $46.1$ & $49.8$ & $57.3$ & $60.3$ & $66.6$ & $74.9$ \\
$36$ & $17.9$ & $19.2$ & $23.3$ & $25.6$ & $30.0$ & $35.3$ & $41.3$ & $47.2$ & $51.0$ & $58.6$ & $61.6$ & $68.0$ & $76.4$ \\
$37$ & $18.6$ & $20.0$ & $24.1$ & $26.5$ & $30.9$ & $36.3$ & $42.4$ & $48.4$ & $52.2$ & $59.9$ & $62.9$ & $69.3$ & $77.8$ \\
$38$ & $19.3$ & $20.7$ & $24.9$ & $27.3$ & $31.8$ & $37.3$ & $43.5$ & $49.5$ & $53.4$ & $61.2$ & $64.2$ & $70.7$ & $79.2$ \\
$39$ & $20.0$ & $21.4$ & $25.7$ & $28.2$ & $32.7$ & $38.3$ & $44.5$ & $50.7$ & $54.6$ & $62.4$ & $65.5$ & $72.1$ & $80.6$ \\
$40$ & $20.7$ & $22.2$ & $26.5$ & $29.1$ & $33.7$ & $39.3$ & $45.6$ & $51.8$ & $55.8$ & $63.7$ & $66.8$ & $73.4$ & $82.1$\\
        \hline
      \end{tabular*}
    }
  \end{center}
\end{table}


\begin{figure}
  %% Creator: Matplotlib, PGF backend
%%
%% To include the figure in your LaTeX document, write
%%   \input{<filename>.pgf}
%%
%% Make sure the required packages are loaded in your preamble
%%   \usepackage{pgf}
%%
%% Also ensure that all the required font packages are loaded; for instance,
%% the lmodern package is sometimes necessary when using math font.
%%   \usepackage{lmodern}
%%
%% Figures using additional raster images can only be included by \input if
%% they are in the same directory as the main LaTeX file. For loading figures
%% from other directories you can use the `import` package
%%   \usepackage{import}
%%
%% and then include the figures with
%%   \import{<path to file>}{<filename>.pgf}
%%
%% Matplotlib used the following preamble
%%   \usepackage[nice]{nicefrac}
%%   \usepackage{amsmath}
%%   \usepackage[utf8]{inputenc}
%%   \DeclareUnicodeCharacter{2212}{\ensuremath{-}}
%%
\begingroup%
\makeatletter%
\begin{pgfpicture}%
\pgfpathrectangle{\pgfpointorigin}{\pgfqpoint{6.380000in}{9.000000in}}%
\pgfusepath{use as bounding box, clip}%
\begin{pgfscope}%
\pgfsetbuttcap%
\pgfsetmiterjoin%
\definecolor{currentfill}{rgb}{1.000000,1.000000,1.000000}%
\pgfsetfillcolor{currentfill}%
\pgfsetlinewidth{0.000000pt}%
\definecolor{currentstroke}{rgb}{1.000000,1.000000,1.000000}%
\pgfsetstrokecolor{currentstroke}%
\pgfsetdash{}{0pt}%
\pgfpathmoveto{\pgfqpoint{0.000000in}{0.000000in}}%
\pgfpathlineto{\pgfqpoint{6.380000in}{0.000000in}}%
\pgfpathlineto{\pgfqpoint{6.380000in}{9.000000in}}%
\pgfpathlineto{\pgfqpoint{0.000000in}{9.000000in}}%
\pgfpathlineto{\pgfqpoint{0.000000in}{0.000000in}}%
\pgfpathclose%
\pgfusepath{fill}%
\end{pgfscope}%
\begin{pgfscope}%
\pgfsetbuttcap%
\pgfsetmiterjoin%
\definecolor{currentfill}{rgb}{1.000000,1.000000,1.000000}%
\pgfsetfillcolor{currentfill}%
\pgfsetlinewidth{0.000000pt}%
\definecolor{currentstroke}{rgb}{0.000000,0.000000,0.000000}%
\pgfsetstrokecolor{currentstroke}%
\pgfsetstrokeopacity{0.000000}%
\pgfsetdash{}{0pt}%
\pgfpathmoveto{\pgfqpoint{0.581943in}{0.509170in}}%
\pgfpathlineto{\pgfqpoint{6.181004in}{0.509170in}}%
\pgfpathlineto{\pgfqpoint{6.181004in}{8.821759in}}%
\pgfpathlineto{\pgfqpoint{0.581943in}{8.821759in}}%
\pgfpathlineto{\pgfqpoint{0.581943in}{0.509170in}}%
\pgfpathclose%
\pgfusepath{fill}%
\end{pgfscope}%
\begin{pgfscope}%
\pgfpathrectangle{\pgfqpoint{0.581943in}{0.509170in}}{\pgfqpoint{5.599061in}{8.312590in}}%
\pgfusepath{clip}%
\pgfsetbuttcap%
\pgfsetroundjoin%
\pgfsetlinewidth{0.250937pt}%
\definecolor{currentstroke}{rgb}{0.680000,0.680000,0.680000}%
\pgfsetstrokecolor{currentstroke}%
\pgfsetdash{{1.000000pt}{1.000000pt}}{0.000000pt}%
\pgfpathmoveto{\pgfqpoint{0.581943in}{0.509170in}}%
\pgfpathlineto{\pgfqpoint{0.581943in}{8.821759in}}%
\pgfusepath{stroke}%
\end{pgfscope}%
\begin{pgfscope}%
\pgfsetbuttcap%
\pgfsetroundjoin%
\definecolor{currentfill}{rgb}{0.000000,0.000000,0.000000}%
\pgfsetfillcolor{currentfill}%
\pgfsetlinewidth{0.803000pt}%
\definecolor{currentstroke}{rgb}{0.000000,0.000000,0.000000}%
\pgfsetstrokecolor{currentstroke}%
\pgfsetdash{}{0pt}%
\pgfsys@defobject{currentmarker}{\pgfqpoint{0.000000in}{-0.048611in}}{\pgfqpoint{0.000000in}{0.000000in}}{%
\pgfpathmoveto{\pgfqpoint{0.000000in}{0.000000in}}%
\pgfpathlineto{\pgfqpoint{0.000000in}{-0.048611in}}%
\pgfusepath{stroke,fill}%
}%
\begin{pgfscope}%
\pgfsys@transformshift{0.581943in}{0.509170in}%
\pgfsys@useobject{currentmarker}{}%
\end{pgfscope}%
\end{pgfscope}%
\begin{pgfscope}%
\definecolor{textcolor}{rgb}{0.000000,0.000000,0.000000}%
\pgfsetstrokecolor{textcolor}%
\pgfsetfillcolor{textcolor}%
\pgftext[x=0.581943in,y=0.411948in,,top]{\color{textcolor}\rmfamily\fontsize{9.000000}{10.800000}\selectfont 1}%
\end{pgfscope}%
\begin{pgfscope}%
\pgfpathrectangle{\pgfqpoint{0.581943in}{0.509170in}}{\pgfqpoint{5.599061in}{8.312590in}}%
\pgfusepath{clip}%
\pgfsetbuttcap%
\pgfsetroundjoin%
\pgfsetlinewidth{0.250937pt}%
\definecolor{currentstroke}{rgb}{0.680000,0.680000,0.680000}%
\pgfsetstrokecolor{currentstroke}%
\pgfsetdash{{1.000000pt}{1.000000pt}}{0.000000pt}%
\pgfpathmoveto{\pgfqpoint{1.574006in}{0.509170in}}%
\pgfpathlineto{\pgfqpoint{1.574006in}{8.821759in}}%
\pgfusepath{stroke}%
\end{pgfscope}%
\begin{pgfscope}%
\pgfsetbuttcap%
\pgfsetroundjoin%
\definecolor{currentfill}{rgb}{0.000000,0.000000,0.000000}%
\pgfsetfillcolor{currentfill}%
\pgfsetlinewidth{0.803000pt}%
\definecolor{currentstroke}{rgb}{0.000000,0.000000,0.000000}%
\pgfsetstrokecolor{currentstroke}%
\pgfsetdash{}{0pt}%
\pgfsys@defobject{currentmarker}{\pgfqpoint{0.000000in}{-0.048611in}}{\pgfqpoint{0.000000in}{0.000000in}}{%
\pgfpathmoveto{\pgfqpoint{0.000000in}{0.000000in}}%
\pgfpathlineto{\pgfqpoint{0.000000in}{-0.048611in}}%
\pgfusepath{stroke,fill}%
}%
\begin{pgfscope}%
\pgfsys@transformshift{1.574006in}{0.509170in}%
\pgfsys@useobject{currentmarker}{}%
\end{pgfscope}%
\end{pgfscope}%
\begin{pgfscope}%
\definecolor{textcolor}{rgb}{0.000000,0.000000,0.000000}%
\pgfsetstrokecolor{textcolor}%
\pgfsetfillcolor{textcolor}%
\pgftext[x=1.574006in,y=0.411948in,,top]{\color{textcolor}\rmfamily\fontsize{9.000000}{10.800000}\selectfont 2}%
\end{pgfscope}%
\begin{pgfscope}%
\pgfpathrectangle{\pgfqpoint{0.581943in}{0.509170in}}{\pgfqpoint{5.599061in}{8.312590in}}%
\pgfusepath{clip}%
\pgfsetbuttcap%
\pgfsetroundjoin%
\pgfsetlinewidth{0.250937pt}%
\definecolor{currentstroke}{rgb}{0.680000,0.680000,0.680000}%
\pgfsetstrokecolor{currentstroke}%
\pgfsetdash{{1.000000pt}{1.000000pt}}{0.000000pt}%
\pgfpathmoveto{\pgfqpoint{2.154326in}{0.509170in}}%
\pgfpathlineto{\pgfqpoint{2.154326in}{8.821759in}}%
\pgfusepath{stroke}%
\end{pgfscope}%
\begin{pgfscope}%
\pgfsetbuttcap%
\pgfsetroundjoin%
\definecolor{currentfill}{rgb}{0.000000,0.000000,0.000000}%
\pgfsetfillcolor{currentfill}%
\pgfsetlinewidth{0.803000pt}%
\definecolor{currentstroke}{rgb}{0.000000,0.000000,0.000000}%
\pgfsetstrokecolor{currentstroke}%
\pgfsetdash{}{0pt}%
\pgfsys@defobject{currentmarker}{\pgfqpoint{0.000000in}{-0.048611in}}{\pgfqpoint{0.000000in}{0.000000in}}{%
\pgfpathmoveto{\pgfqpoint{0.000000in}{0.000000in}}%
\pgfpathlineto{\pgfqpoint{0.000000in}{-0.048611in}}%
\pgfusepath{stroke,fill}%
}%
\begin{pgfscope}%
\pgfsys@transformshift{2.154326in}{0.509170in}%
\pgfsys@useobject{currentmarker}{}%
\end{pgfscope}%
\end{pgfscope}%
\begin{pgfscope}%
\definecolor{textcolor}{rgb}{0.000000,0.000000,0.000000}%
\pgfsetstrokecolor{textcolor}%
\pgfsetfillcolor{textcolor}%
\pgftext[x=2.154326in,y=0.411948in,,top]{\color{textcolor}\rmfamily\fontsize{9.000000}{10.800000}\selectfont 3}%
\end{pgfscope}%
\begin{pgfscope}%
\pgfpathrectangle{\pgfqpoint{0.581943in}{0.509170in}}{\pgfqpoint{5.599061in}{8.312590in}}%
\pgfusepath{clip}%
\pgfsetbuttcap%
\pgfsetroundjoin%
\pgfsetlinewidth{0.250937pt}%
\definecolor{currentstroke}{rgb}{0.680000,0.680000,0.680000}%
\pgfsetstrokecolor{currentstroke}%
\pgfsetdash{{1.000000pt}{1.000000pt}}{0.000000pt}%
\pgfpathmoveto{\pgfqpoint{2.566069in}{0.509170in}}%
\pgfpathlineto{\pgfqpoint{2.566069in}{8.821759in}}%
\pgfusepath{stroke}%
\end{pgfscope}%
\begin{pgfscope}%
\pgfsetbuttcap%
\pgfsetroundjoin%
\definecolor{currentfill}{rgb}{0.000000,0.000000,0.000000}%
\pgfsetfillcolor{currentfill}%
\pgfsetlinewidth{0.803000pt}%
\definecolor{currentstroke}{rgb}{0.000000,0.000000,0.000000}%
\pgfsetstrokecolor{currentstroke}%
\pgfsetdash{}{0pt}%
\pgfsys@defobject{currentmarker}{\pgfqpoint{0.000000in}{-0.048611in}}{\pgfqpoint{0.000000in}{0.000000in}}{%
\pgfpathmoveto{\pgfqpoint{0.000000in}{0.000000in}}%
\pgfpathlineto{\pgfqpoint{0.000000in}{-0.048611in}}%
\pgfusepath{stroke,fill}%
}%
\begin{pgfscope}%
\pgfsys@transformshift{2.566069in}{0.509170in}%
\pgfsys@useobject{currentmarker}{}%
\end{pgfscope}%
\end{pgfscope}%
\begin{pgfscope}%
\definecolor{textcolor}{rgb}{0.000000,0.000000,0.000000}%
\pgfsetstrokecolor{textcolor}%
\pgfsetfillcolor{textcolor}%
\pgftext[x=2.566069in,y=0.411948in,,top]{\color{textcolor}\rmfamily\fontsize{9.000000}{10.800000}\selectfont 4}%
\end{pgfscope}%
\begin{pgfscope}%
\pgfpathrectangle{\pgfqpoint{0.581943in}{0.509170in}}{\pgfqpoint{5.599061in}{8.312590in}}%
\pgfusepath{clip}%
\pgfsetbuttcap%
\pgfsetroundjoin%
\pgfsetlinewidth{0.250937pt}%
\definecolor{currentstroke}{rgb}{0.680000,0.680000,0.680000}%
\pgfsetstrokecolor{currentstroke}%
\pgfsetdash{{1.000000pt}{1.000000pt}}{0.000000pt}%
\pgfpathmoveto{\pgfqpoint{2.885442in}{0.509170in}}%
\pgfpathlineto{\pgfqpoint{2.885442in}{8.821759in}}%
\pgfusepath{stroke}%
\end{pgfscope}%
\begin{pgfscope}%
\pgfsetbuttcap%
\pgfsetroundjoin%
\definecolor{currentfill}{rgb}{0.000000,0.000000,0.000000}%
\pgfsetfillcolor{currentfill}%
\pgfsetlinewidth{0.803000pt}%
\definecolor{currentstroke}{rgb}{0.000000,0.000000,0.000000}%
\pgfsetstrokecolor{currentstroke}%
\pgfsetdash{}{0pt}%
\pgfsys@defobject{currentmarker}{\pgfqpoint{0.000000in}{-0.048611in}}{\pgfqpoint{0.000000in}{0.000000in}}{%
\pgfpathmoveto{\pgfqpoint{0.000000in}{0.000000in}}%
\pgfpathlineto{\pgfqpoint{0.000000in}{-0.048611in}}%
\pgfusepath{stroke,fill}%
}%
\begin{pgfscope}%
\pgfsys@transformshift{2.885442in}{0.509170in}%
\pgfsys@useobject{currentmarker}{}%
\end{pgfscope}%
\end{pgfscope}%
\begin{pgfscope}%
\definecolor{textcolor}{rgb}{0.000000,0.000000,0.000000}%
\pgfsetstrokecolor{textcolor}%
\pgfsetfillcolor{textcolor}%
\pgftext[x=2.885442in,y=0.411948in,,top]{\color{textcolor}\rmfamily\fontsize{9.000000}{10.800000}\selectfont 5}%
\end{pgfscope}%
\begin{pgfscope}%
\pgfpathrectangle{\pgfqpoint{0.581943in}{0.509170in}}{\pgfqpoint{5.599061in}{8.312590in}}%
\pgfusepath{clip}%
\pgfsetbuttcap%
\pgfsetroundjoin%
\pgfsetlinewidth{0.250937pt}%
\definecolor{currentstroke}{rgb}{0.680000,0.680000,0.680000}%
\pgfsetstrokecolor{currentstroke}%
\pgfsetdash{{1.000000pt}{1.000000pt}}{0.000000pt}%
\pgfpathmoveto{\pgfqpoint{3.146389in}{0.509170in}}%
\pgfpathlineto{\pgfqpoint{3.146389in}{8.821759in}}%
\pgfusepath{stroke}%
\end{pgfscope}%
\begin{pgfscope}%
\pgfsetbuttcap%
\pgfsetroundjoin%
\definecolor{currentfill}{rgb}{0.000000,0.000000,0.000000}%
\pgfsetfillcolor{currentfill}%
\pgfsetlinewidth{0.803000pt}%
\definecolor{currentstroke}{rgb}{0.000000,0.000000,0.000000}%
\pgfsetstrokecolor{currentstroke}%
\pgfsetdash{}{0pt}%
\pgfsys@defobject{currentmarker}{\pgfqpoint{0.000000in}{-0.048611in}}{\pgfqpoint{0.000000in}{0.000000in}}{%
\pgfpathmoveto{\pgfqpoint{0.000000in}{0.000000in}}%
\pgfpathlineto{\pgfqpoint{0.000000in}{-0.048611in}}%
\pgfusepath{stroke,fill}%
}%
\begin{pgfscope}%
\pgfsys@transformshift{3.146389in}{0.509170in}%
\pgfsys@useobject{currentmarker}{}%
\end{pgfscope}%
\end{pgfscope}%
\begin{pgfscope}%
\definecolor{textcolor}{rgb}{0.000000,0.000000,0.000000}%
\pgfsetstrokecolor{textcolor}%
\pgfsetfillcolor{textcolor}%
\pgftext[x=3.146389in,y=0.411948in,,top]{\color{textcolor}\rmfamily\fontsize{9.000000}{10.800000}\selectfont 6}%
\end{pgfscope}%
\begin{pgfscope}%
\pgfpathrectangle{\pgfqpoint{0.581943in}{0.509170in}}{\pgfqpoint{5.599061in}{8.312590in}}%
\pgfusepath{clip}%
\pgfsetbuttcap%
\pgfsetroundjoin%
\pgfsetlinewidth{0.250937pt}%
\definecolor{currentstroke}{rgb}{0.680000,0.680000,0.680000}%
\pgfsetstrokecolor{currentstroke}%
\pgfsetdash{{1.000000pt}{1.000000pt}}{0.000000pt}%
\pgfpathmoveto{\pgfqpoint{3.367016in}{0.509170in}}%
\pgfpathlineto{\pgfqpoint{3.367016in}{8.821759in}}%
\pgfusepath{stroke}%
\end{pgfscope}%
\begin{pgfscope}%
\pgfsetbuttcap%
\pgfsetroundjoin%
\definecolor{currentfill}{rgb}{0.000000,0.000000,0.000000}%
\pgfsetfillcolor{currentfill}%
\pgfsetlinewidth{0.803000pt}%
\definecolor{currentstroke}{rgb}{0.000000,0.000000,0.000000}%
\pgfsetstrokecolor{currentstroke}%
\pgfsetdash{}{0pt}%
\pgfsys@defobject{currentmarker}{\pgfqpoint{0.000000in}{-0.048611in}}{\pgfqpoint{0.000000in}{0.000000in}}{%
\pgfpathmoveto{\pgfqpoint{0.000000in}{0.000000in}}%
\pgfpathlineto{\pgfqpoint{0.000000in}{-0.048611in}}%
\pgfusepath{stroke,fill}%
}%
\begin{pgfscope}%
\pgfsys@transformshift{3.367016in}{0.509170in}%
\pgfsys@useobject{currentmarker}{}%
\end{pgfscope}%
\end{pgfscope}%
\begin{pgfscope}%
\definecolor{textcolor}{rgb}{0.000000,0.000000,0.000000}%
\pgfsetstrokecolor{textcolor}%
\pgfsetfillcolor{textcolor}%
\pgftext[x=3.367016in,y=0.411948in,,top]{\color{textcolor}\rmfamily\fontsize{9.000000}{10.800000}\selectfont 7}%
\end{pgfscope}%
\begin{pgfscope}%
\pgfpathrectangle{\pgfqpoint{0.581943in}{0.509170in}}{\pgfqpoint{5.599061in}{8.312590in}}%
\pgfusepath{clip}%
\pgfsetbuttcap%
\pgfsetroundjoin%
\pgfsetlinewidth{0.250937pt}%
\definecolor{currentstroke}{rgb}{0.680000,0.680000,0.680000}%
\pgfsetstrokecolor{currentstroke}%
\pgfsetdash{{1.000000pt}{1.000000pt}}{0.000000pt}%
\pgfpathmoveto{\pgfqpoint{3.558132in}{0.509170in}}%
\pgfpathlineto{\pgfqpoint{3.558132in}{8.821759in}}%
\pgfusepath{stroke}%
\end{pgfscope}%
\begin{pgfscope}%
\pgfsetbuttcap%
\pgfsetroundjoin%
\definecolor{currentfill}{rgb}{0.000000,0.000000,0.000000}%
\pgfsetfillcolor{currentfill}%
\pgfsetlinewidth{0.803000pt}%
\definecolor{currentstroke}{rgb}{0.000000,0.000000,0.000000}%
\pgfsetstrokecolor{currentstroke}%
\pgfsetdash{}{0pt}%
\pgfsys@defobject{currentmarker}{\pgfqpoint{0.000000in}{-0.048611in}}{\pgfqpoint{0.000000in}{0.000000in}}{%
\pgfpathmoveto{\pgfqpoint{0.000000in}{0.000000in}}%
\pgfpathlineto{\pgfqpoint{0.000000in}{-0.048611in}}%
\pgfusepath{stroke,fill}%
}%
\begin{pgfscope}%
\pgfsys@transformshift{3.558132in}{0.509170in}%
\pgfsys@useobject{currentmarker}{}%
\end{pgfscope}%
\end{pgfscope}%
\begin{pgfscope}%
\definecolor{textcolor}{rgb}{0.000000,0.000000,0.000000}%
\pgfsetstrokecolor{textcolor}%
\pgfsetfillcolor{textcolor}%
\pgftext[x=3.558132in,y=0.411948in,,top]{\color{textcolor}\rmfamily\fontsize{9.000000}{10.800000}\selectfont 8}%
\end{pgfscope}%
\begin{pgfscope}%
\pgfpathrectangle{\pgfqpoint{0.581943in}{0.509170in}}{\pgfqpoint{5.599061in}{8.312590in}}%
\pgfusepath{clip}%
\pgfsetbuttcap%
\pgfsetroundjoin%
\pgfsetlinewidth{0.250937pt}%
\definecolor{currentstroke}{rgb}{0.680000,0.680000,0.680000}%
\pgfsetstrokecolor{currentstroke}%
\pgfsetdash{{1.000000pt}{1.000000pt}}{0.000000pt}%
\pgfpathmoveto{\pgfqpoint{3.726708in}{0.509170in}}%
\pgfpathlineto{\pgfqpoint{3.726708in}{8.821759in}}%
\pgfusepath{stroke}%
\end{pgfscope}%
\begin{pgfscope}%
\pgfsetbuttcap%
\pgfsetroundjoin%
\definecolor{currentfill}{rgb}{0.000000,0.000000,0.000000}%
\pgfsetfillcolor{currentfill}%
\pgfsetlinewidth{0.803000pt}%
\definecolor{currentstroke}{rgb}{0.000000,0.000000,0.000000}%
\pgfsetstrokecolor{currentstroke}%
\pgfsetdash{}{0pt}%
\pgfsys@defobject{currentmarker}{\pgfqpoint{0.000000in}{-0.048611in}}{\pgfqpoint{0.000000in}{0.000000in}}{%
\pgfpathmoveto{\pgfqpoint{0.000000in}{0.000000in}}%
\pgfpathlineto{\pgfqpoint{0.000000in}{-0.048611in}}%
\pgfusepath{stroke,fill}%
}%
\begin{pgfscope}%
\pgfsys@transformshift{3.726708in}{0.509170in}%
\pgfsys@useobject{currentmarker}{}%
\end{pgfscope}%
\end{pgfscope}%
\begin{pgfscope}%
\definecolor{textcolor}{rgb}{0.000000,0.000000,0.000000}%
\pgfsetstrokecolor{textcolor}%
\pgfsetfillcolor{textcolor}%
\pgftext[x=3.726708in,y=0.411948in,,top]{\color{textcolor}\rmfamily\fontsize{9.000000}{10.800000}\selectfont 9}%
\end{pgfscope}%
\begin{pgfscope}%
\pgfpathrectangle{\pgfqpoint{0.581943in}{0.509170in}}{\pgfqpoint{5.599061in}{8.312590in}}%
\pgfusepath{clip}%
\pgfsetbuttcap%
\pgfsetroundjoin%
\pgfsetlinewidth{0.250937pt}%
\definecolor{currentstroke}{rgb}{0.680000,0.680000,0.680000}%
\pgfsetstrokecolor{currentstroke}%
\pgfsetdash{{1.000000pt}{1.000000pt}}{0.000000pt}%
\pgfpathmoveto{\pgfqpoint{3.877505in}{0.509170in}}%
\pgfpathlineto{\pgfqpoint{3.877505in}{8.821759in}}%
\pgfusepath{stroke}%
\end{pgfscope}%
\begin{pgfscope}%
\pgfsetbuttcap%
\pgfsetroundjoin%
\definecolor{currentfill}{rgb}{0.000000,0.000000,0.000000}%
\pgfsetfillcolor{currentfill}%
\pgfsetlinewidth{0.803000pt}%
\definecolor{currentstroke}{rgb}{0.000000,0.000000,0.000000}%
\pgfsetstrokecolor{currentstroke}%
\pgfsetdash{}{0pt}%
\pgfsys@defobject{currentmarker}{\pgfqpoint{0.000000in}{-0.048611in}}{\pgfqpoint{0.000000in}{0.000000in}}{%
\pgfpathmoveto{\pgfqpoint{0.000000in}{0.000000in}}%
\pgfpathlineto{\pgfqpoint{0.000000in}{-0.048611in}}%
\pgfusepath{stroke,fill}%
}%
\begin{pgfscope}%
\pgfsys@transformshift{3.877505in}{0.509170in}%
\pgfsys@useobject{currentmarker}{}%
\end{pgfscope}%
\end{pgfscope}%
\begin{pgfscope}%
\definecolor{textcolor}{rgb}{0.000000,0.000000,0.000000}%
\pgfsetstrokecolor{textcolor}%
\pgfsetfillcolor{textcolor}%
\pgftext[x=3.877505in,y=0.411948in,,top]{\color{textcolor}\rmfamily\fontsize{9.000000}{10.800000}\selectfont 10}%
\end{pgfscope}%
\begin{pgfscope}%
\pgfpathrectangle{\pgfqpoint{0.581943in}{0.509170in}}{\pgfqpoint{5.599061in}{8.312590in}}%
\pgfusepath{clip}%
\pgfsetbuttcap%
\pgfsetroundjoin%
\pgfsetlinewidth{0.250937pt}%
\definecolor{currentstroke}{rgb}{0.680000,0.680000,0.680000}%
\pgfsetstrokecolor{currentstroke}%
\pgfsetdash{{1.000000pt}{1.000000pt}}{0.000000pt}%
\pgfpathmoveto{\pgfqpoint{4.013917in}{0.509170in}}%
\pgfpathlineto{\pgfqpoint{4.013917in}{8.821759in}}%
\pgfusepath{stroke}%
\end{pgfscope}%
\begin{pgfscope}%
\pgfsetbuttcap%
\pgfsetroundjoin%
\definecolor{currentfill}{rgb}{0.000000,0.000000,0.000000}%
\pgfsetfillcolor{currentfill}%
\pgfsetlinewidth{0.803000pt}%
\definecolor{currentstroke}{rgb}{0.000000,0.000000,0.000000}%
\pgfsetstrokecolor{currentstroke}%
\pgfsetdash{}{0pt}%
\pgfsys@defobject{currentmarker}{\pgfqpoint{0.000000in}{-0.048611in}}{\pgfqpoint{0.000000in}{0.000000in}}{%
\pgfpathmoveto{\pgfqpoint{0.000000in}{0.000000in}}%
\pgfpathlineto{\pgfqpoint{0.000000in}{-0.048611in}}%
\pgfusepath{stroke,fill}%
}%
\begin{pgfscope}%
\pgfsys@transformshift{4.013917in}{0.509170in}%
\pgfsys@useobject{currentmarker}{}%
\end{pgfscope}%
\end{pgfscope}%
\begin{pgfscope}%
\pgfpathrectangle{\pgfqpoint{0.581943in}{0.509170in}}{\pgfqpoint{5.599061in}{8.312590in}}%
\pgfusepath{clip}%
\pgfsetbuttcap%
\pgfsetroundjoin%
\pgfsetlinewidth{0.250937pt}%
\definecolor{currentstroke}{rgb}{0.680000,0.680000,0.680000}%
\pgfsetstrokecolor{currentstroke}%
\pgfsetdash{{1.000000pt}{1.000000pt}}{0.000000pt}%
\pgfpathmoveto{\pgfqpoint{4.138452in}{0.509170in}}%
\pgfpathlineto{\pgfqpoint{4.138452in}{8.821759in}}%
\pgfusepath{stroke}%
\end{pgfscope}%
\begin{pgfscope}%
\pgfsetbuttcap%
\pgfsetroundjoin%
\definecolor{currentfill}{rgb}{0.000000,0.000000,0.000000}%
\pgfsetfillcolor{currentfill}%
\pgfsetlinewidth{0.803000pt}%
\definecolor{currentstroke}{rgb}{0.000000,0.000000,0.000000}%
\pgfsetstrokecolor{currentstroke}%
\pgfsetdash{}{0pt}%
\pgfsys@defobject{currentmarker}{\pgfqpoint{0.000000in}{-0.048611in}}{\pgfqpoint{0.000000in}{0.000000in}}{%
\pgfpathmoveto{\pgfqpoint{0.000000in}{0.000000in}}%
\pgfpathlineto{\pgfqpoint{0.000000in}{-0.048611in}}%
\pgfusepath{stroke,fill}%
}%
\begin{pgfscope}%
\pgfsys@transformshift{4.138452in}{0.509170in}%
\pgfsys@useobject{currentmarker}{}%
\end{pgfscope}%
\end{pgfscope}%
\begin{pgfscope}%
\pgfpathrectangle{\pgfqpoint{0.581943in}{0.509170in}}{\pgfqpoint{5.599061in}{8.312590in}}%
\pgfusepath{clip}%
\pgfsetbuttcap%
\pgfsetroundjoin%
\pgfsetlinewidth{0.250937pt}%
\definecolor{currentstroke}{rgb}{0.680000,0.680000,0.680000}%
\pgfsetstrokecolor{currentstroke}%
\pgfsetdash{{1.000000pt}{1.000000pt}}{0.000000pt}%
\pgfpathmoveto{\pgfqpoint{4.253013in}{0.509170in}}%
\pgfpathlineto{\pgfqpoint{4.253013in}{8.821759in}}%
\pgfusepath{stroke}%
\end{pgfscope}%
\begin{pgfscope}%
\pgfsetbuttcap%
\pgfsetroundjoin%
\definecolor{currentfill}{rgb}{0.000000,0.000000,0.000000}%
\pgfsetfillcolor{currentfill}%
\pgfsetlinewidth{0.803000pt}%
\definecolor{currentstroke}{rgb}{0.000000,0.000000,0.000000}%
\pgfsetstrokecolor{currentstroke}%
\pgfsetdash{}{0pt}%
\pgfsys@defobject{currentmarker}{\pgfqpoint{0.000000in}{-0.048611in}}{\pgfqpoint{0.000000in}{0.000000in}}{%
\pgfpathmoveto{\pgfqpoint{0.000000in}{0.000000in}}%
\pgfpathlineto{\pgfqpoint{0.000000in}{-0.048611in}}%
\pgfusepath{stroke,fill}%
}%
\begin{pgfscope}%
\pgfsys@transformshift{4.253013in}{0.509170in}%
\pgfsys@useobject{currentmarker}{}%
\end{pgfscope}%
\end{pgfscope}%
\begin{pgfscope}%
\pgfpathrectangle{\pgfqpoint{0.581943in}{0.509170in}}{\pgfqpoint{5.599061in}{8.312590in}}%
\pgfusepath{clip}%
\pgfsetbuttcap%
\pgfsetroundjoin%
\pgfsetlinewidth{0.250937pt}%
\definecolor{currentstroke}{rgb}{0.680000,0.680000,0.680000}%
\pgfsetstrokecolor{currentstroke}%
\pgfsetdash{{1.000000pt}{1.000000pt}}{0.000000pt}%
\pgfpathmoveto{\pgfqpoint{4.359079in}{0.509170in}}%
\pgfpathlineto{\pgfqpoint{4.359079in}{8.821759in}}%
\pgfusepath{stroke}%
\end{pgfscope}%
\begin{pgfscope}%
\pgfsetbuttcap%
\pgfsetroundjoin%
\definecolor{currentfill}{rgb}{0.000000,0.000000,0.000000}%
\pgfsetfillcolor{currentfill}%
\pgfsetlinewidth{0.803000pt}%
\definecolor{currentstroke}{rgb}{0.000000,0.000000,0.000000}%
\pgfsetstrokecolor{currentstroke}%
\pgfsetdash{}{0pt}%
\pgfsys@defobject{currentmarker}{\pgfqpoint{0.000000in}{-0.048611in}}{\pgfqpoint{0.000000in}{0.000000in}}{%
\pgfpathmoveto{\pgfqpoint{0.000000in}{0.000000in}}%
\pgfpathlineto{\pgfqpoint{0.000000in}{-0.048611in}}%
\pgfusepath{stroke,fill}%
}%
\begin{pgfscope}%
\pgfsys@transformshift{4.359079in}{0.509170in}%
\pgfsys@useobject{currentmarker}{}%
\end{pgfscope}%
\end{pgfscope}%
\begin{pgfscope}%
\pgfpathrectangle{\pgfqpoint{0.581943in}{0.509170in}}{\pgfqpoint{5.599061in}{8.312590in}}%
\pgfusepath{clip}%
\pgfsetbuttcap%
\pgfsetroundjoin%
\pgfsetlinewidth{0.250937pt}%
\definecolor{currentstroke}{rgb}{0.680000,0.680000,0.680000}%
\pgfsetstrokecolor{currentstroke}%
\pgfsetdash{{1.000000pt}{1.000000pt}}{0.000000pt}%
\pgfpathmoveto{\pgfqpoint{4.457825in}{0.509170in}}%
\pgfpathlineto{\pgfqpoint{4.457825in}{8.821759in}}%
\pgfusepath{stroke}%
\end{pgfscope}%
\begin{pgfscope}%
\pgfsetbuttcap%
\pgfsetroundjoin%
\definecolor{currentfill}{rgb}{0.000000,0.000000,0.000000}%
\pgfsetfillcolor{currentfill}%
\pgfsetlinewidth{0.803000pt}%
\definecolor{currentstroke}{rgb}{0.000000,0.000000,0.000000}%
\pgfsetstrokecolor{currentstroke}%
\pgfsetdash{}{0pt}%
\pgfsys@defobject{currentmarker}{\pgfqpoint{0.000000in}{-0.048611in}}{\pgfqpoint{0.000000in}{0.000000in}}{%
\pgfpathmoveto{\pgfqpoint{0.000000in}{0.000000in}}%
\pgfpathlineto{\pgfqpoint{0.000000in}{-0.048611in}}%
\pgfusepath{stroke,fill}%
}%
\begin{pgfscope}%
\pgfsys@transformshift{4.457825in}{0.509170in}%
\pgfsys@useobject{currentmarker}{}%
\end{pgfscope}%
\end{pgfscope}%
\begin{pgfscope}%
\pgfpathrectangle{\pgfqpoint{0.581943in}{0.509170in}}{\pgfqpoint{5.599061in}{8.312590in}}%
\pgfusepath{clip}%
\pgfsetbuttcap%
\pgfsetroundjoin%
\pgfsetlinewidth{0.250937pt}%
\definecolor{currentstroke}{rgb}{0.680000,0.680000,0.680000}%
\pgfsetstrokecolor{currentstroke}%
\pgfsetdash{{1.000000pt}{1.000000pt}}{0.000000pt}%
\pgfpathmoveto{\pgfqpoint{4.550195in}{0.509170in}}%
\pgfpathlineto{\pgfqpoint{4.550195in}{8.821759in}}%
\pgfusepath{stroke}%
\end{pgfscope}%
\begin{pgfscope}%
\pgfsetbuttcap%
\pgfsetroundjoin%
\definecolor{currentfill}{rgb}{0.000000,0.000000,0.000000}%
\pgfsetfillcolor{currentfill}%
\pgfsetlinewidth{0.803000pt}%
\definecolor{currentstroke}{rgb}{0.000000,0.000000,0.000000}%
\pgfsetstrokecolor{currentstroke}%
\pgfsetdash{}{0pt}%
\pgfsys@defobject{currentmarker}{\pgfqpoint{0.000000in}{-0.048611in}}{\pgfqpoint{0.000000in}{0.000000in}}{%
\pgfpathmoveto{\pgfqpoint{0.000000in}{0.000000in}}%
\pgfpathlineto{\pgfqpoint{0.000000in}{-0.048611in}}%
\pgfusepath{stroke,fill}%
}%
\begin{pgfscope}%
\pgfsys@transformshift{4.550195in}{0.509170in}%
\pgfsys@useobject{currentmarker}{}%
\end{pgfscope}%
\end{pgfscope}%
\begin{pgfscope}%
\pgfpathrectangle{\pgfqpoint{0.581943in}{0.509170in}}{\pgfqpoint{5.599061in}{8.312590in}}%
\pgfusepath{clip}%
\pgfsetbuttcap%
\pgfsetroundjoin%
\pgfsetlinewidth{0.250937pt}%
\definecolor{currentstroke}{rgb}{0.680000,0.680000,0.680000}%
\pgfsetstrokecolor{currentstroke}%
\pgfsetdash{{1.000000pt}{1.000000pt}}{0.000000pt}%
\pgfpathmoveto{\pgfqpoint{4.636964in}{0.509170in}}%
\pgfpathlineto{\pgfqpoint{4.636964in}{8.821759in}}%
\pgfusepath{stroke}%
\end{pgfscope}%
\begin{pgfscope}%
\pgfsetbuttcap%
\pgfsetroundjoin%
\definecolor{currentfill}{rgb}{0.000000,0.000000,0.000000}%
\pgfsetfillcolor{currentfill}%
\pgfsetlinewidth{0.803000pt}%
\definecolor{currentstroke}{rgb}{0.000000,0.000000,0.000000}%
\pgfsetstrokecolor{currentstroke}%
\pgfsetdash{}{0pt}%
\pgfsys@defobject{currentmarker}{\pgfqpoint{0.000000in}{-0.048611in}}{\pgfqpoint{0.000000in}{0.000000in}}{%
\pgfpathmoveto{\pgfqpoint{0.000000in}{0.000000in}}%
\pgfpathlineto{\pgfqpoint{0.000000in}{-0.048611in}}%
\pgfusepath{stroke,fill}%
}%
\begin{pgfscope}%
\pgfsys@transformshift{4.636964in}{0.509170in}%
\pgfsys@useobject{currentmarker}{}%
\end{pgfscope}%
\end{pgfscope}%
\begin{pgfscope}%
\pgfpathrectangle{\pgfqpoint{0.581943in}{0.509170in}}{\pgfqpoint{5.599061in}{8.312590in}}%
\pgfusepath{clip}%
\pgfsetbuttcap%
\pgfsetroundjoin%
\pgfsetlinewidth{0.250937pt}%
\definecolor{currentstroke}{rgb}{0.680000,0.680000,0.680000}%
\pgfsetstrokecolor{currentstroke}%
\pgfsetdash{{1.000000pt}{1.000000pt}}{0.000000pt}%
\pgfpathmoveto{\pgfqpoint{4.718772in}{0.509170in}}%
\pgfpathlineto{\pgfqpoint{4.718772in}{8.821759in}}%
\pgfusepath{stroke}%
\end{pgfscope}%
\begin{pgfscope}%
\pgfsetbuttcap%
\pgfsetroundjoin%
\definecolor{currentfill}{rgb}{0.000000,0.000000,0.000000}%
\pgfsetfillcolor{currentfill}%
\pgfsetlinewidth{0.803000pt}%
\definecolor{currentstroke}{rgb}{0.000000,0.000000,0.000000}%
\pgfsetstrokecolor{currentstroke}%
\pgfsetdash{}{0pt}%
\pgfsys@defobject{currentmarker}{\pgfqpoint{0.000000in}{-0.048611in}}{\pgfqpoint{0.000000in}{0.000000in}}{%
\pgfpathmoveto{\pgfqpoint{0.000000in}{0.000000in}}%
\pgfpathlineto{\pgfqpoint{0.000000in}{-0.048611in}}%
\pgfusepath{stroke,fill}%
}%
\begin{pgfscope}%
\pgfsys@transformshift{4.718772in}{0.509170in}%
\pgfsys@useobject{currentmarker}{}%
\end{pgfscope}%
\end{pgfscope}%
\begin{pgfscope}%
\pgfpathrectangle{\pgfqpoint{0.581943in}{0.509170in}}{\pgfqpoint{5.599061in}{8.312590in}}%
\pgfusepath{clip}%
\pgfsetbuttcap%
\pgfsetroundjoin%
\pgfsetlinewidth{0.250937pt}%
\definecolor{currentstroke}{rgb}{0.680000,0.680000,0.680000}%
\pgfsetstrokecolor{currentstroke}%
\pgfsetdash{{1.000000pt}{1.000000pt}}{0.000000pt}%
\pgfpathmoveto{\pgfqpoint{4.796155in}{0.509170in}}%
\pgfpathlineto{\pgfqpoint{4.796155in}{8.821759in}}%
\pgfusepath{stroke}%
\end{pgfscope}%
\begin{pgfscope}%
\pgfsetbuttcap%
\pgfsetroundjoin%
\definecolor{currentfill}{rgb}{0.000000,0.000000,0.000000}%
\pgfsetfillcolor{currentfill}%
\pgfsetlinewidth{0.803000pt}%
\definecolor{currentstroke}{rgb}{0.000000,0.000000,0.000000}%
\pgfsetstrokecolor{currentstroke}%
\pgfsetdash{}{0pt}%
\pgfsys@defobject{currentmarker}{\pgfqpoint{0.000000in}{-0.048611in}}{\pgfqpoint{0.000000in}{0.000000in}}{%
\pgfpathmoveto{\pgfqpoint{0.000000in}{0.000000in}}%
\pgfpathlineto{\pgfqpoint{0.000000in}{-0.048611in}}%
\pgfusepath{stroke,fill}%
}%
\begin{pgfscope}%
\pgfsys@transformshift{4.796155in}{0.509170in}%
\pgfsys@useobject{currentmarker}{}%
\end{pgfscope}%
\end{pgfscope}%
\begin{pgfscope}%
\pgfpathrectangle{\pgfqpoint{0.581943in}{0.509170in}}{\pgfqpoint{5.599061in}{8.312590in}}%
\pgfusepath{clip}%
\pgfsetbuttcap%
\pgfsetroundjoin%
\pgfsetlinewidth{0.250937pt}%
\definecolor{currentstroke}{rgb}{0.680000,0.680000,0.680000}%
\pgfsetstrokecolor{currentstroke}%
\pgfsetdash{{1.000000pt}{1.000000pt}}{0.000000pt}%
\pgfpathmoveto{\pgfqpoint{4.869568in}{0.509170in}}%
\pgfpathlineto{\pgfqpoint{4.869568in}{8.821759in}}%
\pgfusepath{stroke}%
\end{pgfscope}%
\begin{pgfscope}%
\pgfsetbuttcap%
\pgfsetroundjoin%
\definecolor{currentfill}{rgb}{0.000000,0.000000,0.000000}%
\pgfsetfillcolor{currentfill}%
\pgfsetlinewidth{0.803000pt}%
\definecolor{currentstroke}{rgb}{0.000000,0.000000,0.000000}%
\pgfsetstrokecolor{currentstroke}%
\pgfsetdash{}{0pt}%
\pgfsys@defobject{currentmarker}{\pgfqpoint{0.000000in}{-0.048611in}}{\pgfqpoint{0.000000in}{0.000000in}}{%
\pgfpathmoveto{\pgfqpoint{0.000000in}{0.000000in}}%
\pgfpathlineto{\pgfqpoint{0.000000in}{-0.048611in}}%
\pgfusepath{stroke,fill}%
}%
\begin{pgfscope}%
\pgfsys@transformshift{4.869568in}{0.509170in}%
\pgfsys@useobject{currentmarker}{}%
\end{pgfscope}%
\end{pgfscope}%
\begin{pgfscope}%
\definecolor{textcolor}{rgb}{0.000000,0.000000,0.000000}%
\pgfsetstrokecolor{textcolor}%
\pgfsetfillcolor{textcolor}%
\pgftext[x=4.869568in,y=0.411948in,,top]{\color{textcolor}\rmfamily\fontsize{9.000000}{10.800000}\selectfont 20}%
\end{pgfscope}%
\begin{pgfscope}%
\pgfpathrectangle{\pgfqpoint{0.581943in}{0.509170in}}{\pgfqpoint{5.599061in}{8.312590in}}%
\pgfusepath{clip}%
\pgfsetbuttcap%
\pgfsetroundjoin%
\pgfsetlinewidth{0.250937pt}%
\definecolor{currentstroke}{rgb}{0.680000,0.680000,0.680000}%
\pgfsetstrokecolor{currentstroke}%
\pgfsetdash{{1.000000pt}{1.000000pt}}{0.000000pt}%
\pgfpathmoveto{\pgfqpoint{4.939399in}{0.509170in}}%
\pgfpathlineto{\pgfqpoint{4.939399in}{8.821759in}}%
\pgfusepath{stroke}%
\end{pgfscope}%
\begin{pgfscope}%
\pgfsetbuttcap%
\pgfsetroundjoin%
\definecolor{currentfill}{rgb}{0.000000,0.000000,0.000000}%
\pgfsetfillcolor{currentfill}%
\pgfsetlinewidth{0.803000pt}%
\definecolor{currentstroke}{rgb}{0.000000,0.000000,0.000000}%
\pgfsetstrokecolor{currentstroke}%
\pgfsetdash{}{0pt}%
\pgfsys@defobject{currentmarker}{\pgfqpoint{0.000000in}{-0.048611in}}{\pgfqpoint{0.000000in}{0.000000in}}{%
\pgfpathmoveto{\pgfqpoint{0.000000in}{0.000000in}}%
\pgfpathlineto{\pgfqpoint{0.000000in}{-0.048611in}}%
\pgfusepath{stroke,fill}%
}%
\begin{pgfscope}%
\pgfsys@transformshift{4.939399in}{0.509170in}%
\pgfsys@useobject{currentmarker}{}%
\end{pgfscope}%
\end{pgfscope}%
\begin{pgfscope}%
\pgfpathrectangle{\pgfqpoint{0.581943in}{0.509170in}}{\pgfqpoint{5.599061in}{8.312590in}}%
\pgfusepath{clip}%
\pgfsetbuttcap%
\pgfsetroundjoin%
\pgfsetlinewidth{0.250937pt}%
\definecolor{currentstroke}{rgb}{0.680000,0.680000,0.680000}%
\pgfsetstrokecolor{currentstroke}%
\pgfsetdash{{1.000000pt}{1.000000pt}}{0.000000pt}%
\pgfpathmoveto{\pgfqpoint{5.005980in}{0.509170in}}%
\pgfpathlineto{\pgfqpoint{5.005980in}{8.821759in}}%
\pgfusepath{stroke}%
\end{pgfscope}%
\begin{pgfscope}%
\pgfsetbuttcap%
\pgfsetroundjoin%
\definecolor{currentfill}{rgb}{0.000000,0.000000,0.000000}%
\pgfsetfillcolor{currentfill}%
\pgfsetlinewidth{0.803000pt}%
\definecolor{currentstroke}{rgb}{0.000000,0.000000,0.000000}%
\pgfsetstrokecolor{currentstroke}%
\pgfsetdash{}{0pt}%
\pgfsys@defobject{currentmarker}{\pgfqpoint{0.000000in}{-0.048611in}}{\pgfqpoint{0.000000in}{0.000000in}}{%
\pgfpathmoveto{\pgfqpoint{0.000000in}{0.000000in}}%
\pgfpathlineto{\pgfqpoint{0.000000in}{-0.048611in}}%
\pgfusepath{stroke,fill}%
}%
\begin{pgfscope}%
\pgfsys@transformshift{5.005980in}{0.509170in}%
\pgfsys@useobject{currentmarker}{}%
\end{pgfscope}%
\end{pgfscope}%
\begin{pgfscope}%
\pgfpathrectangle{\pgfqpoint{0.581943in}{0.509170in}}{\pgfqpoint{5.599061in}{8.312590in}}%
\pgfusepath{clip}%
\pgfsetbuttcap%
\pgfsetroundjoin%
\pgfsetlinewidth{0.250937pt}%
\definecolor{currentstroke}{rgb}{0.680000,0.680000,0.680000}%
\pgfsetstrokecolor{currentstroke}%
\pgfsetdash{{1.000000pt}{1.000000pt}}{0.000000pt}%
\pgfpathmoveto{\pgfqpoint{5.069602in}{0.509170in}}%
\pgfpathlineto{\pgfqpoint{5.069602in}{8.821759in}}%
\pgfusepath{stroke}%
\end{pgfscope}%
\begin{pgfscope}%
\pgfsetbuttcap%
\pgfsetroundjoin%
\definecolor{currentfill}{rgb}{0.000000,0.000000,0.000000}%
\pgfsetfillcolor{currentfill}%
\pgfsetlinewidth{0.803000pt}%
\definecolor{currentstroke}{rgb}{0.000000,0.000000,0.000000}%
\pgfsetstrokecolor{currentstroke}%
\pgfsetdash{}{0pt}%
\pgfsys@defobject{currentmarker}{\pgfqpoint{0.000000in}{-0.048611in}}{\pgfqpoint{0.000000in}{0.000000in}}{%
\pgfpathmoveto{\pgfqpoint{0.000000in}{0.000000in}}%
\pgfpathlineto{\pgfqpoint{0.000000in}{-0.048611in}}%
\pgfusepath{stroke,fill}%
}%
\begin{pgfscope}%
\pgfsys@transformshift{5.069602in}{0.509170in}%
\pgfsys@useobject{currentmarker}{}%
\end{pgfscope}%
\end{pgfscope}%
\begin{pgfscope}%
\pgfpathrectangle{\pgfqpoint{0.581943in}{0.509170in}}{\pgfqpoint{5.599061in}{8.312590in}}%
\pgfusepath{clip}%
\pgfsetbuttcap%
\pgfsetroundjoin%
\pgfsetlinewidth{0.250937pt}%
\definecolor{currentstroke}{rgb}{0.680000,0.680000,0.680000}%
\pgfsetstrokecolor{currentstroke}%
\pgfsetdash{{1.000000pt}{1.000000pt}}{0.000000pt}%
\pgfpathmoveto{\pgfqpoint{5.130515in}{0.509170in}}%
\pgfpathlineto{\pgfqpoint{5.130515in}{8.821759in}}%
\pgfusepath{stroke}%
\end{pgfscope}%
\begin{pgfscope}%
\pgfsetbuttcap%
\pgfsetroundjoin%
\definecolor{currentfill}{rgb}{0.000000,0.000000,0.000000}%
\pgfsetfillcolor{currentfill}%
\pgfsetlinewidth{0.803000pt}%
\definecolor{currentstroke}{rgb}{0.000000,0.000000,0.000000}%
\pgfsetstrokecolor{currentstroke}%
\pgfsetdash{}{0pt}%
\pgfsys@defobject{currentmarker}{\pgfqpoint{0.000000in}{-0.048611in}}{\pgfqpoint{0.000000in}{0.000000in}}{%
\pgfpathmoveto{\pgfqpoint{0.000000in}{0.000000in}}%
\pgfpathlineto{\pgfqpoint{0.000000in}{-0.048611in}}%
\pgfusepath{stroke,fill}%
}%
\begin{pgfscope}%
\pgfsys@transformshift{5.130515in}{0.509170in}%
\pgfsys@useobject{currentmarker}{}%
\end{pgfscope}%
\end{pgfscope}%
\begin{pgfscope}%
\pgfpathrectangle{\pgfqpoint{0.581943in}{0.509170in}}{\pgfqpoint{5.599061in}{8.312590in}}%
\pgfusepath{clip}%
\pgfsetbuttcap%
\pgfsetroundjoin%
\pgfsetlinewidth{0.250937pt}%
\definecolor{currentstroke}{rgb}{0.680000,0.680000,0.680000}%
\pgfsetstrokecolor{currentstroke}%
\pgfsetdash{{1.000000pt}{1.000000pt}}{0.000000pt}%
\pgfpathmoveto{\pgfqpoint{5.188941in}{0.509170in}}%
\pgfpathlineto{\pgfqpoint{5.188941in}{8.821759in}}%
\pgfusepath{stroke}%
\end{pgfscope}%
\begin{pgfscope}%
\pgfsetbuttcap%
\pgfsetroundjoin%
\definecolor{currentfill}{rgb}{0.000000,0.000000,0.000000}%
\pgfsetfillcolor{currentfill}%
\pgfsetlinewidth{0.803000pt}%
\definecolor{currentstroke}{rgb}{0.000000,0.000000,0.000000}%
\pgfsetstrokecolor{currentstroke}%
\pgfsetdash{}{0pt}%
\pgfsys@defobject{currentmarker}{\pgfqpoint{0.000000in}{-0.048611in}}{\pgfqpoint{0.000000in}{0.000000in}}{%
\pgfpathmoveto{\pgfqpoint{0.000000in}{0.000000in}}%
\pgfpathlineto{\pgfqpoint{0.000000in}{-0.048611in}}%
\pgfusepath{stroke,fill}%
}%
\begin{pgfscope}%
\pgfsys@transformshift{5.188941in}{0.509170in}%
\pgfsys@useobject{currentmarker}{}%
\end{pgfscope}%
\end{pgfscope}%
\begin{pgfscope}%
\pgfpathrectangle{\pgfqpoint{0.581943in}{0.509170in}}{\pgfqpoint{5.599061in}{8.312590in}}%
\pgfusepath{clip}%
\pgfsetbuttcap%
\pgfsetroundjoin%
\pgfsetlinewidth{0.250937pt}%
\definecolor{currentstroke}{rgb}{0.680000,0.680000,0.680000}%
\pgfsetstrokecolor{currentstroke}%
\pgfsetdash{{1.000000pt}{1.000000pt}}{0.000000pt}%
\pgfpathmoveto{\pgfqpoint{5.245076in}{0.509170in}}%
\pgfpathlineto{\pgfqpoint{5.245076in}{8.821759in}}%
\pgfusepath{stroke}%
\end{pgfscope}%
\begin{pgfscope}%
\pgfsetbuttcap%
\pgfsetroundjoin%
\definecolor{currentfill}{rgb}{0.000000,0.000000,0.000000}%
\pgfsetfillcolor{currentfill}%
\pgfsetlinewidth{0.803000pt}%
\definecolor{currentstroke}{rgb}{0.000000,0.000000,0.000000}%
\pgfsetstrokecolor{currentstroke}%
\pgfsetdash{}{0pt}%
\pgfsys@defobject{currentmarker}{\pgfqpoint{0.000000in}{-0.048611in}}{\pgfqpoint{0.000000in}{0.000000in}}{%
\pgfpathmoveto{\pgfqpoint{0.000000in}{0.000000in}}%
\pgfpathlineto{\pgfqpoint{0.000000in}{-0.048611in}}%
\pgfusepath{stroke,fill}%
}%
\begin{pgfscope}%
\pgfsys@transformshift{5.245076in}{0.509170in}%
\pgfsys@useobject{currentmarker}{}%
\end{pgfscope}%
\end{pgfscope}%
\begin{pgfscope}%
\pgfpathrectangle{\pgfqpoint{0.581943in}{0.509170in}}{\pgfqpoint{5.599061in}{8.312590in}}%
\pgfusepath{clip}%
\pgfsetbuttcap%
\pgfsetroundjoin%
\pgfsetlinewidth{0.250937pt}%
\definecolor{currentstroke}{rgb}{0.680000,0.680000,0.680000}%
\pgfsetstrokecolor{currentstroke}%
\pgfsetdash{{1.000000pt}{1.000000pt}}{0.000000pt}%
\pgfpathmoveto{\pgfqpoint{5.299091in}{0.509170in}}%
\pgfpathlineto{\pgfqpoint{5.299091in}{8.821759in}}%
\pgfusepath{stroke}%
\end{pgfscope}%
\begin{pgfscope}%
\pgfsetbuttcap%
\pgfsetroundjoin%
\definecolor{currentfill}{rgb}{0.000000,0.000000,0.000000}%
\pgfsetfillcolor{currentfill}%
\pgfsetlinewidth{0.803000pt}%
\definecolor{currentstroke}{rgb}{0.000000,0.000000,0.000000}%
\pgfsetstrokecolor{currentstroke}%
\pgfsetdash{}{0pt}%
\pgfsys@defobject{currentmarker}{\pgfqpoint{0.000000in}{-0.048611in}}{\pgfqpoint{0.000000in}{0.000000in}}{%
\pgfpathmoveto{\pgfqpoint{0.000000in}{0.000000in}}%
\pgfpathlineto{\pgfqpoint{0.000000in}{-0.048611in}}%
\pgfusepath{stroke,fill}%
}%
\begin{pgfscope}%
\pgfsys@transformshift{5.299091in}{0.509170in}%
\pgfsys@useobject{currentmarker}{}%
\end{pgfscope}%
\end{pgfscope}%
\begin{pgfscope}%
\pgfpathrectangle{\pgfqpoint{0.581943in}{0.509170in}}{\pgfqpoint{5.599061in}{8.312590in}}%
\pgfusepath{clip}%
\pgfsetbuttcap%
\pgfsetroundjoin%
\pgfsetlinewidth{0.250937pt}%
\definecolor{currentstroke}{rgb}{0.680000,0.680000,0.680000}%
\pgfsetstrokecolor{currentstroke}%
\pgfsetdash{{1.000000pt}{1.000000pt}}{0.000000pt}%
\pgfpathmoveto{\pgfqpoint{5.351142in}{0.509170in}}%
\pgfpathlineto{\pgfqpoint{5.351142in}{8.821759in}}%
\pgfusepath{stroke}%
\end{pgfscope}%
\begin{pgfscope}%
\pgfsetbuttcap%
\pgfsetroundjoin%
\definecolor{currentfill}{rgb}{0.000000,0.000000,0.000000}%
\pgfsetfillcolor{currentfill}%
\pgfsetlinewidth{0.803000pt}%
\definecolor{currentstroke}{rgb}{0.000000,0.000000,0.000000}%
\pgfsetstrokecolor{currentstroke}%
\pgfsetdash{}{0pt}%
\pgfsys@defobject{currentmarker}{\pgfqpoint{0.000000in}{-0.048611in}}{\pgfqpoint{0.000000in}{0.000000in}}{%
\pgfpathmoveto{\pgfqpoint{0.000000in}{0.000000in}}%
\pgfpathlineto{\pgfqpoint{0.000000in}{-0.048611in}}%
\pgfusepath{stroke,fill}%
}%
\begin{pgfscope}%
\pgfsys@transformshift{5.351142in}{0.509170in}%
\pgfsys@useobject{currentmarker}{}%
\end{pgfscope}%
\end{pgfscope}%
\begin{pgfscope}%
\pgfpathrectangle{\pgfqpoint{0.581943in}{0.509170in}}{\pgfqpoint{5.599061in}{8.312590in}}%
\pgfusepath{clip}%
\pgfsetbuttcap%
\pgfsetroundjoin%
\pgfsetlinewidth{0.250937pt}%
\definecolor{currentstroke}{rgb}{0.680000,0.680000,0.680000}%
\pgfsetstrokecolor{currentstroke}%
\pgfsetdash{{1.000000pt}{1.000000pt}}{0.000000pt}%
\pgfpathmoveto{\pgfqpoint{5.401367in}{0.509170in}}%
\pgfpathlineto{\pgfqpoint{5.401367in}{8.821759in}}%
\pgfusepath{stroke}%
\end{pgfscope}%
\begin{pgfscope}%
\pgfsetbuttcap%
\pgfsetroundjoin%
\definecolor{currentfill}{rgb}{0.000000,0.000000,0.000000}%
\pgfsetfillcolor{currentfill}%
\pgfsetlinewidth{0.803000pt}%
\definecolor{currentstroke}{rgb}{0.000000,0.000000,0.000000}%
\pgfsetstrokecolor{currentstroke}%
\pgfsetdash{}{0pt}%
\pgfsys@defobject{currentmarker}{\pgfqpoint{0.000000in}{-0.048611in}}{\pgfqpoint{0.000000in}{0.000000in}}{%
\pgfpathmoveto{\pgfqpoint{0.000000in}{0.000000in}}%
\pgfpathlineto{\pgfqpoint{0.000000in}{-0.048611in}}%
\pgfusepath{stroke,fill}%
}%
\begin{pgfscope}%
\pgfsys@transformshift{5.401367in}{0.509170in}%
\pgfsys@useobject{currentmarker}{}%
\end{pgfscope}%
\end{pgfscope}%
\begin{pgfscope}%
\pgfpathrectangle{\pgfqpoint{0.581943in}{0.509170in}}{\pgfqpoint{5.599061in}{8.312590in}}%
\pgfusepath{clip}%
\pgfsetbuttcap%
\pgfsetroundjoin%
\pgfsetlinewidth{0.250937pt}%
\definecolor{currentstroke}{rgb}{0.680000,0.680000,0.680000}%
\pgfsetstrokecolor{currentstroke}%
\pgfsetdash{{1.000000pt}{1.000000pt}}{0.000000pt}%
\pgfpathmoveto{\pgfqpoint{5.449888in}{0.509170in}}%
\pgfpathlineto{\pgfqpoint{5.449888in}{8.821759in}}%
\pgfusepath{stroke}%
\end{pgfscope}%
\begin{pgfscope}%
\pgfsetbuttcap%
\pgfsetroundjoin%
\definecolor{currentfill}{rgb}{0.000000,0.000000,0.000000}%
\pgfsetfillcolor{currentfill}%
\pgfsetlinewidth{0.803000pt}%
\definecolor{currentstroke}{rgb}{0.000000,0.000000,0.000000}%
\pgfsetstrokecolor{currentstroke}%
\pgfsetdash{}{0pt}%
\pgfsys@defobject{currentmarker}{\pgfqpoint{0.000000in}{-0.048611in}}{\pgfqpoint{0.000000in}{0.000000in}}{%
\pgfpathmoveto{\pgfqpoint{0.000000in}{0.000000in}}%
\pgfpathlineto{\pgfqpoint{0.000000in}{-0.048611in}}%
\pgfusepath{stroke,fill}%
}%
\begin{pgfscope}%
\pgfsys@transformshift{5.449888in}{0.509170in}%
\pgfsys@useobject{currentmarker}{}%
\end{pgfscope}%
\end{pgfscope}%
\begin{pgfscope}%
\definecolor{textcolor}{rgb}{0.000000,0.000000,0.000000}%
\pgfsetstrokecolor{textcolor}%
\pgfsetfillcolor{textcolor}%
\pgftext[x=5.449888in,y=0.411948in,,top]{\color{textcolor}\rmfamily\fontsize{9.000000}{10.800000}\selectfont 30}%
\end{pgfscope}%
\begin{pgfscope}%
\pgfpathrectangle{\pgfqpoint{0.581943in}{0.509170in}}{\pgfqpoint{5.599061in}{8.312590in}}%
\pgfusepath{clip}%
\pgfsetbuttcap%
\pgfsetroundjoin%
\pgfsetlinewidth{0.250937pt}%
\definecolor{currentstroke}{rgb}{0.680000,0.680000,0.680000}%
\pgfsetstrokecolor{currentstroke}%
\pgfsetdash{{1.000000pt}{1.000000pt}}{0.000000pt}%
\pgfpathmoveto{\pgfqpoint{5.496818in}{0.509170in}}%
\pgfpathlineto{\pgfqpoint{5.496818in}{8.821759in}}%
\pgfusepath{stroke}%
\end{pgfscope}%
\begin{pgfscope}%
\pgfsetbuttcap%
\pgfsetroundjoin%
\definecolor{currentfill}{rgb}{0.000000,0.000000,0.000000}%
\pgfsetfillcolor{currentfill}%
\pgfsetlinewidth{0.803000pt}%
\definecolor{currentstroke}{rgb}{0.000000,0.000000,0.000000}%
\pgfsetstrokecolor{currentstroke}%
\pgfsetdash{}{0pt}%
\pgfsys@defobject{currentmarker}{\pgfqpoint{0.000000in}{-0.048611in}}{\pgfqpoint{0.000000in}{0.000000in}}{%
\pgfpathmoveto{\pgfqpoint{0.000000in}{0.000000in}}%
\pgfpathlineto{\pgfqpoint{0.000000in}{-0.048611in}}%
\pgfusepath{stroke,fill}%
}%
\begin{pgfscope}%
\pgfsys@transformshift{5.496818in}{0.509170in}%
\pgfsys@useobject{currentmarker}{}%
\end{pgfscope}%
\end{pgfscope}%
\begin{pgfscope}%
\pgfpathrectangle{\pgfqpoint{0.581943in}{0.509170in}}{\pgfqpoint{5.599061in}{8.312590in}}%
\pgfusepath{clip}%
\pgfsetbuttcap%
\pgfsetroundjoin%
\pgfsetlinewidth{0.250937pt}%
\definecolor{currentstroke}{rgb}{0.680000,0.680000,0.680000}%
\pgfsetstrokecolor{currentstroke}%
\pgfsetdash{{1.000000pt}{1.000000pt}}{0.000000pt}%
\pgfpathmoveto{\pgfqpoint{5.542258in}{0.509170in}}%
\pgfpathlineto{\pgfqpoint{5.542258in}{8.821759in}}%
\pgfusepath{stroke}%
\end{pgfscope}%
\begin{pgfscope}%
\pgfsetbuttcap%
\pgfsetroundjoin%
\definecolor{currentfill}{rgb}{0.000000,0.000000,0.000000}%
\pgfsetfillcolor{currentfill}%
\pgfsetlinewidth{0.803000pt}%
\definecolor{currentstroke}{rgb}{0.000000,0.000000,0.000000}%
\pgfsetstrokecolor{currentstroke}%
\pgfsetdash{}{0pt}%
\pgfsys@defobject{currentmarker}{\pgfqpoint{0.000000in}{-0.048611in}}{\pgfqpoint{0.000000in}{0.000000in}}{%
\pgfpathmoveto{\pgfqpoint{0.000000in}{0.000000in}}%
\pgfpathlineto{\pgfqpoint{0.000000in}{-0.048611in}}%
\pgfusepath{stroke,fill}%
}%
\begin{pgfscope}%
\pgfsys@transformshift{5.542258in}{0.509170in}%
\pgfsys@useobject{currentmarker}{}%
\end{pgfscope}%
\end{pgfscope}%
\begin{pgfscope}%
\pgfpathrectangle{\pgfqpoint{0.581943in}{0.509170in}}{\pgfqpoint{5.599061in}{8.312590in}}%
\pgfusepath{clip}%
\pgfsetbuttcap%
\pgfsetroundjoin%
\pgfsetlinewidth{0.250937pt}%
\definecolor{currentstroke}{rgb}{0.680000,0.680000,0.680000}%
\pgfsetstrokecolor{currentstroke}%
\pgfsetdash{{1.000000pt}{1.000000pt}}{0.000000pt}%
\pgfpathmoveto{\pgfqpoint{5.586300in}{0.509170in}}%
\pgfpathlineto{\pgfqpoint{5.586300in}{8.821759in}}%
\pgfusepath{stroke}%
\end{pgfscope}%
\begin{pgfscope}%
\pgfsetbuttcap%
\pgfsetroundjoin%
\definecolor{currentfill}{rgb}{0.000000,0.000000,0.000000}%
\pgfsetfillcolor{currentfill}%
\pgfsetlinewidth{0.803000pt}%
\definecolor{currentstroke}{rgb}{0.000000,0.000000,0.000000}%
\pgfsetstrokecolor{currentstroke}%
\pgfsetdash{}{0pt}%
\pgfsys@defobject{currentmarker}{\pgfqpoint{0.000000in}{-0.048611in}}{\pgfqpoint{0.000000in}{0.000000in}}{%
\pgfpathmoveto{\pgfqpoint{0.000000in}{0.000000in}}%
\pgfpathlineto{\pgfqpoint{0.000000in}{-0.048611in}}%
\pgfusepath{stroke,fill}%
}%
\begin{pgfscope}%
\pgfsys@transformshift{5.586300in}{0.509170in}%
\pgfsys@useobject{currentmarker}{}%
\end{pgfscope}%
\end{pgfscope}%
\begin{pgfscope}%
\pgfpathrectangle{\pgfqpoint{0.581943in}{0.509170in}}{\pgfqpoint{5.599061in}{8.312590in}}%
\pgfusepath{clip}%
\pgfsetbuttcap%
\pgfsetroundjoin%
\pgfsetlinewidth{0.250937pt}%
\definecolor{currentstroke}{rgb}{0.680000,0.680000,0.680000}%
\pgfsetstrokecolor{currentstroke}%
\pgfsetdash{{1.000000pt}{1.000000pt}}{0.000000pt}%
\pgfpathmoveto{\pgfqpoint{5.629027in}{0.509170in}}%
\pgfpathlineto{\pgfqpoint{5.629027in}{8.821759in}}%
\pgfusepath{stroke}%
\end{pgfscope}%
\begin{pgfscope}%
\pgfsetbuttcap%
\pgfsetroundjoin%
\definecolor{currentfill}{rgb}{0.000000,0.000000,0.000000}%
\pgfsetfillcolor{currentfill}%
\pgfsetlinewidth{0.803000pt}%
\definecolor{currentstroke}{rgb}{0.000000,0.000000,0.000000}%
\pgfsetstrokecolor{currentstroke}%
\pgfsetdash{}{0pt}%
\pgfsys@defobject{currentmarker}{\pgfqpoint{0.000000in}{-0.048611in}}{\pgfqpoint{0.000000in}{0.000000in}}{%
\pgfpathmoveto{\pgfqpoint{0.000000in}{0.000000in}}%
\pgfpathlineto{\pgfqpoint{0.000000in}{-0.048611in}}%
\pgfusepath{stroke,fill}%
}%
\begin{pgfscope}%
\pgfsys@transformshift{5.629027in}{0.509170in}%
\pgfsys@useobject{currentmarker}{}%
\end{pgfscope}%
\end{pgfscope}%
\begin{pgfscope}%
\pgfpathrectangle{\pgfqpoint{0.581943in}{0.509170in}}{\pgfqpoint{5.599061in}{8.312590in}}%
\pgfusepath{clip}%
\pgfsetbuttcap%
\pgfsetroundjoin%
\pgfsetlinewidth{0.250937pt}%
\definecolor{currentstroke}{rgb}{0.680000,0.680000,0.680000}%
\pgfsetstrokecolor{currentstroke}%
\pgfsetdash{{1.000000pt}{1.000000pt}}{0.000000pt}%
\pgfpathmoveto{\pgfqpoint{5.670515in}{0.509170in}}%
\pgfpathlineto{\pgfqpoint{5.670515in}{8.821759in}}%
\pgfusepath{stroke}%
\end{pgfscope}%
\begin{pgfscope}%
\pgfsetbuttcap%
\pgfsetroundjoin%
\definecolor{currentfill}{rgb}{0.000000,0.000000,0.000000}%
\pgfsetfillcolor{currentfill}%
\pgfsetlinewidth{0.803000pt}%
\definecolor{currentstroke}{rgb}{0.000000,0.000000,0.000000}%
\pgfsetstrokecolor{currentstroke}%
\pgfsetdash{}{0pt}%
\pgfsys@defobject{currentmarker}{\pgfqpoint{0.000000in}{-0.048611in}}{\pgfqpoint{0.000000in}{0.000000in}}{%
\pgfpathmoveto{\pgfqpoint{0.000000in}{0.000000in}}%
\pgfpathlineto{\pgfqpoint{0.000000in}{-0.048611in}}%
\pgfusepath{stroke,fill}%
}%
\begin{pgfscope}%
\pgfsys@transformshift{5.670515in}{0.509170in}%
\pgfsys@useobject{currentmarker}{}%
\end{pgfscope}%
\end{pgfscope}%
\begin{pgfscope}%
\pgfpathrectangle{\pgfqpoint{0.581943in}{0.509170in}}{\pgfqpoint{5.599061in}{8.312590in}}%
\pgfusepath{clip}%
\pgfsetbuttcap%
\pgfsetroundjoin%
\pgfsetlinewidth{0.250937pt}%
\definecolor{currentstroke}{rgb}{0.680000,0.680000,0.680000}%
\pgfsetstrokecolor{currentstroke}%
\pgfsetdash{{1.000000pt}{1.000000pt}}{0.000000pt}%
\pgfpathmoveto{\pgfqpoint{5.710835in}{0.509170in}}%
\pgfpathlineto{\pgfqpoint{5.710835in}{8.821759in}}%
\pgfusepath{stroke}%
\end{pgfscope}%
\begin{pgfscope}%
\pgfsetbuttcap%
\pgfsetroundjoin%
\definecolor{currentfill}{rgb}{0.000000,0.000000,0.000000}%
\pgfsetfillcolor{currentfill}%
\pgfsetlinewidth{0.803000pt}%
\definecolor{currentstroke}{rgb}{0.000000,0.000000,0.000000}%
\pgfsetstrokecolor{currentstroke}%
\pgfsetdash{}{0pt}%
\pgfsys@defobject{currentmarker}{\pgfqpoint{0.000000in}{-0.048611in}}{\pgfqpoint{0.000000in}{0.000000in}}{%
\pgfpathmoveto{\pgfqpoint{0.000000in}{0.000000in}}%
\pgfpathlineto{\pgfqpoint{0.000000in}{-0.048611in}}%
\pgfusepath{stroke,fill}%
}%
\begin{pgfscope}%
\pgfsys@transformshift{5.710835in}{0.509170in}%
\pgfsys@useobject{currentmarker}{}%
\end{pgfscope}%
\end{pgfscope}%
\begin{pgfscope}%
\pgfpathrectangle{\pgfqpoint{0.581943in}{0.509170in}}{\pgfqpoint{5.599061in}{8.312590in}}%
\pgfusepath{clip}%
\pgfsetbuttcap%
\pgfsetroundjoin%
\pgfsetlinewidth{0.250937pt}%
\definecolor{currentstroke}{rgb}{0.680000,0.680000,0.680000}%
\pgfsetstrokecolor{currentstroke}%
\pgfsetdash{{1.000000pt}{1.000000pt}}{0.000000pt}%
\pgfpathmoveto{\pgfqpoint{5.750049in}{0.509170in}}%
\pgfpathlineto{\pgfqpoint{5.750049in}{8.821759in}}%
\pgfusepath{stroke}%
\end{pgfscope}%
\begin{pgfscope}%
\pgfsetbuttcap%
\pgfsetroundjoin%
\definecolor{currentfill}{rgb}{0.000000,0.000000,0.000000}%
\pgfsetfillcolor{currentfill}%
\pgfsetlinewidth{0.803000pt}%
\definecolor{currentstroke}{rgb}{0.000000,0.000000,0.000000}%
\pgfsetstrokecolor{currentstroke}%
\pgfsetdash{}{0pt}%
\pgfsys@defobject{currentmarker}{\pgfqpoint{0.000000in}{-0.048611in}}{\pgfqpoint{0.000000in}{0.000000in}}{%
\pgfpathmoveto{\pgfqpoint{0.000000in}{0.000000in}}%
\pgfpathlineto{\pgfqpoint{0.000000in}{-0.048611in}}%
\pgfusepath{stroke,fill}%
}%
\begin{pgfscope}%
\pgfsys@transformshift{5.750049in}{0.509170in}%
\pgfsys@useobject{currentmarker}{}%
\end{pgfscope}%
\end{pgfscope}%
\begin{pgfscope}%
\pgfpathrectangle{\pgfqpoint{0.581943in}{0.509170in}}{\pgfqpoint{5.599061in}{8.312590in}}%
\pgfusepath{clip}%
\pgfsetbuttcap%
\pgfsetroundjoin%
\pgfsetlinewidth{0.250937pt}%
\definecolor{currentstroke}{rgb}{0.680000,0.680000,0.680000}%
\pgfsetstrokecolor{currentstroke}%
\pgfsetdash{{1.000000pt}{1.000000pt}}{0.000000pt}%
\pgfpathmoveto{\pgfqpoint{5.788218in}{0.509170in}}%
\pgfpathlineto{\pgfqpoint{5.788218in}{8.821759in}}%
\pgfusepath{stroke}%
\end{pgfscope}%
\begin{pgfscope}%
\pgfsetbuttcap%
\pgfsetroundjoin%
\definecolor{currentfill}{rgb}{0.000000,0.000000,0.000000}%
\pgfsetfillcolor{currentfill}%
\pgfsetlinewidth{0.803000pt}%
\definecolor{currentstroke}{rgb}{0.000000,0.000000,0.000000}%
\pgfsetstrokecolor{currentstroke}%
\pgfsetdash{}{0pt}%
\pgfsys@defobject{currentmarker}{\pgfqpoint{0.000000in}{-0.048611in}}{\pgfqpoint{0.000000in}{0.000000in}}{%
\pgfpathmoveto{\pgfqpoint{0.000000in}{0.000000in}}%
\pgfpathlineto{\pgfqpoint{0.000000in}{-0.048611in}}%
\pgfusepath{stroke,fill}%
}%
\begin{pgfscope}%
\pgfsys@transformshift{5.788218in}{0.509170in}%
\pgfsys@useobject{currentmarker}{}%
\end{pgfscope}%
\end{pgfscope}%
\begin{pgfscope}%
\pgfpathrectangle{\pgfqpoint{0.581943in}{0.509170in}}{\pgfqpoint{5.599061in}{8.312590in}}%
\pgfusepath{clip}%
\pgfsetbuttcap%
\pgfsetroundjoin%
\pgfsetlinewidth{0.250937pt}%
\definecolor{currentstroke}{rgb}{0.680000,0.680000,0.680000}%
\pgfsetstrokecolor{currentstroke}%
\pgfsetdash{{1.000000pt}{1.000000pt}}{0.000000pt}%
\pgfpathmoveto{\pgfqpoint{5.825395in}{0.509170in}}%
\pgfpathlineto{\pgfqpoint{5.825395in}{8.821759in}}%
\pgfusepath{stroke}%
\end{pgfscope}%
\begin{pgfscope}%
\pgfsetbuttcap%
\pgfsetroundjoin%
\definecolor{currentfill}{rgb}{0.000000,0.000000,0.000000}%
\pgfsetfillcolor{currentfill}%
\pgfsetlinewidth{0.803000pt}%
\definecolor{currentstroke}{rgb}{0.000000,0.000000,0.000000}%
\pgfsetstrokecolor{currentstroke}%
\pgfsetdash{}{0pt}%
\pgfsys@defobject{currentmarker}{\pgfqpoint{0.000000in}{-0.048611in}}{\pgfqpoint{0.000000in}{0.000000in}}{%
\pgfpathmoveto{\pgfqpoint{0.000000in}{0.000000in}}%
\pgfpathlineto{\pgfqpoint{0.000000in}{-0.048611in}}%
\pgfusepath{stroke,fill}%
}%
\begin{pgfscope}%
\pgfsys@transformshift{5.825395in}{0.509170in}%
\pgfsys@useobject{currentmarker}{}%
\end{pgfscope}%
\end{pgfscope}%
\begin{pgfscope}%
\pgfpathrectangle{\pgfqpoint{0.581943in}{0.509170in}}{\pgfqpoint{5.599061in}{8.312590in}}%
\pgfusepath{clip}%
\pgfsetbuttcap%
\pgfsetroundjoin%
\pgfsetlinewidth{0.250937pt}%
\definecolor{currentstroke}{rgb}{0.680000,0.680000,0.680000}%
\pgfsetstrokecolor{currentstroke}%
\pgfsetdash{{1.000000pt}{1.000000pt}}{0.000000pt}%
\pgfpathmoveto{\pgfqpoint{5.861631in}{0.509170in}}%
\pgfpathlineto{\pgfqpoint{5.861631in}{8.821759in}}%
\pgfusepath{stroke}%
\end{pgfscope}%
\begin{pgfscope}%
\pgfsetbuttcap%
\pgfsetroundjoin%
\definecolor{currentfill}{rgb}{0.000000,0.000000,0.000000}%
\pgfsetfillcolor{currentfill}%
\pgfsetlinewidth{0.803000pt}%
\definecolor{currentstroke}{rgb}{0.000000,0.000000,0.000000}%
\pgfsetstrokecolor{currentstroke}%
\pgfsetdash{}{0pt}%
\pgfsys@defobject{currentmarker}{\pgfqpoint{0.000000in}{-0.048611in}}{\pgfqpoint{0.000000in}{0.000000in}}{%
\pgfpathmoveto{\pgfqpoint{0.000000in}{0.000000in}}%
\pgfpathlineto{\pgfqpoint{0.000000in}{-0.048611in}}%
\pgfusepath{stroke,fill}%
}%
\begin{pgfscope}%
\pgfsys@transformshift{5.861631in}{0.509170in}%
\pgfsys@useobject{currentmarker}{}%
\end{pgfscope}%
\end{pgfscope}%
\begin{pgfscope}%
\definecolor{textcolor}{rgb}{0.000000,0.000000,0.000000}%
\pgfsetstrokecolor{textcolor}%
\pgfsetfillcolor{textcolor}%
\pgftext[x=5.861631in,y=0.411948in,,top]{\color{textcolor}\rmfamily\fontsize{9.000000}{10.800000}\selectfont 40}%
\end{pgfscope}%
\begin{pgfscope}%
\pgfpathrectangle{\pgfqpoint{0.581943in}{0.509170in}}{\pgfqpoint{5.599061in}{8.312590in}}%
\pgfusepath{clip}%
\pgfsetbuttcap%
\pgfsetroundjoin%
\pgfsetlinewidth{0.250937pt}%
\definecolor{currentstroke}{rgb}{0.680000,0.680000,0.680000}%
\pgfsetstrokecolor{currentstroke}%
\pgfsetdash{{1.000000pt}{1.000000pt}}{0.000000pt}%
\pgfpathmoveto{\pgfqpoint{5.896972in}{0.509170in}}%
\pgfpathlineto{\pgfqpoint{5.896972in}{8.821759in}}%
\pgfusepath{stroke}%
\end{pgfscope}%
\begin{pgfscope}%
\pgfsetbuttcap%
\pgfsetroundjoin%
\definecolor{currentfill}{rgb}{0.000000,0.000000,0.000000}%
\pgfsetfillcolor{currentfill}%
\pgfsetlinewidth{0.803000pt}%
\definecolor{currentstroke}{rgb}{0.000000,0.000000,0.000000}%
\pgfsetstrokecolor{currentstroke}%
\pgfsetdash{}{0pt}%
\pgfsys@defobject{currentmarker}{\pgfqpoint{0.000000in}{-0.048611in}}{\pgfqpoint{0.000000in}{0.000000in}}{%
\pgfpathmoveto{\pgfqpoint{0.000000in}{0.000000in}}%
\pgfpathlineto{\pgfqpoint{0.000000in}{-0.048611in}}%
\pgfusepath{stroke,fill}%
}%
\begin{pgfscope}%
\pgfsys@transformshift{5.896972in}{0.509170in}%
\pgfsys@useobject{currentmarker}{}%
\end{pgfscope}%
\end{pgfscope}%
\begin{pgfscope}%
\pgfpathrectangle{\pgfqpoint{0.581943in}{0.509170in}}{\pgfqpoint{5.599061in}{8.312590in}}%
\pgfusepath{clip}%
\pgfsetbuttcap%
\pgfsetroundjoin%
\pgfsetlinewidth{0.250937pt}%
\definecolor{currentstroke}{rgb}{0.680000,0.680000,0.680000}%
\pgfsetstrokecolor{currentstroke}%
\pgfsetdash{{1.000000pt}{1.000000pt}}{0.000000pt}%
\pgfpathmoveto{\pgfqpoint{5.931462in}{0.509170in}}%
\pgfpathlineto{\pgfqpoint{5.931462in}{8.821759in}}%
\pgfusepath{stroke}%
\end{pgfscope}%
\begin{pgfscope}%
\pgfsetbuttcap%
\pgfsetroundjoin%
\definecolor{currentfill}{rgb}{0.000000,0.000000,0.000000}%
\pgfsetfillcolor{currentfill}%
\pgfsetlinewidth{0.803000pt}%
\definecolor{currentstroke}{rgb}{0.000000,0.000000,0.000000}%
\pgfsetstrokecolor{currentstroke}%
\pgfsetdash{}{0pt}%
\pgfsys@defobject{currentmarker}{\pgfqpoint{0.000000in}{-0.048611in}}{\pgfqpoint{0.000000in}{0.000000in}}{%
\pgfpathmoveto{\pgfqpoint{0.000000in}{0.000000in}}%
\pgfpathlineto{\pgfqpoint{0.000000in}{-0.048611in}}%
\pgfusepath{stroke,fill}%
}%
\begin{pgfscope}%
\pgfsys@transformshift{5.931462in}{0.509170in}%
\pgfsys@useobject{currentmarker}{}%
\end{pgfscope}%
\end{pgfscope}%
\begin{pgfscope}%
\pgfpathrectangle{\pgfqpoint{0.581943in}{0.509170in}}{\pgfqpoint{5.599061in}{8.312590in}}%
\pgfusepath{clip}%
\pgfsetbuttcap%
\pgfsetroundjoin%
\pgfsetlinewidth{0.250937pt}%
\definecolor{currentstroke}{rgb}{0.680000,0.680000,0.680000}%
\pgfsetstrokecolor{currentstroke}%
\pgfsetdash{{1.000000pt}{1.000000pt}}{0.000000pt}%
\pgfpathmoveto{\pgfqpoint{5.965140in}{0.509170in}}%
\pgfpathlineto{\pgfqpoint{5.965140in}{8.821759in}}%
\pgfusepath{stroke}%
\end{pgfscope}%
\begin{pgfscope}%
\pgfsetbuttcap%
\pgfsetroundjoin%
\definecolor{currentfill}{rgb}{0.000000,0.000000,0.000000}%
\pgfsetfillcolor{currentfill}%
\pgfsetlinewidth{0.803000pt}%
\definecolor{currentstroke}{rgb}{0.000000,0.000000,0.000000}%
\pgfsetstrokecolor{currentstroke}%
\pgfsetdash{}{0pt}%
\pgfsys@defobject{currentmarker}{\pgfqpoint{0.000000in}{-0.048611in}}{\pgfqpoint{0.000000in}{0.000000in}}{%
\pgfpathmoveto{\pgfqpoint{0.000000in}{0.000000in}}%
\pgfpathlineto{\pgfqpoint{0.000000in}{-0.048611in}}%
\pgfusepath{stroke,fill}%
}%
\begin{pgfscope}%
\pgfsys@transformshift{5.965140in}{0.509170in}%
\pgfsys@useobject{currentmarker}{}%
\end{pgfscope}%
\end{pgfscope}%
\begin{pgfscope}%
\pgfpathrectangle{\pgfqpoint{0.581943in}{0.509170in}}{\pgfqpoint{5.599061in}{8.312590in}}%
\pgfusepath{clip}%
\pgfsetbuttcap%
\pgfsetroundjoin%
\pgfsetlinewidth{0.250937pt}%
\definecolor{currentstroke}{rgb}{0.680000,0.680000,0.680000}%
\pgfsetstrokecolor{currentstroke}%
\pgfsetdash{{1.000000pt}{1.000000pt}}{0.000000pt}%
\pgfpathmoveto{\pgfqpoint{5.998043in}{0.509170in}}%
\pgfpathlineto{\pgfqpoint{5.998043in}{8.821759in}}%
\pgfusepath{stroke}%
\end{pgfscope}%
\begin{pgfscope}%
\pgfsetbuttcap%
\pgfsetroundjoin%
\definecolor{currentfill}{rgb}{0.000000,0.000000,0.000000}%
\pgfsetfillcolor{currentfill}%
\pgfsetlinewidth{0.803000pt}%
\definecolor{currentstroke}{rgb}{0.000000,0.000000,0.000000}%
\pgfsetstrokecolor{currentstroke}%
\pgfsetdash{}{0pt}%
\pgfsys@defobject{currentmarker}{\pgfqpoint{0.000000in}{-0.048611in}}{\pgfqpoint{0.000000in}{0.000000in}}{%
\pgfpathmoveto{\pgfqpoint{0.000000in}{0.000000in}}%
\pgfpathlineto{\pgfqpoint{0.000000in}{-0.048611in}}%
\pgfusepath{stroke,fill}%
}%
\begin{pgfscope}%
\pgfsys@transformshift{5.998043in}{0.509170in}%
\pgfsys@useobject{currentmarker}{}%
\end{pgfscope}%
\end{pgfscope}%
\begin{pgfscope}%
\pgfpathrectangle{\pgfqpoint{0.581943in}{0.509170in}}{\pgfqpoint{5.599061in}{8.312590in}}%
\pgfusepath{clip}%
\pgfsetbuttcap%
\pgfsetroundjoin%
\pgfsetlinewidth{0.250937pt}%
\definecolor{currentstroke}{rgb}{0.680000,0.680000,0.680000}%
\pgfsetstrokecolor{currentstroke}%
\pgfsetdash{{1.000000pt}{1.000000pt}}{0.000000pt}%
\pgfpathmoveto{\pgfqpoint{6.030208in}{0.509170in}}%
\pgfpathlineto{\pgfqpoint{6.030208in}{8.821759in}}%
\pgfusepath{stroke}%
\end{pgfscope}%
\begin{pgfscope}%
\pgfsetbuttcap%
\pgfsetroundjoin%
\definecolor{currentfill}{rgb}{0.000000,0.000000,0.000000}%
\pgfsetfillcolor{currentfill}%
\pgfsetlinewidth{0.803000pt}%
\definecolor{currentstroke}{rgb}{0.000000,0.000000,0.000000}%
\pgfsetstrokecolor{currentstroke}%
\pgfsetdash{}{0pt}%
\pgfsys@defobject{currentmarker}{\pgfqpoint{0.000000in}{-0.048611in}}{\pgfqpoint{0.000000in}{0.000000in}}{%
\pgfpathmoveto{\pgfqpoint{0.000000in}{0.000000in}}%
\pgfpathlineto{\pgfqpoint{0.000000in}{-0.048611in}}%
\pgfusepath{stroke,fill}%
}%
\begin{pgfscope}%
\pgfsys@transformshift{6.030208in}{0.509170in}%
\pgfsys@useobject{currentmarker}{}%
\end{pgfscope}%
\end{pgfscope}%
\begin{pgfscope}%
\pgfpathrectangle{\pgfqpoint{0.581943in}{0.509170in}}{\pgfqpoint{5.599061in}{8.312590in}}%
\pgfusepath{clip}%
\pgfsetbuttcap%
\pgfsetroundjoin%
\pgfsetlinewidth{0.250937pt}%
\definecolor{currentstroke}{rgb}{0.680000,0.680000,0.680000}%
\pgfsetstrokecolor{currentstroke}%
\pgfsetdash{{1.000000pt}{1.000000pt}}{0.000000pt}%
\pgfpathmoveto{\pgfqpoint{6.061665in}{0.509170in}}%
\pgfpathlineto{\pgfqpoint{6.061665in}{8.821759in}}%
\pgfusepath{stroke}%
\end{pgfscope}%
\begin{pgfscope}%
\pgfsetbuttcap%
\pgfsetroundjoin%
\definecolor{currentfill}{rgb}{0.000000,0.000000,0.000000}%
\pgfsetfillcolor{currentfill}%
\pgfsetlinewidth{0.803000pt}%
\definecolor{currentstroke}{rgb}{0.000000,0.000000,0.000000}%
\pgfsetstrokecolor{currentstroke}%
\pgfsetdash{}{0pt}%
\pgfsys@defobject{currentmarker}{\pgfqpoint{0.000000in}{-0.048611in}}{\pgfqpoint{0.000000in}{0.000000in}}{%
\pgfpathmoveto{\pgfqpoint{0.000000in}{0.000000in}}%
\pgfpathlineto{\pgfqpoint{0.000000in}{-0.048611in}}%
\pgfusepath{stroke,fill}%
}%
\begin{pgfscope}%
\pgfsys@transformshift{6.061665in}{0.509170in}%
\pgfsys@useobject{currentmarker}{}%
\end{pgfscope}%
\end{pgfscope}%
\begin{pgfscope}%
\pgfpathrectangle{\pgfqpoint{0.581943in}{0.509170in}}{\pgfqpoint{5.599061in}{8.312590in}}%
\pgfusepath{clip}%
\pgfsetbuttcap%
\pgfsetroundjoin%
\pgfsetlinewidth{0.250937pt}%
\definecolor{currentstroke}{rgb}{0.680000,0.680000,0.680000}%
\pgfsetstrokecolor{currentstroke}%
\pgfsetdash{{1.000000pt}{1.000000pt}}{0.000000pt}%
\pgfpathmoveto{\pgfqpoint{6.092445in}{0.509170in}}%
\pgfpathlineto{\pgfqpoint{6.092445in}{8.821759in}}%
\pgfusepath{stroke}%
\end{pgfscope}%
\begin{pgfscope}%
\pgfsetbuttcap%
\pgfsetroundjoin%
\definecolor{currentfill}{rgb}{0.000000,0.000000,0.000000}%
\pgfsetfillcolor{currentfill}%
\pgfsetlinewidth{0.803000pt}%
\definecolor{currentstroke}{rgb}{0.000000,0.000000,0.000000}%
\pgfsetstrokecolor{currentstroke}%
\pgfsetdash{}{0pt}%
\pgfsys@defobject{currentmarker}{\pgfqpoint{0.000000in}{-0.048611in}}{\pgfqpoint{0.000000in}{0.000000in}}{%
\pgfpathmoveto{\pgfqpoint{0.000000in}{0.000000in}}%
\pgfpathlineto{\pgfqpoint{0.000000in}{-0.048611in}}%
\pgfusepath{stroke,fill}%
}%
\begin{pgfscope}%
\pgfsys@transformshift{6.092445in}{0.509170in}%
\pgfsys@useobject{currentmarker}{}%
\end{pgfscope}%
\end{pgfscope}%
\begin{pgfscope}%
\pgfpathrectangle{\pgfqpoint{0.581943in}{0.509170in}}{\pgfqpoint{5.599061in}{8.312590in}}%
\pgfusepath{clip}%
\pgfsetbuttcap%
\pgfsetroundjoin%
\pgfsetlinewidth{0.250937pt}%
\definecolor{currentstroke}{rgb}{0.680000,0.680000,0.680000}%
\pgfsetstrokecolor{currentstroke}%
\pgfsetdash{{1.000000pt}{1.000000pt}}{0.000000pt}%
\pgfpathmoveto{\pgfqpoint{6.122578in}{0.509170in}}%
\pgfpathlineto{\pgfqpoint{6.122578in}{8.821759in}}%
\pgfusepath{stroke}%
\end{pgfscope}%
\begin{pgfscope}%
\pgfsetbuttcap%
\pgfsetroundjoin%
\definecolor{currentfill}{rgb}{0.000000,0.000000,0.000000}%
\pgfsetfillcolor{currentfill}%
\pgfsetlinewidth{0.803000pt}%
\definecolor{currentstroke}{rgb}{0.000000,0.000000,0.000000}%
\pgfsetstrokecolor{currentstroke}%
\pgfsetdash{}{0pt}%
\pgfsys@defobject{currentmarker}{\pgfqpoint{0.000000in}{-0.048611in}}{\pgfqpoint{0.000000in}{0.000000in}}{%
\pgfpathmoveto{\pgfqpoint{0.000000in}{0.000000in}}%
\pgfpathlineto{\pgfqpoint{0.000000in}{-0.048611in}}%
\pgfusepath{stroke,fill}%
}%
\begin{pgfscope}%
\pgfsys@transformshift{6.122578in}{0.509170in}%
\pgfsys@useobject{currentmarker}{}%
\end{pgfscope}%
\end{pgfscope}%
\begin{pgfscope}%
\pgfpathrectangle{\pgfqpoint{0.581943in}{0.509170in}}{\pgfqpoint{5.599061in}{8.312590in}}%
\pgfusepath{clip}%
\pgfsetbuttcap%
\pgfsetroundjoin%
\pgfsetlinewidth{0.250937pt}%
\definecolor{currentstroke}{rgb}{0.680000,0.680000,0.680000}%
\pgfsetstrokecolor{currentstroke}%
\pgfsetdash{{1.000000pt}{1.000000pt}}{0.000000pt}%
\pgfpathmoveto{\pgfqpoint{6.152089in}{0.509170in}}%
\pgfpathlineto{\pgfqpoint{6.152089in}{8.821759in}}%
\pgfusepath{stroke}%
\end{pgfscope}%
\begin{pgfscope}%
\pgfsetbuttcap%
\pgfsetroundjoin%
\definecolor{currentfill}{rgb}{0.000000,0.000000,0.000000}%
\pgfsetfillcolor{currentfill}%
\pgfsetlinewidth{0.803000pt}%
\definecolor{currentstroke}{rgb}{0.000000,0.000000,0.000000}%
\pgfsetstrokecolor{currentstroke}%
\pgfsetdash{}{0pt}%
\pgfsys@defobject{currentmarker}{\pgfqpoint{0.000000in}{-0.048611in}}{\pgfqpoint{0.000000in}{0.000000in}}{%
\pgfpathmoveto{\pgfqpoint{0.000000in}{0.000000in}}%
\pgfpathlineto{\pgfqpoint{0.000000in}{-0.048611in}}%
\pgfusepath{stroke,fill}%
}%
\begin{pgfscope}%
\pgfsys@transformshift{6.152089in}{0.509170in}%
\pgfsys@useobject{currentmarker}{}%
\end{pgfscope}%
\end{pgfscope}%
\begin{pgfscope}%
\pgfpathrectangle{\pgfqpoint{0.581943in}{0.509170in}}{\pgfqpoint{5.599061in}{8.312590in}}%
\pgfusepath{clip}%
\pgfsetbuttcap%
\pgfsetroundjoin%
\pgfsetlinewidth{0.250937pt}%
\definecolor{currentstroke}{rgb}{0.680000,0.680000,0.680000}%
\pgfsetstrokecolor{currentstroke}%
\pgfsetdash{{1.000000pt}{1.000000pt}}{0.000000pt}%
\pgfpathmoveto{\pgfqpoint{6.181004in}{0.509170in}}%
\pgfpathlineto{\pgfqpoint{6.181004in}{8.821759in}}%
\pgfusepath{stroke}%
\end{pgfscope}%
\begin{pgfscope}%
\pgfsetbuttcap%
\pgfsetroundjoin%
\definecolor{currentfill}{rgb}{0.000000,0.000000,0.000000}%
\pgfsetfillcolor{currentfill}%
\pgfsetlinewidth{0.803000pt}%
\definecolor{currentstroke}{rgb}{0.000000,0.000000,0.000000}%
\pgfsetstrokecolor{currentstroke}%
\pgfsetdash{}{0pt}%
\pgfsys@defobject{currentmarker}{\pgfqpoint{0.000000in}{-0.048611in}}{\pgfqpoint{0.000000in}{0.000000in}}{%
\pgfpathmoveto{\pgfqpoint{0.000000in}{0.000000in}}%
\pgfpathlineto{\pgfqpoint{0.000000in}{-0.048611in}}%
\pgfusepath{stroke,fill}%
}%
\begin{pgfscope}%
\pgfsys@transformshift{6.181004in}{0.509170in}%
\pgfsys@useobject{currentmarker}{}%
\end{pgfscope}%
\end{pgfscope}%
\begin{pgfscope}%
\definecolor{textcolor}{rgb}{0.000000,0.000000,0.000000}%
\pgfsetstrokecolor{textcolor}%
\pgfsetfillcolor{textcolor}%
\pgftext[x=6.181004in,y=0.411948in,,top]{\color{textcolor}\rmfamily\fontsize{9.000000}{10.800000}\selectfont 50}%
\end{pgfscope}%
\begin{pgfscope}%
\definecolor{textcolor}{rgb}{0.000000,0.000000,0.000000}%
\pgfsetstrokecolor{textcolor}%
\pgfsetfillcolor{textcolor}%
\pgftext[x=3.381474in,y=0.245281in,,top]{\color{textcolor}\rmfamily\fontsize{9.000000}{10.800000}\selectfont \(\displaystyle \nu\)}%
\end{pgfscope}%
\begin{pgfscope}%
\pgfpathrectangle{\pgfqpoint{0.581943in}{0.509170in}}{\pgfqpoint{5.599061in}{8.312590in}}%
\pgfusepath{clip}%
\pgfsetbuttcap%
\pgfsetroundjoin%
\pgfsetlinewidth{0.250937pt}%
\definecolor{currentstroke}{rgb}{0.680000,0.680000,0.680000}%
\pgfsetstrokecolor{currentstroke}%
\pgfsetdash{{1.000000pt}{1.000000pt}}{0.000000pt}%
\pgfpathmoveto{\pgfqpoint{0.581943in}{0.509170in}}%
\pgfpathlineto{\pgfqpoint{6.181004in}{0.509170in}}%
\pgfusepath{stroke}%
\end{pgfscope}%
\begin{pgfscope}%
\pgfsetbuttcap%
\pgfsetroundjoin%
\definecolor{currentfill}{rgb}{0.000000,0.000000,0.000000}%
\pgfsetfillcolor{currentfill}%
\pgfsetlinewidth{0.803000pt}%
\definecolor{currentstroke}{rgb}{0.000000,0.000000,0.000000}%
\pgfsetstrokecolor{currentstroke}%
\pgfsetdash{}{0pt}%
\pgfsys@defobject{currentmarker}{\pgfqpoint{-0.048611in}{0.000000in}}{\pgfqpoint{-0.000000in}{0.000000in}}{%
\pgfpathmoveto{\pgfqpoint{-0.000000in}{0.000000in}}%
\pgfpathlineto{\pgfqpoint{-0.048611in}{0.000000in}}%
\pgfusepath{stroke,fill}%
}%
\begin{pgfscope}%
\pgfsys@transformshift{0.581943in}{0.509170in}%
\pgfsys@useobject{currentmarker}{}%
\end{pgfscope}%
\end{pgfscope}%
\begin{pgfscope}%
\definecolor{textcolor}{rgb}{0.000000,0.000000,0.000000}%
\pgfsetstrokecolor{textcolor}%
\pgfsetfillcolor{textcolor}%
\pgftext[x=0.320563in, y=0.465767in, left, base]{\color{textcolor}\rmfamily\fontsize{9.000000}{10.800000}\selectfont \(\displaystyle {0.0}\)}%
\end{pgfscope}%
\begin{pgfscope}%
\pgfpathrectangle{\pgfqpoint{0.581943in}{0.509170in}}{\pgfqpoint{5.599061in}{8.312590in}}%
\pgfusepath{clip}%
\pgfsetbuttcap%
\pgfsetroundjoin%
\pgfsetlinewidth{0.250937pt}%
\definecolor{currentstroke}{rgb}{0.680000,0.680000,0.680000}%
\pgfsetstrokecolor{currentstroke}%
\pgfsetdash{{1.000000pt}{1.000000pt}}{0.000000pt}%
\pgfpathmoveto{\pgfqpoint{0.581943in}{0.786256in}}%
\pgfpathlineto{\pgfqpoint{6.181004in}{0.786256in}}%
\pgfusepath{stroke}%
\end{pgfscope}%
\begin{pgfscope}%
\pgfsetbuttcap%
\pgfsetroundjoin%
\definecolor{currentfill}{rgb}{0.000000,0.000000,0.000000}%
\pgfsetfillcolor{currentfill}%
\pgfsetlinewidth{0.803000pt}%
\definecolor{currentstroke}{rgb}{0.000000,0.000000,0.000000}%
\pgfsetstrokecolor{currentstroke}%
\pgfsetdash{}{0pt}%
\pgfsys@defobject{currentmarker}{\pgfqpoint{-0.048611in}{0.000000in}}{\pgfqpoint{-0.000000in}{0.000000in}}{%
\pgfpathmoveto{\pgfqpoint{-0.000000in}{0.000000in}}%
\pgfpathlineto{\pgfqpoint{-0.048611in}{0.000000in}}%
\pgfusepath{stroke,fill}%
}%
\begin{pgfscope}%
\pgfsys@transformshift{0.581943in}{0.786256in}%
\pgfsys@useobject{currentmarker}{}%
\end{pgfscope}%
\end{pgfscope}%
\begin{pgfscope}%
\definecolor{textcolor}{rgb}{0.000000,0.000000,0.000000}%
\pgfsetstrokecolor{textcolor}%
\pgfsetfillcolor{textcolor}%
\pgftext[x=0.320563in, y=0.742853in, left, base]{\color{textcolor}\rmfamily\fontsize{9.000000}{10.800000}\selectfont \(\displaystyle {0.1}\)}%
\end{pgfscope}%
\begin{pgfscope}%
\pgfpathrectangle{\pgfqpoint{0.581943in}{0.509170in}}{\pgfqpoint{5.599061in}{8.312590in}}%
\pgfusepath{clip}%
\pgfsetbuttcap%
\pgfsetroundjoin%
\pgfsetlinewidth{0.250937pt}%
\definecolor{currentstroke}{rgb}{0.680000,0.680000,0.680000}%
\pgfsetstrokecolor{currentstroke}%
\pgfsetdash{{1.000000pt}{1.000000pt}}{0.000000pt}%
\pgfpathmoveto{\pgfqpoint{0.581943in}{1.063342in}}%
\pgfpathlineto{\pgfqpoint{6.181004in}{1.063342in}}%
\pgfusepath{stroke}%
\end{pgfscope}%
\begin{pgfscope}%
\pgfsetbuttcap%
\pgfsetroundjoin%
\definecolor{currentfill}{rgb}{0.000000,0.000000,0.000000}%
\pgfsetfillcolor{currentfill}%
\pgfsetlinewidth{0.803000pt}%
\definecolor{currentstroke}{rgb}{0.000000,0.000000,0.000000}%
\pgfsetstrokecolor{currentstroke}%
\pgfsetdash{}{0pt}%
\pgfsys@defobject{currentmarker}{\pgfqpoint{-0.048611in}{0.000000in}}{\pgfqpoint{-0.000000in}{0.000000in}}{%
\pgfpathmoveto{\pgfqpoint{-0.000000in}{0.000000in}}%
\pgfpathlineto{\pgfqpoint{-0.048611in}{0.000000in}}%
\pgfusepath{stroke,fill}%
}%
\begin{pgfscope}%
\pgfsys@transformshift{0.581943in}{1.063342in}%
\pgfsys@useobject{currentmarker}{}%
\end{pgfscope}%
\end{pgfscope}%
\begin{pgfscope}%
\definecolor{textcolor}{rgb}{0.000000,0.000000,0.000000}%
\pgfsetstrokecolor{textcolor}%
\pgfsetfillcolor{textcolor}%
\pgftext[x=0.320563in, y=1.019940in, left, base]{\color{textcolor}\rmfamily\fontsize{9.000000}{10.800000}\selectfont \(\displaystyle {0.2}\)}%
\end{pgfscope}%
\begin{pgfscope}%
\pgfpathrectangle{\pgfqpoint{0.581943in}{0.509170in}}{\pgfqpoint{5.599061in}{8.312590in}}%
\pgfusepath{clip}%
\pgfsetbuttcap%
\pgfsetroundjoin%
\pgfsetlinewidth{0.250937pt}%
\definecolor{currentstroke}{rgb}{0.680000,0.680000,0.680000}%
\pgfsetstrokecolor{currentstroke}%
\pgfsetdash{{1.000000pt}{1.000000pt}}{0.000000pt}%
\pgfpathmoveto{\pgfqpoint{0.581943in}{1.340429in}}%
\pgfpathlineto{\pgfqpoint{6.181004in}{1.340429in}}%
\pgfusepath{stroke}%
\end{pgfscope}%
\begin{pgfscope}%
\pgfsetbuttcap%
\pgfsetroundjoin%
\definecolor{currentfill}{rgb}{0.000000,0.000000,0.000000}%
\pgfsetfillcolor{currentfill}%
\pgfsetlinewidth{0.803000pt}%
\definecolor{currentstroke}{rgb}{0.000000,0.000000,0.000000}%
\pgfsetstrokecolor{currentstroke}%
\pgfsetdash{}{0pt}%
\pgfsys@defobject{currentmarker}{\pgfqpoint{-0.048611in}{0.000000in}}{\pgfqpoint{-0.000000in}{0.000000in}}{%
\pgfpathmoveto{\pgfqpoint{-0.000000in}{0.000000in}}%
\pgfpathlineto{\pgfqpoint{-0.048611in}{0.000000in}}%
\pgfusepath{stroke,fill}%
}%
\begin{pgfscope}%
\pgfsys@transformshift{0.581943in}{1.340429in}%
\pgfsys@useobject{currentmarker}{}%
\end{pgfscope}%
\end{pgfscope}%
\begin{pgfscope}%
\definecolor{textcolor}{rgb}{0.000000,0.000000,0.000000}%
\pgfsetstrokecolor{textcolor}%
\pgfsetfillcolor{textcolor}%
\pgftext[x=0.320563in, y=1.297026in, left, base]{\color{textcolor}\rmfamily\fontsize{9.000000}{10.800000}\selectfont \(\displaystyle {0.3}\)}%
\end{pgfscope}%
\begin{pgfscope}%
\pgfpathrectangle{\pgfqpoint{0.581943in}{0.509170in}}{\pgfqpoint{5.599061in}{8.312590in}}%
\pgfusepath{clip}%
\pgfsetbuttcap%
\pgfsetroundjoin%
\pgfsetlinewidth{0.250937pt}%
\definecolor{currentstroke}{rgb}{0.680000,0.680000,0.680000}%
\pgfsetstrokecolor{currentstroke}%
\pgfsetdash{{1.000000pt}{1.000000pt}}{0.000000pt}%
\pgfpathmoveto{\pgfqpoint{0.581943in}{1.617515in}}%
\pgfpathlineto{\pgfqpoint{6.181004in}{1.617515in}}%
\pgfusepath{stroke}%
\end{pgfscope}%
\begin{pgfscope}%
\pgfsetbuttcap%
\pgfsetroundjoin%
\definecolor{currentfill}{rgb}{0.000000,0.000000,0.000000}%
\pgfsetfillcolor{currentfill}%
\pgfsetlinewidth{0.803000pt}%
\definecolor{currentstroke}{rgb}{0.000000,0.000000,0.000000}%
\pgfsetstrokecolor{currentstroke}%
\pgfsetdash{}{0pt}%
\pgfsys@defobject{currentmarker}{\pgfqpoint{-0.048611in}{0.000000in}}{\pgfqpoint{-0.000000in}{0.000000in}}{%
\pgfpathmoveto{\pgfqpoint{-0.000000in}{0.000000in}}%
\pgfpathlineto{\pgfqpoint{-0.048611in}{0.000000in}}%
\pgfusepath{stroke,fill}%
}%
\begin{pgfscope}%
\pgfsys@transformshift{0.581943in}{1.617515in}%
\pgfsys@useobject{currentmarker}{}%
\end{pgfscope}%
\end{pgfscope}%
\begin{pgfscope}%
\definecolor{textcolor}{rgb}{0.000000,0.000000,0.000000}%
\pgfsetstrokecolor{textcolor}%
\pgfsetfillcolor{textcolor}%
\pgftext[x=0.320563in, y=1.574112in, left, base]{\color{textcolor}\rmfamily\fontsize{9.000000}{10.800000}\selectfont \(\displaystyle {0.4}\)}%
\end{pgfscope}%
\begin{pgfscope}%
\pgfpathrectangle{\pgfqpoint{0.581943in}{0.509170in}}{\pgfqpoint{5.599061in}{8.312590in}}%
\pgfusepath{clip}%
\pgfsetbuttcap%
\pgfsetroundjoin%
\pgfsetlinewidth{0.250937pt}%
\definecolor{currentstroke}{rgb}{0.680000,0.680000,0.680000}%
\pgfsetstrokecolor{currentstroke}%
\pgfsetdash{{1.000000pt}{1.000000pt}}{0.000000pt}%
\pgfpathmoveto{\pgfqpoint{0.581943in}{1.894601in}}%
\pgfpathlineto{\pgfqpoint{6.181004in}{1.894601in}}%
\pgfusepath{stroke}%
\end{pgfscope}%
\begin{pgfscope}%
\pgfsetbuttcap%
\pgfsetroundjoin%
\definecolor{currentfill}{rgb}{0.000000,0.000000,0.000000}%
\pgfsetfillcolor{currentfill}%
\pgfsetlinewidth{0.803000pt}%
\definecolor{currentstroke}{rgb}{0.000000,0.000000,0.000000}%
\pgfsetstrokecolor{currentstroke}%
\pgfsetdash{}{0pt}%
\pgfsys@defobject{currentmarker}{\pgfqpoint{-0.048611in}{0.000000in}}{\pgfqpoint{-0.000000in}{0.000000in}}{%
\pgfpathmoveto{\pgfqpoint{-0.000000in}{0.000000in}}%
\pgfpathlineto{\pgfqpoint{-0.048611in}{0.000000in}}%
\pgfusepath{stroke,fill}%
}%
\begin{pgfscope}%
\pgfsys@transformshift{0.581943in}{1.894601in}%
\pgfsys@useobject{currentmarker}{}%
\end{pgfscope}%
\end{pgfscope}%
\begin{pgfscope}%
\definecolor{textcolor}{rgb}{0.000000,0.000000,0.000000}%
\pgfsetstrokecolor{textcolor}%
\pgfsetfillcolor{textcolor}%
\pgftext[x=0.320563in, y=1.851199in, left, base]{\color{textcolor}\rmfamily\fontsize{9.000000}{10.800000}\selectfont \(\displaystyle {0.5}\)}%
\end{pgfscope}%
\begin{pgfscope}%
\pgfpathrectangle{\pgfqpoint{0.581943in}{0.509170in}}{\pgfqpoint{5.599061in}{8.312590in}}%
\pgfusepath{clip}%
\pgfsetbuttcap%
\pgfsetroundjoin%
\pgfsetlinewidth{0.250937pt}%
\definecolor{currentstroke}{rgb}{0.680000,0.680000,0.680000}%
\pgfsetstrokecolor{currentstroke}%
\pgfsetdash{{1.000000pt}{1.000000pt}}{0.000000pt}%
\pgfpathmoveto{\pgfqpoint{0.581943in}{2.171688in}}%
\pgfpathlineto{\pgfqpoint{6.181004in}{2.171688in}}%
\pgfusepath{stroke}%
\end{pgfscope}%
\begin{pgfscope}%
\pgfsetbuttcap%
\pgfsetroundjoin%
\definecolor{currentfill}{rgb}{0.000000,0.000000,0.000000}%
\pgfsetfillcolor{currentfill}%
\pgfsetlinewidth{0.803000pt}%
\definecolor{currentstroke}{rgb}{0.000000,0.000000,0.000000}%
\pgfsetstrokecolor{currentstroke}%
\pgfsetdash{}{0pt}%
\pgfsys@defobject{currentmarker}{\pgfqpoint{-0.048611in}{0.000000in}}{\pgfqpoint{-0.000000in}{0.000000in}}{%
\pgfpathmoveto{\pgfqpoint{-0.000000in}{0.000000in}}%
\pgfpathlineto{\pgfqpoint{-0.048611in}{0.000000in}}%
\pgfusepath{stroke,fill}%
}%
\begin{pgfscope}%
\pgfsys@transformshift{0.581943in}{2.171688in}%
\pgfsys@useobject{currentmarker}{}%
\end{pgfscope}%
\end{pgfscope}%
\begin{pgfscope}%
\definecolor{textcolor}{rgb}{0.000000,0.000000,0.000000}%
\pgfsetstrokecolor{textcolor}%
\pgfsetfillcolor{textcolor}%
\pgftext[x=0.320563in, y=2.128285in, left, base]{\color{textcolor}\rmfamily\fontsize{9.000000}{10.800000}\selectfont \(\displaystyle {0.6}\)}%
\end{pgfscope}%
\begin{pgfscope}%
\pgfpathrectangle{\pgfqpoint{0.581943in}{0.509170in}}{\pgfqpoint{5.599061in}{8.312590in}}%
\pgfusepath{clip}%
\pgfsetbuttcap%
\pgfsetroundjoin%
\pgfsetlinewidth{0.250937pt}%
\definecolor{currentstroke}{rgb}{0.680000,0.680000,0.680000}%
\pgfsetstrokecolor{currentstroke}%
\pgfsetdash{{1.000000pt}{1.000000pt}}{0.000000pt}%
\pgfpathmoveto{\pgfqpoint{0.581943in}{2.448774in}}%
\pgfpathlineto{\pgfqpoint{6.181004in}{2.448774in}}%
\pgfusepath{stroke}%
\end{pgfscope}%
\begin{pgfscope}%
\pgfsetbuttcap%
\pgfsetroundjoin%
\definecolor{currentfill}{rgb}{0.000000,0.000000,0.000000}%
\pgfsetfillcolor{currentfill}%
\pgfsetlinewidth{0.803000pt}%
\definecolor{currentstroke}{rgb}{0.000000,0.000000,0.000000}%
\pgfsetstrokecolor{currentstroke}%
\pgfsetdash{}{0pt}%
\pgfsys@defobject{currentmarker}{\pgfqpoint{-0.048611in}{0.000000in}}{\pgfqpoint{-0.000000in}{0.000000in}}{%
\pgfpathmoveto{\pgfqpoint{-0.000000in}{0.000000in}}%
\pgfpathlineto{\pgfqpoint{-0.048611in}{0.000000in}}%
\pgfusepath{stroke,fill}%
}%
\begin{pgfscope}%
\pgfsys@transformshift{0.581943in}{2.448774in}%
\pgfsys@useobject{currentmarker}{}%
\end{pgfscope}%
\end{pgfscope}%
\begin{pgfscope}%
\definecolor{textcolor}{rgb}{0.000000,0.000000,0.000000}%
\pgfsetstrokecolor{textcolor}%
\pgfsetfillcolor{textcolor}%
\pgftext[x=0.320563in, y=2.405371in, left, base]{\color{textcolor}\rmfamily\fontsize{9.000000}{10.800000}\selectfont \(\displaystyle {0.7}\)}%
\end{pgfscope}%
\begin{pgfscope}%
\pgfpathrectangle{\pgfqpoint{0.581943in}{0.509170in}}{\pgfqpoint{5.599061in}{8.312590in}}%
\pgfusepath{clip}%
\pgfsetbuttcap%
\pgfsetroundjoin%
\pgfsetlinewidth{0.250937pt}%
\definecolor{currentstroke}{rgb}{0.680000,0.680000,0.680000}%
\pgfsetstrokecolor{currentstroke}%
\pgfsetdash{{1.000000pt}{1.000000pt}}{0.000000pt}%
\pgfpathmoveto{\pgfqpoint{0.581943in}{2.725860in}}%
\pgfpathlineto{\pgfqpoint{6.181004in}{2.725860in}}%
\pgfusepath{stroke}%
\end{pgfscope}%
\begin{pgfscope}%
\pgfsetbuttcap%
\pgfsetroundjoin%
\definecolor{currentfill}{rgb}{0.000000,0.000000,0.000000}%
\pgfsetfillcolor{currentfill}%
\pgfsetlinewidth{0.803000pt}%
\definecolor{currentstroke}{rgb}{0.000000,0.000000,0.000000}%
\pgfsetstrokecolor{currentstroke}%
\pgfsetdash{}{0pt}%
\pgfsys@defobject{currentmarker}{\pgfqpoint{-0.048611in}{0.000000in}}{\pgfqpoint{-0.000000in}{0.000000in}}{%
\pgfpathmoveto{\pgfqpoint{-0.000000in}{0.000000in}}%
\pgfpathlineto{\pgfqpoint{-0.048611in}{0.000000in}}%
\pgfusepath{stroke,fill}%
}%
\begin{pgfscope}%
\pgfsys@transformshift{0.581943in}{2.725860in}%
\pgfsys@useobject{currentmarker}{}%
\end{pgfscope}%
\end{pgfscope}%
\begin{pgfscope}%
\definecolor{textcolor}{rgb}{0.000000,0.000000,0.000000}%
\pgfsetstrokecolor{textcolor}%
\pgfsetfillcolor{textcolor}%
\pgftext[x=0.320563in, y=2.682458in, left, base]{\color{textcolor}\rmfamily\fontsize{9.000000}{10.800000}\selectfont \(\displaystyle {0.8}\)}%
\end{pgfscope}%
\begin{pgfscope}%
\pgfpathrectangle{\pgfqpoint{0.581943in}{0.509170in}}{\pgfqpoint{5.599061in}{8.312590in}}%
\pgfusepath{clip}%
\pgfsetbuttcap%
\pgfsetroundjoin%
\pgfsetlinewidth{0.250937pt}%
\definecolor{currentstroke}{rgb}{0.680000,0.680000,0.680000}%
\pgfsetstrokecolor{currentstroke}%
\pgfsetdash{{1.000000pt}{1.000000pt}}{0.000000pt}%
\pgfpathmoveto{\pgfqpoint{0.581943in}{3.002947in}}%
\pgfpathlineto{\pgfqpoint{6.181004in}{3.002947in}}%
\pgfusepath{stroke}%
\end{pgfscope}%
\begin{pgfscope}%
\pgfsetbuttcap%
\pgfsetroundjoin%
\definecolor{currentfill}{rgb}{0.000000,0.000000,0.000000}%
\pgfsetfillcolor{currentfill}%
\pgfsetlinewidth{0.803000pt}%
\definecolor{currentstroke}{rgb}{0.000000,0.000000,0.000000}%
\pgfsetstrokecolor{currentstroke}%
\pgfsetdash{}{0pt}%
\pgfsys@defobject{currentmarker}{\pgfqpoint{-0.048611in}{0.000000in}}{\pgfqpoint{-0.000000in}{0.000000in}}{%
\pgfpathmoveto{\pgfqpoint{-0.000000in}{0.000000in}}%
\pgfpathlineto{\pgfqpoint{-0.048611in}{0.000000in}}%
\pgfusepath{stroke,fill}%
}%
\begin{pgfscope}%
\pgfsys@transformshift{0.581943in}{3.002947in}%
\pgfsys@useobject{currentmarker}{}%
\end{pgfscope}%
\end{pgfscope}%
\begin{pgfscope}%
\definecolor{textcolor}{rgb}{0.000000,0.000000,0.000000}%
\pgfsetstrokecolor{textcolor}%
\pgfsetfillcolor{textcolor}%
\pgftext[x=0.320563in, y=2.959544in, left, base]{\color{textcolor}\rmfamily\fontsize{9.000000}{10.800000}\selectfont \(\displaystyle {0.9}\)}%
\end{pgfscope}%
\begin{pgfscope}%
\pgfpathrectangle{\pgfqpoint{0.581943in}{0.509170in}}{\pgfqpoint{5.599061in}{8.312590in}}%
\pgfusepath{clip}%
\pgfsetbuttcap%
\pgfsetroundjoin%
\pgfsetlinewidth{0.250937pt}%
\definecolor{currentstroke}{rgb}{0.680000,0.680000,0.680000}%
\pgfsetstrokecolor{currentstroke}%
\pgfsetdash{{1.000000pt}{1.000000pt}}{0.000000pt}%
\pgfpathmoveto{\pgfqpoint{0.581943in}{3.280033in}}%
\pgfpathlineto{\pgfqpoint{6.181004in}{3.280033in}}%
\pgfusepath{stroke}%
\end{pgfscope}%
\begin{pgfscope}%
\pgfsetbuttcap%
\pgfsetroundjoin%
\definecolor{currentfill}{rgb}{0.000000,0.000000,0.000000}%
\pgfsetfillcolor{currentfill}%
\pgfsetlinewidth{0.803000pt}%
\definecolor{currentstroke}{rgb}{0.000000,0.000000,0.000000}%
\pgfsetstrokecolor{currentstroke}%
\pgfsetdash{}{0pt}%
\pgfsys@defobject{currentmarker}{\pgfqpoint{-0.048611in}{0.000000in}}{\pgfqpoint{-0.000000in}{0.000000in}}{%
\pgfpathmoveto{\pgfqpoint{-0.000000in}{0.000000in}}%
\pgfpathlineto{\pgfqpoint{-0.048611in}{0.000000in}}%
\pgfusepath{stroke,fill}%
}%
\begin{pgfscope}%
\pgfsys@transformshift{0.581943in}{3.280033in}%
\pgfsys@useobject{currentmarker}{}%
\end{pgfscope}%
\end{pgfscope}%
\begin{pgfscope}%
\definecolor{textcolor}{rgb}{0.000000,0.000000,0.000000}%
\pgfsetstrokecolor{textcolor}%
\pgfsetfillcolor{textcolor}%
\pgftext[x=0.320563in, y=3.236630in, left, base]{\color{textcolor}\rmfamily\fontsize{9.000000}{10.800000}\selectfont \(\displaystyle {1.0}\)}%
\end{pgfscope}%
\begin{pgfscope}%
\pgfpathrectangle{\pgfqpoint{0.581943in}{0.509170in}}{\pgfqpoint{5.599061in}{8.312590in}}%
\pgfusepath{clip}%
\pgfsetbuttcap%
\pgfsetroundjoin%
\pgfsetlinewidth{0.250937pt}%
\definecolor{currentstroke}{rgb}{0.680000,0.680000,0.680000}%
\pgfsetstrokecolor{currentstroke}%
\pgfsetdash{{1.000000pt}{1.000000pt}}{0.000000pt}%
\pgfpathmoveto{\pgfqpoint{0.581943in}{3.557119in}}%
\pgfpathlineto{\pgfqpoint{6.181004in}{3.557119in}}%
\pgfusepath{stroke}%
\end{pgfscope}%
\begin{pgfscope}%
\pgfsetbuttcap%
\pgfsetroundjoin%
\definecolor{currentfill}{rgb}{0.000000,0.000000,0.000000}%
\pgfsetfillcolor{currentfill}%
\pgfsetlinewidth{0.803000pt}%
\definecolor{currentstroke}{rgb}{0.000000,0.000000,0.000000}%
\pgfsetstrokecolor{currentstroke}%
\pgfsetdash{}{0pt}%
\pgfsys@defobject{currentmarker}{\pgfqpoint{-0.048611in}{0.000000in}}{\pgfqpoint{-0.000000in}{0.000000in}}{%
\pgfpathmoveto{\pgfqpoint{-0.000000in}{0.000000in}}%
\pgfpathlineto{\pgfqpoint{-0.048611in}{0.000000in}}%
\pgfusepath{stroke,fill}%
}%
\begin{pgfscope}%
\pgfsys@transformshift{0.581943in}{3.557119in}%
\pgfsys@useobject{currentmarker}{}%
\end{pgfscope}%
\end{pgfscope}%
\begin{pgfscope}%
\definecolor{textcolor}{rgb}{0.000000,0.000000,0.000000}%
\pgfsetstrokecolor{textcolor}%
\pgfsetfillcolor{textcolor}%
\pgftext[x=0.320563in, y=3.513717in, left, base]{\color{textcolor}\rmfamily\fontsize{9.000000}{10.800000}\selectfont \(\displaystyle {1.1}\)}%
\end{pgfscope}%
\begin{pgfscope}%
\pgfpathrectangle{\pgfqpoint{0.581943in}{0.509170in}}{\pgfqpoint{5.599061in}{8.312590in}}%
\pgfusepath{clip}%
\pgfsetbuttcap%
\pgfsetroundjoin%
\pgfsetlinewidth{0.250937pt}%
\definecolor{currentstroke}{rgb}{0.680000,0.680000,0.680000}%
\pgfsetstrokecolor{currentstroke}%
\pgfsetdash{{1.000000pt}{1.000000pt}}{0.000000pt}%
\pgfpathmoveto{\pgfqpoint{0.581943in}{3.834206in}}%
\pgfpathlineto{\pgfqpoint{6.181004in}{3.834206in}}%
\pgfusepath{stroke}%
\end{pgfscope}%
\begin{pgfscope}%
\pgfsetbuttcap%
\pgfsetroundjoin%
\definecolor{currentfill}{rgb}{0.000000,0.000000,0.000000}%
\pgfsetfillcolor{currentfill}%
\pgfsetlinewidth{0.803000pt}%
\definecolor{currentstroke}{rgb}{0.000000,0.000000,0.000000}%
\pgfsetstrokecolor{currentstroke}%
\pgfsetdash{}{0pt}%
\pgfsys@defobject{currentmarker}{\pgfqpoint{-0.048611in}{0.000000in}}{\pgfqpoint{-0.000000in}{0.000000in}}{%
\pgfpathmoveto{\pgfqpoint{-0.000000in}{0.000000in}}%
\pgfpathlineto{\pgfqpoint{-0.048611in}{0.000000in}}%
\pgfusepath{stroke,fill}%
}%
\begin{pgfscope}%
\pgfsys@transformshift{0.581943in}{3.834206in}%
\pgfsys@useobject{currentmarker}{}%
\end{pgfscope}%
\end{pgfscope}%
\begin{pgfscope}%
\definecolor{textcolor}{rgb}{0.000000,0.000000,0.000000}%
\pgfsetstrokecolor{textcolor}%
\pgfsetfillcolor{textcolor}%
\pgftext[x=0.320563in, y=3.790803in, left, base]{\color{textcolor}\rmfamily\fontsize{9.000000}{10.800000}\selectfont \(\displaystyle {1.2}\)}%
\end{pgfscope}%
\begin{pgfscope}%
\pgfpathrectangle{\pgfqpoint{0.581943in}{0.509170in}}{\pgfqpoint{5.599061in}{8.312590in}}%
\pgfusepath{clip}%
\pgfsetbuttcap%
\pgfsetroundjoin%
\pgfsetlinewidth{0.250937pt}%
\definecolor{currentstroke}{rgb}{0.680000,0.680000,0.680000}%
\pgfsetstrokecolor{currentstroke}%
\pgfsetdash{{1.000000pt}{1.000000pt}}{0.000000pt}%
\pgfpathmoveto{\pgfqpoint{0.581943in}{4.111292in}}%
\pgfpathlineto{\pgfqpoint{6.181004in}{4.111292in}}%
\pgfusepath{stroke}%
\end{pgfscope}%
\begin{pgfscope}%
\pgfsetbuttcap%
\pgfsetroundjoin%
\definecolor{currentfill}{rgb}{0.000000,0.000000,0.000000}%
\pgfsetfillcolor{currentfill}%
\pgfsetlinewidth{0.803000pt}%
\definecolor{currentstroke}{rgb}{0.000000,0.000000,0.000000}%
\pgfsetstrokecolor{currentstroke}%
\pgfsetdash{}{0pt}%
\pgfsys@defobject{currentmarker}{\pgfqpoint{-0.048611in}{0.000000in}}{\pgfqpoint{-0.000000in}{0.000000in}}{%
\pgfpathmoveto{\pgfqpoint{-0.000000in}{0.000000in}}%
\pgfpathlineto{\pgfqpoint{-0.048611in}{0.000000in}}%
\pgfusepath{stroke,fill}%
}%
\begin{pgfscope}%
\pgfsys@transformshift{0.581943in}{4.111292in}%
\pgfsys@useobject{currentmarker}{}%
\end{pgfscope}%
\end{pgfscope}%
\begin{pgfscope}%
\definecolor{textcolor}{rgb}{0.000000,0.000000,0.000000}%
\pgfsetstrokecolor{textcolor}%
\pgfsetfillcolor{textcolor}%
\pgftext[x=0.320563in, y=4.067889in, left, base]{\color{textcolor}\rmfamily\fontsize{9.000000}{10.800000}\selectfont \(\displaystyle {1.3}\)}%
\end{pgfscope}%
\begin{pgfscope}%
\pgfpathrectangle{\pgfqpoint{0.581943in}{0.509170in}}{\pgfqpoint{5.599061in}{8.312590in}}%
\pgfusepath{clip}%
\pgfsetbuttcap%
\pgfsetroundjoin%
\pgfsetlinewidth{0.250937pt}%
\definecolor{currentstroke}{rgb}{0.680000,0.680000,0.680000}%
\pgfsetstrokecolor{currentstroke}%
\pgfsetdash{{1.000000pt}{1.000000pt}}{0.000000pt}%
\pgfpathmoveto{\pgfqpoint{0.581943in}{4.388378in}}%
\pgfpathlineto{\pgfqpoint{6.181004in}{4.388378in}}%
\pgfusepath{stroke}%
\end{pgfscope}%
\begin{pgfscope}%
\pgfsetbuttcap%
\pgfsetroundjoin%
\definecolor{currentfill}{rgb}{0.000000,0.000000,0.000000}%
\pgfsetfillcolor{currentfill}%
\pgfsetlinewidth{0.803000pt}%
\definecolor{currentstroke}{rgb}{0.000000,0.000000,0.000000}%
\pgfsetstrokecolor{currentstroke}%
\pgfsetdash{}{0pt}%
\pgfsys@defobject{currentmarker}{\pgfqpoint{-0.048611in}{0.000000in}}{\pgfqpoint{-0.000000in}{0.000000in}}{%
\pgfpathmoveto{\pgfqpoint{-0.000000in}{0.000000in}}%
\pgfpathlineto{\pgfqpoint{-0.048611in}{0.000000in}}%
\pgfusepath{stroke,fill}%
}%
\begin{pgfscope}%
\pgfsys@transformshift{0.581943in}{4.388378in}%
\pgfsys@useobject{currentmarker}{}%
\end{pgfscope}%
\end{pgfscope}%
\begin{pgfscope}%
\definecolor{textcolor}{rgb}{0.000000,0.000000,0.000000}%
\pgfsetstrokecolor{textcolor}%
\pgfsetfillcolor{textcolor}%
\pgftext[x=0.320563in, y=4.344975in, left, base]{\color{textcolor}\rmfamily\fontsize{9.000000}{10.800000}\selectfont \(\displaystyle {1.4}\)}%
\end{pgfscope}%
\begin{pgfscope}%
\pgfpathrectangle{\pgfqpoint{0.581943in}{0.509170in}}{\pgfqpoint{5.599061in}{8.312590in}}%
\pgfusepath{clip}%
\pgfsetbuttcap%
\pgfsetroundjoin%
\pgfsetlinewidth{0.250937pt}%
\definecolor{currentstroke}{rgb}{0.680000,0.680000,0.680000}%
\pgfsetstrokecolor{currentstroke}%
\pgfsetdash{{1.000000pt}{1.000000pt}}{0.000000pt}%
\pgfpathmoveto{\pgfqpoint{0.581943in}{4.665465in}}%
\pgfpathlineto{\pgfqpoint{6.181004in}{4.665465in}}%
\pgfusepath{stroke}%
\end{pgfscope}%
\begin{pgfscope}%
\pgfsetbuttcap%
\pgfsetroundjoin%
\definecolor{currentfill}{rgb}{0.000000,0.000000,0.000000}%
\pgfsetfillcolor{currentfill}%
\pgfsetlinewidth{0.803000pt}%
\definecolor{currentstroke}{rgb}{0.000000,0.000000,0.000000}%
\pgfsetstrokecolor{currentstroke}%
\pgfsetdash{}{0pt}%
\pgfsys@defobject{currentmarker}{\pgfqpoint{-0.048611in}{0.000000in}}{\pgfqpoint{-0.000000in}{0.000000in}}{%
\pgfpathmoveto{\pgfqpoint{-0.000000in}{0.000000in}}%
\pgfpathlineto{\pgfqpoint{-0.048611in}{0.000000in}}%
\pgfusepath{stroke,fill}%
}%
\begin{pgfscope}%
\pgfsys@transformshift{0.581943in}{4.665465in}%
\pgfsys@useobject{currentmarker}{}%
\end{pgfscope}%
\end{pgfscope}%
\begin{pgfscope}%
\definecolor{textcolor}{rgb}{0.000000,0.000000,0.000000}%
\pgfsetstrokecolor{textcolor}%
\pgfsetfillcolor{textcolor}%
\pgftext[x=0.320563in, y=4.622062in, left, base]{\color{textcolor}\rmfamily\fontsize{9.000000}{10.800000}\selectfont \(\displaystyle {1.5}\)}%
\end{pgfscope}%
\begin{pgfscope}%
\pgfpathrectangle{\pgfqpoint{0.581943in}{0.509170in}}{\pgfqpoint{5.599061in}{8.312590in}}%
\pgfusepath{clip}%
\pgfsetbuttcap%
\pgfsetroundjoin%
\pgfsetlinewidth{0.250937pt}%
\definecolor{currentstroke}{rgb}{0.680000,0.680000,0.680000}%
\pgfsetstrokecolor{currentstroke}%
\pgfsetdash{{1.000000pt}{1.000000pt}}{0.000000pt}%
\pgfpathmoveto{\pgfqpoint{0.581943in}{4.942551in}}%
\pgfpathlineto{\pgfqpoint{6.181004in}{4.942551in}}%
\pgfusepath{stroke}%
\end{pgfscope}%
\begin{pgfscope}%
\pgfsetbuttcap%
\pgfsetroundjoin%
\definecolor{currentfill}{rgb}{0.000000,0.000000,0.000000}%
\pgfsetfillcolor{currentfill}%
\pgfsetlinewidth{0.803000pt}%
\definecolor{currentstroke}{rgb}{0.000000,0.000000,0.000000}%
\pgfsetstrokecolor{currentstroke}%
\pgfsetdash{}{0pt}%
\pgfsys@defobject{currentmarker}{\pgfqpoint{-0.048611in}{0.000000in}}{\pgfqpoint{-0.000000in}{0.000000in}}{%
\pgfpathmoveto{\pgfqpoint{-0.000000in}{0.000000in}}%
\pgfpathlineto{\pgfqpoint{-0.048611in}{0.000000in}}%
\pgfusepath{stroke,fill}%
}%
\begin{pgfscope}%
\pgfsys@transformshift{0.581943in}{4.942551in}%
\pgfsys@useobject{currentmarker}{}%
\end{pgfscope}%
\end{pgfscope}%
\begin{pgfscope}%
\definecolor{textcolor}{rgb}{0.000000,0.000000,0.000000}%
\pgfsetstrokecolor{textcolor}%
\pgfsetfillcolor{textcolor}%
\pgftext[x=0.320563in, y=4.899148in, left, base]{\color{textcolor}\rmfamily\fontsize{9.000000}{10.800000}\selectfont \(\displaystyle {1.6}\)}%
\end{pgfscope}%
\begin{pgfscope}%
\pgfpathrectangle{\pgfqpoint{0.581943in}{0.509170in}}{\pgfqpoint{5.599061in}{8.312590in}}%
\pgfusepath{clip}%
\pgfsetbuttcap%
\pgfsetroundjoin%
\pgfsetlinewidth{0.250937pt}%
\definecolor{currentstroke}{rgb}{0.680000,0.680000,0.680000}%
\pgfsetstrokecolor{currentstroke}%
\pgfsetdash{{1.000000pt}{1.000000pt}}{0.000000pt}%
\pgfpathmoveto{\pgfqpoint{0.581943in}{5.219637in}}%
\pgfpathlineto{\pgfqpoint{6.181004in}{5.219637in}}%
\pgfusepath{stroke}%
\end{pgfscope}%
\begin{pgfscope}%
\pgfsetbuttcap%
\pgfsetroundjoin%
\definecolor{currentfill}{rgb}{0.000000,0.000000,0.000000}%
\pgfsetfillcolor{currentfill}%
\pgfsetlinewidth{0.803000pt}%
\definecolor{currentstroke}{rgb}{0.000000,0.000000,0.000000}%
\pgfsetstrokecolor{currentstroke}%
\pgfsetdash{}{0pt}%
\pgfsys@defobject{currentmarker}{\pgfqpoint{-0.048611in}{0.000000in}}{\pgfqpoint{-0.000000in}{0.000000in}}{%
\pgfpathmoveto{\pgfqpoint{-0.000000in}{0.000000in}}%
\pgfpathlineto{\pgfqpoint{-0.048611in}{0.000000in}}%
\pgfusepath{stroke,fill}%
}%
\begin{pgfscope}%
\pgfsys@transformshift{0.581943in}{5.219637in}%
\pgfsys@useobject{currentmarker}{}%
\end{pgfscope}%
\end{pgfscope}%
\begin{pgfscope}%
\definecolor{textcolor}{rgb}{0.000000,0.000000,0.000000}%
\pgfsetstrokecolor{textcolor}%
\pgfsetfillcolor{textcolor}%
\pgftext[x=0.320563in, y=5.176234in, left, base]{\color{textcolor}\rmfamily\fontsize{9.000000}{10.800000}\selectfont \(\displaystyle {1.7}\)}%
\end{pgfscope}%
\begin{pgfscope}%
\pgfpathrectangle{\pgfqpoint{0.581943in}{0.509170in}}{\pgfqpoint{5.599061in}{8.312590in}}%
\pgfusepath{clip}%
\pgfsetbuttcap%
\pgfsetroundjoin%
\pgfsetlinewidth{0.250937pt}%
\definecolor{currentstroke}{rgb}{0.680000,0.680000,0.680000}%
\pgfsetstrokecolor{currentstroke}%
\pgfsetdash{{1.000000pt}{1.000000pt}}{0.000000pt}%
\pgfpathmoveto{\pgfqpoint{0.581943in}{5.496724in}}%
\pgfpathlineto{\pgfqpoint{6.181004in}{5.496724in}}%
\pgfusepath{stroke}%
\end{pgfscope}%
\begin{pgfscope}%
\pgfsetbuttcap%
\pgfsetroundjoin%
\definecolor{currentfill}{rgb}{0.000000,0.000000,0.000000}%
\pgfsetfillcolor{currentfill}%
\pgfsetlinewidth{0.803000pt}%
\definecolor{currentstroke}{rgb}{0.000000,0.000000,0.000000}%
\pgfsetstrokecolor{currentstroke}%
\pgfsetdash{}{0pt}%
\pgfsys@defobject{currentmarker}{\pgfqpoint{-0.048611in}{0.000000in}}{\pgfqpoint{-0.000000in}{0.000000in}}{%
\pgfpathmoveto{\pgfqpoint{-0.000000in}{0.000000in}}%
\pgfpathlineto{\pgfqpoint{-0.048611in}{0.000000in}}%
\pgfusepath{stroke,fill}%
}%
\begin{pgfscope}%
\pgfsys@transformshift{0.581943in}{5.496724in}%
\pgfsys@useobject{currentmarker}{}%
\end{pgfscope}%
\end{pgfscope}%
\begin{pgfscope}%
\definecolor{textcolor}{rgb}{0.000000,0.000000,0.000000}%
\pgfsetstrokecolor{textcolor}%
\pgfsetfillcolor{textcolor}%
\pgftext[x=0.320563in, y=5.453321in, left, base]{\color{textcolor}\rmfamily\fontsize{9.000000}{10.800000}\selectfont \(\displaystyle {1.8}\)}%
\end{pgfscope}%
\begin{pgfscope}%
\pgfpathrectangle{\pgfqpoint{0.581943in}{0.509170in}}{\pgfqpoint{5.599061in}{8.312590in}}%
\pgfusepath{clip}%
\pgfsetbuttcap%
\pgfsetroundjoin%
\pgfsetlinewidth{0.250937pt}%
\definecolor{currentstroke}{rgb}{0.680000,0.680000,0.680000}%
\pgfsetstrokecolor{currentstroke}%
\pgfsetdash{{1.000000pt}{1.000000pt}}{0.000000pt}%
\pgfpathmoveto{\pgfqpoint{0.581943in}{5.773810in}}%
\pgfpathlineto{\pgfqpoint{6.181004in}{5.773810in}}%
\pgfusepath{stroke}%
\end{pgfscope}%
\begin{pgfscope}%
\pgfsetbuttcap%
\pgfsetroundjoin%
\definecolor{currentfill}{rgb}{0.000000,0.000000,0.000000}%
\pgfsetfillcolor{currentfill}%
\pgfsetlinewidth{0.803000pt}%
\definecolor{currentstroke}{rgb}{0.000000,0.000000,0.000000}%
\pgfsetstrokecolor{currentstroke}%
\pgfsetdash{}{0pt}%
\pgfsys@defobject{currentmarker}{\pgfqpoint{-0.048611in}{0.000000in}}{\pgfqpoint{-0.000000in}{0.000000in}}{%
\pgfpathmoveto{\pgfqpoint{-0.000000in}{0.000000in}}%
\pgfpathlineto{\pgfqpoint{-0.048611in}{0.000000in}}%
\pgfusepath{stroke,fill}%
}%
\begin{pgfscope}%
\pgfsys@transformshift{0.581943in}{5.773810in}%
\pgfsys@useobject{currentmarker}{}%
\end{pgfscope}%
\end{pgfscope}%
\begin{pgfscope}%
\definecolor{textcolor}{rgb}{0.000000,0.000000,0.000000}%
\pgfsetstrokecolor{textcolor}%
\pgfsetfillcolor{textcolor}%
\pgftext[x=0.320563in, y=5.730407in, left, base]{\color{textcolor}\rmfamily\fontsize{9.000000}{10.800000}\selectfont \(\displaystyle {1.9}\)}%
\end{pgfscope}%
\begin{pgfscope}%
\pgfpathrectangle{\pgfqpoint{0.581943in}{0.509170in}}{\pgfqpoint{5.599061in}{8.312590in}}%
\pgfusepath{clip}%
\pgfsetbuttcap%
\pgfsetroundjoin%
\pgfsetlinewidth{0.250937pt}%
\definecolor{currentstroke}{rgb}{0.680000,0.680000,0.680000}%
\pgfsetstrokecolor{currentstroke}%
\pgfsetdash{{1.000000pt}{1.000000pt}}{0.000000pt}%
\pgfpathmoveto{\pgfqpoint{0.581943in}{6.050896in}}%
\pgfpathlineto{\pgfqpoint{6.181004in}{6.050896in}}%
\pgfusepath{stroke}%
\end{pgfscope}%
\begin{pgfscope}%
\pgfsetbuttcap%
\pgfsetroundjoin%
\definecolor{currentfill}{rgb}{0.000000,0.000000,0.000000}%
\pgfsetfillcolor{currentfill}%
\pgfsetlinewidth{0.803000pt}%
\definecolor{currentstroke}{rgb}{0.000000,0.000000,0.000000}%
\pgfsetstrokecolor{currentstroke}%
\pgfsetdash{}{0pt}%
\pgfsys@defobject{currentmarker}{\pgfqpoint{-0.048611in}{0.000000in}}{\pgfqpoint{-0.000000in}{0.000000in}}{%
\pgfpathmoveto{\pgfqpoint{-0.000000in}{0.000000in}}%
\pgfpathlineto{\pgfqpoint{-0.048611in}{0.000000in}}%
\pgfusepath{stroke,fill}%
}%
\begin{pgfscope}%
\pgfsys@transformshift{0.581943in}{6.050896in}%
\pgfsys@useobject{currentmarker}{}%
\end{pgfscope}%
\end{pgfscope}%
\begin{pgfscope}%
\definecolor{textcolor}{rgb}{0.000000,0.000000,0.000000}%
\pgfsetstrokecolor{textcolor}%
\pgfsetfillcolor{textcolor}%
\pgftext[x=0.320563in, y=6.007493in, left, base]{\color{textcolor}\rmfamily\fontsize{9.000000}{10.800000}\selectfont \(\displaystyle {2.0}\)}%
\end{pgfscope}%
\begin{pgfscope}%
\pgfpathrectangle{\pgfqpoint{0.581943in}{0.509170in}}{\pgfqpoint{5.599061in}{8.312590in}}%
\pgfusepath{clip}%
\pgfsetbuttcap%
\pgfsetroundjoin%
\pgfsetlinewidth{0.250937pt}%
\definecolor{currentstroke}{rgb}{0.680000,0.680000,0.680000}%
\pgfsetstrokecolor{currentstroke}%
\pgfsetdash{{1.000000pt}{1.000000pt}}{0.000000pt}%
\pgfpathmoveto{\pgfqpoint{0.581943in}{6.327982in}}%
\pgfpathlineto{\pgfqpoint{6.181004in}{6.327982in}}%
\pgfusepath{stroke}%
\end{pgfscope}%
\begin{pgfscope}%
\pgfsetbuttcap%
\pgfsetroundjoin%
\definecolor{currentfill}{rgb}{0.000000,0.000000,0.000000}%
\pgfsetfillcolor{currentfill}%
\pgfsetlinewidth{0.803000pt}%
\definecolor{currentstroke}{rgb}{0.000000,0.000000,0.000000}%
\pgfsetstrokecolor{currentstroke}%
\pgfsetdash{}{0pt}%
\pgfsys@defobject{currentmarker}{\pgfqpoint{-0.048611in}{0.000000in}}{\pgfqpoint{-0.000000in}{0.000000in}}{%
\pgfpathmoveto{\pgfqpoint{-0.000000in}{0.000000in}}%
\pgfpathlineto{\pgfqpoint{-0.048611in}{0.000000in}}%
\pgfusepath{stroke,fill}%
}%
\begin{pgfscope}%
\pgfsys@transformshift{0.581943in}{6.327982in}%
\pgfsys@useobject{currentmarker}{}%
\end{pgfscope}%
\end{pgfscope}%
\begin{pgfscope}%
\definecolor{textcolor}{rgb}{0.000000,0.000000,0.000000}%
\pgfsetstrokecolor{textcolor}%
\pgfsetfillcolor{textcolor}%
\pgftext[x=0.320563in, y=6.284580in, left, base]{\color{textcolor}\rmfamily\fontsize{9.000000}{10.800000}\selectfont \(\displaystyle {2.1}\)}%
\end{pgfscope}%
\begin{pgfscope}%
\pgfpathrectangle{\pgfqpoint{0.581943in}{0.509170in}}{\pgfqpoint{5.599061in}{8.312590in}}%
\pgfusepath{clip}%
\pgfsetbuttcap%
\pgfsetroundjoin%
\pgfsetlinewidth{0.250937pt}%
\definecolor{currentstroke}{rgb}{0.680000,0.680000,0.680000}%
\pgfsetstrokecolor{currentstroke}%
\pgfsetdash{{1.000000pt}{1.000000pt}}{0.000000pt}%
\pgfpathmoveto{\pgfqpoint{0.581943in}{6.605069in}}%
\pgfpathlineto{\pgfqpoint{6.181004in}{6.605069in}}%
\pgfusepath{stroke}%
\end{pgfscope}%
\begin{pgfscope}%
\pgfsetbuttcap%
\pgfsetroundjoin%
\definecolor{currentfill}{rgb}{0.000000,0.000000,0.000000}%
\pgfsetfillcolor{currentfill}%
\pgfsetlinewidth{0.803000pt}%
\definecolor{currentstroke}{rgb}{0.000000,0.000000,0.000000}%
\pgfsetstrokecolor{currentstroke}%
\pgfsetdash{}{0pt}%
\pgfsys@defobject{currentmarker}{\pgfqpoint{-0.048611in}{0.000000in}}{\pgfqpoint{-0.000000in}{0.000000in}}{%
\pgfpathmoveto{\pgfqpoint{-0.000000in}{0.000000in}}%
\pgfpathlineto{\pgfqpoint{-0.048611in}{0.000000in}}%
\pgfusepath{stroke,fill}%
}%
\begin{pgfscope}%
\pgfsys@transformshift{0.581943in}{6.605069in}%
\pgfsys@useobject{currentmarker}{}%
\end{pgfscope}%
\end{pgfscope}%
\begin{pgfscope}%
\definecolor{textcolor}{rgb}{0.000000,0.000000,0.000000}%
\pgfsetstrokecolor{textcolor}%
\pgfsetfillcolor{textcolor}%
\pgftext[x=0.320563in, y=6.561666in, left, base]{\color{textcolor}\rmfamily\fontsize{9.000000}{10.800000}\selectfont \(\displaystyle {2.2}\)}%
\end{pgfscope}%
\begin{pgfscope}%
\pgfpathrectangle{\pgfqpoint{0.581943in}{0.509170in}}{\pgfqpoint{5.599061in}{8.312590in}}%
\pgfusepath{clip}%
\pgfsetbuttcap%
\pgfsetroundjoin%
\pgfsetlinewidth{0.250937pt}%
\definecolor{currentstroke}{rgb}{0.680000,0.680000,0.680000}%
\pgfsetstrokecolor{currentstroke}%
\pgfsetdash{{1.000000pt}{1.000000pt}}{0.000000pt}%
\pgfpathmoveto{\pgfqpoint{0.581943in}{6.882155in}}%
\pgfpathlineto{\pgfqpoint{6.181004in}{6.882155in}}%
\pgfusepath{stroke}%
\end{pgfscope}%
\begin{pgfscope}%
\pgfsetbuttcap%
\pgfsetroundjoin%
\definecolor{currentfill}{rgb}{0.000000,0.000000,0.000000}%
\pgfsetfillcolor{currentfill}%
\pgfsetlinewidth{0.803000pt}%
\definecolor{currentstroke}{rgb}{0.000000,0.000000,0.000000}%
\pgfsetstrokecolor{currentstroke}%
\pgfsetdash{}{0pt}%
\pgfsys@defobject{currentmarker}{\pgfqpoint{-0.048611in}{0.000000in}}{\pgfqpoint{-0.000000in}{0.000000in}}{%
\pgfpathmoveto{\pgfqpoint{-0.000000in}{0.000000in}}%
\pgfpathlineto{\pgfqpoint{-0.048611in}{0.000000in}}%
\pgfusepath{stroke,fill}%
}%
\begin{pgfscope}%
\pgfsys@transformshift{0.581943in}{6.882155in}%
\pgfsys@useobject{currentmarker}{}%
\end{pgfscope}%
\end{pgfscope}%
\begin{pgfscope}%
\definecolor{textcolor}{rgb}{0.000000,0.000000,0.000000}%
\pgfsetstrokecolor{textcolor}%
\pgfsetfillcolor{textcolor}%
\pgftext[x=0.320563in, y=6.838752in, left, base]{\color{textcolor}\rmfamily\fontsize{9.000000}{10.800000}\selectfont \(\displaystyle {2.3}\)}%
\end{pgfscope}%
\begin{pgfscope}%
\pgfpathrectangle{\pgfqpoint{0.581943in}{0.509170in}}{\pgfqpoint{5.599061in}{8.312590in}}%
\pgfusepath{clip}%
\pgfsetbuttcap%
\pgfsetroundjoin%
\pgfsetlinewidth{0.250937pt}%
\definecolor{currentstroke}{rgb}{0.680000,0.680000,0.680000}%
\pgfsetstrokecolor{currentstroke}%
\pgfsetdash{{1.000000pt}{1.000000pt}}{0.000000pt}%
\pgfpathmoveto{\pgfqpoint{0.581943in}{7.159241in}}%
\pgfpathlineto{\pgfqpoint{6.181004in}{7.159241in}}%
\pgfusepath{stroke}%
\end{pgfscope}%
\begin{pgfscope}%
\pgfsetbuttcap%
\pgfsetroundjoin%
\definecolor{currentfill}{rgb}{0.000000,0.000000,0.000000}%
\pgfsetfillcolor{currentfill}%
\pgfsetlinewidth{0.803000pt}%
\definecolor{currentstroke}{rgb}{0.000000,0.000000,0.000000}%
\pgfsetstrokecolor{currentstroke}%
\pgfsetdash{}{0pt}%
\pgfsys@defobject{currentmarker}{\pgfqpoint{-0.048611in}{0.000000in}}{\pgfqpoint{-0.000000in}{0.000000in}}{%
\pgfpathmoveto{\pgfqpoint{-0.000000in}{0.000000in}}%
\pgfpathlineto{\pgfqpoint{-0.048611in}{0.000000in}}%
\pgfusepath{stroke,fill}%
}%
\begin{pgfscope}%
\pgfsys@transformshift{0.581943in}{7.159241in}%
\pgfsys@useobject{currentmarker}{}%
\end{pgfscope}%
\end{pgfscope}%
\begin{pgfscope}%
\definecolor{textcolor}{rgb}{0.000000,0.000000,0.000000}%
\pgfsetstrokecolor{textcolor}%
\pgfsetfillcolor{textcolor}%
\pgftext[x=0.320563in, y=7.115839in, left, base]{\color{textcolor}\rmfamily\fontsize{9.000000}{10.800000}\selectfont \(\displaystyle {2.4}\)}%
\end{pgfscope}%
\begin{pgfscope}%
\pgfpathrectangle{\pgfqpoint{0.581943in}{0.509170in}}{\pgfqpoint{5.599061in}{8.312590in}}%
\pgfusepath{clip}%
\pgfsetbuttcap%
\pgfsetroundjoin%
\pgfsetlinewidth{0.250937pt}%
\definecolor{currentstroke}{rgb}{0.680000,0.680000,0.680000}%
\pgfsetstrokecolor{currentstroke}%
\pgfsetdash{{1.000000pt}{1.000000pt}}{0.000000pt}%
\pgfpathmoveto{\pgfqpoint{0.581943in}{7.436328in}}%
\pgfpathlineto{\pgfqpoint{6.181004in}{7.436328in}}%
\pgfusepath{stroke}%
\end{pgfscope}%
\begin{pgfscope}%
\pgfsetbuttcap%
\pgfsetroundjoin%
\definecolor{currentfill}{rgb}{0.000000,0.000000,0.000000}%
\pgfsetfillcolor{currentfill}%
\pgfsetlinewidth{0.803000pt}%
\definecolor{currentstroke}{rgb}{0.000000,0.000000,0.000000}%
\pgfsetstrokecolor{currentstroke}%
\pgfsetdash{}{0pt}%
\pgfsys@defobject{currentmarker}{\pgfqpoint{-0.048611in}{0.000000in}}{\pgfqpoint{-0.000000in}{0.000000in}}{%
\pgfpathmoveto{\pgfqpoint{-0.000000in}{0.000000in}}%
\pgfpathlineto{\pgfqpoint{-0.048611in}{0.000000in}}%
\pgfusepath{stroke,fill}%
}%
\begin{pgfscope}%
\pgfsys@transformshift{0.581943in}{7.436328in}%
\pgfsys@useobject{currentmarker}{}%
\end{pgfscope}%
\end{pgfscope}%
\begin{pgfscope}%
\definecolor{textcolor}{rgb}{0.000000,0.000000,0.000000}%
\pgfsetstrokecolor{textcolor}%
\pgfsetfillcolor{textcolor}%
\pgftext[x=0.320563in, y=7.392925in, left, base]{\color{textcolor}\rmfamily\fontsize{9.000000}{10.800000}\selectfont \(\displaystyle {2.5}\)}%
\end{pgfscope}%
\begin{pgfscope}%
\pgfpathrectangle{\pgfqpoint{0.581943in}{0.509170in}}{\pgfqpoint{5.599061in}{8.312590in}}%
\pgfusepath{clip}%
\pgfsetbuttcap%
\pgfsetroundjoin%
\pgfsetlinewidth{0.250937pt}%
\definecolor{currentstroke}{rgb}{0.680000,0.680000,0.680000}%
\pgfsetstrokecolor{currentstroke}%
\pgfsetdash{{1.000000pt}{1.000000pt}}{0.000000pt}%
\pgfpathmoveto{\pgfqpoint{0.581943in}{7.713414in}}%
\pgfpathlineto{\pgfqpoint{6.181004in}{7.713414in}}%
\pgfusepath{stroke}%
\end{pgfscope}%
\begin{pgfscope}%
\pgfsetbuttcap%
\pgfsetroundjoin%
\definecolor{currentfill}{rgb}{0.000000,0.000000,0.000000}%
\pgfsetfillcolor{currentfill}%
\pgfsetlinewidth{0.803000pt}%
\definecolor{currentstroke}{rgb}{0.000000,0.000000,0.000000}%
\pgfsetstrokecolor{currentstroke}%
\pgfsetdash{}{0pt}%
\pgfsys@defobject{currentmarker}{\pgfqpoint{-0.048611in}{0.000000in}}{\pgfqpoint{-0.000000in}{0.000000in}}{%
\pgfpathmoveto{\pgfqpoint{-0.000000in}{0.000000in}}%
\pgfpathlineto{\pgfqpoint{-0.048611in}{0.000000in}}%
\pgfusepath{stroke,fill}%
}%
\begin{pgfscope}%
\pgfsys@transformshift{0.581943in}{7.713414in}%
\pgfsys@useobject{currentmarker}{}%
\end{pgfscope}%
\end{pgfscope}%
\begin{pgfscope}%
\definecolor{textcolor}{rgb}{0.000000,0.000000,0.000000}%
\pgfsetstrokecolor{textcolor}%
\pgfsetfillcolor{textcolor}%
\pgftext[x=0.320563in, y=7.670011in, left, base]{\color{textcolor}\rmfamily\fontsize{9.000000}{10.800000}\selectfont \(\displaystyle {2.6}\)}%
\end{pgfscope}%
\begin{pgfscope}%
\pgfpathrectangle{\pgfqpoint{0.581943in}{0.509170in}}{\pgfqpoint{5.599061in}{8.312590in}}%
\pgfusepath{clip}%
\pgfsetbuttcap%
\pgfsetroundjoin%
\pgfsetlinewidth{0.250937pt}%
\definecolor{currentstroke}{rgb}{0.680000,0.680000,0.680000}%
\pgfsetstrokecolor{currentstroke}%
\pgfsetdash{{1.000000pt}{1.000000pt}}{0.000000pt}%
\pgfpathmoveto{\pgfqpoint{0.581943in}{7.990500in}}%
\pgfpathlineto{\pgfqpoint{6.181004in}{7.990500in}}%
\pgfusepath{stroke}%
\end{pgfscope}%
\begin{pgfscope}%
\pgfsetbuttcap%
\pgfsetroundjoin%
\definecolor{currentfill}{rgb}{0.000000,0.000000,0.000000}%
\pgfsetfillcolor{currentfill}%
\pgfsetlinewidth{0.803000pt}%
\definecolor{currentstroke}{rgb}{0.000000,0.000000,0.000000}%
\pgfsetstrokecolor{currentstroke}%
\pgfsetdash{}{0pt}%
\pgfsys@defobject{currentmarker}{\pgfqpoint{-0.048611in}{0.000000in}}{\pgfqpoint{-0.000000in}{0.000000in}}{%
\pgfpathmoveto{\pgfqpoint{-0.000000in}{0.000000in}}%
\pgfpathlineto{\pgfqpoint{-0.048611in}{0.000000in}}%
\pgfusepath{stroke,fill}%
}%
\begin{pgfscope}%
\pgfsys@transformshift{0.581943in}{7.990500in}%
\pgfsys@useobject{currentmarker}{}%
\end{pgfscope}%
\end{pgfscope}%
\begin{pgfscope}%
\definecolor{textcolor}{rgb}{0.000000,0.000000,0.000000}%
\pgfsetstrokecolor{textcolor}%
\pgfsetfillcolor{textcolor}%
\pgftext[x=0.320563in, y=7.947098in, left, base]{\color{textcolor}\rmfamily\fontsize{9.000000}{10.800000}\selectfont \(\displaystyle {2.7}\)}%
\end{pgfscope}%
\begin{pgfscope}%
\pgfpathrectangle{\pgfqpoint{0.581943in}{0.509170in}}{\pgfqpoint{5.599061in}{8.312590in}}%
\pgfusepath{clip}%
\pgfsetbuttcap%
\pgfsetroundjoin%
\pgfsetlinewidth{0.250937pt}%
\definecolor{currentstroke}{rgb}{0.680000,0.680000,0.680000}%
\pgfsetstrokecolor{currentstroke}%
\pgfsetdash{{1.000000pt}{1.000000pt}}{0.000000pt}%
\pgfpathmoveto{\pgfqpoint{0.581943in}{8.267587in}}%
\pgfpathlineto{\pgfqpoint{6.181004in}{8.267587in}}%
\pgfusepath{stroke}%
\end{pgfscope}%
\begin{pgfscope}%
\pgfsetbuttcap%
\pgfsetroundjoin%
\definecolor{currentfill}{rgb}{0.000000,0.000000,0.000000}%
\pgfsetfillcolor{currentfill}%
\pgfsetlinewidth{0.803000pt}%
\definecolor{currentstroke}{rgb}{0.000000,0.000000,0.000000}%
\pgfsetstrokecolor{currentstroke}%
\pgfsetdash{}{0pt}%
\pgfsys@defobject{currentmarker}{\pgfqpoint{-0.048611in}{0.000000in}}{\pgfqpoint{-0.000000in}{0.000000in}}{%
\pgfpathmoveto{\pgfqpoint{-0.000000in}{0.000000in}}%
\pgfpathlineto{\pgfqpoint{-0.048611in}{0.000000in}}%
\pgfusepath{stroke,fill}%
}%
\begin{pgfscope}%
\pgfsys@transformshift{0.581943in}{8.267587in}%
\pgfsys@useobject{currentmarker}{}%
\end{pgfscope}%
\end{pgfscope}%
\begin{pgfscope}%
\definecolor{textcolor}{rgb}{0.000000,0.000000,0.000000}%
\pgfsetstrokecolor{textcolor}%
\pgfsetfillcolor{textcolor}%
\pgftext[x=0.320563in, y=8.224184in, left, base]{\color{textcolor}\rmfamily\fontsize{9.000000}{10.800000}\selectfont \(\displaystyle {2.8}\)}%
\end{pgfscope}%
\begin{pgfscope}%
\pgfpathrectangle{\pgfqpoint{0.581943in}{0.509170in}}{\pgfqpoint{5.599061in}{8.312590in}}%
\pgfusepath{clip}%
\pgfsetbuttcap%
\pgfsetroundjoin%
\pgfsetlinewidth{0.250937pt}%
\definecolor{currentstroke}{rgb}{0.680000,0.680000,0.680000}%
\pgfsetstrokecolor{currentstroke}%
\pgfsetdash{{1.000000pt}{1.000000pt}}{0.000000pt}%
\pgfpathmoveto{\pgfqpoint{0.581943in}{8.544673in}}%
\pgfpathlineto{\pgfqpoint{6.181004in}{8.544673in}}%
\pgfusepath{stroke}%
\end{pgfscope}%
\begin{pgfscope}%
\pgfsetbuttcap%
\pgfsetroundjoin%
\definecolor{currentfill}{rgb}{0.000000,0.000000,0.000000}%
\pgfsetfillcolor{currentfill}%
\pgfsetlinewidth{0.803000pt}%
\definecolor{currentstroke}{rgb}{0.000000,0.000000,0.000000}%
\pgfsetstrokecolor{currentstroke}%
\pgfsetdash{}{0pt}%
\pgfsys@defobject{currentmarker}{\pgfqpoint{-0.048611in}{0.000000in}}{\pgfqpoint{-0.000000in}{0.000000in}}{%
\pgfpathmoveto{\pgfqpoint{-0.000000in}{0.000000in}}%
\pgfpathlineto{\pgfqpoint{-0.048611in}{0.000000in}}%
\pgfusepath{stroke,fill}%
}%
\begin{pgfscope}%
\pgfsys@transformshift{0.581943in}{8.544673in}%
\pgfsys@useobject{currentmarker}{}%
\end{pgfscope}%
\end{pgfscope}%
\begin{pgfscope}%
\definecolor{textcolor}{rgb}{0.000000,0.000000,0.000000}%
\pgfsetstrokecolor{textcolor}%
\pgfsetfillcolor{textcolor}%
\pgftext[x=0.320563in, y=8.501270in, left, base]{\color{textcolor}\rmfamily\fontsize{9.000000}{10.800000}\selectfont \(\displaystyle {2.9}\)}%
\end{pgfscope}%
\begin{pgfscope}%
\pgfpathrectangle{\pgfqpoint{0.581943in}{0.509170in}}{\pgfqpoint{5.599061in}{8.312590in}}%
\pgfusepath{clip}%
\pgfsetbuttcap%
\pgfsetroundjoin%
\pgfsetlinewidth{0.250937pt}%
\definecolor{currentstroke}{rgb}{0.680000,0.680000,0.680000}%
\pgfsetstrokecolor{currentstroke}%
\pgfsetdash{{1.000000pt}{1.000000pt}}{0.000000pt}%
\pgfpathmoveto{\pgfqpoint{0.581943in}{8.821759in}}%
\pgfpathlineto{\pgfqpoint{6.181004in}{8.821759in}}%
\pgfusepath{stroke}%
\end{pgfscope}%
\begin{pgfscope}%
\pgfsetbuttcap%
\pgfsetroundjoin%
\definecolor{currentfill}{rgb}{0.000000,0.000000,0.000000}%
\pgfsetfillcolor{currentfill}%
\pgfsetlinewidth{0.803000pt}%
\definecolor{currentstroke}{rgb}{0.000000,0.000000,0.000000}%
\pgfsetstrokecolor{currentstroke}%
\pgfsetdash{}{0pt}%
\pgfsys@defobject{currentmarker}{\pgfqpoint{-0.048611in}{0.000000in}}{\pgfqpoint{-0.000000in}{0.000000in}}{%
\pgfpathmoveto{\pgfqpoint{-0.000000in}{0.000000in}}%
\pgfpathlineto{\pgfqpoint{-0.048611in}{0.000000in}}%
\pgfusepath{stroke,fill}%
}%
\begin{pgfscope}%
\pgfsys@transformshift{0.581943in}{8.821759in}%
\pgfsys@useobject{currentmarker}{}%
\end{pgfscope}%
\end{pgfscope}%
\begin{pgfscope}%
\definecolor{textcolor}{rgb}{0.000000,0.000000,0.000000}%
\pgfsetstrokecolor{textcolor}%
\pgfsetfillcolor{textcolor}%
\pgftext[x=0.320563in, y=8.778357in, left, base]{\color{textcolor}\rmfamily\fontsize{9.000000}{10.800000}\selectfont \(\displaystyle {3.0}\)}%
\end{pgfscope}%
\begin{pgfscope}%
\definecolor{textcolor}{rgb}{0.000000,0.000000,0.000000}%
\pgfsetstrokecolor{textcolor}%
\pgfsetfillcolor{textcolor}%
\pgftext[x=0.265007in,y=4.665465in,,bottom,rotate=90.000000]{\color{textcolor}\rmfamily\fontsize{9.000000}{10.800000}\selectfont Quantili normalizzati della distribuzione del \(\displaystyle \chi^2\)}%
\end{pgfscope}%
\begin{pgfscope}%
\pgfpathrectangle{\pgfqpoint{0.581943in}{0.509170in}}{\pgfqpoint{5.599061in}{8.312590in}}%
\pgfusepath{clip}%
\pgfsetrectcap%
\pgfsetroundjoin%
\pgfsetlinewidth{1.254687pt}%
\definecolor{currentstroke}{rgb}{0.000000,0.000000,0.000000}%
\pgfsetstrokecolor{currentstroke}%
\pgfsetdash{}{0pt}%
\pgfpathmoveto{\pgfqpoint{0.581943in}{0.509174in}}%
\pgfpathlineto{\pgfqpoint{1.412265in}{0.510465in}}%
\pgfpathlineto{\pgfqpoint{1.659117in}{0.513169in}}%
\pgfpathlineto{\pgfqpoint{1.838646in}{0.517208in}}%
\pgfpathlineto{\pgfqpoint{1.995734in}{0.522970in}}%
\pgfpathlineto{\pgfqpoint{2.130381in}{0.530097in}}%
\pgfpathlineto{\pgfqpoint{2.253807in}{0.538795in}}%
\pgfpathlineto{\pgfqpoint{2.366013in}{0.548765in}}%
\pgfpathlineto{\pgfqpoint{2.478218in}{0.560919in}}%
\pgfpathlineto{\pgfqpoint{2.590424in}{0.575428in}}%
\pgfpathlineto{\pgfqpoint{2.691409in}{0.590611in}}%
\pgfpathlineto{\pgfqpoint{2.792394in}{0.607877in}}%
\pgfpathlineto{\pgfqpoint{2.893379in}{0.627265in}}%
\pgfpathlineto{\pgfqpoint{2.994364in}{0.648784in}}%
\pgfpathlineto{\pgfqpoint{3.095349in}{0.672415in}}%
\pgfpathlineto{\pgfqpoint{3.207555in}{0.701098in}}%
\pgfpathlineto{\pgfqpoint{3.319760in}{0.732247in}}%
\pgfpathlineto{\pgfqpoint{3.431966in}{0.765744in}}%
\pgfpathlineto{\pgfqpoint{3.544172in}{0.801446in}}%
\pgfpathlineto{\pgfqpoint{3.667598in}{0.843077in}}%
\pgfpathlineto{\pgfqpoint{3.802245in}{0.891047in}}%
\pgfpathlineto{\pgfqpoint{3.936891in}{0.941379in}}%
\pgfpathlineto{\pgfqpoint{4.082759in}{0.998200in}}%
\pgfpathlineto{\pgfqpoint{4.251067in}{1.066213in}}%
\pgfpathlineto{\pgfqpoint{4.441817in}{1.145739in}}%
\pgfpathlineto{\pgfqpoint{4.688669in}{1.251218in}}%
\pgfpathlineto{\pgfqpoint{5.563873in}{1.627510in}}%
\pgfpathlineto{\pgfqpoint{5.788285in}{1.720442in}}%
\pgfpathlineto{\pgfqpoint{6.001475in}{1.806306in}}%
\pgfpathlineto{\pgfqpoint{6.181004in}{1.876530in}}%
\pgfpathlineto{\pgfqpoint{6.181004in}{1.876530in}}%
\pgfusepath{stroke}%
\end{pgfscope}%
\begin{pgfscope}%
\pgfpathrectangle{\pgfqpoint{0.581943in}{0.509170in}}{\pgfqpoint{5.599061in}{8.312590in}}%
\pgfusepath{clip}%
\pgfsetrectcap%
\pgfsetroundjoin%
\pgfsetlinewidth{1.254687pt}%
\definecolor{currentstroke}{rgb}{0.000000,0.000000,0.000000}%
\pgfsetstrokecolor{currentstroke}%
\pgfsetdash{}{0pt}%
\pgfpathmoveto{\pgfqpoint{0.581943in}{0.509605in}}%
\pgfpathlineto{\pgfqpoint{0.851236in}{0.510740in}}%
\pgfpathlineto{\pgfqpoint{1.053207in}{0.513658in}}%
\pgfpathlineto{\pgfqpoint{1.221515in}{0.518129in}}%
\pgfpathlineto{\pgfqpoint{1.367382in}{0.524071in}}%
\pgfpathlineto{\pgfqpoint{1.502029in}{0.531795in}}%
\pgfpathlineto{\pgfqpoint{1.614235in}{0.540304in}}%
\pgfpathlineto{\pgfqpoint{1.726440in}{0.551022in}}%
\pgfpathlineto{\pgfqpoint{1.838646in}{0.564150in}}%
\pgfpathlineto{\pgfqpoint{1.939631in}{0.578149in}}%
\pgfpathlineto{\pgfqpoint{2.040616in}{0.594322in}}%
\pgfpathlineto{\pgfqpoint{2.141601in}{0.612747in}}%
\pgfpathlineto{\pgfqpoint{2.242586in}{0.633454in}}%
\pgfpathlineto{\pgfqpoint{2.343571in}{0.656454in}}%
\pgfpathlineto{\pgfqpoint{2.444556in}{0.681701in}}%
\pgfpathlineto{\pgfqpoint{2.545542in}{0.709143in}}%
\pgfpathlineto{\pgfqpoint{2.657747in}{0.742099in}}%
\pgfpathlineto{\pgfqpoint{2.769953in}{0.777512in}}%
\pgfpathlineto{\pgfqpoint{2.882158in}{0.815211in}}%
\pgfpathlineto{\pgfqpoint{3.005585in}{0.859093in}}%
\pgfpathlineto{\pgfqpoint{3.140231in}{0.909544in}}%
\pgfpathlineto{\pgfqpoint{3.274878in}{0.962339in}}%
\pgfpathlineto{\pgfqpoint{3.431966in}{1.026412in}}%
\pgfpathlineto{\pgfqpoint{3.611495in}{1.102225in}}%
\pgfpathlineto{\pgfqpoint{3.824686in}{1.194790in}}%
\pgfpathlineto{\pgfqpoint{4.161303in}{1.343788in}}%
\pgfpathlineto{\pgfqpoint{4.587684in}{1.532112in}}%
\pgfpathlineto{\pgfqpoint{4.834537in}{1.638686in}}%
\pgfpathlineto{\pgfqpoint{5.047727in}{1.728407in}}%
\pgfpathlineto{\pgfqpoint{5.249697in}{1.811008in}}%
\pgfpathlineto{\pgfqpoint{5.440447in}{1.886624in}}%
\pgfpathlineto{\pgfqpoint{5.619976in}{1.955502in}}%
\pgfpathlineto{\pgfqpoint{5.799505in}{2.022051in}}%
\pgfpathlineto{\pgfqpoint{5.979034in}{2.086196in}}%
\pgfpathlineto{\pgfqpoint{6.158563in}{2.147894in}}%
\pgfpathlineto{\pgfqpoint{6.181004in}{2.155433in}}%
\pgfpathlineto{\pgfqpoint{6.181004in}{2.155433in}}%
\pgfusepath{stroke}%
\end{pgfscope}%
\begin{pgfscope}%
\pgfpathrectangle{\pgfqpoint{0.581943in}{0.509170in}}{\pgfqpoint{5.599061in}{8.312590in}}%
\pgfusepath{clip}%
\pgfsetrectcap%
\pgfsetroundjoin%
\pgfsetlinewidth{1.254687pt}%
\definecolor{currentstroke}{rgb}{0.000000,0.000000,0.000000}%
\pgfsetstrokecolor{currentstroke}%
\pgfsetdash{}{0pt}%
\pgfpathmoveto{\pgfqpoint{0.581943in}{0.520065in}}%
\pgfpathlineto{\pgfqpoint{0.705369in}{0.525956in}}%
\pgfpathlineto{\pgfqpoint{0.828795in}{0.534148in}}%
\pgfpathlineto{\pgfqpoint{0.941001in}{0.543888in}}%
\pgfpathlineto{\pgfqpoint{1.041986in}{0.554752in}}%
\pgfpathlineto{\pgfqpoint{1.142971in}{0.567799in}}%
\pgfpathlineto{\pgfqpoint{1.243956in}{0.583204in}}%
\pgfpathlineto{\pgfqpoint{1.333721in}{0.598990in}}%
\pgfpathlineto{\pgfqpoint{1.423485in}{0.616814in}}%
\pgfpathlineto{\pgfqpoint{1.513250in}{0.636695in}}%
\pgfpathlineto{\pgfqpoint{1.614235in}{0.661464in}}%
\pgfpathlineto{\pgfqpoint{1.715220in}{0.688729in}}%
\pgfpathlineto{\pgfqpoint{1.816205in}{0.718459in}}%
\pgfpathlineto{\pgfqpoint{1.917190in}{0.750523in}}%
\pgfpathlineto{\pgfqpoint{2.018175in}{0.784764in}}%
\pgfpathlineto{\pgfqpoint{2.130381in}{0.825210in}}%
\pgfpathlineto{\pgfqpoint{2.242586in}{0.867919in}}%
\pgfpathlineto{\pgfqpoint{2.366013in}{0.917237in}}%
\pgfpathlineto{\pgfqpoint{2.500659in}{0.973427in}}%
\pgfpathlineto{\pgfqpoint{2.657747in}{1.041568in}}%
\pgfpathlineto{\pgfqpoint{2.837276in}{1.122024in}}%
\pgfpathlineto{\pgfqpoint{3.072908in}{1.230294in}}%
\pgfpathlineto{\pgfqpoint{3.791024in}{1.562056in}}%
\pgfpathlineto{\pgfqpoint{4.004215in}{1.657276in}}%
\pgfpathlineto{\pgfqpoint{4.206185in}{1.745021in}}%
\pgfpathlineto{\pgfqpoint{4.396935in}{1.825356in}}%
\pgfpathlineto{\pgfqpoint{4.576464in}{1.898508in}}%
\pgfpathlineto{\pgfqpoint{4.744772in}{1.964799in}}%
\pgfpathlineto{\pgfqpoint{4.913081in}{2.028798in}}%
\pgfpathlineto{\pgfqpoint{5.081389in}{2.090456in}}%
\pgfpathlineto{\pgfqpoint{5.249697in}{2.149749in}}%
\pgfpathlineto{\pgfqpoint{5.418006in}{2.206674in}}%
\pgfpathlineto{\pgfqpoint{5.586314in}{2.261244in}}%
\pgfpathlineto{\pgfqpoint{5.754623in}{2.313488in}}%
\pgfpathlineto{\pgfqpoint{5.922931in}{2.363443in}}%
\pgfpathlineto{\pgfqpoint{6.091240in}{2.411159in}}%
\pgfpathlineto{\pgfqpoint{6.181004in}{2.435709in}}%
\pgfpathlineto{\pgfqpoint{6.181004in}{2.435709in}}%
\pgfusepath{stroke}%
\end{pgfscope}%
\begin{pgfscope}%
\pgfpathrectangle{\pgfqpoint{0.581943in}{0.509170in}}{\pgfqpoint{5.599061in}{8.312590in}}%
\pgfusepath{clip}%
\pgfsetrectcap%
\pgfsetroundjoin%
\pgfsetlinewidth{1.254687pt}%
\definecolor{currentstroke}{rgb}{0.000000,0.000000,0.000000}%
\pgfsetstrokecolor{currentstroke}%
\pgfsetdash{}{0pt}%
\pgfpathmoveto{\pgfqpoint{0.581943in}{0.620197in}}%
\pgfpathlineto{\pgfqpoint{0.682928in}{0.646548in}}%
\pgfpathlineto{\pgfqpoint{0.783913in}{0.675276in}}%
\pgfpathlineto{\pgfqpoint{0.884898in}{0.706508in}}%
\pgfpathlineto{\pgfqpoint{0.974663in}{0.736456in}}%
\pgfpathlineto{\pgfqpoint{1.064427in}{0.768512in}}%
\pgfpathlineto{\pgfqpoint{1.154192in}{0.802692in}}%
\pgfpathlineto{\pgfqpoint{1.255177in}{0.843641in}}%
\pgfpathlineto{\pgfqpoint{1.356162in}{0.887100in}}%
\pgfpathlineto{\pgfqpoint{1.468367in}{0.937997in}}%
\pgfpathlineto{\pgfqpoint{1.603014in}{1.001816in}}%
\pgfpathlineto{\pgfqpoint{1.760102in}{1.078735in}}%
\pgfpathlineto{\pgfqpoint{1.939631in}{1.169138in}}%
\pgfpathlineto{\pgfqpoint{2.231366in}{1.319094in}}%
\pgfpathlineto{\pgfqpoint{2.556762in}{1.485608in}}%
\pgfpathlineto{\pgfqpoint{2.758732in}{1.586587in}}%
\pgfpathlineto{\pgfqpoint{2.938261in}{1.674006in}}%
\pgfpathlineto{\pgfqpoint{3.106570in}{1.753575in}}%
\pgfpathlineto{\pgfqpoint{3.263658in}{1.825532in}}%
\pgfpathlineto{\pgfqpoint{3.420746in}{1.895112in}}%
\pgfpathlineto{\pgfqpoint{3.577833in}{1.962218in}}%
\pgfpathlineto{\pgfqpoint{3.734921in}{2.026793in}}%
\pgfpathlineto{\pgfqpoint{3.892009in}{2.088810in}}%
\pgfpathlineto{\pgfqpoint{4.049097in}{2.148269in}}%
\pgfpathlineto{\pgfqpoint{4.206185in}{2.205190in}}%
\pgfpathlineto{\pgfqpoint{4.363273in}{2.259611in}}%
\pgfpathlineto{\pgfqpoint{4.520361in}{2.311582in}}%
\pgfpathlineto{\pgfqpoint{4.677449in}{2.361164in}}%
\pgfpathlineto{\pgfqpoint{4.834537in}{2.408426in}}%
\pgfpathlineto{\pgfqpoint{4.991625in}{2.453440in}}%
\pgfpathlineto{\pgfqpoint{5.159933in}{2.499264in}}%
\pgfpathlineto{\pgfqpoint{5.328241in}{2.542696in}}%
\pgfpathlineto{\pgfqpoint{5.496550in}{2.583835in}}%
\pgfpathlineto{\pgfqpoint{5.664858in}{2.622783in}}%
\pgfpathlineto{\pgfqpoint{5.844387in}{2.662022in}}%
\pgfpathlineto{\pgfqpoint{6.023916in}{2.699000in}}%
\pgfpathlineto{\pgfqpoint{6.181004in}{2.729592in}}%
\pgfpathlineto{\pgfqpoint{6.181004in}{2.729592in}}%
\pgfusepath{stroke}%
\end{pgfscope}%
\begin{pgfscope}%
\pgfpathrectangle{\pgfqpoint{0.581943in}{0.509170in}}{\pgfqpoint{5.599061in}{8.312590in}}%
\pgfusepath{clip}%
\pgfsetrectcap%
\pgfsetroundjoin%
\pgfsetlinewidth{1.254687pt}%
\definecolor{currentstroke}{rgb}{0.000000,0.000000,0.000000}%
\pgfsetstrokecolor{currentstroke}%
\pgfsetdash{}{0pt}%
\pgfpathmoveto{\pgfqpoint{0.581943in}{0.790498in}}%
\pgfpathlineto{\pgfqpoint{0.694148in}{0.840864in}}%
\pgfpathlineto{\pgfqpoint{0.806354in}{0.893652in}}%
\pgfpathlineto{\pgfqpoint{0.918560in}{0.948784in}}%
\pgfpathlineto{\pgfqpoint{1.041986in}{1.011980in}}%
\pgfpathlineto{\pgfqpoint{1.176633in}{1.083667in}}%
\pgfpathlineto{\pgfqpoint{1.322500in}{1.164001in}}%
\pgfpathlineto{\pgfqpoint{1.524470in}{1.278142in}}%
\pgfpathlineto{\pgfqpoint{1.849867in}{1.462016in}}%
\pgfpathlineto{\pgfqpoint{2.063057in}{1.579756in}}%
\pgfpathlineto{\pgfqpoint{2.231366in}{1.670134in}}%
\pgfpathlineto{\pgfqpoint{2.388454in}{1.751877in}}%
\pgfpathlineto{\pgfqpoint{2.534321in}{1.825309in}}%
\pgfpathlineto{\pgfqpoint{2.680188in}{1.896178in}}%
\pgfpathlineto{\pgfqpoint{2.814835in}{1.959227in}}%
\pgfpathlineto{\pgfqpoint{2.949482in}{2.019953in}}%
\pgfpathlineto{\pgfqpoint{3.084129in}{2.078336in}}%
\pgfpathlineto{\pgfqpoint{3.218775in}{2.134379in}}%
\pgfpathlineto{\pgfqpoint{3.353422in}{2.188104in}}%
\pgfpathlineto{\pgfqpoint{3.488069in}{2.239546in}}%
\pgfpathlineto{\pgfqpoint{3.622716in}{2.288753in}}%
\pgfpathlineto{\pgfqpoint{3.768583in}{2.339604in}}%
\pgfpathlineto{\pgfqpoint{3.914450in}{2.387979in}}%
\pgfpathlineto{\pgfqpoint{4.060318in}{2.433964in}}%
\pgfpathlineto{\pgfqpoint{4.206185in}{2.477651in}}%
\pgfpathlineto{\pgfqpoint{4.352052in}{2.519132in}}%
\pgfpathlineto{\pgfqpoint{4.509140in}{2.561446in}}%
\pgfpathlineto{\pgfqpoint{4.666228in}{2.601428in}}%
\pgfpathlineto{\pgfqpoint{4.823316in}{2.639194in}}%
\pgfpathlineto{\pgfqpoint{4.991625in}{2.677326in}}%
\pgfpathlineto{\pgfqpoint{5.159933in}{2.713179in}}%
\pgfpathlineto{\pgfqpoint{5.339462in}{2.749055in}}%
\pgfpathlineto{\pgfqpoint{5.518991in}{2.782637in}}%
\pgfpathlineto{\pgfqpoint{5.709741in}{2.815963in}}%
\pgfpathlineto{\pgfqpoint{5.900490in}{2.847023in}}%
\pgfpathlineto{\pgfqpoint{6.102460in}{2.877610in}}%
\pgfpathlineto{\pgfqpoint{6.181004in}{2.888903in}}%
\pgfpathlineto{\pgfqpoint{6.181004in}{2.888903in}}%
\pgfusepath{stroke}%
\end{pgfscope}%
\begin{pgfscope}%
\pgfpathrectangle{\pgfqpoint{0.581943in}{0.509170in}}{\pgfqpoint{5.599061in}{8.312590in}}%
\pgfusepath{clip}%
\pgfsetrectcap%
\pgfsetroundjoin%
\pgfsetlinewidth{1.254687pt}%
\definecolor{currentstroke}{rgb}{0.000000,0.000000,0.000000}%
\pgfsetstrokecolor{currentstroke}%
\pgfsetdash{}{0pt}%
\pgfpathmoveto{\pgfqpoint{0.581943in}{4.175863in}}%
\pgfpathlineto{\pgfqpoint{0.682928in}{4.211141in}}%
\pgfpathlineto{\pgfqpoint{0.772692in}{4.240060in}}%
\pgfpathlineto{\pgfqpoint{0.851236in}{4.263164in}}%
\pgfpathlineto{\pgfqpoint{0.929780in}{4.283949in}}%
\pgfpathlineto{\pgfqpoint{1.008324in}{4.302180in}}%
\pgfpathlineto{\pgfqpoint{1.075648in}{4.315619in}}%
\pgfpathlineto{\pgfqpoint{1.142971in}{4.326933in}}%
\pgfpathlineto{\pgfqpoint{1.210294in}{4.336056in}}%
\pgfpathlineto{\pgfqpoint{1.277618in}{4.342972in}}%
\pgfpathlineto{\pgfqpoint{1.344941in}{4.347728in}}%
\pgfpathlineto{\pgfqpoint{1.412265in}{4.350448in}}%
\pgfpathlineto{\pgfqpoint{1.490809in}{4.351360in}}%
\pgfpathlineto{\pgfqpoint{1.580573in}{4.350148in}}%
\pgfpathlineto{\pgfqpoint{1.692779in}{4.346329in}}%
\pgfpathlineto{\pgfqpoint{1.793764in}{4.340694in}}%
\pgfpathlineto{\pgfqpoint{1.894749in}{4.332900in}}%
\pgfpathlineto{\pgfqpoint{1.995734in}{4.322982in}}%
\pgfpathlineto{\pgfqpoint{2.107940in}{4.309740in}}%
\pgfpathlineto{\pgfqpoint{2.242586in}{4.291505in}}%
\pgfpathlineto{\pgfqpoint{2.388454in}{4.269371in}}%
\pgfpathlineto{\pgfqpoint{2.545542in}{4.243197in}}%
\pgfpathlineto{\pgfqpoint{2.736291in}{4.209005in}}%
\pgfpathlineto{\pgfqpoint{2.971923in}{4.164289in}}%
\pgfpathlineto{\pgfqpoint{3.342202in}{4.091339in}}%
\pgfpathlineto{\pgfqpoint{3.858348in}{3.989968in}}%
\pgfpathlineto{\pgfqpoint{4.161303in}{3.932828in}}%
\pgfpathlineto{\pgfqpoint{4.430596in}{3.884259in}}%
\pgfpathlineto{\pgfqpoint{4.688669in}{3.839972in}}%
\pgfpathlineto{\pgfqpoint{4.935522in}{3.799823in}}%
\pgfpathlineto{\pgfqpoint{5.182374in}{3.761905in}}%
\pgfpathlineto{\pgfqpoint{5.429227in}{3.726232in}}%
\pgfpathlineto{\pgfqpoint{5.676079in}{3.692778in}}%
\pgfpathlineto{\pgfqpoint{5.922931in}{3.661493in}}%
\pgfpathlineto{\pgfqpoint{6.181004in}{3.631024in}}%
\pgfpathlineto{\pgfqpoint{6.181004in}{3.631024in}}%
\pgfusepath{stroke}%
\end{pgfscope}%
\begin{pgfscope}%
\pgfpathrectangle{\pgfqpoint{0.581943in}{0.509170in}}{\pgfqpoint{5.599061in}{8.312590in}}%
\pgfusepath{clip}%
\pgfsetrectcap%
\pgfsetroundjoin%
\pgfsetlinewidth{1.254687pt}%
\definecolor{currentstroke}{rgb}{0.000000,0.000000,0.000000}%
\pgfsetstrokecolor{currentstroke}%
\pgfsetdash{}{0pt}%
\pgfpathmoveto{\pgfqpoint{0.581943in}{6.014866in}}%
\pgfpathlineto{\pgfqpoint{0.671707in}{5.989973in}}%
\pgfpathlineto{\pgfqpoint{0.761472in}{5.962840in}}%
\pgfpathlineto{\pgfqpoint{0.851236in}{5.933353in}}%
\pgfpathlineto{\pgfqpoint{0.941001in}{5.901423in}}%
\pgfpathlineto{\pgfqpoint{1.030765in}{5.866991in}}%
\pgfpathlineto{\pgfqpoint{1.120530in}{5.830043in}}%
\pgfpathlineto{\pgfqpoint{1.210294in}{5.790628in}}%
\pgfpathlineto{\pgfqpoint{1.300059in}{5.748876in}}%
\pgfpathlineto{\pgfqpoint{1.401044in}{5.699417in}}%
\pgfpathlineto{\pgfqpoint{1.524470in}{5.636213in}}%
\pgfpathlineto{\pgfqpoint{1.737661in}{5.523852in}}%
\pgfpathlineto{\pgfqpoint{1.984513in}{5.391253in}}%
\pgfpathlineto{\pgfqpoint{2.377233in}{5.179774in}}%
\pgfpathlineto{\pgfqpoint{2.579203in}{5.073613in}}%
\pgfpathlineto{\pgfqpoint{2.758732in}{4.981666in}}%
\pgfpathlineto{\pgfqpoint{2.927041in}{4.897925in}}%
\pgfpathlineto{\pgfqpoint{3.084129in}{4.822153in}}%
\pgfpathlineto{\pgfqpoint{3.241217in}{4.748838in}}%
\pgfpathlineto{\pgfqpoint{3.398304in}{4.678083in}}%
\pgfpathlineto{\pgfqpoint{3.544172in}{4.614734in}}%
\pgfpathlineto{\pgfqpoint{3.690039in}{4.553674in}}%
\pgfpathlineto{\pgfqpoint{3.835906in}{4.494910in}}%
\pgfpathlineto{\pgfqpoint{3.981774in}{4.438432in}}%
\pgfpathlineto{\pgfqpoint{4.127641in}{4.384213in}}%
\pgfpathlineto{\pgfqpoint{4.273508in}{4.332219in}}%
\pgfpathlineto{\pgfqpoint{4.430596in}{4.278662in}}%
\pgfpathlineto{\pgfqpoint{4.587684in}{4.227568in}}%
\pgfpathlineto{\pgfqpoint{4.744772in}{4.178865in}}%
\pgfpathlineto{\pgfqpoint{4.901860in}{4.132477in}}%
\pgfpathlineto{\pgfqpoint{5.058948in}{4.088323in}}%
\pgfpathlineto{\pgfqpoint{5.227256in}{4.043400in}}%
\pgfpathlineto{\pgfqpoint{5.395565in}{4.000844in}}%
\pgfpathlineto{\pgfqpoint{5.563873in}{3.960552in}}%
\pgfpathlineto{\pgfqpoint{5.743402in}{3.919952in}}%
\pgfpathlineto{\pgfqpoint{5.922931in}{3.881689in}}%
\pgfpathlineto{\pgfqpoint{6.102460in}{3.845643in}}%
\pgfpathlineto{\pgfqpoint{6.181004in}{3.830541in}}%
\pgfpathlineto{\pgfqpoint{6.181004in}{3.830541in}}%
\pgfusepath{stroke}%
\end{pgfscope}%
\begin{pgfscope}%
\pgfpathrectangle{\pgfqpoint{0.581943in}{0.509170in}}{\pgfqpoint{5.599061in}{8.312590in}}%
\pgfusepath{clip}%
\pgfsetrectcap%
\pgfsetroundjoin%
\pgfsetlinewidth{1.254687pt}%
\definecolor{currentstroke}{rgb}{0.000000,0.000000,0.000000}%
\pgfsetstrokecolor{currentstroke}%
\pgfsetdash{}{0pt}%
\pgfpathmoveto{\pgfqpoint{1.563407in}{8.831759in}}%
\pgfpathlineto{\pgfqpoint{1.659117in}{8.638460in}}%
\pgfpathlineto{\pgfqpoint{1.748881in}{8.461763in}}%
\pgfpathlineto{\pgfqpoint{1.838646in}{8.290016in}}%
\pgfpathlineto{\pgfqpoint{1.928411in}{8.123393in}}%
\pgfpathlineto{\pgfqpoint{2.006954in}{7.981891in}}%
\pgfpathlineto{\pgfqpoint{2.085498in}{7.844418in}}%
\pgfpathlineto{\pgfqpoint{2.175263in}{7.691864in}}%
\pgfpathlineto{\pgfqpoint{2.265027in}{7.543913in}}%
\pgfpathlineto{\pgfqpoint{2.354792in}{7.400667in}}%
\pgfpathlineto{\pgfqpoint{2.444556in}{7.261989in}}%
\pgfpathlineto{\pgfqpoint{2.534321in}{7.127862in}}%
\pgfpathlineto{\pgfqpoint{2.624085in}{6.998079in}}%
\pgfpathlineto{\pgfqpoint{2.713850in}{6.872578in}}%
\pgfpathlineto{\pgfqpoint{2.803615in}{6.751220in}}%
\pgfpathlineto{\pgfqpoint{2.893379in}{6.633923in}}%
\pgfpathlineto{\pgfqpoint{2.983144in}{6.520554in}}%
\pgfpathlineto{\pgfqpoint{3.072908in}{6.411004in}}%
\pgfpathlineto{\pgfqpoint{3.162673in}{6.305166in}}%
\pgfpathlineto{\pgfqpoint{3.252437in}{6.202926in}}%
\pgfpathlineto{\pgfqpoint{3.342202in}{6.104172in}}%
\pgfpathlineto{\pgfqpoint{3.431966in}{6.008796in}}%
\pgfpathlineto{\pgfqpoint{3.521731in}{5.916692in}}%
\pgfpathlineto{\pgfqpoint{3.611495in}{5.827754in}}%
\pgfpathlineto{\pgfqpoint{3.701260in}{5.741880in}}%
\pgfpathlineto{\pgfqpoint{3.791024in}{5.658969in}}%
\pgfpathlineto{\pgfqpoint{3.880789in}{5.578921in}}%
\pgfpathlineto{\pgfqpoint{3.970553in}{5.501640in}}%
\pgfpathlineto{\pgfqpoint{4.060318in}{5.427034in}}%
\pgfpathlineto{\pgfqpoint{4.161303in}{5.346185in}}%
\pgfpathlineto{\pgfqpoint{4.262288in}{5.268478in}}%
\pgfpathlineto{\pgfqpoint{4.363273in}{5.193792in}}%
\pgfpathlineto{\pgfqpoint{4.464258in}{5.122009in}}%
\pgfpathlineto{\pgfqpoint{4.565243in}{5.053014in}}%
\pgfpathlineto{\pgfqpoint{4.666228in}{4.986699in}}%
\pgfpathlineto{\pgfqpoint{4.767213in}{4.922956in}}%
\pgfpathlineto{\pgfqpoint{4.868198in}{4.861685in}}%
\pgfpathlineto{\pgfqpoint{4.969183in}{4.802786in}}%
\pgfpathlineto{\pgfqpoint{5.081389in}{4.740012in}}%
\pgfpathlineto{\pgfqpoint{5.193595in}{4.679925in}}%
\pgfpathlineto{\pgfqpoint{5.305800in}{4.622408in}}%
\pgfpathlineto{\pgfqpoint{5.418006in}{4.567346in}}%
\pgfpathlineto{\pgfqpoint{5.530212in}{4.514631in}}%
\pgfpathlineto{\pgfqpoint{5.653638in}{4.459231in}}%
\pgfpathlineto{\pgfqpoint{5.777064in}{4.406412in}}%
\pgfpathlineto{\pgfqpoint{5.900490in}{4.356049in}}%
\pgfpathlineto{\pgfqpoint{6.023916in}{4.308024in}}%
\pgfpathlineto{\pgfqpoint{6.147343in}{4.262221in}}%
\pgfpathlineto{\pgfqpoint{6.181004in}{4.250101in}}%
\pgfpathlineto{\pgfqpoint{6.181004in}{4.250101in}}%
\pgfusepath{stroke}%
\end{pgfscope}%
\begin{pgfscope}%
\pgfpathrectangle{\pgfqpoint{0.581943in}{0.509170in}}{\pgfqpoint{5.599061in}{8.312590in}}%
\pgfusepath{clip}%
\pgfsetrectcap%
\pgfsetroundjoin%
\pgfsetlinewidth{1.254687pt}%
\definecolor{currentstroke}{rgb}{0.000000,0.000000,0.000000}%
\pgfsetstrokecolor{currentstroke}%
\pgfsetdash{}{0pt}%
\pgfpathmoveto{\pgfqpoint{2.901075in}{8.831759in}}%
\pgfpathlineto{\pgfqpoint{2.960702in}{8.689997in}}%
\pgfpathlineto{\pgfqpoint{3.028026in}{8.534644in}}%
\pgfpathlineto{\pgfqpoint{3.095349in}{8.384097in}}%
\pgfpathlineto{\pgfqpoint{3.173893in}{8.214333in}}%
\pgfpathlineto{\pgfqpoint{3.252437in}{8.050674in}}%
\pgfpathlineto{\pgfqpoint{3.330981in}{7.892880in}}%
\pgfpathlineto{\pgfqpoint{3.409525in}{7.740735in}}%
\pgfpathlineto{\pgfqpoint{3.488069in}{7.594027in}}%
\pgfpathlineto{\pgfqpoint{3.566613in}{7.452550in}}%
\pgfpathlineto{\pgfqpoint{3.645157in}{7.316106in}}%
\pgfpathlineto{\pgfqpoint{3.723701in}{7.184505in}}%
\pgfpathlineto{\pgfqpoint{3.802245in}{7.057562in}}%
\pgfpathlineto{\pgfqpoint{3.880789in}{6.935103in}}%
\pgfpathlineto{\pgfqpoint{3.959333in}{6.816958in}}%
\pgfpathlineto{\pgfqpoint{4.037877in}{6.702964in}}%
\pgfpathlineto{\pgfqpoint{4.116421in}{6.592964in}}%
\pgfpathlineto{\pgfqpoint{4.194964in}{6.486806in}}%
\pgfpathlineto{\pgfqpoint{4.273508in}{6.384348in}}%
\pgfpathlineto{\pgfqpoint{4.352052in}{6.285448in}}%
\pgfpathlineto{\pgfqpoint{4.430596in}{6.189973in}}%
\pgfpathlineto{\pgfqpoint{4.509140in}{6.097795in}}%
\pgfpathlineto{\pgfqpoint{4.587684in}{6.008790in}}%
\pgfpathlineto{\pgfqpoint{4.666228in}{5.922839in}}%
\pgfpathlineto{\pgfqpoint{4.744772in}{5.839828in}}%
\pgfpathlineto{\pgfqpoint{4.834537in}{5.748419in}}%
\pgfpathlineto{\pgfqpoint{4.924301in}{5.660551in}}%
\pgfpathlineto{\pgfqpoint{5.014066in}{5.576074in}}%
\pgfpathlineto{\pgfqpoint{5.103830in}{5.494845in}}%
\pgfpathlineto{\pgfqpoint{5.193595in}{5.416727in}}%
\pgfpathlineto{\pgfqpoint{5.283359in}{5.341592in}}%
\pgfpathlineto{\pgfqpoint{5.373124in}{5.269313in}}%
\pgfpathlineto{\pgfqpoint{5.462888in}{5.199773in}}%
\pgfpathlineto{\pgfqpoint{5.552653in}{5.132857in}}%
\pgfpathlineto{\pgfqpoint{5.642417in}{5.068457in}}%
\pgfpathlineto{\pgfqpoint{5.743402in}{4.998887in}}%
\pgfpathlineto{\pgfqpoint{5.844387in}{4.932229in}}%
\pgfpathlineto{\pgfqpoint{5.945372in}{4.868351in}}%
\pgfpathlineto{\pgfqpoint{6.046358in}{4.807125in}}%
\pgfpathlineto{\pgfqpoint{6.147343in}{4.748432in}}%
\pgfpathlineto{\pgfqpoint{6.181004in}{4.729410in}}%
\pgfpathlineto{\pgfqpoint{6.181004in}{4.729410in}}%
\pgfusepath{stroke}%
\end{pgfscope}%
\begin{pgfscope}%
\pgfpathrectangle{\pgfqpoint{0.581943in}{0.509170in}}{\pgfqpoint{5.599061in}{8.312590in}}%
\pgfusepath{clip}%
\pgfsetrectcap%
\pgfsetroundjoin%
\pgfsetlinewidth{1.254687pt}%
\definecolor{currentstroke}{rgb}{0.000000,0.000000,0.000000}%
\pgfsetstrokecolor{currentstroke}%
\pgfsetdash{}{0pt}%
\pgfpathmoveto{\pgfqpoint{3.827533in}{8.831759in}}%
\pgfpathlineto{\pgfqpoint{3.892009in}{8.672255in}}%
\pgfpathlineto{\pgfqpoint{3.959333in}{8.511166in}}%
\pgfpathlineto{\pgfqpoint{4.026656in}{8.355435in}}%
\pgfpathlineto{\pgfqpoint{4.093979in}{8.204867in}}%
\pgfpathlineto{\pgfqpoint{4.161303in}{8.059273in}}%
\pgfpathlineto{\pgfqpoint{4.228626in}{7.918473in}}%
\pgfpathlineto{\pgfqpoint{4.295950in}{7.782293in}}%
\pgfpathlineto{\pgfqpoint{4.363273in}{7.650566in}}%
\pgfpathlineto{\pgfqpoint{4.430596in}{7.523131in}}%
\pgfpathlineto{\pgfqpoint{4.497920in}{7.399832in}}%
\pgfpathlineto{\pgfqpoint{4.565243in}{7.280522in}}%
\pgfpathlineto{\pgfqpoint{4.643787in}{7.146178in}}%
\pgfpathlineto{\pgfqpoint{4.722331in}{7.016851in}}%
\pgfpathlineto{\pgfqpoint{4.800875in}{6.892333in}}%
\pgfpathlineto{\pgfqpoint{4.879419in}{6.772426in}}%
\pgfpathlineto{\pgfqpoint{4.957963in}{6.656939in}}%
\pgfpathlineto{\pgfqpoint{5.036507in}{6.545692in}}%
\pgfpathlineto{\pgfqpoint{5.115051in}{6.438511in}}%
\pgfpathlineto{\pgfqpoint{5.193595in}{6.335231in}}%
\pgfpathlineto{\pgfqpoint{5.272139in}{6.235694in}}%
\pgfpathlineto{\pgfqpoint{5.350683in}{6.139747in}}%
\pgfpathlineto{\pgfqpoint{5.429227in}{6.047248in}}%
\pgfpathlineto{\pgfqpoint{5.507770in}{5.958057in}}%
\pgfpathlineto{\pgfqpoint{5.586314in}{5.872042in}}%
\pgfpathlineto{\pgfqpoint{5.664858in}{5.789078in}}%
\pgfpathlineto{\pgfqpoint{5.743402in}{5.709043in}}%
\pgfpathlineto{\pgfqpoint{5.821946in}{5.631821in}}%
\pgfpathlineto{\pgfqpoint{5.911711in}{5.546872in}}%
\pgfpathlineto{\pgfqpoint{6.001475in}{5.465297in}}%
\pgfpathlineto{\pgfqpoint{6.091240in}{5.386945in}}%
\pgfpathlineto{\pgfqpoint{6.181004in}{5.311675in}}%
\pgfpathlineto{\pgfqpoint{6.181004in}{5.311675in}}%
\pgfusepath{stroke}%
\end{pgfscope}%
\begin{pgfscope}%
\pgfpathrectangle{\pgfqpoint{0.581943in}{0.509170in}}{\pgfqpoint{5.599061in}{8.312590in}}%
\pgfusepath{clip}%
\pgfsetbuttcap%
\pgfsetroundjoin%
\pgfsetlinewidth{1.254687pt}%
\definecolor{currentstroke}{rgb}{0.501961,0.501961,0.501961}%
\pgfsetstrokecolor{currentstroke}%
\pgfsetdash{{4.625000pt}{2.000000pt}}{0.000000pt}%
\pgfpathmoveto{\pgfqpoint{0.581943in}{7.198625in}}%
\pgfpathlineto{\pgfqpoint{0.671707in}{7.077649in}}%
\pgfpathlineto{\pgfqpoint{0.761472in}{6.960408in}}%
\pgfpathlineto{\pgfqpoint{0.851236in}{6.846786in}}%
\pgfpathlineto{\pgfqpoint{0.941001in}{6.736672in}}%
\pgfpathlineto{\pgfqpoint{1.030765in}{6.629958in}}%
\pgfpathlineto{\pgfqpoint{1.120530in}{6.526538in}}%
\pgfpathlineto{\pgfqpoint{1.210294in}{6.426311in}}%
\pgfpathlineto{\pgfqpoint{1.300059in}{6.329178in}}%
\pgfpathlineto{\pgfqpoint{1.389823in}{6.235044in}}%
\pgfpathlineto{\pgfqpoint{1.479588in}{6.143816in}}%
\pgfpathlineto{\pgfqpoint{1.580573in}{6.044547in}}%
\pgfpathlineto{\pgfqpoint{1.681558in}{5.948718in}}%
\pgfpathlineto{\pgfqpoint{1.782543in}{5.856212in}}%
\pgfpathlineto{\pgfqpoint{1.883528in}{5.766912in}}%
\pgfpathlineto{\pgfqpoint{1.984513in}{5.680707in}}%
\pgfpathlineto{\pgfqpoint{2.085498in}{5.597491in}}%
\pgfpathlineto{\pgfqpoint{2.186483in}{5.517159in}}%
\pgfpathlineto{\pgfqpoint{2.287469in}{5.439612in}}%
\pgfpathlineto{\pgfqpoint{2.388454in}{5.364753in}}%
\pgfpathlineto{\pgfqpoint{2.489439in}{5.292489in}}%
\pgfpathlineto{\pgfqpoint{2.590424in}{5.222730in}}%
\pgfpathlineto{\pgfqpoint{2.702629in}{5.148052in}}%
\pgfpathlineto{\pgfqpoint{2.814835in}{5.076245in}}%
\pgfpathlineto{\pgfqpoint{2.927041in}{5.007198in}}%
\pgfpathlineto{\pgfqpoint{3.039246in}{4.940805in}}%
\pgfpathlineto{\pgfqpoint{3.151452in}{4.876964in}}%
\pgfpathlineto{\pgfqpoint{3.263658in}{4.815578in}}%
\pgfpathlineto{\pgfqpoint{3.375863in}{4.756551in}}%
\pgfpathlineto{\pgfqpoint{3.499289in}{4.694239in}}%
\pgfpathlineto{\pgfqpoint{3.622716in}{4.634556in}}%
\pgfpathlineto{\pgfqpoint{3.746142in}{4.577393in}}%
\pgfpathlineto{\pgfqpoint{3.869568in}{4.522641in}}%
\pgfpathlineto{\pgfqpoint{3.992994in}{4.470201in}}%
\pgfpathlineto{\pgfqpoint{4.127641in}{4.415513in}}%
\pgfpathlineto{\pgfqpoint{4.262288in}{4.363339in}}%
\pgfpathlineto{\pgfqpoint{4.396935in}{4.313562in}}%
\pgfpathlineto{\pgfqpoint{4.531581in}{4.266072in}}%
\pgfpathlineto{\pgfqpoint{4.677449in}{4.217084in}}%
\pgfpathlineto{\pgfqpoint{4.823316in}{4.170530in}}%
\pgfpathlineto{\pgfqpoint{4.969183in}{4.126289in}}%
\pgfpathlineto{\pgfqpoint{5.115051in}{4.084245in}}%
\pgfpathlineto{\pgfqpoint{5.272139in}{4.041301in}}%
\pgfpathlineto{\pgfqpoint{5.429227in}{4.000649in}}%
\pgfpathlineto{\pgfqpoint{5.597535in}{3.959500in}}%
\pgfpathlineto{\pgfqpoint{5.765843in}{3.920701in}}%
\pgfpathlineto{\pgfqpoint{5.934152in}{3.884117in}}%
\pgfpathlineto{\pgfqpoint{6.113681in}{3.847394in}}%
\pgfpathlineto{\pgfqpoint{6.181004in}{3.834206in}}%
\pgfpathlineto{\pgfqpoint{6.181004in}{3.834206in}}%
\pgfusepath{stroke}%
\end{pgfscope}%
\begin{pgfscope}%
\pgfpathrectangle{\pgfqpoint{0.581943in}{0.509170in}}{\pgfqpoint{5.599061in}{8.312590in}}%
\pgfusepath{clip}%
\pgfsetbuttcap%
\pgfsetroundjoin%
\pgfsetlinewidth{1.254687pt}%
\definecolor{currentstroke}{rgb}{0.501961,0.501961,0.501961}%
\pgfsetstrokecolor{currentstroke}%
\pgfsetdash{{4.625000pt}{2.000000pt}}{0.000000pt}%
\pgfpathmoveto{\pgfqpoint{1.563699in}{0.499170in}}%
\pgfpathlineto{\pgfqpoint{1.659117in}{0.590344in}}%
\pgfpathlineto{\pgfqpoint{1.760102in}{0.683578in}}%
\pgfpathlineto{\pgfqpoint{1.861087in}{0.773581in}}%
\pgfpathlineto{\pgfqpoint{1.962072in}{0.860464in}}%
\pgfpathlineto{\pgfqpoint{2.063057in}{0.944335in}}%
\pgfpathlineto{\pgfqpoint{2.164042in}{1.025299in}}%
\pgfpathlineto{\pgfqpoint{2.265027in}{1.103457in}}%
\pgfpathlineto{\pgfqpoint{2.366013in}{1.178905in}}%
\pgfpathlineto{\pgfqpoint{2.466998in}{1.251738in}}%
\pgfpathlineto{\pgfqpoint{2.567983in}{1.322046in}}%
\pgfpathlineto{\pgfqpoint{2.680188in}{1.397312in}}%
\pgfpathlineto{\pgfqpoint{2.792394in}{1.469684in}}%
\pgfpathlineto{\pgfqpoint{2.904600in}{1.539275in}}%
\pgfpathlineto{\pgfqpoint{3.016805in}{1.606190in}}%
\pgfpathlineto{\pgfqpoint{3.129011in}{1.670533in}}%
\pgfpathlineto{\pgfqpoint{3.241217in}{1.732403in}}%
\pgfpathlineto{\pgfqpoint{3.353422in}{1.791894in}}%
\pgfpathlineto{\pgfqpoint{3.476848in}{1.854697in}}%
\pgfpathlineto{\pgfqpoint{3.600275in}{1.914849in}}%
\pgfpathlineto{\pgfqpoint{3.723701in}{1.972462in}}%
\pgfpathlineto{\pgfqpoint{3.847127in}{2.027645in}}%
\pgfpathlineto{\pgfqpoint{3.970553in}{2.080498in}}%
\pgfpathlineto{\pgfqpoint{4.105200in}{2.135616in}}%
\pgfpathlineto{\pgfqpoint{4.239847in}{2.188201in}}%
\pgfpathlineto{\pgfqpoint{4.374493in}{2.238370in}}%
\pgfpathlineto{\pgfqpoint{4.509140in}{2.286233in}}%
\pgfpathlineto{\pgfqpoint{4.655008in}{2.335607in}}%
\pgfpathlineto{\pgfqpoint{4.800875in}{2.382527in}}%
\pgfpathlineto{\pgfqpoint{4.946742in}{2.427117in}}%
\pgfpathlineto{\pgfqpoint{5.092610in}{2.469491in}}%
\pgfpathlineto{\pgfqpoint{5.249697in}{2.512774in}}%
\pgfpathlineto{\pgfqpoint{5.406785in}{2.553745in}}%
\pgfpathlineto{\pgfqpoint{5.563873in}{2.592528in}}%
\pgfpathlineto{\pgfqpoint{5.732182in}{2.631787in}}%
\pgfpathlineto{\pgfqpoint{5.900490in}{2.668803in}}%
\pgfpathlineto{\pgfqpoint{6.080019in}{2.705961in}}%
\pgfpathlineto{\pgfqpoint{6.181004in}{2.725860in}}%
\pgfpathlineto{\pgfqpoint{6.181004in}{2.725860in}}%
\pgfusepath{stroke}%
\end{pgfscope}%
\begin{pgfscope}%
\pgfsetrectcap%
\pgfsetmiterjoin%
\pgfsetlinewidth{0.803000pt}%
\definecolor{currentstroke}{rgb}{0.000000,0.000000,0.000000}%
\pgfsetstrokecolor{currentstroke}%
\pgfsetdash{}{0pt}%
\pgfpathmoveto{\pgfqpoint{0.581943in}{0.509170in}}%
\pgfpathlineto{\pgfqpoint{0.581943in}{8.821759in}}%
\pgfusepath{stroke}%
\end{pgfscope}%
\begin{pgfscope}%
\pgfsetrectcap%
\pgfsetmiterjoin%
\pgfsetlinewidth{0.803000pt}%
\definecolor{currentstroke}{rgb}{0.000000,0.000000,0.000000}%
\pgfsetstrokecolor{currentstroke}%
\pgfsetdash{}{0pt}%
\pgfpathmoveto{\pgfqpoint{6.181004in}{0.509170in}}%
\pgfpathlineto{\pgfqpoint{6.181004in}{8.821759in}}%
\pgfusepath{stroke}%
\end{pgfscope}%
\begin{pgfscope}%
\pgfsetrectcap%
\pgfsetmiterjoin%
\pgfsetlinewidth{0.803000pt}%
\definecolor{currentstroke}{rgb}{0.000000,0.000000,0.000000}%
\pgfsetstrokecolor{currentstroke}%
\pgfsetdash{}{0pt}%
\pgfpathmoveto{\pgfqpoint{0.581943in}{0.509170in}}%
\pgfpathlineto{\pgfqpoint{6.181004in}{0.509170in}}%
\pgfusepath{stroke}%
\end{pgfscope}%
\begin{pgfscope}%
\pgfsetrectcap%
\pgfsetmiterjoin%
\pgfsetlinewidth{0.803000pt}%
\definecolor{currentstroke}{rgb}{0.000000,0.000000,0.000000}%
\pgfsetstrokecolor{currentstroke}%
\pgfsetdash{}{0pt}%
\pgfpathmoveto{\pgfqpoint{0.581943in}{8.821759in}}%
\pgfpathlineto{\pgfqpoint{6.181004in}{8.821759in}}%
\pgfusepath{stroke}%
\end{pgfscope}%
\begin{pgfscope}%
\pgfsetbuttcap%
\pgfsetmiterjoin%
\definecolor{currentfill}{rgb}{1.000000,1.000000,1.000000}%
\pgfsetfillcolor{currentfill}%
\pgfsetlinewidth{0.501875pt}%
\definecolor{currentstroke}{rgb}{1.000000,1.000000,1.000000}%
\pgfsetstrokecolor{currentstroke}%
\pgfsetdash{}{0pt}%
\pgfpathmoveto{\pgfqpoint{4.207989in}{1.021479in}}%
\pgfpathlineto{\pgfqpoint{4.610618in}{1.021479in}}%
\pgfpathlineto{\pgfqpoint{4.610618in}{1.242590in}}%
\pgfpathlineto{\pgfqpoint{4.207989in}{1.242590in}}%
\pgfpathlineto{\pgfqpoint{4.207989in}{1.021479in}}%
\pgfpathclose%
\pgfusepath{stroke,fill}%
\end{pgfscope}%
\begin{pgfscope}%
\definecolor{textcolor}{rgb}{0.000000,0.000000,0.000000}%
\pgfsetstrokecolor{textcolor}%
\pgfsetfillcolor{textcolor}%
\pgftext[x=4.409303in,y=1.132034in,,]{\color{textcolor}\rmfamily\fontsize{9.000000}{10.800000}\selectfont \(\displaystyle 0.001\)}%
\end{pgfscope}%
\begin{pgfscope}%
\pgfsetbuttcap%
\pgfsetmiterjoin%
\definecolor{currentfill}{rgb}{1.000000,1.000000,1.000000}%
\pgfsetfillcolor{currentfill}%
\pgfsetlinewidth{0.501875pt}%
\definecolor{currentstroke}{rgb}{1.000000,1.000000,1.000000}%
\pgfsetstrokecolor{currentstroke}%
\pgfsetdash{}{0pt}%
\pgfpathmoveto{\pgfqpoint{4.207989in}{1.343203in}}%
\pgfpathlineto{\pgfqpoint{4.610618in}{1.343203in}}%
\pgfpathlineto{\pgfqpoint{4.610618in}{1.564314in}}%
\pgfpathlineto{\pgfqpoint{4.207989in}{1.564314in}}%
\pgfpathlineto{\pgfqpoint{4.207989in}{1.343203in}}%
\pgfpathclose%
\pgfusepath{stroke,fill}%
\end{pgfscope}%
\begin{pgfscope}%
\definecolor{textcolor}{rgb}{0.000000,0.000000,0.000000}%
\pgfsetstrokecolor{textcolor}%
\pgfsetfillcolor{textcolor}%
\pgftext[x=4.409303in,y=1.453759in,,]{\color{textcolor}\rmfamily\fontsize{9.000000}{10.800000}\selectfont \(\displaystyle 0.010\)}%
\end{pgfscope}%
\begin{pgfscope}%
\pgfsetbuttcap%
\pgfsetmiterjoin%
\definecolor{currentfill}{rgb}{1.000000,1.000000,1.000000}%
\pgfsetfillcolor{currentfill}%
\pgfsetlinewidth{0.501875pt}%
\definecolor{currentstroke}{rgb}{1.000000,1.000000,1.000000}%
\pgfsetstrokecolor{currentstroke}%
\pgfsetdash{}{0pt}%
\pgfpathmoveto{\pgfqpoint{4.207989in}{1.719919in}}%
\pgfpathlineto{\pgfqpoint{4.610618in}{1.719919in}}%
\pgfpathlineto{\pgfqpoint{4.610618in}{1.941030in}}%
\pgfpathlineto{\pgfqpoint{4.207989in}{1.941030in}}%
\pgfpathlineto{\pgfqpoint{4.207989in}{1.719919in}}%
\pgfpathclose%
\pgfusepath{stroke,fill}%
\end{pgfscope}%
\begin{pgfscope}%
\definecolor{textcolor}{rgb}{0.000000,0.000000,0.000000}%
\pgfsetstrokecolor{textcolor}%
\pgfsetfillcolor{textcolor}%
\pgftext[x=4.409303in,y=1.830474in,,]{\color{textcolor}\rmfamily\fontsize{9.000000}{10.800000}\selectfont \(\displaystyle 0.050\)}%
\end{pgfscope}%
\begin{pgfscope}%
\pgfsetbuttcap%
\pgfsetmiterjoin%
\definecolor{currentfill}{rgb}{1.000000,1.000000,1.000000}%
\pgfsetfillcolor{currentfill}%
\pgfsetlinewidth{0.501875pt}%
\definecolor{currentstroke}{rgb}{1.000000,1.000000,1.000000}%
\pgfsetstrokecolor{currentstroke}%
\pgfsetdash{}{0pt}%
\pgfpathmoveto{\pgfqpoint{4.207989in}{2.164535in}}%
\pgfpathlineto{\pgfqpoint{4.610618in}{2.164535in}}%
\pgfpathlineto{\pgfqpoint{4.610618in}{2.385646in}}%
\pgfpathlineto{\pgfqpoint{4.207989in}{2.385646in}}%
\pgfpathlineto{\pgfqpoint{4.207989in}{2.164535in}}%
\pgfpathclose%
\pgfusepath{stroke,fill}%
\end{pgfscope}%
\begin{pgfscope}%
\definecolor{textcolor}{rgb}{0.000000,0.000000,0.000000}%
\pgfsetstrokecolor{textcolor}%
\pgfsetfillcolor{textcolor}%
\pgftext[x=4.409303in,y=2.275091in,,]{\color{textcolor}\rmfamily\fontsize{9.000000}{10.800000}\selectfont \(\displaystyle 0.159\)}%
\end{pgfscope}%
\begin{pgfscope}%
\pgfsetbuttcap%
\pgfsetmiterjoin%
\definecolor{currentfill}{rgb}{1.000000,1.000000,1.000000}%
\pgfsetfillcolor{currentfill}%
\pgfsetlinewidth{0.501875pt}%
\definecolor{currentstroke}{rgb}{1.000000,1.000000,1.000000}%
\pgfsetstrokecolor{currentstroke}%
\pgfsetdash{}{0pt}%
\pgfpathmoveto{\pgfqpoint{4.207989in}{2.424275in}}%
\pgfpathlineto{\pgfqpoint{4.610618in}{2.424275in}}%
\pgfpathlineto{\pgfqpoint{4.610618in}{2.645387in}}%
\pgfpathlineto{\pgfqpoint{4.207989in}{2.645387in}}%
\pgfpathlineto{\pgfqpoint{4.207989in}{2.424275in}}%
\pgfpathclose%
\pgfusepath{stroke,fill}%
\end{pgfscope}%
\begin{pgfscope}%
\definecolor{textcolor}{rgb}{0.000000,0.000000,0.000000}%
\pgfsetstrokecolor{textcolor}%
\pgfsetfillcolor{textcolor}%
\pgftext[x=4.409303in,y=2.534831in,,]{\color{textcolor}\rmfamily\fontsize{9.000000}{10.800000}\selectfont \(\displaystyle 0.250\)}%
\end{pgfscope}%
\begin{pgfscope}%
\pgfsetbuttcap%
\pgfsetmiterjoin%
\definecolor{currentfill}{rgb}{1.000000,1.000000,1.000000}%
\pgfsetfillcolor{currentfill}%
\pgfsetlinewidth{0.501875pt}%
\definecolor{currentstroke}{rgb}{1.000000,1.000000,1.000000}%
\pgfsetstrokecolor{currentstroke}%
\pgfsetdash{}{0pt}%
\pgfpathmoveto{\pgfqpoint{4.207989in}{3.777459in}}%
\pgfpathlineto{\pgfqpoint{4.610618in}{3.777459in}}%
\pgfpathlineto{\pgfqpoint{4.610618in}{3.998571in}}%
\pgfpathlineto{\pgfqpoint{4.207989in}{3.998571in}}%
\pgfpathlineto{\pgfqpoint{4.207989in}{3.777459in}}%
\pgfpathclose%
\pgfusepath{stroke,fill}%
\end{pgfscope}%
\begin{pgfscope}%
\definecolor{textcolor}{rgb}{0.000000,0.000000,0.000000}%
\pgfsetstrokecolor{textcolor}%
\pgfsetfillcolor{textcolor}%
\pgftext[x=4.409303in,y=3.888015in,,]{\color{textcolor}\rmfamily\fontsize{9.000000}{10.800000}\selectfont \(\displaystyle 0.750\)}%
\end{pgfscope}%
\begin{pgfscope}%
\pgfsetbuttcap%
\pgfsetmiterjoin%
\definecolor{currentfill}{rgb}{1.000000,1.000000,1.000000}%
\pgfsetfillcolor{currentfill}%
\pgfsetlinewidth{0.501875pt}%
\definecolor{currentstroke}{rgb}{1.000000,1.000000,1.000000}%
\pgfsetstrokecolor{currentstroke}%
\pgfsetdash{}{0pt}%
\pgfpathmoveto{\pgfqpoint{4.207989in}{4.175220in}}%
\pgfpathlineto{\pgfqpoint{4.610618in}{4.175220in}}%
\pgfpathlineto{\pgfqpoint{4.610618in}{4.396331in}}%
\pgfpathlineto{\pgfqpoint{4.207989in}{4.396331in}}%
\pgfpathlineto{\pgfqpoint{4.207989in}{4.175220in}}%
\pgfpathclose%
\pgfusepath{stroke,fill}%
\end{pgfscope}%
\begin{pgfscope}%
\definecolor{textcolor}{rgb}{0.000000,0.000000,0.000000}%
\pgfsetstrokecolor{textcolor}%
\pgfsetfillcolor{textcolor}%
\pgftext[x=4.409303in,y=4.285776in,,]{\color{textcolor}\rmfamily\fontsize{9.000000}{10.800000}\selectfont \(\displaystyle 0.841\)}%
\end{pgfscope}%
\begin{pgfscope}%
\pgfsetbuttcap%
\pgfsetmiterjoin%
\definecolor{currentfill}{rgb}{1.000000,1.000000,1.000000}%
\pgfsetfillcolor{currentfill}%
\pgfsetlinewidth{0.501875pt}%
\definecolor{currentstroke}{rgb}{1.000000,1.000000,1.000000}%
\pgfsetstrokecolor{currentstroke}%
\pgfsetdash{}{0pt}%
\pgfpathmoveto{\pgfqpoint{4.207989in}{5.050164in}}%
\pgfpathlineto{\pgfqpoint{4.610618in}{5.050164in}}%
\pgfpathlineto{\pgfqpoint{4.610618in}{5.271275in}}%
\pgfpathlineto{\pgfqpoint{4.207989in}{5.271275in}}%
\pgfpathlineto{\pgfqpoint{4.207989in}{5.050164in}}%
\pgfpathclose%
\pgfusepath{stroke,fill}%
\end{pgfscope}%
\begin{pgfscope}%
\definecolor{textcolor}{rgb}{0.000000,0.000000,0.000000}%
\pgfsetstrokecolor{textcolor}%
\pgfsetfillcolor{textcolor}%
\pgftext[x=4.409303in,y=5.160719in,,]{\color{textcolor}\rmfamily\fontsize{9.000000}{10.800000}\selectfont \(\displaystyle 0.950\)}%
\end{pgfscope}%
\begin{pgfscope}%
\pgfsetbuttcap%
\pgfsetmiterjoin%
\definecolor{currentfill}{rgb}{1.000000,1.000000,1.000000}%
\pgfsetfillcolor{currentfill}%
\pgfsetlinewidth{0.501875pt}%
\definecolor{currentstroke}{rgb}{1.000000,1.000000,1.000000}%
\pgfsetstrokecolor{currentstroke}%
\pgfsetdash{}{0pt}%
\pgfpathmoveto{\pgfqpoint{4.207989in}{6.104969in}}%
\pgfpathlineto{\pgfqpoint{4.610618in}{6.104969in}}%
\pgfpathlineto{\pgfqpoint{4.610618in}{6.326080in}}%
\pgfpathlineto{\pgfqpoint{4.207989in}{6.326080in}}%
\pgfpathlineto{\pgfqpoint{4.207989in}{6.104969in}}%
\pgfpathclose%
\pgfusepath{stroke,fill}%
\end{pgfscope}%
\begin{pgfscope}%
\definecolor{textcolor}{rgb}{0.000000,0.000000,0.000000}%
\pgfsetstrokecolor{textcolor}%
\pgfsetfillcolor{textcolor}%
\pgftext[x=4.409303in,y=6.215525in,,]{\color{textcolor}\rmfamily\fontsize{9.000000}{10.800000}\selectfont \(\displaystyle 0.990\)}%
\end{pgfscope}%
\begin{pgfscope}%
\pgfsetbuttcap%
\pgfsetmiterjoin%
\definecolor{currentfill}{rgb}{1.000000,1.000000,1.000000}%
\pgfsetfillcolor{currentfill}%
\pgfsetlinewidth{0.501875pt}%
\definecolor{currentstroke}{rgb}{1.000000,1.000000,1.000000}%
\pgfsetstrokecolor{currentstroke}%
\pgfsetdash{}{0pt}%
\pgfpathmoveto{\pgfqpoint{4.207989in}{7.452425in}}%
\pgfpathlineto{\pgfqpoint{4.610618in}{7.452425in}}%
\pgfpathlineto{\pgfqpoint{4.610618in}{7.673537in}}%
\pgfpathlineto{\pgfqpoint{4.207989in}{7.673537in}}%
\pgfpathlineto{\pgfqpoint{4.207989in}{7.452425in}}%
\pgfpathclose%
\pgfusepath{stroke,fill}%
\end{pgfscope}%
\begin{pgfscope}%
\definecolor{textcolor}{rgb}{0.000000,0.000000,0.000000}%
\pgfsetstrokecolor{textcolor}%
\pgfsetfillcolor{textcolor}%
\pgftext[x=4.409303in,y=7.562981in,,]{\color{textcolor}\rmfamily\fontsize{9.000000}{10.800000}\selectfont \(\displaystyle 0.999\)}%
\end{pgfscope}%
\end{pgfpicture}%
\makeatother%
\endgroup%

  \caption{Quantili della distribuzione del $\chisq$ normalizzati al numero
  $\nu$ di gradi di libertà. Le due linee tratteggiare rappresentano
  l'approssimazione Gaussiana a $\pm 1~\sigma$, che, come si vede, è
  ragionevolmente accurata per $\nu \geq 10$.}
  \label{fig:chi2_quantiles}
\end{figure}
