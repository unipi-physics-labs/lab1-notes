\chapter{Complementi di statistica}
\label{chap:complementi}


\section{La funzione generatrice dei momenti}

La \emph{funzione generatrice dei momenti} $\momgenf(t)$ per una variabile
casuale $x$ è definita come
\begin{align}
  \dccases{\momgenf(t) = \expect{e^{xt}}}%
          {\sum_k e^{x_kt} \prob{x_k}}%
          {\intinf e^{xt} p(x) \diff}
\end{align}
Sviluppando in serie l'esponenziale attorno al punto $t = 0$ possiamo scrivere
\begin{align*}
  \momgenf(t) & = \expect{e^{xt}} =
  \expect{\sum_{n = 1}^{\infty} \frac{(xt)^n}{n!}} =
  \sum_{n = 1}^{\infty} \frac{\expect{x^n}t^n}{n!} = 
  \sum_{n = 1}^{\infty} \frac{\momalg{n}t^n}{n!} =
  1 + \momalg{1}t + \frac{\momalg{2}t^2}{2} + \frac{\momalg{3}t^3}{6} +
  \cdots
\end{align*}
e, derivando rispetto a $t$, si hanno le relazioni notevoli
\begin{align} \label{eq:mgf_mean}
  \momalg{n} = \td[n]{\momgenf}{t}{0}
  \quad\text{e, in particolare}\quad
  \mu = \td{\momgenf}{t}{0}.
\end{align}
(Da cui il nome di funzione generatrice dei momenti.)

\begin{examplebox}
  \begin{example}
    Sia $x$ l'uscita di un dado equo a sei facce. La funzione generatrice dei
    momenti è
    \begin{align*}
      \momgenf(t) = \sum_{k = 1}^{6} e^{x_kt} \prob{x_k} =
      \sum_{k = 1}^{6} e^{kt} \prob{k} =
      \frac{1}{6}\sum_{k = 1}^{6} e^{kt}
    \end{align*}
    e, calcolando la derivata prima nel punto $t = 0$ otteniamo banalmente
    la definizione di media~\eqref{eq:mean}
    \begin{align*}
      \mu = \td{\momgenf}{t}{0} =
      \at{\frac{1}{6} \sum_{k = 1}^{6} ke^{kt}}{t = 0}\mkern10mu = 
      \frac{1}{6} \sum_{k = 1}^{6} k.
    \end{align*}
  \end{example}
  
  \begin{example}
    Sia $x$ una variabile casuale continua distribuita uniformemente tra $0$ ed
    $1$. Si ha
    \begin{align*}
      \momgenf(t) = \int_{0}^{1}e^{xt}p(x) \diff =
      \int_{0}^{1}e^{xt} \diff = \frac{1}{t} \int_{0}^{t} e^y \diff[y] =
      \frac{e^t - 1}{t}.
    \end{align*}
    Si tratta di un caso interessante, in quanto la derivata è una forma
    indeterminata per $t \rightarrow 0$, che si può calcolare passando al
    limite ed utilizzando il teorema di de L'H\^opital
    \begin{align*}
      \mu = \lim_{t \rightarrow 0} \td{\momgenf}{t}{t} =
      \lim_{t \rightarrow 0} \frac{te^t - (e^t - 1)}{t^2} =
      \lim_{t \rightarrow 0} \frac{te^t}{2t} = \frac{1}{2}.
    \end{align*}
    Giusto, ma significativamente più complicato che non applicare la
    definizione~\eqref{eq:mean}.
  \end{example}
\end{examplebox}

Analogamente, se la variabile casuale $x$ ha media definita $\mu$, si possono
generare i momenti centrali a partire dalla funzione generatrice
\begin{align*}
  \momgenf[x - \mu](t) & = \expect{e^{(x - \mu)t}} =
  \expect{\sum_{n = 1}^{\infty} \frac{(x-\mu)^nt^n}{n!}} =
  \sum_{n = 1}^{\infty} \frac{\expect{(x - \mu)^n}t^n}{n!} =
  \sum_{n = 1}^{\infty} \frac{\momcen{n}t^n}{n!} =
  1 + \momcen{1}t + \frac{\momcen{2}t^2}{2} + \cdots
\end{align*}
calcolando, come prima, le derivate nel punto $t = 0$:
\begin{align} 
  \momcen{n} = \td[n]{\momgenf[x - \mu]}{t}{0}
  \quad\text{e ancora}\quad
  \sigma^2 = \td[2]{\momgenf[x - \mu]}{t}{0}.
\end{align}


\begin{examplebox}
  \begin{example}
    Sia $x$ l'uscita di un dado equo a sei facce. La funzione generatrice dei
    momenti centrali e le derivate rilevanti sono
    \begin{align*}
      \momgenf[x - \mu](t) &= e^{-\mu t} \sum_{k = 1}^{6} e^{x_kt} \prob{x_k} =
      e^{-\mu t} \sum_{k = 1}^{6} e^{kt} \prob{k} =
      \frac{e^{-\mu t}}{6} \sum_{k = 1}^{6} e^{kt}\\
      \td{\momgenf[x - \mu]}{t}{t} & =
      \frac{e^{-\mu t}}{6} \left[-\mu  \sum_{k = 1}^{6} e^{kt} +
        \sum_{k = 1}^{6} ke^{kt} \right] = 
      \frac{e^{-\mu t}}{6} \sum_{k = 1}^{6} (k - \mu) e^{kt}\\
      \td[2]{\momgenf[x - \mu]}{t}{t} & = 
      \frac{e^{-\mu t}}{6} \left[ -\mu \sum_{k = 1}^{6} (k - \mu) e^{kt}
        + \sum_{k = 1}^{6} k(k - \mu)  e^{kt} \right] = 
      \frac{e^{-\mu t}}{6} \sum_{k = 1}^{6} (k - \mu)^2 e^{kt},
    \end{align*}
    da cui otteniamo il risultato già noto
    \begin{align*}
      \sigma^2 = \td[2]{\momgenf[x - \mu](t)}{t}{0} = 
      \frac{1}{6}\sum_{k = 1}^{6} (k - \mu)^2.
    \end{align*}
  \end{example}

  \begin{example}
    Sia $x$ una variabile casuale continua distribuita uniformemente tra $0$ ed
    $1$. La funzione generatrice dei momenti centrali è
    \begin{align*}
      \momgenf[x - \mu](t) = e^{-\mu t} \int_{0}^{1}e^{xt} \diff =
      \frac{e^{-\frac{t}{2}} (e^t - 1)}{t} =
      \frac{e^{\frac{t}{2}} - e^{-\frac{t}{2}}}{t} = 
      \frac{2\sinh(t/2)}{t}
    \end{align*}
    (abbiamo sfruttato il fatto che $\mu = 1/2$) e le derivate prima e seconda
    si scrivono come
    \begin{align*}
      \td{\momgenf[x - \mu](t)}{t}{} & =
      \frac{t\cosh(t/2) - 2\sinh(t/2)}{t^2}\\
      \td[2]{\momgenf[x - \mu](t)}{t}{} & = 
      \frac{\frac{t^3}{2}\sinh(t/2) -
        2t\left[t\cosh(t/2) - 2\sinh(t/2)\right]}{t^4} =
       \frac{(\frac{t^2}{2} + 4)\sinh(t/2) - 2t\cosh(t/2) }{t^3}.
    \end{align*}
    La derivata seconda è ancora una forma indeterminata per cui, come prima,
    calcoliamo il limite per $t \rightarrow 0$ usando il teorema di
    de L'H\^opital:
    \begin{align*}
      \sigma^2 = \lim_{t \rightarrow 0} \td[2]{\momgenf[x - \mu](t)}{t}{} = 
      \lim_{t \rightarrow 0} \frac{\frac{t^2}{4}\cosh(t/2)}{3t^2} = \frac{1}{12}
    \end{align*}
    che, di nuovo, è il risultato noto.
  \end{example}
\end{examplebox}

Gli esempi appena illustrati mostrano come la funzione generatrice dei momenti
non sia necessariamente la strada più semplice per il calcolo della media
e della varianza (o più in generale dei momenti di ordine superiore) di una
funzione di distribuzione data.

Sotto ipotesi del tutto generali i momenti definiscono univocamente la
funzione di distribuzione, nel senso che si può dimostrare che se due
funzioni di distribuzione hanno momenti uguali a tutti gli ordini, allora
sono identiche. 


\section{La distribuzione $t$ di Student}


\summary
