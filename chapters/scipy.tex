\chapter{Glossario di \scipy}
\label{sec:scipy}

Nello stesso spirito dell'appendice~\ref{sec:numpy}, riportiamo qui un breve
glossario di \scipy, arricchito da una serie di piccoli frammenti di codice che
illustrano problemi specifici incontrati nelle dispense.


\section{Funzioni speciali}

Il modulo \scipymodule{special} fornisce una vasta gamma di funzioni matematiche.
(Quelle relative al calcolo combinatorio, in particolare, sono degne di nota,
perché permettono di fare uso dell'aritmetica intera di \python, che per sua
natura ha precisione arbitraria, in tutti i casi in cui uno abbia bisogno di una
risposta non approssimata.

\vfill

\scipycmd{special.comb}{
restituisce il coef\-fi\-cien\-te binomiale
\begin{align*}
  \binom{n}{k} = \frac{n!}{k! (n - k!)}.
\end{align*}
Di \foreign{default} il risultato è restituito (in forma approssimata) in virgola
mobile. Se specifichiamo l'argomento \code{exact=True}, allora il calcolo
sfrutta la precisione illimitata dell'aritmetica intera di \python.
}

\scipycmd{special.erf}{%
implementazione della \foreign{error function} che può inter-operare nativamente con
\nparray. Nell'esempio a fianco abbiamo ricavato di nuovo la
regola~\eqref{eq:65-95-99rule} utilizzando il fatto che
\begin{align*}
  P(-z_0 \leq z \leq z_0) = \erf{\frac{z_0}{\sqrt{2\pi}}}.
\end{align*}
}

\scipycmd{special.factorial}{%
restituisce il fattoriale del primo argomento
\begin{align*}
  n! = n(n-1)(n-2) \cdots 1.
\end{align*}
Di \foreign{default} il risultato è restituito (in forma approssimata) in virgola
mobile. Se specifichiamo l'argomento \code{exact=True}, allora il calcolo
sfrutta la precisione illimitata dell'aritmetica intera di \python.
}

\scipycmd{special.gamma}{%
restituisce il valore della funzione $\Gamma$ dell'argomento. La funzione
$\Gamma$ soddisfa la relazione
\begin{align*}
  \Gamma (n + 1) = n!,
\end{align*}
ed è la stessa funzione che interviene nella definizione della distribuzione
del $\chisq$.}


\section{Funzioni statistiche}

Il modulo \href{https://docs.scipy.org/doc/scipy/reference/stats.html}{scipy.stats}
contiene, tra le altre cose, una vasta collezioni di distribuzioni di probabilità,
discrete e continue. In particolare, gli oggetti restituiti dalle funzioni
descritte in questa sezione rappresentano vere e proprie \emph{variabili causali},
in tutta la loro ricchezza e complessità, e forniscono tutte le interfacce
necessarie ad inter-operare con esse nei modi in cui abbiamo studiato.

Più in dettaglio, ogni variabile casuale restituita dalle funzioni di \code{scipy.stat}
mette a disposizione, tra gli altri, i seguenti metodi fondamentali:
\begin{itemize}
  \item \pyfunc{pmf} o \pyfunc{pdf} restituiscono rispettivamente la probabilità
  (\foreign{probability mass function}, per le variabili casuali discrete) o la
  densità di probabilità (\foreign{probability density function}, per le variabili
  casuali continue);
  \item \pyfunc{cdf} restituisce la funzione cumulativa
  (cfr. sezione~\ref{sec:funzione_cumulativa});
  \item \pyfunc{ppf} restituisce la \foreign{percent point function}, o funzione
  cumulativa inversa (cfr. sezione~\ref{sec:funzione_cumulativa});
  \item \pyfunc{rvs} restituisce un \foreign{array} di valori numerici, di lunghezza
  arbitraria, ottenuto campionando la distribuzione di probabilità in
  questione---essenzialmente la stessa cosa che forniscono le funzioni del
  modulo \npmodule{random}.
\end{itemize}


\scipycmd{stats.binom}{%
restituisce una variabile casuale discreta con funzione di distribuzione
binomiale (cfr. sezione~\ref{sec:distribuzione_binomiale}).}

\scipycmd{stats.chi2}{%
restituisce una variabile casuale discreta distribuita come un $\chisq$
(cfr. sezione~\ref{sec:pdf_chi2}). Fissato il numero di gradi di libertà,
la funzione cumulativa e la \foreign{percent point function} si possono
utilizzare per passare dal valore ottenuto al \pvalue\ corrispondente
e viceversa (i.e., per calcolare il valore del $\chisq$ che corrisponde ad
un determinato \pvalue).}

\scipycmd{stats.norm}{
restituisce una variabile casuale continua con funzione di distribuzione Gaussiana
(cfr. sezione~\ref{sec:distribuzione_gauss}). Nell'esempio a fianco abbiamo
utilizzato la densità di probabilità per calcolare
\begin{align*}
  \gausspdf{0; \mu=0, \sigma=1} = \frac{1}{\sqrt{2\pi}}
\end{align*}
e la funzione cumulativa per ricavare di nuovo la regola~\eqref{eq:65-95-99rule}.}

\scipycmd{stats.poisson}{%
restituisce una variabile casuale discreta con funzione di distribuzione
Poissoniana (cfr. sezione~\ref{sec:poisson_pdf}).}


\section{Ottimizzazione}

Il modulo \scipymodule{optimize} contiene tutto ciò che serve per risolvere
problemi di ottimizzazione, minimizzazione di funzioni, e ricerca di zeri.

Abbiamo descritto in dettaglio la funzione \pyfunc{curve_fit} nelle
sezioni~\ref{sec:fit_numerici} e \ref{sec:dettagli_curve_fit}, ma la
documentazione del modulo vale senza dubbio un'occhiata.
