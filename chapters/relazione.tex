\chapter{La relazione di laboratorio}

Lo scopo della relazione di laboratorio è quella di descrivere, in modo più
chiaro possibile, le procedure utilizzate e i risultati ottenuti durante
l'esperienza. Comunicare il risultato del proprio lavoro è fondamentale nella
ricerca scientifica, e le relazioni di laboratorio sono un primo passo verso
l'acquisizione delle adeguate capacità di comunicazione.
Chi non ha mai scritto una relazione (ad esempio alle scuole superiori), può
incontrare alcune difficoltà iniziali. In questo capitolo sono quindi
riportati alcuni suggerimenti e linee guida per la preparazione delle relazioni.


\section{Considerazioni generali}
\label{sec:relazione_considerazioni}

Nella stesura della relazione, sono fondamentali la \emph{completezza},
la \emph{concisione} e la \emph{chiarezza}. Chi legge il testo, ad esempio un
docente o un altro studente, deve capire cosa è stato fatto e come sono stati
ottenuti i risultati.
Uno dei punti chiave del metodo scientifico è la riproducibilità dei
risultati. \`E quindi fondamentale che, leggendo la relazione, un altro
sperimentatore abbia le informazioni necessarie per poter ripetere esattamente
la misura, quindi è importante riportare le informazioni in modo completo,
descrivendo ad esempio l'apparato sperimentale oppure le procedure seguite
durante l'esecuzione dell'esperienza.

Bisogna tuttavia tenere presente la brevità e la concisione della relazione.
La relazione infatti \emph{non è un trattato di fisica}. E' giusto introdurre
brevemente i concetti di fisica e le equazioni utilizzate nella relazione, ma
senza dilungarsi troppo su tutti i dettagli del fenomeno che si vuole studiare.
Inoltre, se la relazione è stata scritta in modo chiaro, sarà più facile
rileggerla a distanza di tempo, ad esempio durante la preparazione all'esame
di fine corso.

Tenendo presente questi aspetti, bisogna sottolineare tuttavia che la relazione
\emph{non è un diario delle attività svolte in laboratorio!} Bisogna senza
dubbio spiegare come sono state effettuate le misure e le difficoltà
incontrate, ma evitando i dettagli non rilevanti. Ad esempio, è opportuno
ridurre al minimo i commenti soggettivi e mantenere lo stile asciutto.
Raccontare le scelte sperimentali adottate nel condurre una certa misura può
essere utile (ad esempio gli accorgimenti utilizzati per misurare al meglio i
periodi di oscillazione di un pendolo), ma solo se sono funzionali alla
comprensione dell'esperienza e dei suoi risultati. E' molto più importante
concentrarsi \emph{sulle misure effettuate, sull'analisi dei dati e sui
  risultati ottenuti}.


\section{Carta, penna e calamaio, oppure il computer?}

Al momento di scrivere le prime relazioni, ci si trova in dubbio se prepararle
al computer o scriverle a penna su un foglio. Esistono molte soluzioni per
preparare la relazione in formato elettronico, a partire dai vari elaboratori
di testo oggi disponibili per \linux\ o per \windows, oppure preparare un
documento \LaTeX. \emph{Il punto chiave non è la forma in cui viene presentata
  la relazione, ma il suo contenuto}. Se non ci si sente sicuri nell'utilizzo
del computer, si può scrivere la relazione a mano senza problemi, magari
inserendo dei grafici realizzati al computer se necessario. \`E importante
tenere presente che \emph{scrivere la relazione al computer non corrisponde a
  una valutazione più alta}. L'unica accortezza che si richiede quando la
relazione è redatta a mano, è che sia scritta con una grafia comprensibile!


\section{La struttura della relazione}
\label{sec:relazione_struttura}

Nella letteratura scientifica, gli articoli, o \foreign{paper}, sono basati su uno
schema ben definito, che tende a seguire la cosiddetta struttura \foreign{IMRaD}
(\foreign{Introduction, Methods, Results, and Discussion}). Questo schema
organizzativo è il frutto di un'evoluzione di stile nel corso del tempo,
mirata sempre di più a ottimizzare la comunicazione dei risultati scientifici.
La produzione scientifica ha una crescita continua, e ogni giorno centinaia di
nuovi articoli sono pubblicati in ogni disciplina. Avere uno schema predefinito
aiuta gli scienziati a recuperare rapidamente le informazioni utili, senza
dover passare troppo tempo a scovare i metodi o i risultati di un lavoro. Ad
esempio sappiamo che nell'\foreign{Abstract} sono menzionati i punti essenziali di
un lavoro, oppure che i dettagli circa l'apparato sperimentale sono discussi
in una sezione chiamata \foreign{Methods} o simili.

In maniera simile, anche la relazione di laboratorio, che è una forma
semplificata di \foreign{report} scientifico, può seguire uno schema definito,
che può essere molto utile soprattutto nel corso delle prime esperienze di
laboratorio. Ecco un possibile schema da seguire per la stesura della relazione:
\begin{enumerate}
\item Informazioni generali;
\item Scopo dell'esperienza;
\item Cenni teorici;
\item Apparato sperimentale;
\item Descrizione delle misure;
\item Analisi dei dati;
\item Conclusioni;
\end{enumerate}


\subsection{Informazioni generali}

All'inizio della relazione è ovviamente importante riportare il titolo
dell'esperienza (ad esempio "Misura dell'accelerazione di gravità
utilizzando un pendolo semplice"), la data e il nome dello studente o degli
studenti del gruppo che ha condotto l'esperienza.


\subsection{Scopo dell'esperienza}

Spiegare, in un paio di righe, lo scopo dell'esperienza e la metodologia che
si intende utilizzare. Ad esempio misurare la densità di alcuni oggetti
misurandone massa e volume.


\subsection{Cenni teorici}

Descrivere, \emph{brevemente}, i concetti fisici e il fenomeno che si sta
studiando e mostrare la legge o le leggi che si intendono verificare. I cenni
di teoria vanno citati in poche righe, e non si devono ricavare o dimostrare
le equazioni utilizzate. \`E fondamentale introdurre le formule che si
utilizzeranno nell'esperienza, scrivendole se necessario anche nella forma
usata durante l'esperienza (ad esempio, passare da $T$ a $T^{2}$ per evidenziare
andamenti costanti o lineari). Un paio di paragrafi sono necessari per questa
sezione.

In questa sezione si deve, ove possibile, riportare anche un disegno che
schematizzi le grandezze in gioco, ad esempio lo schema di un pendolo semplice,
avendo cura di annotare con opportune lettere le grandezze descritte nelle
equazioni (ad esempio lunghezze e angoli). Non è importante che il disegno
sia raffinato, tridimensionale, o a colori, ma che descriva fedelmente il
fenomeno. Quindi non serve perdere tempo a cercare su Internet un disegno molto
bello: basta anche uno schema disegnato a mano, purché sia chiaro.


\subsection{Apparato sperimentale e strumenti}

Si tratta di una sezione molto importante, in cui si richiede di descrivere
prima di tutto gli strumenti utilizzati (calibri, bilancia, ecc), riportando
per ciascuno la risoluzione. Bisogna poi riportare il materiale a disposizione
(ad esempio molle, pendoli ecc).


\subsection{Descrizione delle misure}

Descrivere le misure effettuate, le unità di misura e le incertezze associate
a ciascuna grandezza. \`E buona norma riportare i dati grezzi, quando non sono
in numero eccessivo, in opportune tabelle. Nelle tabelle devono essere
riportati in modo chiaro le misure con le opportune unità di misura e
incertezze. Attenzione: bisogna dire cosa è stato fatto davvero (difficoltà
e imprevisti compresi), non quello che sarebbe dovuto succedere in un mondo
ideale!


\subsection{Analisi dei dati}

Descrivere in questa sezione metodi di analisi utilizzati (ad esempio i
\fit), includendo le tabelle con i dati raccolti e gli eventuali grafici.
\`E molto importante descrivere in modo preciso i metodi di analisi utilizzati:
ad esempio "metodo di fit basato sui minimi quadrati condotto utilizzando la
funzione \pyfunc{curve_fit} di \python", piuttosto che un generico
"\fit\ lineare" oppure "\fit\ con \python". Quando si effettuano fit o test
sulla bontà di un fit è sempre opportuno riportare il valore del
$\chi^{2}$, i gradi di libertà e la probabilità associata ad esso.
Bisogna sempre riportare come è stata stimata e propagata l'incertezza sulle
varie quantità, e ovviamente anche per i risultati non bisogna mai
dimenticarsi le unità di misura e le incertezze. In questa sezione
naturalmente vanno inseriti i grafici costruiti a partire dai dati raccolti.


\subsection{Conclusioni}

Si tratta di una delle sezioni più importanti e che al tempo stesso
richiedono maggiore esperienza. In questa sezione si riportano infatti in modo
chiaro i risultati finali dell'esperienza. Bisogna quindi discutere i risultati
ottenuti dall'analisi dei dati, ad esempio la bontà di un risultato in
termini di errore relativo e accordo con il modello teorico, oppure l'accordo
o il disaccordo con una legge teorica utilizzando la probabilità associata al
test del $\chi^{2}$.

Bisogna supportare le conclusioni con stime quantitative, ad esempio valutando
la differenza dati-teoria alla luce delle incertezze di misura. \`E anche
possibile discutere possibili idee per migliorare le misure o la stima degli
errori.


\section{Cosa non dimenticare, e cosa non fare}
\label{sec:relazione_daevitare}

Una volta redatta la relazione seguendo lo schema appena descritto, prima di
consegnarla conviene sempre controllare di non aver fatto alcuni errori o
aver dimenticato delle informazioni importanti.


\subsection{Cose da non dimenticare}

\begin{itemize}
\item Propagare sempre gli errori (possibilmente in modo corretto);
\item Riportare sempre un disegno dell'apparato sperimentale;
\item Descrivere sempre gli strumenti utilizzati;
\item Includere sempre i grafici rilevanti e, se possibile, tutti i dati
  raccolti;
\item Riportare sempre nelle tabelle, o sugli assi dei grafici, il nome della
  quantità misurata e la sua unità di misura;
\item Riportare sempre con completezza i risultati di un fit, o di un test del
  $\chi^{2}$;
\item Evidenziare sempre opportunamente i risultati finali dell'esperienza.
\end{itemize}


\subsection{Cose da non fare}

\begin{itemize}
\item Non riportare mai la misura di una quantità senza riportare le unità
  di misura o l'incertezza associata;
\item Non riportare mai in modo errato le cifre significative;
\item Non riportare mai i valori delle misure direttamente sugli assi;
\item Non discutere mai in maniera troppo qualitativa i risultati---non dire,
  ad esempio, che "l'accordo con la teoria è abbastanza buono\ldots";
\item Non usare mai vocaboli ed espressioni colorite di dubbia correttezza,
  o gergali, come "graficare", "plottare", "misure sospette", "fare senso", e
  simili.
\end{itemize}


\section{In conclusione\ldots}

Scrivere una relazione di laboratorio fa parte del bagaglio culturale da
acquisire quando si lavora nel mondo della ricerca e non solo. \`E importante
ricordarsi i punti descritti in questo capitolo, e applicarli man mano che
diventeranno sempre più automatici e naturali.
Ogni esperienza di laboratorio è un'occasione nuova per esercitarsi anche da
questo punto di vista, in modo da imparare a descrivere il proprio lavoro in
modo sempre più chiaro e rigoroso.
