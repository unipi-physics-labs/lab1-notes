\chapter{Primi passi: misure ed incertezze}
\label{sec:primi_passi}

Il concetto di \emph{incertezza} o \emph{errore} di misura gioca un ruolo centrale
nelle scienze sperimentali. Visto che non è possibile misurare una grandezza fisica
con accuratezza infinita, il risultato di una misura manca di una parte importante
del proprio contenuto quantitativo in assenza di una stima (implicita o esplicita)
dell'incertezza ad essa associata; quest'ultima è elemento imprescindibile per
poter confrontare una misura con una predizione teorica o con un'altra misura.

Nel linguaggio comune è piuttosto inusuale fare menzione esplicita delle
incertezze di misura: vi immaginate comperare mezzo kilogrammo di pane al
supermercato e chiedere all'inserviente
"potrebbe tagliarlo con una tolleranza di 10 grammi, per cortesia?".
Eppure il fatto che, a fronte di una richiesta di mezzo kilogrammo, 600~grammi
sarebbero probabilmente accettabili dalla maggior parte dei clienti, mentre
2~kilogrammi no, ci dice che in fondo la nozione di errore di misura deve
essere in qualche modo connaturata con il nostro senso comune.

In Fisica, e nelle scienze sperimentali in genere, si cerca tipicamente di
evitare ambiguità ed usare le parole in un modo per quanto possibile
quantitativo e ben definito. Ciò nonostante anche in letteratura scientifica
si trovano occasionalmente diciture in cui l'errore di misura è omesso---di
solito volontariamente e consapevolmente. Anche in questo caso, come vedremo,
è il bagaglio di conoscenze comuni (che cominceremo a sviluppare in questo
capitolo) a guidarci.

In questo primo capitolo introdurremo in parallelo due nozioni di incertezza:
quella \emph{massima} e quella \emph{statistica}---la prima perché non
richiede nessuna nozione pregressa ed è utile per introdurre alcuni concetti
fondamentali, e la seconda perché è quella \emph{corretta} e che
utilizzeremo sin dall'inizio (anche se la giustificheremo rigorosamente solo
nel capitolo~\ref{sec:teoria_dei_campioni}). Ove possibile utilizzeremo la prima
per giustificare intuitivamente, per analogia, alcune \emph{ricette} che daremo
inizialmente senza dimostrazione a proposito della seconda.


\section{Unità di misura e ordini di grandezza}
\label{sec:unita_di_misura}

L'accuratezza delle misure raggiunta in alcune aree della Fisica è niente di
meno che stupefacente. In elettrodinamica quantistica, una delle teorie fisiche
meglio verificate sperimentalmente, il momento magnetico dell'elettrone è
misurato a meglio di una parte su $10^{12}$~\cite{PhysRevA.83.052122}---e la
misura è in accordo con i calcoli teorici, che hanno a loro volta raggiunto un
livello di sofisticazione tale da consentire di avere incertezze dello stesso
ordine di grandezza~\cite{PhysRevLett.109.111807}. Tanto per fissare le idee:
sarebbe come misurare la distanza tra Londra e New York con una precisione di
$1~\mu$m (un millesimo di mm): non sfuggirà la portata dei problemi operativi
connessi con il raggiungimento di un tale livello di riproducibilità, che
costituiscono una delle aree di studio delle moderna metrologia.


\subsection{Il Sistema Internazionale di unità di misura}

Il primo sforzo sistematico dell'età moderna in direzione di un sistema
standard di unità di misura è costituito dalla definizione, nella Francia
della fine del XVIII secolo, del sistema metrico decimale---basato sulle
unità standard di \emph{metro} e \emph{kilogrammo}. Definiti inizialmente
come la decimilionesima parte della lunghezza del meridiano terrestre%
\footnote{La cosa ci porterebbe troppo lontano, ma il lettore curioso è
  incoraggiato a leggere la referenza~\cite{dagostini_metro} per una
  discussione interessante e non convenzionale sull'origine storica della
  definizione del metro.}
e come la massa di $1000$~cm$^3$ di acqua alla temperatura di fusione del
ghiaccio, il metro ed il kilogrammo sono stati ridefiniti circa un secolo più
tardi sulla base di due manufatti (il metro campione ed il kilogrammo campione)
realizzati in una lega di platino-iridio e conservati a Parigi, con un certo
numero di repliche \emph{identiche} distribuite ai principali istituti di
metrologia. \`E chiaro che l'utilizzo di manufatti fisici come unità
fondamentali costituisce di per sé un limite alla riproducibilità delle
misure fisiche. Tanto per fare un esempio, il kilogrammo campione, conservato
insieme a sei copie di riferimento presso il \foreign{Bureau International des Poids et Mesures},
è soggetto ad una contaminazione superficiale al livello di $\sim 1~\mu$g
all'anno. Questa contaminazione, largamente reversibile, rende periodicamente
necessarie sofisticate operazioni di pulizia. Nonostante tutti gli accorgimenti
del caso, tra 1889 al 1989 si è osservata una variazione relativa di massa del
kilogrammo campione rispetto alle sei copie al livello di alcune decine di
$\mu$g~\cite{si_unit_mass}.

Ormai adottato quasi universalmente (sia pure con importanti eccezioni), il
Sistema Internazionale (SI) di unità di misura costituisce ad oggi lo stato
dell'arte nella materia. Il SI include 7 unità di base~\cite{si_brochure}
per la misura di altrettante grandezze fisiche: lunghezza, massa, tempo,
corrente elettrica, temperatura termodinamica, quantità di sostanza ed intensità
luminosa---come mostrato nella tabella~\ref{tab:unita_si}.

\begin{table}[!htb]
  \tablehstack{
    \begin{tabular}{llll}
      \hline
      Quantità & Simbolo & Unità & Simbolo\\
      \hline
      \hline
      Lunghezza & $l$ & metro & m\\
      Massa & $m$ & kilogrammo & kg\\
      Tempo & $t$ & secondo & s\\
      Corrente elettrica & $I$, $i$ & ampere & A\\
      Temperatura & $T$ & kelvin & K\\
      Quantità di sostanza & $n$ & mole & mol \\
      Intensità luminosa & $I_\nu$ & candela & cd\\
      \hline
    \end{tabular}
  }{
    \caption{Tabella delle 7 grandezze fisiche di base, con relative unità di
      misura, su cui è fondato il Sistema Internazionale.
      (Adattato da~\cite{si_brochure}.)}
    \label{tab:unita_si}
  }
\end{table}

Storicamente, le unità fondamentali sono state definite in vario modo---utilizzando
manufatti come il metro ed il kilogrammo campione, stati specifici della
materia come il punto triplo dell'acqua, costanti fondamentali della natura come
la velocità della luce, o prescrizioni sperimentali idealizzate.
A partire dal 20 maggio 2019, \emph{il Sistema Internazionale è definito sulla
base del valore numerico di 7 costanti fondamentali della Natura, che si
assumono note esattamente}. In particolare, il SI è il sistema in cui:
\begin{itemize}
\item la frequenza della transizione iper-fine dello stato fondamentale non
  perturbato dell'atomo di $^{133}$Cs vale esattamente
  $\Delta\nu_{Cs} = 9\,192\,631\,770$~Hz;
\item la velocità della luce nel vuoto vale esattamente $c = 299\,792\,458$~m~s$^{-1}$;
\item la costante di Planck vale esattamente $h = 6.626\,070\,15 \times 10^{-34}$~J~s;
\item la carica elementare vale esattamente $e = 1.602\,176\,634 \times 10^{-19}$~C;
\item la costante di Boltzmann vale esattamente $k = 1.380\,649 \times 10^{-23}$~J~K$^{-1}$;
\item il numero di Avogadro vale esattamente $N_A = 6.022\,140\,76 \times 10^{23}$~mol$^{-1}$;
\item l'efficacia luminosa della radiazione monocromatica di frequenza
  $540 \times 10^{12}$~Hz vale esattamente $K_{cd} = 683$~lm~W$^{-1}$.
\end{itemize}
Così, ad esempio, il secondo corrisponde alla durata di $9\,192\,631\,770$
periodi della transizione tra i due livelli iper-fini dello stato fondamentale
dell'atomo di $^{133}$Cs, ed il metro è la lunghezza percorsa dalla luce nel
vuoto in un intervallo di tempo pari a $1/299\,792\,458$~s.

Arrivare ad un sistema in cui tutte le unità fondamentali sono definite in
termini di \emph{invarianti} della natura, affrancandosi dalla necessità di
ricorrere a manufatti, costituisce un traguardo fondamentale per la metrologia
moderna, che ha richiesto uno sforzo lungo più di un secolo.
Ed il Sistema Internazionale è in continua evoluzione, nel tentativo di rispondere
alle esigenze sempre più stringenti dalla scienza e della tecnologia moderna.


\subsection{Unità derivate ed unità riconosciute}

Alcune unità di misura derivate hanno, nel Sistema Internazionale, nomi speciali,
che corrispondono ad una forma convenientemente compatta di combinazioni specifiche
delle 7 unità di base. Una lista parziale di queste unità derivate è presentata nella
tabella~\ref{tab:unita_derivate_si}. (Notate che, ove l'unità di misura prenda il
nome da una persona---tipicamente un illustre scienziato---si scrive convenzionalmente
con la prima lettera minuscola.)

\begin{table}[!htb]
  \tablehstack{
    \begin{tabular}{llll}
      \hline
      Quantità & Unità & Simbolo & Espressione SI\\
      \hline
      \hline
      Angolo & radiante & rad & -\\
      Angolo solido & steradiante & sr & -\\
      Frequenza & hertz & Hz & s$^{-1}$ \\
      Forza & newton & N & m~kg~s$^{-2}$ \\
      Pressione & pascal & Pa & m$^{-1}$~kg~s$^{-2}$ \\
      Energia & joule & J & m$^2$~kg~s$^{-2}$ \\
      Potenza & watt & W & m$^2$~kg~s$^{-3}$ \\
      Temperatura Celsius & grado Celsius & $^\circ$C & K\\
      Carica elettrica & coulomb & C & s~A \\
      Potenziale elettrico & volt & V & m$^2$~kg~s$^{-3}$~A$^{-1}$\\
      Capacità & farad & F & m$^{-2}$~kg$^{-1}$~s$^{4}$~A$^{2}$ \\
      Resistenza & ohm & $\Omega$ & m$^{2}$~kg~s$^{-3}$~A$^{-2}$ \\
      Attività & becquerel & Bq & s$^{-1}$\\
      \hline
    \end{tabular}
  }{
    \caption{Lista (non esaustiva) delle unità derivate che hanno nomi
      speciali nel Sistema Internazionale. Le corrispondenti espressioni in
      termini delle 7 unità fondamentali del Sistema Internazionale sono
      mostrate esplicitamente nell'ultima colonna.
      (Adattato da~\cite{si_brochure}.)
      \label{tab:unita_derivate_si}}
  }
\end{table}

In generale l'espressione, in termini delle unità fondamentali del SI, per una
generica grandezza derivata si ricava, più o meno banalmente, da una
qualsiasi legge fisica che coinvolga l'unità derivata. Così, dalla legge di
Newton
\begin{align*}
  F = ma \quad \text{[m~kg~s$^{-2}$]}
\end{align*}
sappiamo che la forza è una massa (kg) per un'accelerazione (m~s$^{-2}$),
per cui il newton corrisponde a m~kg~s$^{-2}$; una pressione, d'altra parte,
è una forza per unità di superficie (m$^2$)
\begin{align*}
  p = \frac{F}{A} \quad \text{[m$^{-1}$~kg~s$^{-2}$]}
\end{align*}
per cui il pascal corrisponde a m$^{-1}$~kg~s$^{-2}$; l'energia (o il lavoro) è
il prodotto di una forza per uno spostamento (m),
\begin{align*}
L = F s \quad \text{[m$^2$~kg~s$^{-2}$]}
\end{align*}
per cui si misura in m$^2$~kg~s$^{-2}$, e così via.

Vi è infine un certo numero di unità di misura che, pur non facendo parte
del Sistema Internazionale, sono così comunemente usate (ad esempio il
litro per le misure di volume) o sono così profondamente radicate nella
nostra cultura (ad esempio le ore, i minuti ed i secondi per le misure di tempo)
da essere ufficialmente riconosciute ed accettate, come illustrato nella
tabella~\ref{tab:unita_non_si}.
(Per inciso: l'atmosfera, che corrisponde a $101325$~Pa, non fa parte delle
unità riconosciute.)

\begin{table}[!htb]
  \tablehstack{
    \begin{tabular}{llll}
      \hline
      Quantità & Unità & Simbolo & Conversione SI\\
      \hline
      \hline
      Tempo & minuto & min & $1~\text{min} = 60~\text{s}$\\
      & ora & h & $1~\text{h} = 3600~\text{s}$\\
      & giorno & d & $1~\text{d} = 86\,400~\text{s}$\\
      Volume & litro & l & $1~\text{l} = 1~\text{dm}^3$\\
      Massa & tonnellata & t & $1~\text{t} = 1000~\text{kg}$\\
      Energia & electronvolt & eV &
      $1~\text{eV} \approx 1.602 \times 10^{-19} ~\text{J}$ \\
      & erg & erg & $1~\text{erg} = 10^{-7}~\text{J}$ \\
      Pressione & bar & bar & $1~\text{bar} = 10^5~\text{Pa}$ \\
      & mm di Hg & mmHg & $1~\text{mmHg} \approx 133.3~\text{Pa}$ \\
      \hline
    \end{tabular}
  }{
    \caption{Lista (non esaustiva) delle unità che, pur non facendo parte
      del Sistema Internazionale, sono riconosciute in quanto culturalmente o
      storicamente rilevanti. (Adattato da~\cite{si_brochure}.)
      \label{tab:unita_non_si}}
  }
\end{table}


\subsection{Multipli, sottomultipli ed ordini di grandezza}

Le grandezze oggetto di studio in Fisica possono differire tra loro di
numerosi ordini di grandezza. Così, mentre il raggio di un protone è
dell'ordine di $10^{-15}$~m, le sorgenti astronomiche più lontane si
trovano a distanze di svariati Gpc (gigaparsec), cioè a distanze dell'ordine
di~$10^{26}$~m.
Analogamente, mentre la vita media dei bosoni $W$ e $Z_0$ è dell'ordine di
$10^{-25}$~s, l'età del nostro universo è dell'ordine di 10~miliardi di
anni, o $10^{17}$~s. In entrambi i casi la \emph{gamma dinamica} corrisponde a
più di $40$ ordini di grandezza.
A questo scopo il Sistema Internazionale definisce una serie di prefissi
per facilitare la scrittura di grandezze il cui valore sia molto più piccolo
o molto più grande della relativa unità di misura, come mostrato in
tabella~\ref{tab:prefissi_si}.

\begin{table}[!htb]
  \tablehstack{
    \begin{tabular}{llllll}
      \hline
      Fattore & Nome & Simbolo & Fattore & Nome & Simbolo\\
      \hline
      \hline
      $10^{1}$  & deca  & da & $10^{-1}$ & deci & d\\
      $10^{2}$  & hecto & h  & $10^{-2}$ & centi & c\\
      $10^{3}$  & kilo  & k  & $10^{-3}$ & milli & m\\
      $10^{6}$  & mega  & M  & $10^{-6}$ & micro & $\mu$\\
      $10^{9}$  & giga  & G  & $10^{-9}$ & nano & n\\
      $10^{12}$ & tera  & T  & $10^{-12}$ & pico & p\\
      $10^{15}$ & peta  & P  & $10^{-15}$ & femto & f\\
      $10^{18}$ & exa   & E  & $10^{-18}$ & atto & a\\
      $10^{21}$ & zetta & Z  & $10^{-21}$ & zepto & z\\
      $10^{24}$ & yotta & Y  & $10^{-24}$ & yocto & y\\
      \hline
    \end{tabular}
  }{
    \caption{Prefissi per i multipli e sottomultipli decimali delle unità di
      misura del Sistema Internazionale. Il kg si trova nella situazione
      peculiare di rappresentare una delle unità fondamentali pur essendo
      preceduto dal prefisso k. (Adattato da~\cite{si_brochure}.)
      \label{tab:prefissi_si}}
  }
\end{table}

\section{Cenni di analisi dimensionale}
\label{sec:calcolo_dimensionale}

Quando si ha a che fare con un sistema fisico descritto da grandezze non
adimensionali, le equazioni che lo descrivono debbono soddisfare il requisito
di base di essere \emph{dimensionalmente corrette}---cioè le dimensioni
fisiche dei due membri devono essere omogenee. In caso contrario l'equazione
è banalmente sbagliata. Controllare sistematicamente le dimensioni fisiche
in ogni passaggio di un calcolo permette spesso di evitare errori banali.

Nel seguito indicheremo le dimensioni fisiche di una generica grandezza
con il simbolo corrispondente in tabella~\ref{tab:unita_si} racchiuso in
parentesi quadre. Così, e.g.,  diremo che il raggio della Terra ha le
dimensioni di una lunghezza, che indicheremo con $[l]$. Le grandezze derivate
seguono le regole del prodotto, e le dimensioni fisiche di una velocità,
e.g., sono quelle di una lunghezza per l'inverso di un tempo, o $[l~t^{-1}]$.

Supponiamo allora di essere interessati alla relazione che lega il periodo
$T$ di un pendolo alla sua lunghezza $l$. Se qualcuno scrivesse
\begin{align*}
  T = 2\pi \frac{l}{g} \quad \text{(dimensionalmente non corretta)}
\end{align*}
potremmo smentire l'affermazione, anche ignorando le leggi fondamentali della
meccanica, sulla base del fatto che le dimensioni fisiche del membro di destra
dell'equazione sono quelle di un tempo al quadrato e non quelle di un tempo
\begin{align*}
  [t] \neq [l] \times [l~t^{-2}]^{-1} = [t^{2}]
\end{align*}

L'analisi dimensionale, se utilizzata opportunamente, può anche avere potere
predittivo. Nel caso specifico, se ipotizziamo che il periodo $T$ possa
dipendere dalla lunghezza $l$ del pendolo e dalla sua massa $m$, oltre che
dall'accelerazione di gravità $g$, ci troviamo con il problema di capire
quali siano le combinazioni funzionali di queste tre grandezze che forniscono
una quantità con le dimensioni fisiche giuste. Se partiamo dall'ipotesi di
lavoro (arbitraria, ma non irragionevole) che il periodo cercato si possa
scrivere nella forma
\begin{align*}
  T = l^{\alpha_1} m^{\alpha_2} g^{\alpha_3},
\end{align*}
allora da un punto di vista dimensionale possiamo scrivere
\begin{align*}
  [t] = [l]^{\alpha_1} \times [m]^{\alpha_2} \times [l~t^{-2}]^{\alpha_3} =
  [l]^{\alpha_1 + \alpha_3} \times [m]^{\alpha_2} \times [t]^{-2\alpha_3},
\end{align*}
che a sua volta può essere trasformato nel sistema di equazioni
\begin{align*}
  \begin{cases}
    \alpha_1 + \alpha_3 = 0\\
    \alpha_2 = 0\\
    -2\alpha_3 = 1
  \end{cases}
  \text{ossia} \quad
  \begin{cases}
    \alpha_1 = 1/2\\
    \alpha_2 = 0\\
    \alpha_3 = -1/2.
  \end{cases}
\end{align*}
In questo caso specifico l'analisi dimensionale da sola ci permette dunque di
concludere che il periodo del pendolo non può dipendere dalla massa (a meno
che non inseriamo nel problema dall'esterno un'ulteriore grandezza che abbia
la massa nelle sue dimensioni fisiche, ad esempio nella forma di un coefficiente
di attrito viscoso) e che l'equazione cercata deve essere nella forma
\begin{align*}
  T \propto \sqrt{\frac{l}{g}}.
\end{align*}
Come sappiamo la conclusione è corretta ed il fattore di proporzionalità
è, nel limite di piccole oscillazioni, $2\pi$. Per completezza, come
vedremo nella sezione~\ref{sec:piccole_oscillazioni}, nel caso di ampiezza
iniziale $\theta_0$ non nulla si ha un ulteriore fattore moltiplicativo
adimensionale, funzione di $\theta_0$ che non si può ovviamente ricavare per
via dimensionale.

In generale, se siamo interessati alla relazione funzionale che lega una
generica grandezza $y$ ad una serie di grandezze $x_1, x_2 \ldots x_n$
possiamo---anche se non vi è garanzia che questo funzioni---provare a
scrivere la grandezza cercata nella forma
\begin{align}
  y = x_1^{\alpha_1}x_2^{\alpha_2} \ldots x_n^{\alpha_n}.
\end{align}
Il requisito che l'equazione scritta sopra sia dimensionalmente corretta può
essere trasformato in un sistema di equazioni lineari per gli esponenti
$\alpha_1, \alpha_2 \ldots \alpha_n$ che a sua volta può fornire informazioni
non banali sul sistema che stiamo studiando, come illustrato
nell'esempio~\ref{exp:calcolo_dimensionale_keplero}.
Per completezza gli argomenti che abbiamo illustrato brevemente in questa
sezione si possono sviluppare in modo rigoroso e costituiscono il contenuto del
celebre teorema~$\pi$~\cite{teorema_pi}.


\begin{examplebox}[htb]
  \begin{example}[terza legge di Keplero]
    \label{exp:calcolo_dimensionale_keplero}
    Ci proponiamo di ricavare per via dimensionale la terza legge di Keplero,
    ossia la relazione che lega il periodo orbitale $T$ di un pianeta al
    semiasse maggiore $a$ della sua orbita. Assumeremo che gli ulteriori
    parametri che entrano nel problema siano la costante di gravitazione
    universale $G$ e la massa del Sole $m_\odot$ e che possiamo scrivere la
    relazione cercata come $T = G^{\alpha_1} m_\odot^{\alpha_2} a^{\alpha_3}$.
    (Se a questo punto vi state chiedendo perché non abbiamo incluso la
    massa del pianeta nella lista\ldots\ si tratta di un'ottima domanda, cui
    l'analisi dimensionale da sola non fornisce una risposta soddisfacente,
    ma procediamo ugualmente e vediamo dove ci porta il nostro ragionamento.)
    Le unità di misura di $G$ si possono ricavare, e.g. dalla legge di
    gravitazione universale
    \begin{align}
      F = G \frac{m_1 m_2}{r^2} \quad \text{da cui} \quad
      G = \frac{Fr^2}{m_1 m_2}
    \end{align}
    e sono, nel sistema internazionale m$^3$~kg$^{-1}$~s$^{-2}$. Ragionando come
    prima si ha allora
    \begin{align*}
      [t] = [l^3~m^{-1}~t^{-2}]^{\alpha_1} \times [m]^{\alpha_2} \times [l]^{\alpha_3} =
      [l]^{3\alpha_1 + \alpha_3} \times [m]^{\alpha_2 - \alpha_1} \times [t]^{-2\alpha1},
    \end{align*}
    da cui
     \begin{align}\label{eq:legge_di_keplero}
      \begin{cases}
        3\alpha_1 + \alpha_3 = 0\\
        \alpha_2 - \alpha_1 = 0\\
        -2\alpha_1 = 1
      \end{cases}
      \quad \text{ossia} \quad
      \begin{cases}
        \alpha_1 = -1/2\\
        \alpha_2 = -1/2\\
        \alpha_3 = 3/2.
      \end{cases}
      \quad \text{e infine} \quad
      T \propto \frac{a^{\frac{3}{2}}}{\sqrt{Gm_\odot}}.
     \end{align}
     La conclusione (tutt'altro che ovvia) è corretta ed il fattore di
     proporzionalità è di nuovo $2\pi$---ma, va da sé, si tratta di una
     semplice coincidenza.
  \end{example}
\end{examplebox}


\section{Il concetto di incertezza di misura}
\label{sec:errore_max}

Nella sua formulazione più elementare, il concetto di \emph{errore massimo}
è legato alla domanda: "qual è il più piccolo intervallo che contiene
con certezza il valore numerico della quantità che sto misurando?"
In altre parole, se scriviamo il risultato della misura di una generica
grandezza fisica $x$ come:
\begin{align}\label{eq:errore_max}
  x = \hat{x} \pm \Delta x~\text{[unità di misura]}
\end{align}
intendendo $\Delta x$ come errore massimo, allora in effetti stiamo dicendo che
l'intervallo $[\hat{x} - \Delta x, \hat{x} + \Delta x]$ è il più piccolo
intervallo possibile che ci dia la certezza di includere il valore (incognito)
di~$x$. Se scriviamo, ad esempio, $l = 12.4 \pm 0.2$~cm, stiamo dicendo di
essere certi che $l$ sia compreso tra $12.2$ e~$12.6$~cm, e che l'intervallo
$[12.2, 12.6]$~cm è il più piccolo che ci possa dare tale certezza.

Gli ingredienti della~\eqref{eq:errore_max} hanno ciascuno un significato ben
preciso che è bene fissare il prima possibile:
\begin{itemize}
\item $x$ è il valore, incognito, della grandezza che vogliamo misurare, che
  chiameremo \emph{misurando} (cercando deliberatamente di evitare l'espressione
  \emph{valore vero}, che pure si trova usata in letteratura);
\item $\hat{x}$ è la \emph{miglior stima} di $x$ che possiamo fornire a
  partire dai dati a nostra disposizione---che chiameremo anche
  \emph{valore centrale} o \emph{migliore stima} della misura;
\item $\Delta x$ è l'incertezza di misura---o più precisamente, in questo
  contesto, l'\emph{errore massimo}, nel senso che abbiamo
  precisato sopra. (Notiamo che, per definizione, $\Delta x$ rappresenta la
  lunghezza di un intervallo ed è perciò una grandezza definita positiva.)
\end{itemize}
Vale la pena sottolineare che il risultato di una misura non ha senso se non
sono indicate, ove necessario, le unità di misura.

Quando il risultato di una misura fluttua, la nozione di errore massimo porta
in sé una contraddizione logica dovuta al fatto che, da un punto di vista
operativo, non è possibile garantire a priori che una nuova misura della
stessa grandezza non fornisca un valore al di fuori dell'intervallo iniziale di
incertezza. Ove questo accada siamo costretti ad allargare tale intervallo---ci
troviamo, cioè, nella situazione assurda in cui acquisire nuova informazione
può solo peggiorare (o, al massimo, lasciare invariato) il nostro stato di
conoscenza, nella misura in cui esso è determinato dall'incertezza di misura.

Questo è il motivo per cui in pratica non useremo \emph{mai} il concetto
di errore massimo, ma utilizzeremo fin da subito quello che comunemente va
sotto il nome di errore statistico:
\begin{align}\label{eq:errore_stat}
  x = \hat{x} \pm \sigma_x~\text{[unità di misura]}.
\end{align}
La~\eqref{eq:errore_stat} ha un significato fondamentalmente diverso
dalla~\eqref{eq:errore_max}: come vedremo in dettaglio nel
capitolo~\ref{sec:teoria_dei_campioni} essa definisce un intervallo che non ci
dà la \emph{certezza}, ma solo una \emph{probabilità} ben definita di
contenere il valore del misurando. (Purtroppo introdurremo il concetto di
probabilità solo nel capitolo~\ref{sec:probabilita}.)
A questo livello potrebbe sembrare una mera questione di notazione (una
$\sigma$ al posto di una $\Delta$), ma si tratta in realtà di una differenza
profonda e carica di conseguenze.


\subsection{Errore assoluto ed errore relativo}
\label{sec:errore_assoluto_relativo}

Definiamo \emph{errore relativo} (detto anche \emph{errore percentuale}) il
rapporto tra l'incertezza della misura (detta anche \emph{errore assoluto}) ed
il suo valore centrale. In particolare, utilizzando la nostra notazione:
\begin{align}\label{eq:errore_max_relativo}
  \text{Errore relativo} := \frac{\sigma_x}{\abs{\hat{x}}}.
\end{align}
Poiché l'incertezza ha, per ovvie ragioni, le stesse dimensioni fisiche della
misura a cui si riferisce, l'errore relativo è una quantità adimensionale
(e definita positiva).

\begin{examplebox}
  \begin{example}
    Si misura una lunghezza $l$ ottenendo il risultato $l = 12.4 \pm 0.2$~cm.
    L'errore relativo è in questo caso $0.016$, vale a dire $1.6\%$.
  \end{example}
\end{examplebox}


\subsection{Precisione ed accuratezza}

In letteratura si trova la distinzione tra i concetti di \emph{accuratezza},
intesa come accordo tra il valore della misura e quello del misurando, e di
\emph{precisione}, intesa come consistenza tra risultati di successive misure
della stessa quantità nelle medesime condizioni.

Si tratta di una distinzione molto rilevante, poiché un dato strumento può
essere molto accurato ma poco preciso (se fornisce in media la risposta giusta,
ma con fluttuazioni rilevanti tra misurazioni successive) o---il che è
potenzialmente più pericoloso---molto preciso ma poco accurato (se fornisce
in modo estremamente riproducibile la risposta sbagliata). Avremo tempo di
tornare sull'argomento in seguito, ma diciamo fin da subito che cercheremo
scrupolosamente di usare questi due termini nell'accezione appena indicata.

\pgffigfour[!b]{precisione_accuratezza_1}{precisione_accuratezza_2}{precisione_accuratezza_3}{precisione_accuratezza_4}{
  Illustrazione grafica dei concetti di precisione ed accuratezza
  (adattato da \url{https://www.roma1.infn.it/~dagos/BMS/node116.html}).
  In tutte e quattro le figure la linea verticale rappresenta il misurando
  mentre la curva serve a dare un'idea, che a
  questo livello è puramente qualitativa, delle fluttuazioni delle misure
  attorno al misurando stesso. Va da sé che gli aggettivi "buono" e
  "scarso" debbono essere intesi in senso relativo, come metrica di
  confronto tra le varie situazioni.
}

Per completezza, la figura~\ref{fig:precisione_accuratezza_1_four} illustra
graficamente i concetti di precisione ed accuratezza che abbiamo appena
definito. In ciascuno dei quattro grafici la curva rappresenta in modo
qualitativo le fluttuazioni delle misure attorno al misurando (incognito), che
è a sua volta indicato dalla linea verticale. Con leggero abuso di
linguaggio potremmo dire che la precisione ha a che vedere con la larghezza
della curva, mentre l'accuratezza ha a che vedere con la vicinanza tra il
massimo della curva ed il misurando.


\section{Cifre significative e convenzioni di scrittura}
\label{sec:cifre_significative}

Prima ancora di capire come si stima in pratica l'incertezza massima di misura,
ci soffermiamo per un attimo su alcune questioni, semplici ma importanti, di
notazione. In un articolo scientifico non troverete mai scritto (sperabilmente)
che la massa di un oggetto è $74.562572 \pm 0.024894$~kg. Ed il motivo è
semplice: se l'errore di misura è dell'ordine di 25~g, è palesemente
assurdo scrivere il valore centrale fino al~$\mu$g (cioè fino al millesimo di
mg!). Le ultime tre cifre (almeno) sono irrilevanti---semplice rumore numerico.

Facciamo un passo indietro. La questione si può inquadrare quantitativamente
attraverso la nozione di \emph{cifra significativa}, che si definisce mediante
tre semplici regole, illustrate dall'esempio~\ref{exp:cifre_significative}.
Dato un valore generico:
\begin{itemize}
\item la cifra più significativa è quella più a sinistra diversa
  da zero;
\item la cifra meno significativa è quella più a destra (inclusi gli
  zeri a destra del separatore decimale, ma fatta eccezione per gli zeri che
  definiscono l'ordine di grandezza per i numeri interi);
\item tutte le cifre comprese tra la più significativa e la
  meno significativa sono significative.
\end{itemize}

\begin{examplebox}
  \hstack[0.8]{\begin{example}\label{exp:cifre_significative}
      Negli esempi qui di fianco indichiamo con una sottolineatura la cifra meno
      significativa, con una sopralineatura quella più significativa e tra
      parentesi quadre il numero di cifre significative, in una serie di
      situazioni tipiche. (Si tratta di una cosa di importanza basilare, per cui
      assicuratevi di dominare l'argomento prima di procedere.)
  \end{example}}{
    \vspace*{7pt}
    \rule{12pt}{0pt}%
    \begin{tabular}{ll}
      $\overline{3}11\underline{5}$ & [$4$]\\
      $\overline{3}212.\underline{5}$ & [$5$]\\
      $0.0\overline{3}2\underline{5}$ & [$3$]\\
      $0.0\overline{3}0\underline{0}$ & [$3$]\\
      $\overline{3}00.0\underline{0}$ & [$5$]
    \end{tabular}
  }
\end{examplebox}

Torniamo allora al nostro problema originale. La nozione di cifra significativa
è strettamente connessa con quella di arrotondamento del valore centrale
della misura ed il numero di cifre significative con cui si deve scrivere tale
valore è determinato dall'incertezza associata. Più precisamente:
\emph{si scrive l'incertezza di misura con una o al massimo due cifre
  significative (arrotondando opportunamente), e si arrotonda il valore
  centrale in modo da essere consistente, in termini di cifre decimali, con
  l'incertezza associata}.

Così $74.563 \pm 0.025$~kg e $74.56 \pm 0.02$~kg sono entrambe scritture
accettabili per la massa dell'oggetto con cui abbiamo aperto questa sezione.
(Va da sé che, quando i risultati di misure, dirette o indirette, vengono
usati per stimare grandezze derivate, è buona norma avere cura che gli
arrotondamenti effettuati nei passaggi intermedi non introducano errori
significativi nel risultato finale.) Il lettore è incoraggiato ad esaminare
scrupolosamente l'esempio~\ref{exp:errori_cifre_significative}, che rappresenta
un breve campionario di modi in cui \emph{non} scrivere il risultato della
misura.

\begin{examplebox}
  \begin{example}\label{exp:errori_cifre_significative}
    La seguente è una breve lista di errori tipici che si possono compiere;
    i valori tra parentesi quadre indicano la (o una) possibile versione
    corretta.

    \medskip
    \begin{tabular}{l@{\hskip 18pt}l@{\hskip 18pt}l}
      $3.436 \pm 0.542$ & incertezza con troppe (tre) cifre significative &
      [$3.44\pm0.54$]\\
      $3.4 \pm 0.54$ & incertezza con una cifra significativa di troppo &
      [$3.4 \pm 0.5$]\\
      & oppure valore centrale con una cifra significativa mancante &
      [$3.40 \pm 0.54$]\\
      $3.44 \pm 0.5$ & valore centrale con una cifra significativa di troppo &
      [$3.4 \pm 0.5$]\\
      & oppure incertezza con una cifra significativa mancante &
      [$3.44 \pm 0.50$]\\
      $3000 \pm 100$ & scrittura ambigua per gli zeri meno significativi &
      [$(3.0 \pm 0.1)\times 10^3$]
    \end{tabular}
  \end{example}
\end{examplebox}

La capacità di scrivere correttamente il risultato di una misura è di
fondamentale importanza e va acquisita il prima possibile. Troncare l'incertezza
di misura a una o due cifre significative può a prima vista apparire come una
genuina perdita di informazione, ma è importante realizzare subito che così
non è, perché le cifre che tronchiamo
\emph{non contengono nessuna informazione}.
Se misuriamo la lunghezza di un tavolo con un metro a nastro non c'è nessuna
informazione utile nella cifra corrispondente ad $1~\mu$m; quando ci pesiamo su
una bilancia pesapersone non c'è nessuna informazione utile nella cifra
corrispondente al decimo di grammo. Per cui, lo ripetiamo perché non vi
è rischio di enfatizzare troppo il concetto,
\emph{scrivere un'incertezza con più di due cifre significative non ha alcun
  senso}. Nella maggior parte dei casi una è sufficiente.

Per completezza notiamo che non è inconsueto in letteratura che l'incertezza
di misura venga (consapevolmente) omessa. In questi casi la precisione del
risultato sperimentale è, almeno parzialmente, implicita nel modo in cui il
risultato stesso è scritto, e si può assumere che l'errore sia da
considerarsi pari ad una unità della cifra meno significativa del valore
centrale. Così i valori $5.0$~m, e $5.00$~m hanno significati diversi:
il primo è da leggersi come $5.0 \pm 0.1$~m, mentre il secondo come
$5.00 \pm 0.01$~m. Inutile a dirsi, riportare esplicitamente l'errore di misura
è il modo migliore per evitare ambiguità di sorta.


\subsection{Alcune precisazioni sugli arrotondamenti}

Nella sezione precedente abbiamo accennato sommariamente al problema degli
arrotondamenti, ma ci sono un paio di dettagli che vale la pena sviscerare
prima di procedere.

Tecnicamente per \emph{arrotondamento} si intende il processo di riduzione
del numero di cifre significative con cui si rappresenta una quantità fisica.
L'arrotondamento può essere per difetto, se il valore arrotondato è minore
di quello originale, e per eccesso in caso contrario. Ove non si specifichi
altrimenti, si dà per inteso che gli arrotondamenti si eseguono in modo
che il valore arrotondato sia quello più vicino a quello originale. Così, se
vogliamo arrotondare a 2~cifre significative si ha
\begin{align*}
  3.21 & \rightarrow 3.2 \quad \text{(arrotondamento per difetto)} \\
  3.27 & \rightarrow 3.3 \quad \text{(arrotondamento per eccesso).}
\end{align*}

Tecnicamente questa prescrizione non è ben definita quando i valori
arrotondati per eccesso e per difetto sono equidistanti dal valore originale,
e.g.,
\begin{align*}
  3.25 \rightarrow\ ? \quad \text{(arrotondamento per difetto o per eccesso?).}
\end{align*}
In questi casi arrotondare sempre per difetto o sempre per eccesso porterebbe
ad una sottostima o una sovrastima sistematica (e potenzialmente pericolosa)
delle misure---che è per ovvie ragioni da evitare. Una prescrizione possibile
è allora arrotondare per difetto se la cifra a sinistra del $5$ è pari
(come è il $2$ in questo caso) ed arrotondare per eccesso se la cifra è
dispari%
\footnote{Tecnicamente l'idea funziona se la frequenza delle cifre pari è
  uguale a quella delle cifre dispari---cosa che, pur essendo apparentemente
  ragionevole, a rigore non è garantita, come vedremo nella
  sezione~\ref{sec:legge_di_benford}.
  Ma abbiamo discusso il problema a sufficienza ed è arrivato il momento di
  dichiararci soddisfatti e procedere.}%
:
\begin{align*}
  3.25 & \rightarrow 3.2 \quad \text{(arrotondamento per difetto)} \\
  3.35 & \rightarrow 3.4 \quad \text{(arrotondamento per eccesso).}
\end{align*}


\section{Digressione: \python\ come calcolatore tascabile}

Interrompiamo per un attimo il filo della discussione per introdurre \python,
il linguaggio di programmazione che ci accompagnerà lungo tutto il nostro
viaggio. Per poter essere utilizzati, gli esempi inclusi in questa sezione, e
quelli che seguiranno, presuppongono che abbiate un'installazione funzionante
di \python. Sottolineiamo che queste dispense non hanno la benché minima pretesa
di essere un corso auto-contenuto di \python, ed il lettore che non fosse già
familiare con la materia avrà bisogno senza dubbio di risorse aggiuntive per
mettersi in condizione di utilizzare gli esempi mostrati nel seguito---ma il
\foreign{web} è letteralmente stracolmo di istruzioni e \emph{tutorial} al
proposito, per cui non dovrebbe essere difficile trovare quello che fa più al
caso vostro.

Nella modalità di utilizzo più semplice, \python\ può essere usato in modo
\emph{interattivo}---ovvero lanciando l'interprete da linea di comando ed
inserendo comandi \python\ (validi) nell'interprete stesso. La cosa è
interessante perché in questa modalità l'interprete \emph{risponde} ad
ogni comando stampando sullo schermo il risultato del comando stesso ogni qual
volta si preme il tasto di invio, cosa che può risultare utile di tanto in
tanto---specialmente quando si sperimenta.

L'esempio che segue illustra l'utilizzo di \python\ come una sorta di calcolatore
tascabile. Notiamo in particolare:
\begin{itemize}
\item le quattro operazioni elementari sono implementate attraverso gli
  operatori \cchar{+}, \cchar{-}, \cchar{*} e \cchar{/}, con le consuete regole
  di precedenza;
\item l'operatore \cchar{**} (che ha precedenza sui quattro elencati sopra)
  corrisponde all'elevamento a potenza;
\item la funzione \code{round()} permette di arrotondare un numero reale
  all'intero più vicino.
\end{itemize}

\begin{Verbatim}
[lbaldini@nbbaldini ~]$ python
Python 3.7.4 (default, Jul  9 2019, 16:32:37)
[GCC 9.1.1 20190503 (Red Hat 9.1.1-1)] on linux
Type "help", "copyright", "credits" or "license" for more information.
>>> (3 + 3) / 2.2
2.727272727272727
>>> 2.4 * 15.3
36.72
>>> 3.**2.
9.0
>>> round(2.49)
2
>>> round(2.51)
3
>>>
\end{Verbatim}

Chi non avesse già familiarità con la cosa è vivamente incoraggiato ad
aprire un interprete \python\ e sperimentare. Non vi è rischio di rompere
niente ed è un buon modo per prepararsi al lavoro vero che arriverà a breve!

Nella stragrande maggioranza dei casi, tuttavia, non lavoreremo in modalità
interattiva; scriveremo viceversa la nostra serie di comandi in un \emph{file}
di testo nella forma di un \emph{programma} che l'interprete esegue
sequenzialmente. Questo ha il vantaggio che il lavoro che facciamo non va
perduto al termine della sessione interattiva, ma può essere salvato,
modificato, condiviso ed eseguito all'occorrenza. Essere capaci di scrivere ed
eseguire il nostro primo programma, per quanto semplice, è il secondo passo
fondamentale verso l'arte della programmazione. Da qui in poi la strada è in
discesa.

Come vedremo, gli strumenti che l'interprete \python\ mette a disposizione di
\emph{default} all'avvio non ci basteranno. Dovremo utilizzare una serie
di librerie, sia dalla libreria standard di \python\ che da pacchetti esterni,
che implementano funzionalità più avanzate, e che vedremo via via. La
libreria \code{math}, ad esempio, mette a disposizione una serie di costanti
(e.g., $\pi$) e di funzioni matematiche (esponenziale, logaritmi, funzioni
trigonometriche) che ci saranno utili in seguito.

\begin{snippet}[htb!]
  \bigskip % This is ugly and should be taken care of automagically.
  \hstack[0.45]{\begin{Verbatim}[label=\makebox{\href{https://github.com/unipi-physics-labs/lab1-notes/tree/main/snippy/rounding.py}{https://github.com/.../rounding.py}},commandchars=\\\{\}]
\PY{k+kn}{import}\PY{+w}{ }\PY{n+nn}{math}

\PY{n}{a} \PY{o}{=} \PY{l+m+mf}{2.34}
\PY{c+c1}{\PYZsh{} Various roundings...}
\PY{n+nb}{print}\PY{p}{(}\PY{n+nb}{round}\PY{p}{(}\PY{n}{a}\PY{p}{)}\PY{p}{)}
\PY{n+nb}{print}\PY{p}{(}\PY{n}{math}\PY{o}{.}\PY{n}{floor}\PY{p}{(}\PY{n}{a}\PY{p}{)}\PY{p}{)}
\PY{n+nb}{print}\PY{p}{(}\PY{n}{math}\PY{o}{.}\PY{n}{ceil}\PY{p}{(}\PY{n}{a}\PY{p}{)}\PY{p}{)}
\PY{n+nb}{print}\PY{p}{(}\PY{n}{math}\PY{o}{.}\PY{n}{trunc}\PY{p}{(}\PY{n}{a}\PY{p}{)}\PY{p}{)}
\PY{c+c1}{\PYZsh{} ...and a bit of trigonometry.}
\PY{n+nb}{print}\PY{p}{(}\PY{n}{math}\PY{o}{.}\PY{n}{sin}\PY{p}{(}\PY{n}{math}\PY{o}{.}\PY{n}{pi} \PY{o}{/} \PY{l+m+mf}{4.0}\PY{p}{)}\PY{p}{)}
\PY{n+nb}{print}\PY{p}{(}\PY{l+m+mf}{1.0} \PY{o}{/} \PY{n}{math}\PY{o}{.}\PY{n}{sqrt}\PY{p}{(}\PY{l+m+mf}{2.0}\PY{p}{)}\PY{p}{)}

[Output]
2
2
3
2
0.7071067811865475
0.7071067811865475
\end{Verbatim}
}{
    \caption{Illustrazione di un piccolo sottoinsieme delle funzioni che
      la libreria standard \code{math} di \python\ mette a disposizione.
      Per ciò che riguarda l'arrotondamento ad intero dei numeri in virgola
      mobile: \code{round()} arrotonda all'intero più vicino,
      \code{math.floor()} e \code{math.ceil()} arrotondano per difetto e per
      eccesso, rispettivamente, e \code{math.trunc()} tronca la parte decimale.
    }
    \label{snip:rounding}
  }
\end{snippet}


\section{Stima dell'incertezza in condizioni di ripetitività}
\label{sec:stima_errore_max}

Torniamo adesso alla discussione delle incertezze di misura che abbiamo
temporaneamente messo da parte dopo la sezione~\ref{sec:errore_max}.
Quando una misura è fatta più volte in condizioni di ripetitività si
hanno due casi tipici: (i) la dispersione delle misure è nulla---cioè
otteniamo sempre lo stesso risultato---oppure (ii) il valore ottenuto
fluttua---cioè misure successive forniscono risultati in generale diversi
tra loro.


\subsection{Misure in regime di dispersione nulla: la risoluzione strumentale}

La \emph{risoluzione} di uno strumento è la più piccola \emph{variazione}
della quantità da misurare che è possibile apprezzare con lo strumento
stesso%
\footnote{Notiamo, per completezza, che la risoluzione in generale non coincide
con la più piccola quantità che uno strumento può apprezzare (che
si dice \emph{sensibilità}). Notiamo anche che la risoluzione non coincide
necessariamente con l'unità di formato dello strumento (e.g., la divisione
più piccola di una scala graduata).}.
Così la risoluzione di un metro a nastro è $1$~mm, quella di un calibro
ventesimale $0.05$~mm e quella di un calibro Palmer $0.01$~mm, tanto
per fare alcuni esempi (vedi appendice~\ref{sec:strumenti}). In pratica capita
di frequente che la risoluzione dello strumento che utilizziamo non sia nota
a priori, e vedremo nel seguito alcuni modi comunemente usati per stimarla, ove
ciò fosse necessario.

\emph{Quando eseguiamo una misura in condizioni di dispersione nulla, cioè
  misure successive della stessa quantità forniscono sempre lo stesso
  risultato, almeno nei casi più semplici si può assumere la risoluzione
  strumentale (o metà della risoluzione strumentale) come errore di misura}.
(Avvertiamo fin da subito che torneremo sulla questione nella
sezione~\ref{sec:errore_campionamenti_singoli} e la risposta formalmente
corretta richiederà un fattore $\sqrt{12}$ al denominatore, ma per il momento
la cosa non ci interessa.)

\begin{examplebox}
  \begin{example}\label{exp:metro_a_nastro}
    Se misuriamo la lunghezza $l$ di un tavolo con un metro a nastro (con una
    risoluzione di $1$~mm) ottenendo il valore $189.4$~cm, scriveremo la nostra
    misura come $l = 189.4 \pm 0.1$~cm.
  \end{example}

  \begin{example}\label{exp:termometro_digitale}
    Misuriamo la temperatura $T$ dell'acqua in un recipiente con un termometro
    digitale ed il \emph{display} indica $34.5^\circ$C. In mancanza di
    informazioni aggiuntive (ad esempio fornite dal \emph{data sheet} dello
    strumento), dobbiamo assumere che tutte le cifre siano significative e che
    la risoluzione dello strumento sia $0.1^\circ$C. Scriveremo dunque il
    risultato della misura come $T = 34.5 \pm 0.1^\circ$C.
  \end{example}
\end{examplebox}

Per quanto apparentemente banali, gli esempi~\ref{exp:metro_a_nastro}
e~\ref{exp:termometro_digitale} non devono trarre in inganno: la vita è
spesso più complicata e a volte identificare l'accuratezza di una misura
(nel senso della distanza dal misurando) con la risoluzione dello strumento
utilizzato può portare ad errori apprezzabili.
In una determinata situazione potremmo avere difficoltà pratiche nella
lettura del metro a nastro o nella definizione stessa della lunghezza da
misurare (ad esempio per problemi di messa a fuoco nelle misure di ottica).
Oppure il nostro termometro potrebbe non essere correttamente calibrato (di un
fattore che non conosciamo) per cui, dato lo stesso oggetto, ci fornisce sempre
la stessa temperatura, ma quest'ultima è sistematicamente sbagliata.
(Torneremo brevemente sulla questione dei cosiddetti errori sistematici nella
sezione~\ref{sec:errori_sistematici}.)


\subsection{Un semplice esperimento: l'incertezza di misura con il metro a
  nastro}
\label{sec:interpolazione_metro_a_nastro}

A corredo della discussione sulla stima dell'incertezza di misura, in
figura~\ref{fig:interpolazione_metro_2} mostriamo i risultati di un semplice
esperimento in cui si è chiesto ad un gruppo di $33$ studenti di Fisica del
primo anno di misurare indipendentemente con un metro a nastro,
\emph{interpolando tra le divisioni al meglio delle loro capacità},
la lunghezza $l$ del lato di una lastrina di ottone.

La lunghezza stessa era stata preliminarmente (ed accuratamente) misurata con
un calibro cinquantesimale ottenendo il valore $l = 4.323 \pm 0.002$~cm.
Si tratta di una situazione interessante in cui, dal punto di vista della misura
con il metro a nastro, il misurando è noto a priori con precisione
\emph{infinita}---la risoluzione del calibro cinquantesimale è molto minore
di quella del metro a nastro, e la cosa può essere utilizzata per ricavare
una serie di conseguenze sulla misura stessa. Tenetelo a mente, perché l'idea
di \emph{calibrare} uno strumento relativamente poco preciso con uno molto
più preciso è utilizzata molto comunemente in Fisica sperimentale.

\pgffigone{interpolazione_metro_2}{
  Istogramma delle lunghezze $l$ del lato di una lastrina di ottone misurate
  indipendentemente da un gruppo di $33$ studenti, interpolando ad occhio tra le
  divisioni di un metro a nastro. La freccia indica il valore misurato con il
  calibro cinquantesimale, che per i nostri scopi coincide a tutti gli effetti
  con il misurando, mentre le due linee verticali tratteggiate indicano
  l'intervallo al di fuori del quale siamo di fronte ad un banale errore di
  lettura.
}

L'istogramma in figura~\ref{fig:interpolazione_metro_2} (chi non sapesse
cosa è un istogramma può saltare per un attimo alla
sezione~\ref{sec:barre_e_istogrammi}) è interessante sotto
diversi aspetti. Se non avessimo richiesto di interpolare tra le tacche del
metro a nastro ci aspetteremmo di essere in regime di dispersione nulla (cioè
tutti avrebbero dovuto leggere $l = 4.3$~cm, ovvero il valore corrispondente
alla divisione del metro più vicina al misurando), ma è chiaro che qui
il panorama è estremamente più complicato:
\begin{itemize}
\item il picco più alto è proprio in corrispondenza del valore misurato
  con il calibro cinquantesimale---una frazione significativa degli studenti
  (per la precisione $9$ su $33$) ha fornito una misura con una precisione
  di $0.1$~mm o meno, il che dimostra la possibilità di interpolare ben al
  di sotto della risoluzione strumentale (a patto che si facciano le cose con
  cura);
\item vi sono due picchi intermedi a $l = 4.3$~mm ($4$ studenti) e $l = 4.4$~mm
  (3 studenti)---in questo caso evidentemente gli studenti hanno deciso di non
  interpolare e di leggere direttamente il valore della misura come la divisione
  più vicina sul metro (e il primo gruppo ha letto giusto, mentre il secondo
  no);
\item $9$ studenti su $33$ hanno riportato un valore che dista più di mezza
  divisione ($0.5$~mm) dal misurando, e $3$ hanno sbagliato di più di una
  divisione ($1$~mm)---tutti questi casi, e sono circa il $25\%$ costituiscono
  banali errori di lettura.
\end{itemize}

Ora, è chiaro che se anche in una situazione apparentemente tanto semplice
il risultato è così variegato, la nostra ricetta iniziale di prendere
la risoluzione strumentale come stima dell'incertezza di misura va ponderata
con attenzione caso per caso. Sicuramente sarebbe sbagliata, per motivi diversi,
per la maggior parte degli studenti che hanno preso parte a questo
esperimento.


\subsection{Fluttuazioni casuali: stima a posteriori dell'incertezza}

Il caso in cui i valori misurati sono soggetti a fluttuazioni statistiche
è chiaramente, dal nostro punto di vista, più interessante---e quello in
cui la nozione di errore massimo diventa problematica. Le fluttuazioni possono
essere dovute alle caratteristiche dello strumento di misura (ad esempio il
rumore in un dispositivo elettronico), alle proprietà intrinseche del sistema
fisico sotto studio (ad esempio effetti quanto-meccanici), oppure ad una
combinazione dei due fattori.

In situazioni di questo tipo possiamo stimare a posteriori l'incertezza
in base alla dispersione attorno al valor medio di misure successive eseguite
in condizioni di ripetitività. Se eseguiamo, cioè, $n$ misure
$x_1, x_2 \ldots x_n$ della grandezza $x$ cui siamo interessati, a questo
livello potrebbe sembrare ragionevole prendere la media aritmetica delle misure
stesse come valore centrale e la semidispersione come errore massimo:
\begin{align}\label{eq:errore_max_fluttuazioni}
  \hat{x} = \frac{1}{n} \sum_{i = 1}^n x_i \quad \text{e} \quad
  \Delta x = \frac{x_{\max} - x_{\min}}{2}.
\end{align}

Per quanto apparentemente sensata, si tratta di nuovo di una ricetta che
possiamo utilizzare per avere un'idea dell'incertezza di misura in alcune
situazioni semplici, ma che è afflitta dai problemi insanabili descritti
alla fine della sezione~\ref{sec:errore_max}. Il primo è che, data una serie
finita di misure, nessuno ci assicura che misure successive non possano cadere
al di fuori della semidispersione iniziale---il che, a rigore, contraddice la
nostra definizione di errore massimo. Il secondo è che all'aumentare del
numero $n$ di misure la semidispersione non può che aumentare, il che ci
lascia nella situazione apparentemente assurda in cui eseguire nuove misure
(cioè aggiungere informazione) causa un incremento dell'incertezza.

Come abbiamo detto all'inizio, in pratica non utilizzeremo mai l'errore massimo,
ed è giunto dunque il momento di dare la prima ricetta priva di
dimostrazione. In presenza di fluttuazioni casuali stimeremo a posteriori
l'errore statistico con quella che prende il nome di
\emph{deviazione standard della media}
\begin{align}\label{eq:stdev_media_intro}
  \sigma_x = \sqrt{\frac{1}{n(n - 1)}\sum_{i = 1}^n (x_i - \hat{x})^2}.
\end{align}
Senza scendere nei dettagli, la~\eqref{eq:stdev_media_intro} costituisce una
sorta di valor medio degli scarti quadratici delle singole misure rispetto
al loro valor medio. Se vi state chiedendo il senso di utilizzare la media
quadratica anziché quella aritmetica, anticipiamo che quest'ultima sarebbe
identicamente nulla a causa del fatto che le fluttuazioni positive e quelle
negative tendono a compensarsi:
\begin{align*}
  \sum_{i = 1}^n (x_i - \hat{x}) = \sum_{i = 1}^n x_i - n \hat{x} =
  n \left( \frac{1}{n} \sum_{i = 1}^n x_i - \hat{x} \right) =
  n (\hat{x} - \hat{x}) = 0.
\end{align*}
Avremo occasione di tornare sull'argomento e sviscerarlo in dettaglio nella
sezione~\ref{sec:deviazione_standard_media}.

\begin{examplebox}
  \begin{example}\label{exp:pendolo_errore}
    Vogliamo misurare il periodo $T$ di un pendolo ed abbiamo a disposizione
    un cronometro digitale con risoluzione di $0.01$~s. Si tratta di un caso
    interessante, in cui l'apparato di misura è costituito, per così dire,
    dal combinato del cronometro e della persona che fisicamente lo utilizza.
    Non è banale quantificare a priori l'incertezza di misura, ma possiamo
    sospettare che le fluttuazioni del tempo di reazione umano rappresentino il
    contributo più rilevante.
    Eseguiamo dunque $5$ misurazioni del periodo ottenendo i valori:
    $2.12$~s, $2.22$~s, $2.16$~s, $2.10$~s e $2.15$~s. Seguendo
    la~\eqref{eq:errore_max_fluttuazioni}, avremmo $T = 2.15 \pm 0.06$~s.
    In realtà la risposta corretta è la~\eqref{eq:stdev_media_intro}, che
    fornisce $T = 2.15 \pm 0.02$~s, o anche $T = 2.15 \pm 0.02$~s.
    (Come atteso, l'errore statistico è più piccolo dell'errore massimo.)
  \end{example}
\end{examplebox}

\begin{snippet}[htb!]
  \bigskip % This is ugly and should be taken care of automagically.
  \hstack[0.47]{\begin{Verbatim}[label=\makebox{\href{https://github.com/unipi-physics-labs/statnotes/tree/main/snippy/mean_stdev.py}{https://github.com/.../mean\_stdev.py}},commandchars=\\\{\}]
\PY{k+kn}{import}\PY{+w}{ }\PY{n+nn}{numpy}\PY{+w}{ }\PY{k}{as}\PY{+w}{ }\PY{n+nn}{np}

\PY{c+c1}{\PYZsh{} Input data.}
\PY{n}{T} \PY{o}{=} \PY{p}{[}\PY{l+m+mf}{2.12}\PY{p}{,} \PY{l+m+mf}{2.22}\PY{p}{,} \PY{l+m+mf}{2.16}\PY{p}{,} \PY{l+m+mf}{2.10}\PY{p}{,} \PY{l+m+mf}{2.15}\PY{p}{]}
\PY{c+c1}{\PYZsh{} Calculation of basic metrics.}
\PY{n}{n} \PY{o}{=} \PY{n+nb}{len}\PY{p}{(}\PY{n}{T}\PY{p}{)}
\PY{n}{mean} \PY{o}{=} \PY{n}{np}\PY{o}{.}\PY{n}{mean}\PY{p}{(}\PY{n}{T}\PY{p}{)}
\PY{n}{stdevm} \PY{o}{=} \PY{n}{np}\PY{o}{.}\PY{n}{std}\PY{p}{(}\PY{n}{T}\PY{p}{,} \PY{n}{ddof}\PY{o}{=}\PY{l+m+mi}{1}\PY{p}{)} \PY{o}{/} \PY{n}{np}\PY{o}{.}\PY{n}{sqrt}\PY{p}{(}\PY{n}{n}\PY{p}{)}
\PY{n}{max\PYZus{}err} \PY{o}{=} \PY{p}{(}\PY{n+nb}{max}\PY{p}{(}\PY{n}{T}\PY{p}{)} \PY{o}{\PYZhy{}} \PY{n+nb}{min}\PY{p}{(}\PY{n}{T}\PY{p}{)}\PY{p}{)} \PY{o}{/} \PY{l+m+mf}{2.0}
\PY{c+c1}{\PYZsh{} Print out stuff...}
\PY{n+nb}{print}\PY{p}{(}\PY{l+s+sa}{f}\PY{l+s+s1}{\PYZsq{}}\PY{l+s+s1}{n = }\PY{l+s+si}{\PYZob{}}\PY{n}{n}\PY{l+s+si}{\PYZcb{}}\PY{l+s+s1}{\PYZsq{}}\PY{p}{)}
\PY{n+nb}{print}\PY{p}{(}\PY{l+s+sa}{f}\PY{l+s+s1}{\PYZsq{}}\PY{l+s+s1}{mean = }\PY{l+s+si}{\PYZob{}}\PY{n}{mean}\PY{l+s+si}{:}\PY{l+s+s1}{.2f}\PY{l+s+si}{\PYZcb{}}\PY{l+s+s1}{\PYZsq{}}\PY{p}{)}
\PY{n+nb}{print}\PY{p}{(}\PY{l+s+sa}{f}\PY{l+s+s1}{\PYZsq{}}\PY{l+s+s1}{sigma = }\PY{l+s+si}{\PYZob{}}\PY{n}{stdevm}\PY{l+s+si}{:}\PY{l+s+s1}{.2f}\PY{l+s+si}{\PYZcb{}}\PY{l+s+s1}{\PYZsq{}}\PY{p}{)}
\PY{n+nb}{print}\PY{p}{(}\PY{l+s+sa}{f}\PY{l+s+s1}{\PYZsq{}}\PY{l+s+s1}{max\PYZus{}err = }\PY{l+s+si}{\PYZob{}}\PY{n}{max\PYZus{}err}\PY{l+s+si}{:}\PY{l+s+s1}{.2f}\PY{l+s+si}{\PYZcb{}}\PY{l+s+s1}{\PYZsq{}}\PY{p}{)}

[Output]
n = 5
mean = 2.15
sigma = 0.02
max_err = 0.06
\end{Verbatim}
}{
    \caption{Illustrazione del calcolo della media, della semidispersione e
      della deviazione standard della media per le misure di periodo
      nell'esempio~\ref{exp:pendolo_errore},
      utilizzando la libreria~\numpy\ di \python. (Potete confrontare
      direttamente i risultati ottenuti.)
      Il lettore è incoraggiato a consultare la documentazione online di
      \numpy\ per una spiegazione esaustiva delle funzioni che abbiamo
      utilizzato. Se la linea~8 (calcolo della deviazione standard della media)
      in particolare vi sembra criptica, non preoccupatevi, perché dopo
      aver letto la sezione~\ref{sec:deviazione_standard_media} vi sarà
      chiarissima. Infine: vi siete chiesti come mai la semidispersione
      calcolata non è \emph{esattamente} $0.06$?
    }
    \label{snip:mean_stdev}
  }
\end{snippet}


\subsection{Si può ridurre l'incertezza di misura al di sotto della
  risoluzione strumentale?}
\label{sec:misure_sotto_risoluzione}

Il titolo di questa sezione può sembrare provocatorio, ma la risposta,
in generale, è affermativa---ed in effetti abbiamo già visto con
l'esperimento descritto nella sezione~\ref{sec:interpolazione_metro_a_nastro}
che interpolare tra le tacche di un metro a nastro è una via possibile.
C'è un'altra situazione tipica in cui si può battere (per così dire) la
risoluzione strumentale, quella cioè in cui si ha a disposizione un certo
numero di copie \emph{identiche} (o abbastanza simili da poter essere
considerate tali) dell'oggetto o della grandezza che vogliamo misurare.
In tal caso si può fare una misura diretta della somma di queste copie
e \emph{spalmare} la risoluzione strumentale su di esse, dividendo il
valore centrale e l'errore per il loro numero, come illustrato negli
esempi~\ref{exp:misura_spessore_foglio} e~\ref{exp:misura_peso_spillo}.

\begin{examplebox}
  \begin{example}\label{exp:misura_spessore_foglio}
    Si vuole misurare lo spessore $s$ di un foglio di carta disponendo solo di
    un metro a nastro. A tale scopo si misura lo spessore $h$ di una risma di
    $500$ fogli, ottenendo il valore $h = 40 \pm 1$~mm. Dividendo tutto per
    $500$, possiamo scrivere $s = 0.080 \pm 0.002$~mm, con un errore massimo
    $500$ volte più piccolo della risoluzione del metro a nastro.
    (Notiamo che il risultato dipende dall'assunzione implicita che i 500~fogli
    abbiano lo stesso spessore nei limiti dell'incertezza sperimentale.)
  \end{example}

  \begin{example}\label{exp:misura_peso_spillo}
    Si vuole misurare il peso di uno spillo con una bilancia con risoluzione
    di $1$~g. Come prima, se si dispone di un gran numero di spilli uguali,
    possiamo pesarli tutti insieme e dividere il valore centrale e l'incertezza
    di misura sul peso complessivo per questo numero.
  \end{example}
\end{examplebox}

A complemento di questa sezione descriviamo il risultato di un semplice
esperimento in cui si è chiesto ad un gruppo di studenti di misurare
ripetutamente (per dieci volte) la durata di (i) un singolo periodo e (ii)
di $10$ periodi del moto armonico di un cerchietto colorato simulato su un
calcolatore portatile e proiettato su uno schermo. (Il vantaggio di utilizzare
una simulazione anziché un oggetto reale è la possibilità di controllare
il periodo con una precisione maggiore di tutte le incertezze in gioco
nell'esperimento---che nel nostro linguaggio significa essenzialmente conoscere
il valore del misurando). Va da sé che l'idea di base è che, dividendo
le misure di $10$ periodi per $10$, ci aspettiamo di ottenere una misura più
precisa che non misurando il periodo singolo direttamente.

\pgffigtwo[!b]{periodo_pendolo1}{periodo_pendolo10}{
  Distribuzione dei valori del periodo di un moto armonico di un cerchietto
  proiettato su uno schermo, ottenuti misurando direttamente un periodo singolo
  (a sinistra) e dividendo per $10$ la misura di $10$ periodi (a destra).
  Nel secondo caso l'effetto delle fluttuazioni del tempo di reazione umano
  è significativamente ridotto (e, incidentalmente, appare anche un piccolo
  picco secondario a $\nicefrac{9}{10}$ del valore vero, dovuto evidentemente
  ad un banale errore di conteggio).
}

In figura~\ref{fig:periodo_pendolo1_periodo_pendolo10} sono mostrate (sulla
stessa scala orizzontale per evidenziare la differenza) le distribuzioni
su un campione di $93$ studenti (per un totale di $930$ valori) delle misure
di $1$ e $10$~periodi---queste ultime ovviamente divise per $10$. Le prime
variano di qualche decimo di secondo (cioè qualche decina di volte la
risoluzione strumentale, a conferma che la misura è dominata dalle
fluttuazioni del tempo di reazione dello sperimentatore) e sono anche
sistematicamente sottostimate rispetto al valore \emph{vero}, che è
$1.76$~s come indicato dalla freccia verticale.
Le altre tendono, almeno per la maggior parte, a fluttuare significativamente
di meno (diciamo qualche centesimo di secondo) intorno al valore
vero---esattamente come ci aspettavamo.

Questo semplice esempio va estrapolato con un minimo di cura (e caso per caso)
a situazioni diverse da quella descritta. In particolare se dovessimo misurare
in questo modo il periodo di un pendolo, il cui moto non è armonico per
angoli sufficientemente grandi, dovremmo fare attenzione ai possibili
effetti dello smorzamento, che potrebbe provocare una diminuzione apprezzabile
del periodo su tempi abbastanza lunghi. Purtuttavia l'idea di base è buona
ed utilizzabile in pratica.

Concludiamo notando come, nell'istogramma di destra in
figura~\ref{fig:periodo_pendolo1_periodo_pendolo10}, appaia un piccolo picco
secondario, contenente poche decine di misurazioni, a sinistra di quello
principale. La posizione del picco è intorno a $1.58$~s, che corrisponde
con buona precisione a $\nicefrac{9}{10}$ del valore vero $1.76$~s.
Si tratta banalmente di un errore di conteggio (alcuni studenti hanno misurato
il periodo corrispondente a $9$ periodi anziché $10$), che ci ricorda come
sia sempre importante fare attenzione, perché di fronte agli errori
(non intesi come incertezze) tutta la nostra teoria crolla.


\section{Andamento asintotico dell'errore statistico}
\label{sec:andamento_asintotico_stat}

Torniamo per un attimo a guardare la~\eqref{eq:stdev_media_intro}. C'è una
domanda fondamentale che possiamo farci (e a cui possiamo già rispondere),
ovvero: \emph{come scala l'errore statistico al crescere del numero di
misure?} La cosa, come è facile immaginare, è di interesse non solo
accademico, perché è intimamente legata alla questione generale del progetto
degli esperimenti---una parte essenziale del quale è stimare quante misure
dobbiamo fare (ovverosia, per quanto tempo dobbiamo prendere dati) per
raggiungere il livello di precisione voluto.

Con un leggero abuso di notazione potremmo dire che siamo interessati a
studiare il comportamento del limite
\begin{align*}
  \lim_{n \rightarrow \infty} \sqrt{\frac{1}{n(n - 1)}\sum_{i = 1}^n (x_i - \hat{x})^2},
\end{align*}
ma è importante notare fin dall'inizio, per evitare confusione, che non
si tratta di un limite nel senso usuale del termine (come formalizzato,
e.g., nel corso di analisi matematica)%
\footnote{Vedremo più in dettaglio
nella sezione~\ref{sec:prop_definizione_freq} il concetto della convergenza
statistica, ma per questo dovremo prima acquisire i concetti di base della
teoria della probabilità.}, perché le fluttuazioni delle misure attorno
al misurando (o, per quel che conta, attorno al valor medio) hanno un carattere
squisitamente aleatorio e non sono predicibili a priori.

Osserviamo per prima cosa la sommatoria all'interno della radice quadrata. Come
abbiamo detto, il valore del termine $i$-esimo fluttuerà casualmente e non
è prevedibile a priori; tuttavia è ragionevole supporre che,
trattandosi di una somma di $n$ termini definiti positivi, essa tenda a
crescere linearmente con $n$, ovvero:
\begin{align*}
  \lim_{n \rightarrow \infty} \sum_{i = 1}^n (x_i - \hat{x})^2 \propto n.
\end{align*}
(In altre parole: più termini sommo, più è grande la somma.) Allora il
limite cercato è semplice da calcolare utilizzando le regole consuete:
\begin{align}\label{eq:errore_radice_n}
  \lim_{n \rightarrow \infty} \sqrt{\frac{1}{n(n - 1)}\sum_{i = 1}^n (x_i - \hat{x})^2}
  \propto \lim_{n \rightarrow \infty} \sqrt{\frac{n}{n(n - 1)}} =
  \lim_{n \rightarrow \infty} \frac{1}{\sqrt{n}}.
\end{align}
In altre parole:
\emph{l'errore statistico decresce come $\nicefrac{1}{\sqrt{n}}$} al
crescere del numero $n$ delle misure. Una formulazione equivalente, ed
altrettanto utile, di questo principio di base è la seguente:
\emph{per aumentare la precisione di un fattore $c$ è necessario aumentare
il numero di misure di un fattore $c^2$.} Tenetelo bene in mente perché si
tratta di un'affermazione di validità estremamente più generale di quanto
non possa apparire in questo momento.

\begin{examplebox}
  \begin{example}
    Si misura il periodo $T$ di un pendolo con un cronometro digitale e la
    distribuzione a posteriori di un certo numero di misure di prova indica che
    l'incertezza (statistica) dovuta alle fluttuazioni del tempo di reazione
    dello sperimentatore è $0.05$~s (5 centesimi di secondo). Qual è il
    numero $n$ di misure singole che dobbiamo eseguire per arrivare ad una
    precisione di $0.001$~s (1 millesimo di secondo) sul valor medio?

    La risposta è data banalmente da
    \begin{align*}
      \frac{0.05}{\sqrt{n}} = 0.001 \quad \text{ovvero} \quad
      n = \left(\frac{0.05}{0.001}\right)^2 = 2500.
    \end{align*}
    (Ovverosia: dobbiamo eseguire 2500 misure.)
  \end{example}

  \begin{example}
    Un esperimento per la misura della massa $m$ di una nuova particella
    raggiunge una precisione $\nicefrac{\sigma_{m}}{m} = 5\%$ in un anno di
    presa dati. Per quanto tempo ancora si deve continuare ad operare (nelle
    stesse condizioni) se si vuole raggiungere una precisione dell'$1\%$?

    La risposta è semplice: dato che l'errore statistico scala come
    $\nicefrac{1}{\sqrt{n}}$, per abbattere $\sigma_m$ di un fattore 5
    dobbiamo aumentare la statistica di un fattore $5^2 = 25$---quindi dobbiamo
    prendere dati per altri 24 anni.

    (Per inciso: questo è un caso in cui, probabilmente, avrebbe più senso
    pensare ad un \foreign{upgrade} dell'esperimento per accumulare statistica
    più velocemente.)
  \end{example}
\end{examplebox}

Notiamo esplicitamente che, benché la~\eqref{eq:errore_radice_n} sembri
suggerire che sia possibile raggiungere un'incertezza di misura arbitrariamente
piccola (e, almeno in linea di principio, anche nulla) semplicemente
accumulando più dati, questo non è in generale vero a causa
dell'impossibilità di controllare esattamente tutte le condizioni al contorno
che potenzialmente influenzano la nostra misura. Questa impossibilità causa
l'insorgenza di un nuovo tipo di incertezza, che è intrinsecamente diversa
da quella statistica e che non può essere mitigata semplicemente eseguendo un
numero maggiore di misure, come vedremo nella
sezione~\ref{sec:errori_sistematici}.


\subsection{L'andamento dell'errore statistico in un semplice esperimento}

Supponiamo di fare il seguente semplice esperimento: misuriamo per 10 volte
il periodo di un pendolo con un cronometro digitale e calcoliamo il valor medio
$\hat{T}$ delle misure e l'incertezza statistica associata $\sigma_T$ secondo
la~\eqref{eq:stdev_media_intro}. Poi eseguiamo altre 10 misure e ricalcoliamo
$\sigma_T$ utilizzando 20 misure. Poi eseguiamo altre 10 misure e così via
fin quando non ci annoiamo (e.g., dopo 150 misure), calcolando ogni volta
l'incertezza sulla base di tutte le $n$ misure a disposizione.
Come apparirà il grafico di $\sigma_T$ in funzione di $n$?

\pgffigone{errore_asintotico}{
  Incertezza statistica, stimata secondo la~\eqref{eq:stdev_media_intro},
  in funzione del numero di misure effettuate per due realizzazioni
  indipendenti del semplice esperimento (misura del periodo di un pendolo)
  descritto nel corpo del testo. La linea tratteggiata rappresenta
  l'andamento atteso in media, che è proporzionale a
  $\nicefrac{1}{\sqrt{n}}$.}

La risposta è in figura~\ref{fig:errore_asintotico}, in cui sono mostrate
due realizzazioni indipendenti dell'esperimento appena descritto.
Come vedete (e come era logico aspettarsi) le due realizzazioni sono diverse
tra di loro, e nessuna delle due coincide esattamente con l'andamento atteso
in media (rappresentato, quest'ultimo, dalla linea tratteggiata). Eppure
entrambe decrescono (come è giusto che sia) al crescere di $n$ ed entrambe
sembrano, almeno qualitativamente, avvicinarsi progressivamente all'andamento
asintotico atteso al crescere del numero di misure.

Torneremo a discutere più in dettaglio il concetto della convergenza statistica
nella sezione~\ref{sec:prop_definizione_freq}, ma per il momento cominciamo ad
abituarci a questo nuovo tipo di limite, e teniamolo a mente.


\section{Digressione: sviluppi in serie}
\label{sec:sviluppi_in_serie}

A costo di interrompere il flusso naturale del discorso, in questa sezione
ci soffermiamo per un attimo su un argomento intimamente connesso con il
concetto di approssimazione che utilizzeremo frequentemente in seguito.
Sotto ipotesi ragionevoli una generica funzione di una variabile reale $x$
può essere rappresentata come la somma infinita
\begin{align}\label{eq:sviluppo_taylor}
  f(x) = \sum_{k=0}^\infty \frac{1}{k!}\td[k]{f}{x}{x_0} (x - x_0)^k =
  f(x_0) + \td{f}{x}{x_0} (x - x_0) + \frac{1}{2}\td[2]{f}{x}{x_0} (x - x_0)^2 +
  \cdots,
\end{align}
che prende il nome di
\emph{sviluppo in serie di Taylor di $f$ intorno al punto $x_0$}. Non lasciatevi
ingannare dall'apparente complicatezza della~\eqref{eq:sviluppo_taylor}.
L'idea di fondo è che, se la quantità $(x - x_0)$ è abbastanza
piccola---cioè se $x$ è abbastanza vicino ad $x_0$, allora i termini della
serie sono via via più piccoli, e la funzione di partenza può essere
approssimata efficacemente con una somma finita di un numero piccolo di addendi.

\begin{examplebox}
  \begin{example}\label{exp:sviluppo_taylor_seno}
    Utilizziamo lo sviluppo in serie di $\sin(x)$ attorno al punto $x_0 = 0$
    per approssimare il valore della funzione stessa nel punto $x = 0.1$~rad.
    Si tratta di uno degli sviluppi in serie più semplici in assoluto,
    poiché tutte le derivate di ordine pari sono della forma $\pm \sin(x)$,
    per cui si annullano in $0$, e tutte le derivate di ordine dispari sono
    della forma $\pm \cos(x)$, e calcolate in $0$ danno $\pm 1$, per cui:
    \begin{align*}
      \sin(x) = \sum_{k=0}^\infty  \frac{(-1)^k}{(2k + 1)!} \, x^{2k + 1} =
      x - \frac{x^3}{6} + \frac{x^5}{120} -
      \frac{x^7}{5040} \cdots
    \end{align*}
    Potete verificare per calcolo diretto che per $x = 0.1$ il primo termine
    della serie fornisce un risultato che differisce di meno di due parti in
    $10^3$ da quello esatto. Il secondo temine della serie è dell'ordine di
    $10^{-4}$, il terzo di $10^{-8}$ ed il quarto di $10^{-11}$.
  \end{example}

  \begin{example}
    Un corpo si muove lungo una retta di moto uniformemente accelerato.
    Se sviluppiamo la legge oraria $x(t)$ in serie di Taylor attorno al
    punto $t = 0$ si ha
    \begin{align*}
      x(t) = x(0) + \td{x}{t}{0} t + \frac{1}{2}\td[2]{x}{t}{0} t^2 +
      \frac{1}{6}\td[3]{x}{t}{0} t^3 + \cdots =
      x(0) + \dot{x}(0) t + \frac{1}{2}\ddot{x}(0) t^2
      + \frac{1}{6}\dddot{x}(0)t^3 + \cdots
    \end{align*}
    (Abbiamo utilizzato la consueta notazione con i puntini per indicare le
    derivate rispetto al tempo.) Notiamo che se il moto è uniformemente
    accelerato, la derivata seconda è costante, e tutte le derivate di
    ordine superiore si annullano. Se allora, con ovvio significato dei
    termini, poniamo $x(0) = x_0$, $\dot{x}(0) = v_0$ e $\ddot{x}(0) = a$
    otteniamo la relazione (esatta, poiché, come appena detto, le derivate
    al di là della terza si annullano):
    \begin{align*}
      x(t) = x_0 + v_0t + \frac{1}{2}at^2.
    \end{align*}
  \end{example}
\end{examplebox}

L'esempio~\ref{exp:sviluppo_taylor_seno} illustra come il primo termine della
serie sia sufficiente in molti casi di interesse concreto. Lo
sviluppo in serie di Taylor troncato al primo ordine
\begin{align}\label{eq:sviluppo_taylor_lineare}
  f(x) \approx f(x_0) + \td{f}{x}{x_0} (x - x_0)
\end{align}
ha una semplice interpretazione geometrica, come mostrato in
figura~\ref{fig:sviluppo_taylor_lineare}: essenzialmente equivale ad
approssimare il valore della funzione in un generico punto $x$ con quello della
retta tangente alla funzione stessa nel punto $x_0$ attorno al quale si esegue
lo sviluppo. Chiaramente l'approssimazione è tanto migliore quanto più $x$
ed $x_0$ sono vicini.

\pgffigone{sviluppo_taylor_lineare}{
  Interpretazione geometrica dello sviluppo in serie di Taylor troncato al
  primo ordine per una funzione di variabile reale. La derivata della funzione
  calcolata nel punto $x_0$ coincide con il coefficiente angolare della retta
  tangente alla funzione in $x_0$---vale a dire la tangente dell'angolo che la
  retta stessa forma con l'asse delle $x$. L'approssimazione al prim'ordine
  diventa in generale tanto peggiore quanto più la quantità $(x - x_0)$ è
  grande.
}


\subsection{Gli sviluppi in serie come ausilio al calcolo}

Essere capaci di portare a termine con almeno un paio di cifre significative
calcoli non troppo complicati senza l'utilizzo di carta e penna---o, peggio
ancora, calcolatrice o \foreign{personal computer}---fa parte delle capacità che
un Fisico deve acquisire prima o poi nella sua carriera. Gli sviluppi in serie
di Taylor (tipicamente al primo ordine) possono costituire un ausilio pratico
notevole allo scopo.

A titolo di esempio sviluppiamo al primo ordine la funzione
$f(x) = (1 + x)^\alpha$ attorno al punto $x = 0$. Ci serve la derivata prima
valutata, appunto, in $x = 0$
\begin{align}
  \td{f}{x}{x} = \alpha (1 + x)^{\alpha - 1} \quad \text{da cui} \quad
  \td{f}{x}{0} = \alpha \quad \text{e} \quad
  f(x) \approx 1 + \alpha x.
\end{align}
Questo ci permette di derivare banalmente un certo numero di relazioni
\begin{align}
  \frac{1}{1 + x} \approx 1 - x, \quad
  \sqrt{1 + x} \approx 1 + \frac{1}{2}x \quad \text{e} \quad
  (1 + x)^2 \approx 1 + 2x
\end{align}
che sono utili in pratica poiché per le persone è più facile eseguire
\emph{a mente} addizioni e sottrazioni che non moltiplicazioni, divisioni,
estrazioni di radici ed elevamenti a potenza.

\begin{examplebox}
  \begin{example}[per ingannare il tempo al distributore]
    Un litro di gasolio costa $1.124$~\euro; quanti litri possiamo mettere nel
    serbatoio con $10$~euro? La risposta è semplice
    \begin{align*}
      \frac{10}{1.124} = \frac{10}{1 + 0.124}
      \approx 10 \times (1 - 0.124) = 10 \times 0.876 = 8.76.
    \end{align*}
    La risposta esatta è $8.90$, e l'errore relativo
    della nostra approssimazione è $1.5\%$. Non male.
  \end{example}
  \begin{example}
    Una lente ha potere diottrico nominale $\nicefrac{1}{f} = 12~$m$^{-1}$;
    quanto vale la distanza focale?
    \begin{align*}
      f = \frac{1}{12} = \frac{1}{10}\times\frac{1}{1.2} =
      \frac{1}{10}\times\frac{1}{1 + 0.2} \approx \frac{1}{10} \times 0.8 =
      0.08~\text{m} = 8~\text{cm}.
    \end{align*}
    Il risultato esatto è $f = 0.8\overline{3}$, ed il nostro errore
    relativo è del $4\%$.
  \end{example}
\end{examplebox}


\section{Primo approccio alla propagazione degli errori}
\label{sec:propagazione_errrore_max}

Riprendiamo il filo del discorso che avevamo momentaneamente interrotto.
Adesso che abbiamo nel nostro bagaglio di conoscenze il concetto di incertezza
di misura e sappiamo come stimarla---per lo meno nelle situazioni più
semplici---per le misure dirette, siamo pronti ad affrontare il problema
generale di come l'incertezza di misura si propaghi nelle misure indirette.

Pur consci della potenziale ambiguità di questo approccio svilupperemo
la maggior parte del formalismo nel contesto dell'errore massimo, che sappiamo
essere statisticamente scorretto, ma che allo stesso tempo non richiede alcuna
nozione di teoria (che ancora non avremmo) delle probabilità. Forniremo in
parallelo la ricetta corretta per la propagazione dell'errore
statistico, che utilizzeremo sin dall'inizio ma giustificheremo rigorosamente
solo nella sezione~\ref{sec:prop_errore_stat}.


\subsection{Problema generale della propagazione dell'errore massimo}

Dato un certo numero di grandezze \emph{indipendenti}%
\footnote{La richiesta dell'indipendenza delle grandezze, che in questo
  momento non siamo attrezzati per discutere in maniera quantitativa, è
  di fondamentale importanza per tutta la discussione sviluppata in questa
  sezione. Si tratta di un aspetto su cui torneremo in seguito ma che è bene
  tenere a mente sin dall'inizio.}
$x, y, z\ldots$ (con errori massimi $\Delta x, \Delta y, \Delta z\ldots$) ed
una funzione $f(x, y, z\ldots)$ delle grandezze stesse, vorremmo scrivere il
risultato della nostra misura indiretta come $f = \hat{f} \pm \Delta f$ in modo
da essere certi che l'intervallo $[\hat{f} - \Delta f, \hat{f} + \Delta f]$
contenga il misurando $f$---così come siamo certi che l'intervallo
$[\hat{x} - \Delta x, \hat{x} + \Delta x]$ contenga il valore di $x$ e così
via.
\`E ragionevole prendere come valore centrale della misura indiretta il valore
della funzione calcolata in corrispondenza dei valori centrali delle
grandezze di partenza
\begin{align*}
  \hat{f} = f(\hat{x}, \hat{y}, \hat{z}\ldots).
\end{align*}
Per stimare l'errore associato, invece, dovremmo calcolare i valori minimo e
massimo che la funzione $f(x, y, z\ldots)$ assume al variare di $x, y, z\ldots$
nei rispettivi intervalli di incertezza massima---e prenderne la
semidispersione. Nel caso generale si tratta di un problema niente affatto
banale. Fortunatamente, come vedremo tra un attimo, nella maggior parte dei
casi di interesse pratico la soluzione del corrispondente problema linearizzato
fornisce una risposta sufficientemente accurata.


\subsection{Propagazione dell'errore nella somma e nella differenza}

Cominciamo con il caso più semplice, ovvero la propagazione dell'errore
massimo sulla somma $S = x + y$. I casi estremi sono quelli in cui le
incertezze su $x$ e $y$ contribuiscono nello stesso verso:
\begin{align*}
  S_{\max} &= (\hat{x} + \Delta x) + (\hat{y} + \Delta y) =
  (\hat{x} + \hat{y}) + (\Delta x + \Delta y)\\
  S_{\min} &= (\hat{x} - \Delta x) + (\hat{y} - \Delta y) =
  (\hat{x} + \hat{y}) - (\Delta x + \Delta y).
\end{align*}
Ne segue banalmente che in una somma gli errori massimi si sommano, cioè
si ha:
\begin{align}\label{eq:errore_max_somma}
  \hat{S} = \frac{S_{\max} + S_{\min}}{2} = \hat{x} + \hat{y}
  \quad \text{e} \quad
  \Delta S = \frac{S_{\max} - S_{\min}}{2} = \Delta x + \Delta y.
\end{align}

Per analogia potremmo allora essere tentati di pensare che in una differenza gli
errori si sottraggano. Invece, se proviamo a fare lo stesso esercizio sulla
quantità $D = x - y$
\begin{align*}
  D_{\max} &= (\hat{x} + \Delta x) - (\hat{y} - \Delta y) =
  (\hat{x} - \hat{y}) + (\Delta x + \Delta y)\\
  D_{\min} &= (\hat{x} - \Delta x) - (\hat{y} + \Delta y) =
  (\hat{x} - \hat{y}) - (\Delta x + \Delta y)
\end{align*}
scopriamo che gli errori massimi si sommano anche in una differenza:
\begin{align}\label{eq:errore_max_differenza}
  \hat{D} = \frac{D_{\max} + D_{\min}}{2} = \hat{x} - \hat{y}
  \quad \text{e} \quad
  \Delta D = \frac{D_{\max} - D_{\min}}{2} = \Delta x + \Delta y.
\end{align}
(Se siete veramente tentati di scrivere $\Delta D = \Delta x - \Delta y$
pensate a cosa succederebbe nel caso $\Delta x = \Delta y$. Quando le grandezze
misurate sono indipendenti gli errori purtroppo non si sottraggono mai.)

Senza indugiare oltre, diciamo subito che la risposta giusta è simile a
quella appena ottenuta nel contesto dell'errore massimo, con l'accorgimento
di sostituire le somme nelle due formule delle incertezze con le corrispondenti
somme in quadratura:
\begin{align}\label{eq:errore_stat_somma_differenza}
  \sigma_S = \sigma_D = \sqrt{\sigma_x^2 + \sigma_y^2}.
\end{align}
Tenetelo bene a mente perché si tratta, come vedremo, di un principio del
tutto generale: \emph{per grandezze indipendenti gli errori si sommano
in quadratura.}


\begin{examplebox}
  \begin{example}\label{exp:errore_max_somma_differenza}
    Abbiamo misurato con un calibro ventesimale le due lunghezze
    $a = 2.50 \pm 0.05$~mm e $b = 2.35 \pm 0.05$~mm e vogliamo propagare
    l'errore sulla somma $S$ e sulla differenza $D$.
    Se applicassimo la~\eqref{eq:errore_max_somma} e
    la~\eqref{eq:errore_max_differenza} otterremmo $S = 4.85 \pm 0.10$~mm e
    $D = 0.15 \pm 0.10$~mm.
    In realtà sappiamo che se le misure sono indipendenti,
    la~\eqref{eq:errore_stat_somma_differenza} fornisce
    è $\sigma_S = \sigma_D = 0.07$~mm.

    Notiamo come l'errore relativo sulla differenza sia del $\sim 50\%$,
    nonostante le due grandezze di partenza siano misurate al $\sim 2\%$.
    Si tratta di un problema tipico quando si sottraggono misure vicine tra
    loro.
  \end{example}
\end{examplebox}


\subsection{Propagazione dell'errore nel prodotto e nel quoziente}

Il problema della propagazione dell'errore sulla somma e sulla differenza
è peculiare per il fatto che la soluzione esatta si può scrivere in
modo compatto.
Nel caso del prodotto $P = xy$ (assumendo, senza perdere in generalità che
$x$ ed $y$ siano positivi) i casi estremi si scrivono come
\begin{align*}
  P_{\max} &= (\hat{x} + \Delta x)(\hat{y} + \Delta y) =
  \hat{x}\hat{y} + (\hat{x}\Delta y + \hat{y}\Delta x + \Delta x\Delta y)\\
  P_{\min} &= (\hat{x} - \Delta x)(\hat{y} - \Delta y) =
  \hat{x}\hat{y} - (\hat{x}\Delta y + \hat{y}\Delta x - \Delta x\Delta y)
\end{align*}
e la novità rispetto al caso precedente è che adesso appaiono termini
contenenti il prodotto $\Delta x\Delta y$ degli errori sulle due
grandezze---che con un leggero abuso di notazione potremmo chiamare
infinitesimi del second'ordine. Questi nuovi termini rovinano tutto poiché
non si elidono nel calcolo del valore centrale
\begin{align*}
  \hat{P} = \frac{P_{\max} + P_{\min}}{2} = \hat{x}\hat{y} + \Delta x\Delta y
  \neq \hat{x}\hat{y}.
\end{align*}
Nella situazione (tipica, ma da verificare sempre caso per caso) in cui gli
errori relativi sono piccoli, cioè $\Delta x \ll \abs{\hat{x}}$ e
$\Delta y \ll \abs{\hat{y}}$, questi termini possono essere trascurati per
ottenere la relazione approssimata
\begin{align}\label{eq:errore_max_prodotto}
  \hat{P} = \frac{P_{\max} + P_{\min}}{2} \approx \hat{x}\hat{y}
  \quad \text{e} \quad
  \Delta P = \frac{P_{\max} - P_{\min}}{2} \approx
  \abs{\hat{x}}\Delta y + \abs{\hat{y}}\Delta x
\end{align}
(abbiamo ripristinato i valori assoluti che sono necessari per garantire che
$\Delta P$ sia positivo).
Va da sé che, in analogia con quanto visto per la somma e la differenza,
la risposta corretta è:
\begin{align}
  \sigma_P \approx \sqrt{\hat{x}^2 \sigma_y^2 + \hat{y}^2 \sigma_x^2}.
\end{align}

Vale la pena ricordare che $\Delta P$ (o $\sigma_P$) rappresenta un errore di
misura, e come tale non si scrive con più di due cifre significative---per
cui conoscerlo al livello di qualche \% è tipicamente più che sufficiente.
In molti casi i termini di tipo $\Delta x\Delta y$ (o $\sigma_x\sigma_y$) sono
molto più piccoli della prima cifra affetta da errore e costituiscono
essenzialmente rumore numerico.

\begin{examplebox}
  \begin{example}\label{exp:errore_max_prodotto}
    Vogliamo propagare l'errore sul prodotto $P = ab$ delle due grandezze
    dell'esempio~\ref{exp:errore_max_somma_differenza}. I due termini
    della~\eqref{eq:errore_max_prodotto} sono
    $\hat{a}\Delta b \approx 0.12$~mm$^2$
    e $\hat{b}\Delta a \approx 0.12$~mm$^2$; per confronto, il termine che
    abbiamo trascurato è $\Delta a\Delta b \approx 0.0025$~mm$^2$, ossia
    circa $50$ volte più piccolo. Possiamo dunque sommare le incertezze in
    quadratura e scrivere $P = 5.87 \pm 0.17$~mm$^2$.
    (Notiamo esplicitamente che il termine $\Delta a\Delta b$ non influisce
    sulle cifre significative con cui abbiamo scritto la misura.)
  \end{example}
\end{examplebox}

La propagazione dell'errore massimo sul quoziente $Q = x/y$ è del tutto
analoga, e ci limitiamo qui a fornire il risultato:
\begin{align}\label{eq:errore_max_quoziente}
  \hat{Q} = \frac{Q_{\max} + Q_{\min}}{2} \approx \frac{\hat{x}}{\hat{y}}
  \quad \text{e} \quad
  \Delta Q = \frac{Q_{\max} - Q_{\min}}{2} \approx
  \frac{\abs{\hat{x}}\Delta y + \abs{\hat{y}}\Delta x}{\hat{y}^2}
  \quad \text{e} \quad
  \sigma_Q \approx \frac{\sqrt{\hat{x}^2 \sigma_y^2 + \hat{y}^2 \sigma_x^2}}{\hat{y}^2}.
\end{align}

Una conseguenza interessante dell'espressione appena scritta è che,
\emph{Quando dividiamo una grandezza affetta da incertezza per un numero senza
  errore, si divide per il numero stesso sia il valore centrale che l'incertezza
  di partenza.}
Questo è il principio alla base della possibilità di fare misure (indirette)
con un errore massimo inferiore alla risoluzione strumentale, come già
illustrato nella sezione~\ref{sec:misure_sotto_risoluzione}.

Notiamo anche che, quando si propagano gli errori su un prodotto o un quoziente
è comodo lavorare con gli errori relativi, poiché, come è facile
dimostrare per calcolo diretto
\begin{align}
  \frac{\sigma_P}{\abs{\hat{P}}} = \frac{\sigma_Q}{\abs{\hat{Q}}} \approx
  \sqrt{\frac{\sigma_x^2}{\hat{x}^2} + \frac{\sigma_y^2}{\hat{y}^2}},
\end{align}
ovvero \emph{nei prodotti e nei quozienti gli errori massimi relativi si
 sommano (ovviamente in quadratura).}


\subsection{Propagazione dell'errore nel caso generale}

Generalizziamo quanto detto fino ad ora per le quattro operazioni elementari,
cominciando dal caso di una funzione generica $f(x)$ di una sola
grandezza $x$ (misurata), con errore massimo associato $\Delta x$. Possiamo
sviluppare $f$ in serie di Taylor al prim'ordine attorno al punto $\hat{x}$
\begin{align*}
  f(\hat{x} \pm \Delta x) \approx
  f(\hat{x}) \pm \td{f}{x}{\hat{x}}\,\Delta x
\end{align*}
da cui si ottiene banalmente
\begin{align}\label{eq:errore_max_generale_unidimensionale}
  \hat{f} = \frac{f_{\max} + f_{\min}}{2} \approx f(\hat{x})
  \quad \text{e} \quad
  \Delta f = \frac{f_{\max} - f_{\min}}{2} \approx
  \abs{\td{f}{x}{\hat{x}}}\Delta x
  \quad \text{da cui} \quad
  \sigma_f = \abs{\td{f}{x}{\hat{x}}}\sigma_x.
\end{align}
Il simbolo $\approx$ ci ricorda che
la~\eqref{eq:errore_max_generale_unidimensionale}
è una relazione che vale approssimativamente nell'ipotesi in cui i termini di
ordine superiore nello sviluppo di Taylor siano trascurabili. Geometricamente
questo equivale a richiedere, come mostrato in
figura~\ref{fig:linearizzazione_errore_max}, che la retta tangente ad $f$
nel punto $\hat{x}$ sia una buona approssimazione di $f$ nell'intervallo
$[\hat{x} - \Delta x, \hat{x} + \Delta x]$.

\pgffigone[!htb]{linearizzazione_errore_max}{
  Rappresentazione grafica della linearizzazione della propagazione dell'errore
  su una generica funzione di una
  variabile~\eqref{eq:errore_max_generale_unidimensionale}.
  La derivata della funzione $f(x)$ nel punto $\hat{x}$ coincide con
  il coefficiente angolare della retta tangente ad $f(x)$ e passante per il
  punto $P = (\hat{x}, f(\hat{x}))$ e l'approssimazione lineare è tanto più
  accurata quanto più la tangente è vicina alla funzione nell'intervallo
  $[\hat{x} - \Delta x, \hat{x} + \Delta x]$.
}

Come detto in precedenza questa assunzione, che dipende sia dall'incertezza
su $x$ che dalla forma della funzione $f(x)$ in un intorno di $\hat{x}$, va
verificata caso per caso---un caso banale in cui il nostro sviluppo in serie
troncato al prim'ordine non funziona è quello in cui $f'(\hat{x}) = 0$, come
illustrato nell'esempio~\ref{exp:errore_max_derivata_nulla}.

\begin{examplebox}
  \begin{example}
    Dato un angolo (misurato)
    $\theta = \hat{\theta} \pm \sigma_\theta = (30 \pm 1)^\circ$ si vuole
    propagare l'errore sulla funzione $\sin\theta$. Possiamo utilizzare
    direttamente la~\eqref{eq:errore_max_generale_unidimensionale}, che nel
    caso specifico fornisce:
    \begin{align*}
      \Delta(\sin\theta) \approx
      \abs{\td{\sin}{\theta}{\hat{\theta}}}\sigma_\theta =
      \abs{\cos\hat{\theta}}\sigma_\theta.
    \end{align*}
    Con l'accortezza necessaria di convertire $\sigma_\theta$ in radianti
    (le derivate delle funzioni trigonometriche sono quelle note solo se
    gli angoli sono misurati in radianti) possiamo scrivere
    $\sin\theta = 0.500 \pm  0.015$.
  \end{example}

  \begin{example}\label{exp:errore_max_derivata_nulla}
    Dato un angolo (misurato)
    $\theta = \hat{\theta} \pm \sigma_\theta = (90 \pm 5)^\circ$ si vuole
    propagare l'errore sulla funzione $\sin\theta$. In questo caso la derivata
    prima si annulla nel punto $\hat{\theta}$ per cui il nostro formalismo è
    inutilizzabile (si tratta di un caso in cui, per ovvie ragioni, i termini
    di ordine superiore nello sviluppo di Taylor non sono trascurabili).
    Allora possiamo stimare esplicitamente i casi estremi:
    \begin{align*}
      \sin(\hat\theta - \sigma_\theta) = \sin(\hat\theta + \sigma_\theta)
      \approx 0.996
      \quad \text{e} \quad \sin\theta = 1.000 \pm  0.004.
    \end{align*}
    Alternativamente possiamo estendere lo sviluppo in serie di Taylor
    all'ordine successivo
    \begin{align*}
      \sigma_{(\sin\theta)} \approx
      \frac{1}{2}\abs{\td[2]{\sin}{\theta}{\hat{\theta}}} \sigma_\theta^2
      = \frac{1}{2} \abs{\sin\hat{\theta}} \sigma_\theta^2 \approx 0.004
    \end{align*}
    (che è in accordo con il risultato ottenuto per calcolo diretto.)
    Notiamo anche che questo esempio è peculiare per un altro motivo:
    dato che $\sin\theta \leq 1$, ci troviamo in un caso in cui l'intervallo
    di incertezza sulla nostra grandezza derivata è asimmetrico. Più
    correttamente dovremmo allora scrivere $\sin\theta = 1.000^{+0.000}_{-0.004}$.
  \end{example}
\end{examplebox}

Siamo finalmente pronti per l'arma finale---ovverosia la formula generale di
propagazione delle incertezze nel caso multidimensionale.
I risultati precedenti si estendono più o meno banalmente (modulo uno
sviluppo di Taylor al prim'ordine per una funzione di più variabili) al caso
generale di una funzione $f(x, y, z\ldots)$ delle grandezze $x, y, z\ldots$
\begin{align}\label{eq:intro_err_stat_multidimensionale}
  \sigma^2_f \approx
  \left(\pd{f}{x}{\hat{x}, \hat{y}, \hat{z}\ldots}\right)^2 \sigma^2_x +
  \left(\pd{f}{y}{\hat{x}, \hat{y}, \hat{z}\ldots}\right)^2 \sigma^2_y +
  \left(\pd{f}{z}{\hat{x}, \hat{y}, \hat{z}\ldots}\right)^2 \sigma^2_z + \cdots
\end{align}

Il lettore può verificare facilmente che la
\eqref{eq:intro_err_stat_multidimensionale} permette di derivare le
formule che abbiamo ricavato mediante calcolo diretto per la propagazione
dell'errore massimo nelle quattro operazioni elementari.
Così per la somma si ha, nel nostro linguaggio:
\begin{align*}
  f(x) = x + y
  \quad
  \pd{f}{x}{\hat{x}, \hat{y}} = 1
  \quad
  \pd{f}{y}{\hat{x}, \hat{y}} = 1
  \quad
  \sigma^2_f = \sigma_x^2 + \sigma_y^2
  \quad \text{ovvero} \quad
  \sigma_f = \sqrt{\sigma_x^2 + \sigma_y^2},
\end{align*}
che è il risultato che avevamo già derivato. (In questo caso il risultato
è esatto perché la nostra funzione è lineare in entrambe le variabili, per
cui tutti i termini successivi dello sviluppo in serie di Taylor si annullano.)

La~\eqref{eq:intro_err_stat_multidimensionale} permette anche di ricavare una
formula compatta per la propagazione dell'errore relativo
su una grandezza del tipo $G(x, y, z\ldots) = x^\alpha y^\beta z^\gamma \cdots$
In questo caso le derivate parziali sono banali, e.g.
\begin{align*}
  \pd{G}{x}{\hat{x}, \hat{y}, \hat{z}\ldots} =
  \alpha \hat{x}^{\alpha - 1} \hat{y}^\beta \hat{z}^\gamma \cdots
\end{align*}
(ed espressioni equivalenti per le derivate rispetto alle altre grandezze), per
cui mettendo tutto insieme:
\begin{align}\label{eq:errore_relativo_generale}
  \frac{\sigma_G}{\abs{\hat{G}}} = \sqrt{
  \left(\alpha\frac{\sigma_x}{\abs{\hat{x}}}\right)^2 +
  \left(\beta\frac{\sigma_y}{\abs{\hat{y}}}\right)^2 +
  \left(\gamma\frac{\sigma_z}{\abs{\hat{z}}}\right)^2 + \cdots
  }
\end{align}
\emph{Quando cioè si ha a che fare con prodotti e quozienti di grandezze
  elevate ad un esponente generico, gli errori relativi si sommano pesati con
  il modulo del proprio esponente}, come illustrato
nell'esempio~\ref{exp:misura_g_pendolo}.

\begin{examplebox}

  \begin{example}\label{exp:misura_g_pendolo}
    Vogliamo misurare l'accelerazione di gravità $g$ al livello del suolo
    utilizzando un pendolo semplice e sfruttando la relazione che lega il
    periodo alla lunghezza del pendolo stesso in approssimazione di piccole
    oscillazioni
    \begin{align}\label{eq:periodo_pendolo}
      T = 2\pi\sqrt{\frac{l}{g}}.
    \end{align}
    Misuriamo dunque direttamente $l = 1.005 \pm 0.006$~m e
    $T = 2.02 \pm 0.01$~s. La relazione scritta sopra si può invertire per
    ottenere un'espressione per $g$ in funzione di $l$ e $T$
    \begin{align*}
      g(l, T) = 4\pi^2\frac{l}{T^2}
    \end{align*}
    da cui possiamo calcolare il valore centrale $\hat{g} = 9.72$~m~s$^{-2}$.
    Non ci rimane che propagare l'errore su $g$ utilizzando
    la~\eqref{eq:intro_err_stat_multidimensionale}---cosa che
    facciamo pedissequamente, calcolando le derivate
    \begin{align*}
      \pd{g}{l}{l, T} = \frac{4\pi^2}{T^2} \quad\text{e}\quad
      \pd{g}{T}{l, T} = -\frac{8\pi^2l}{T^3},
    \end{align*}
    e mettendo tutto insieme
    \begin{align*}
      \sigma_g = \frac{4\pi^2}{\hat{T}^2}\sqrt{
        \left(\sigma_l^2 + \frac{4\hat{l}^2}{\hat{T}^2}\sigma_T^2\right)
      } \approx 0.11~\text{m~s}^{-2}.
    \end{align*}
    Scriveremo dunque il risultato come $g = 9.72 \pm 0.11$~m~s$^{-2}$, che
    è compatibile con il valore noto di $9.81$~m~s$^{-2}$.
    Notiamo, per inciso, che saremmo arrivato allo stesso risultato più
    facilmente propagando l'errore relativo---cosa che è lasciata al lettore
    some esercizio.
  \end{example}
\end{examplebox}

A proposito dell'esempio~\ref{exp:misura_g_pendolo}, il lettore più attento
avrà notato che non abbiamo specificato le condizioni in cui il nostro modello
è applicabile---cioè non ci siamo chiesti esplicitamente che cosa
significhi lavorare in approssimazione di piccole oscillazioni. Ci torneremo
nella sezione~\ref{sec:piccole_oscillazioni}, ma anticipiamo che i termini
della questione dipendono dal contesto specifico in cui la misura è
effettuata, per cui affermazioni del tipo "siamo in regime di piccole
oscillazioni per $\theta_0 < 5^\circ$", che potete aver sentito alle scuole
superiori, non hanno alcun senso.


\section{Compatibilità e relazioni d'ordine tra misure sperimentali}
\label{sec:compatibilita_err_max}

Nel contesto dell'errore massimo due grandezze $x_1 = \hat{x_1} \pm \Delta x_1$
e $x_2 = \hat{x_2} \pm \Delta x_2$ si dicono \emph{compatibili} se
l'intersezione tra i rispettivi intervalli di incertezza è non nulla, come
mostrato graficamente in figura~\ref{fig:compatibilita}.

\begin{figure}[htb!]
  \autohstack{\begin{tikzpicture}
  \node at (0, 0) {};
  \pgfmathsetmacro{\xa}{3.}
  \pgfmathsetmacro{\xb}{5.}
  \pgfmathsetmacro{\ya}{-1.5}
  \pgfmathsetmacro{\yb}{-3.5}
  \pgfmathsetmacro{\dx}{0.2}
  \pgfmathsetmacro{\qx}{1.0}
  \pgfmathsetmacro{\dy}{1.2}
  \draw[very thick] (\xa, \ya+\dy) -- (\xa, \ya-\dy);
  \draw[very thick] (\xa-\dx, \ya+\dy) -- (\xa+\dx, \ya+\dy);
  \draw[very thick] (\xa-\dx, \ya-\dy) -- (\xa+\dx, \ya-\dy);
  \fill (\xa, \ya) circle [radius=3pt];
  \node[anchor=west] at (\xa + \dx, \ya) {$\hat{x_1}$};
  \draw[thick,{<[scale=1.5]}-{>[scale=1.5]}]
  (\xa - \qx, \ya+\dy) -- (\xa - \qx, \ya);
  \node[anchor=east] at (\xa - \qx, \ya+0.5*\dy) {$\Delta x_1$};
  
  \draw[very thick] (\xb, \yb+\dy) -- (\xb, \yb-\dy);
  \draw[very thick] (\xb-\dx, \yb+\dy) -- (\xb+\dx, \yb+\dy);
  \draw[very thick] (\xb-\dx, \yb-\dy) -- (\xb+\dx, \yb-\dy);
  \fill (\xb, \yb) circle [radius=3pt];
  \node[anchor=west] at (\xb + \dx, \yb) {$\hat{x_2}$};
  \draw[thick,{<[scale=1.5]}-{>[scale=1.5]}]
  (\xb + \qx, \yb+\dy) -- (\xb + \qx, \yb);
  \node[anchor=west] at (\xb + \qx, \yb+0.5*\dy) {$\Delta x_2$};

  \draw[style=dotted] (\xa, \ya-\dy) -- (\xb, \ya-\dy);
  \draw[style=dotted] (\xa, \yb+\dy) -- (\xb, \yb+\dy);
\end{tikzpicture}
}{
    \caption{Illustrazione grafica del significato geometrico della
      compatibilità (nel senso dell'errore massimo) tra due grandezze
      misurate. Gli intervalli di incertezza sono convenzionalmente
      rappresentati, come vedremo nel seguito, da barre d'errore. Due misure
      sono compatibili quando i rispettivi intervalli di incertezza hanno
      intersezione non nulla---ovvero, quando le barre d'errore
      \emph{si toccano}---come appunto in questa figura.}
    \label{fig:compatibilita}
  }
\end{figure}

Equivalentemente possiamo dire che due misure sono compatibili tra loro se la
loro differenza è compatibile con zero. Poiché l'incertezza su una
differenza è pari alla somma delle incertezze, questo si traduce nella
condizione
\begin{align}\label{eq:compat_err_max}
  \abs{\hat{x_2} - \hat{x_1}} \leq (\Delta x_1 + \Delta x_2)
  \quad \text{(errore massimo, i.e., sbagliato)}.
\end{align}

Nel linguaggio dell'errore statistico i contorni della nozione di
compatibilità sono in un certo senso più sfumati a causa della definizione
probabilistica del concetto di incertezza di misura. Formalmente
l'equivalente della~\eqref{eq:compat_err_max} è
\begin{align}\label{eq:compat_err_stat}
  \abs{\hat{x_2} - \hat{x_1}} \leq \sqrt{\sigma^2_{x_1} + \sigma^2_{x_2}}
  \quad \text{(errore statistico, i.e., corretto)},
\end{align}
ma in pratica la definizione di compatibilità è solitamente meno stringente
della~\eqref{eq:compat_err_stat}, e quando lavoriamo con l'errore statistico
non è inusuale dire che due misure sono compatibili anche se distano due barre
d'errore equivalenti. Avremo occasione di affrontare il discorso in modo
organico a tempo debito. (Va da sé che se la~\eqref{eq:compat_err_stat} è
verificata, allora due misure sono compatibili nel senso dell'errore
statistico.)

\begin{examplebox}
  \begin{example}
    Due gruppi sperimentali misurano l'accelerazione di gravità al livello del
    suolo ottenendo i valori $g_1 = 9.8 \pm 0.1$~m~s$^{-2}$ e
    $g_2 = 9.84 \pm 0.06$~m~s$^{-2}$. La differenza tra le due misure vale
    $g_2 - g_1 = 0.04 \pm 0.12$~m~s$^{-2}$, per cui i due valori sono
    compatibili.
  \end{example}
\end{examplebox}

Prima di andare avanti, vale la pena fare una breve carrellata di espressioni
da evitare accuratamente in quanto ambigue o, peggio ancora, prive di
significato quantitativo. Non dite o scrivete mai per nessuna ragione che:
\begin{itemize}
\item \emph{due misure sono confrontabili}: se due grandezze sono omogenee si
  possono sempre confrontare tra di loro, ma questo non ci dice niente sul fatto
  che esse siano compatibili o meno;
\item \emph{due misure sono vicine}: esattamente come non ha senso dire che
  una grandezza è piccola o grande in assoluto, così non ha senso dire
  che due grandezze misurate sono vicine o lontane in assoluto;
\item \emph{due misure molto sono vicine}: qui l'avverbio non fa che
  peggiorare le cose\ldots
\item \emph{due misure sono congruenti}: il concetto di congruenza è
  puramente geometrico e non ha nessun significato nel contesto dell'analisi
  dei dati;
\item \emph{due misure sono in buon accordo}: si tratta di un'espressione
  generica che si presta a diverse interpretazioni.
\end{itemize}
Detto più concisamente: quando confrontate due grandezze limitatevi a dire
che esse sono compatibili o incompatibili. Non è difficile da ricordare e non
serve altro.

Il concetto operativo di incertezza di misura che abbiamo sviluppato in questa
sezione, ed il legame indissolubile che abbiamo stabilito tra misure ed
incertezze, ci forniscono uno spunto di riflessione per inquadrare sotto una
luce nuova alcuni aggettivi di uso comune. Adesso che siamo in grado di dire
se due grandezze fisiche sono compatibili, possiamo anche stabilire una
relazione d'ordine tra grandezze---dire, cioè, se una è più grande
dell'altra. Ma la cosa fondamentale da sottolineare è che in Fisica non ha
senso dire che il valore numerico di una generica grandezza è piccolo o grande
\emph{in assoluto}. Ogni cosa può essere definita grande o piccola solo in
relazione ad un'altra grandezza ad essa omogenea. Questo vale sia per i valori
centrali che per le incertezze ad essi associate.

Cogliamo infine l'occasione per notare che in Fisica si usa dire che una
grandezza $x$ è \emph{molto più grande} o \emph{molto più piccola} di una
grandezza (omogenea) $y$, e si scrive
\begin{align}\label{eq:gg_ll}
  x \gg y \quad \text{oppure} \quad x \ll y,
\end{align}
se $x$ ed $y$ differiscono tra loro in valore numerico di più di un ordine di
grandezza, ovverosia che $x \geq 10y$ o $x \leq \nicefrac{y}{10}$.
(Va da sé che c'è un certo grado di arbitrarietà nel numero $10$.)
Ha sicuramente senso dire che $80 \gg 10$ anche se, a rigore,
la~\eqref{eq:gg_ll} non è verificata, ma non che $15 \gg 10$.) Questa
caratterizzazione ci sarà utile nella discussione che segue sulle
approssimazioni in Fisica.


\section{Il ruolo delle approssimazioni in Fisica}
\label{sec:approssimazioni}

Le approssimazioni sono onnipresenti in Fisica. Sono la chiave con cui problemi
intrattabili o eccessivamente complicati nella loro formulazione completa
possono essere semplificati per trovare soluzioni compatte sotto opportune
condizioni. In Fisica sperimentale le approssimazioni, se utilizzate in
modo opportuno, possono semplificare enormemente il processo di misura e
di interpretazione dei dati in quanto:
\begin{itemize}
\item permettono, ove giustificato, di trascurare i termini non importanti
  nella propagazione delle incertezze;
\item permettono, ove giustificato, di evitare misure, dirette o indirette, di
  una o più grandezze potenzialmente rilevanti per il problema, sulla base di
  semplici considerazioni sugli ordini di grandezza;
\item permettono, ove giustificato, di utilizzare modelli semplificati per
  l'interpretazione dei dati sperimentali.
\end{itemize}

Armati del bagaglio di conoscenze che abbiamo accumulato in questo primo
capitolo possiamo sviscerare la questione con alcuni esempi concreti. La cosa
fondamentale, che è ovvia ancora prima di iniziare la discussione, è che
\emph{qualsiasi approssimazione decidiamo di fare, sia essa nel processo di
  misura o di analisi dei dati, deve essere opportunamente verificata---a
  priori o a posteriori}.
In altre parole: dobbiamo essere capaci di tradurre in una formula matematica
le condizioni di validità dell'approssimazione stessa, e verificare
quest'ultima nel contesto del problema che stiamo studiando.
Aggiungiamo anche che, in generale, la validità o meno di un'approssimazione
data non può prescindere dalla precisione dell'apparato sperimentale che
utilizziamo per la misura. Il principio generale da cui ci faremo guidare
è che \emph{un effetto è trascurabile se non si può misurare}.


\subsection{Approssimazioni e propagazione delle incertezze}

Sappiamo già che la formula generale (linearizzata) di propagazione delle
incertezze~\eqref{eq:intro_err_stat_multidimensionale} è per sua natura
approssimata. Quando propaghiamo gli errori capita sovente di trovarsi nella
situazione in cui alcuni termini sono più piccoli di altri. Ove ciò accada
è chiaramente conveniente trascurarli \emph{ab initio}, per snellire i
calcoli.

Supponiamo di voler stimare il volume $V = l^2h$ di una lastrina quadrata di
lato~$l$ e spessore~$h$. Se l'errore relativo sullo spessore è molto più
grande di quello sul lato (ad esempio se misuriamo i due con lo stesso strumento
ma $h \ll l$), allora possiamo trascurare il secondo nella formula di
propagazione
\begin{align*}
  \frac{\sigma_V}{V} =
  \sqrt{4\frac{\sigma^2_l}{l^2} + \frac{\sigma^2_h}{h^2}} \approx
  \frac{\sigma_h}{h}.
\end{align*}

\begin{examplebox}
  \begin{example}
    Se le dimensioni misurate della lastrina in questione sono
    $l = 100.00 \pm 0.02$~mm e $h = 1.00 \pm 0.01$~mm, i.e., l'errore relativo
    sullo spessore $\nicefrac{\sigma_h}{h} = 1\%$ è molto (50 volte) più
    grande di quello sul lato $\nicefrac{\sigma_l}{l} = 0.02\%$. Possiamo
    dunque trascurare il secondo nella propagazione delle incertezze, e
    l'errore relativo sul volume sarà dell'$1\%$, i.e., scriveremo
    $V = 10.0 \pm 0.1$~cm$^3$. (Provate a fare il conto senza trascurare il
    primo termine e vi renderete conto che l'incertezza non cambia
    apprezzabilmente.)
  \end{example}
\end{examplebox}

Questa è una lezione importante:
\emph{padroneggiare gli ordini di grandezza nella propagazione degli errori
  ed essere capaci di trascurare i termini che possono essere trascurati
  è un'abilità estremamente utile da coltivare, che nella pratica di
  tutti i giorni semplifica enormemente la vita.}


\subsection{Volume di un solido irregolare}

Consideriamo di nuovo la lastrina della sezione precedente, ma supponiamo
che essa abbia al centro un foro di raggio~$r$. Intuitivamente se il foro è
\emph{abbastanza piccolo} possiamo pensare di poterlo trascurare nella
determinazione del volume---e risparmiarci dunque una misura. Ma come possiamo
scrivere rigorosamente la condizione che deve essere verificata perché la
nostra approssimazione sia lecita?

Potremmo essere tentati di dire che la condizione è che il volume del foro
sia molto più piccolo del volume totale del parallelepipedo a base
quadrata, ovvero
\begin{align}
  \pi r^2 h \ll l^2 h \quad \text{e quindi} \quad
  r \ll \frac{l}{\sqrt{\pi}} \quad \text{(sbagliato)}
\end{align}
Questo non può essere corretto: non importa quanto piccolo sia il foro---se
misuriamo il volume del parallelepipedo con una precisione abbastanza spinta
prima o poi ne apprezzeremo l'effetto.

\begin{examplebox}
  \begin{example}
    Se il raggio del foro centrale è $r \approx 10$~mm ed il lato della
    lastrina è, come prima, $l = 100.00 \pm 0.02$~mm siamo nella situazione
    in cui $r \ll l$ (o, per quel che conta, $r \ll \nicefrac{l}{\sqrt{\pi}}$),
    ma se andiamo a calcolare il volume che andremmo a sottrarre ai
    $10.0 \pm 0.1$~cm$^3$ calcolati prima, otterremo un valore di
    $0.314$~cm$^3$, che sposterebbe il valore centrale significativamente
    (più di $3$~volte la barra d'errore). \emph{Ergo}: il foro non è trascurabile.
  \end{example}
\end{examplebox}

Quello che dobbiamo richiedere, invece, è che il volume del foro sia molto
più piccolo dell'\emph{incertezza} sul volume del parallelepipedo, ovverosia
\begin{align}
  \pi r^2 h \ll l^2 h \sqrt{4\frac{\sigma^2_l}{l^2} + \frac{\sigma^2_h}{h^2}}
  \quad \text{e cioè} \quad
  r \ll \frac{l}{\sqrt{\pi}}
  \left( 4\frac{\sigma^2_l}{l^2} + \frac{\sigma^2_h}{h^2} \right)^\frac{1}{4}
  \approx \frac{l}{\sqrt{\pi}} \sqrt{\frac{\sigma_h}{h}}
  \quad \text{(corretto)}.
\end{align}
(Se l'errore relativo su $h$ è, come nel nostro esempio, l'1\%, la condizione
corretta è $10$ volte più stringente di quella sbagliata che abbiamo
scritto sopra.) Se possiamo assicurarci che questa condizione è verificata
sulla base di una stima, anche grossolana, degli ordini di grandezza, allora
possiamo tranquillamente ignorare il foro nella stima del volume
\emph{senza eseguire una misura diretta}. Il fatto che chiamiamo quest'ultima
una stima e non vi associamo un'incertezza di misura non è in contraddizione
con quanto abbiamo detto fino a questo momento perché, in questo contesto,
siamo interessati solamente a distinguere $5$~mm da mezzo~mm o $5$~cm, e non
ci interessano le sfumature intermedie.

\begin{examplebox}
  \begin{example}
    Nel caso in questione la condizione per cui il foro può essere trascurato
    ai fini della stima del volume si legge $r \ll 3$~mm (che è consistente
    con il fatto che, come detto prima, un foro di raggio $10$~mm non è
    trascurabile). Notiamo esplicitamente che la condizione non dipende solo da
    $r$ ed $l$, ma anche dalle incertezze di misura sui termini non
    trascurabili (in questo caso essenzialmente $h$).
  \end{example}
\end{examplebox}


\subsection{Misure di indice di rifrazione}

Si vuole stimare l'indice di rifrazione $n$ di un materiale trasparente,
immerso in aria, misurando l'angolo di incidenza e di rifrazione sulla
superficie del materiale stesso e sfruttando la legge di Snell
\begin{align*}
  n = \frac{\sin\theta_i}{\sin\theta_r} \; n_\text{aria}.
\end{align*}
Ora, il valore tabulato dell'indice di rifrazione dell'aria in condizioni
normali è $1,0002926$---ovvero, differisce da $1$ per circa $3 \times 10^{-4}$.
Allora possiamo semplicemente assumere $n_\text{aria} = 1$?

Prima ancora di rispondere ricordiamo, se ce ne fosse bisogno, che non c'è
niente di speciale nel valore numerico $3 \times 10^{-4}$. In assoluto, esso non
è né piccolo né grande, per cui la decisione di trascurarlo o meno può
essere presa solo in relazione a qualcos'altro.
In questo caso specifico la cosa è piuttosto semplice: l'errore relativo che
introduciamo su $n$ ponendo $n_\text{aria} = 1$ è proprio $3 \times 10^{-4}$,
per cui l'approssimazione è lecita se l'incertezza relativa su $n$
(determinata dalle incertezze di misura sugli angoli di incidenza e di
rifrazione) è molto più grande:
\begin{align*}
  n_\text{aria} - 1 \ll \frac{\sigma_n}{n}.
\end{align*}


\subsection{Focale di una lente convergente}

Supponiamo di voler misurare la lunghezza focale $f$ di una lente convergente
sfruttando la legge fondamentale delle lenti sottili
\begin{align}\label{eq:lenti_sottili}
  \frac{1}{p} + \frac{1}{q} = \frac{1}{f},
\end{align}
in cui $p$ è la distanza tra la sorgente e la lente, e $q$ è la distanza
tra la lente e l'immagine. In generale la~\eqref{eq:lenti_sottili} indica
che per stimare $f$ dobbiamo misurare sia $p$ che $q$, ma suggerisce anche
immediatamente che se la sorgente fosse all'infinito, la lunghezza focale
coinciderebbe con la distanza $q$ tra la lente e lo schermo, e noi potremmo
cavarcela con una misura anziché due. Esattamente come nel caso delle piccole
oscillazioni ci chiediamo allora quale sia la condizione che deve essere
verificata perché possiamo trascurare $p$
nella~\eqref{eq:lenti_sottili}.

Se siete tentati di rispondere $p \gg q$, fate lo sforzo di trattenervi e
rileggete con attenzione la sezione precedente: il fatto che $p$ sia molto
più grande di $q$ non basta a dire che $\nicefrac{1}{p}$ può essere
trascurato, se non diciamo niente sull'incertezza $\sigma_q$.

Supponiamo allora di disporre di una misura $q = \hat{q} \pm \sigma_q$ della
distanza lente-immagine e di una \emph{stima}, anche grossolana, $\hat{p}$
della distanza sorgente-lente---per fissare le idee possiamo immaginare
che la sorgente sia una plafoniera fissata al soffitto di una stanza e che noi
siamo seduti alla scrivania con la nostra lente in mano: in questa situazione
possiamo dire, e.g., che $p$ è circa $2$~m senza fare nessuna misura.
Guardando più da vicino la~\eqref{eq:lenti_sottili} è chiaro che $p$ può
essere trascurato se $\nicefrac{1}{p}$ è molto più piccolo dell'errore su
$\nicefrac{1}{q}$, ovvero
\begin{align}
  \frac{1}{\hat{p}} \ll \frac{\sigma_q}{\hat{q}^2}
  \quad \text{o, ancora} \quad
  \hat{p} \gg \frac{\hat{q}^2}{\sigma_q}.
\end{align}

\begin{examplebox}
  \begin{example}
    Se $q = 10.0 \pm 0.2$~cm, la sorgente deve essere ad una distanza molto
    maggiore di $5$~m dalla lente perché $p$ possa essere trascurato. Potete
    verificare direttamente che, in caso contrario, il termine $\nicefrac{1}{p}$
    dà un contributo alla stima della focale non trascurabile rispetto
    all'incertezza sulla focale stessa.
  \end{example}
\end{examplebox}


\subsection{Le piccole oscillazioni di un pendolo}
\label{sec:piccole_oscillazioni}

Consideriamo l'equazione del moto del pendolo semplice:
\begin{align}\label{eq:moto_pendolo}
  \ddot\theta(t) + \omega_0^2\sin\theta(t) = 0.
\end{align}
Vi sarà sicuramente capitato di sentir dire che per piccole oscillazioni è
lecito utilizzare lo sviluppo in serie del seno al prim'ordine
$\sin\theta \approx \theta$, col che l'equazione del moto viene linearizzata
e le oscillazioni diventano \emph{armoniche}---cioè il periodo di oscillazione
non dipende dall'ampiezza.  \`E probabile che abbiate anche sentito dire che
la condizione di piccole oscillazioni si scrive come
\begin{align*}
  \theta \ll 1 \quad
  \text{(condizione di piccole oscillazioni fisicamente insoddisfacente)}
\end{align*}
e, se siete stati particolarmente sfortunati, che in pratica sono \emph{piccoli}
tutti gli angolo sotto i $5^\circ$ (o al sotto di qualche altro numero
ugualmente arbitrario). Ora, le ultime due affermazioni sono ovviamente
insoddisfacenti---non esiste un valore \emph{magico} al di sotto del quale le
oscillazioni diventano improvvisamente piccole.

Torniamo dunque al nostro pendolo. La~\eqref{eq:moto_pendolo} non si può
risolvere in forma chiusa, ma è possibile ottenere una formula generale
per il periodo in funzione dell'ampiezza di oscillazione $\theta_0$ nella forma
di uno sviluppo di Taylor:
\begin{align}\label{eq:periodo_pendolo_theta}
  T(\theta_0) = T_0 \left( 1 + \frac{1}{16}\theta_0^2 +
  \frac{11}{3072}\theta_0^4 + \frac{173}{737\,280}\theta_0^6 +
  \frac{22\,931}{1\,321\,205\,760}\theta_0^8 +
  \cdots \right) \quad \text{con} \quad
  T_0 = 2\pi\sqrt{\frac{l}{g}}.
\end{align}

\pgffigone{piccole_oscillazioni}{
  Andamento del periodo di oscillazione di un pendolo (o meglio, del rapporto
  $T(\theta_0)/T$ tra il periodo ed il suo valore asintotico per piccole
  oscillazioni) in funzione dell'ampiezza $\theta_0$. La linea continua
  rappresenta la somma dei primi quattro termini dello
  sviluppo~\eqref{eq:periodo_pendolo_theta}---fino a quello in $\theta_0^6$
  incluso. Le tre linee tratteggiate indicano le somme parziali dei termini di
  ordine inferiore---quello costante (approssimazione di piccole oscillazioni),
  quello in $\theta_0^2$ e quello in $\theta_0^4$.
}

La domanda che ci facciamo è allora: dato un valore fissato di $\theta_0$,
posso considerare le oscillazioni piccole e trascurare tutti i termini dello
sviluppo di ordine superiore al primo? O, nel nostro nuovo linguaggio: dato
un valore fissato di $\theta_0$ posso misurare la variazione di periodo indotta
dal fatto che l'ampiezza di oscillazione non è nulla? (Ricordate: se non posso
misurare l'effetto, allora posso trascurarlo.) Posta in questi termini la
questione, la risposta è semplice: se voglio considerare le oscillazioni
piccole devo assicurarmi che i termini di correzione al periodo che
trascuro---ed in particolare il primo, che è il più grande---siano molto
più piccoli dell'incertezza di misura sul periodo stesso, vale a dire:
\begin{align}\label{eq:condizione_piccole_oscillazioni}
  \frac{T_0\theta_0^2}{16} \ll \sigma_T
  \quad \text{ovvero} \quad
  \frac{\theta_0^2}{16} \ll \frac{\sigma_T}{T_0}
  \quad \text{o, ancora} \quad
  \theta_0 \ll 4 \sqrt{\frac{\sigma_T}{T_0}}.
\end{align}
L'ultima relazione che abbiamo scritto ci dà una definizione operativa di
cosa significhi "piccole oscillazioni" nel contesto della nostra misura.
Come avevamo anticipato, la condizione non si scrive in astratto, ma dipende
dalla precisione dell'apparato sperimentale. L'angolo oltre il quale le
oscillazioni cessano di essere piccole dipende dalla risoluzione dello
strumento che uso. Se l'errore relativo di misura è dell'ordine dell'$1\%$
(che è un valore tipico se si usa un cronometro manuale), si cominciano ad
apprezzare i termini anarmonici per $\theta_0 \approx 20^\circ$; ma se è di
$10^{-5}$ (che non è impossibile da ottenere con un microcontrollore
relativamente poco sofisticato), allora $\theta_0 \approx 1^\circ$ non è
affatto piccolo.


\section{Errori sistematici}
\label{sec:errori_sistematici}

Nella sezione~\ref{sec:errore_max} abbiamo introdotto la distinzione tra le
nozioni di precisione ed accuratezza. Torniamo adesso su una domanda che
abbiamo toccato brevemente più volte, ovverosia: immaginando di poter
migliorare a piacimento la precisione di una misura, possiamo affermare che
siamo in grado di migliorare a piacimento anche la sua accuratezza? In altre
parole, possiamo avvicinarci arbitrariamente al valore del misurando
semplicemente facendo in modo di ridurre le fluttuazioni statistiche della
misura (dove possibile) e migliorando la risoluzione dello strumento?
La risposta, in generale, è no, ed il motivo di fondo è l'esistenza degli
\emph{errori sistematici}.

Gli errori sistematici possono essere causati da svariati fattori, tra cui:
problemi di calibrazione degli strumenti di misura o utilizzo improprio degli
strumenti stessi, condizioni ambientali o fattori esterni non completamente
sotto controllo, errori ed incertezze di modellizzazione nelle misure indirette.
Corrispondentemente, la zoologia degli errori sistematici è molto variegata e
in questa sezione ci limiteremo a fare alcune considerazioni generali
riesaminando criticamente alcuni degli esempi visti in precedenza.

\pgffigone{errore_stat_sys}{
  Illustrazione qualitativa dell'andamento asintotico delle incertezze
  statistiche e sistematiche in funzione del numero di misure $n$ (cfr. la
  sezione~\ref{sec:andamento_asintotico_stat}). Mentre l'incertezza statistica
  diminuisce come $\nicefrac{1}{\sqrt{n}}$, quella sistematica è costante,
  per cui esiste un regime in cui continuare ad eseguire misure non ha
  alcun beneficio in termini di accuratezza della misura.
  }

L'aspetto fondamentale e caratterizzante della questione è il fatto che le
incertezze sistematiche non si possono abbattere, come avviene invece per quelle
statistiche, semplicemente continuando a prendere dati. Mentre l'errore
statistico decresce in media come $\nicefrac{1}{\sqrt{n}}$, quello sistematico
è per definizione indipendente da $n$ e, quando i due divengono dello stesso
ordine di grandezza, è banalmente giunto il momento in cui continuare a
prendere dati non porta più alcun beneficio per quel che riguarda
l'accuratezza della misura.


\subsection{Errori sistematici strumentali}

Spesso gli errori sistematici hanno origine strumentale, ad esempio una non
perfetta calibrazione dell'apparato di misura. Questo è particolarmente vero
negli esperimenti di frontiera, in cui gli strumenti sono estremamente
complessi e sono utilizzati al limite delle proprie possibilità.

\pgffigone[!b]{risposta_strumento}{
  Esempio illustrativo della funzione di risposta di uno strumento generico.
  La diagonale indica la situazione in cui la lettura è identica al valore
  (incognito) del misurando su tutto l'intervallo operativo (cioè lo
  strumento è perfettamente calibrato). In pratica la risposta non è mai
  ideale, e si possono dare errori di zero, errori di scala o, più in generale
  deviazioni dalla diagonale di forma arbitraria. La funzione fondamentale
  della calibrazione di uno strumento è di fare in modo che queste deviazioni
  siano il più piccole possibili e sotto controllo.
}

Ogni strumento è caratterizzato da una \emph{funzione di risposta}, che
rappresenta essenzialmente il valore di lettura dello strumento stesso in
funzione del valore (incognito) del misurando
(figura~\ref{fig:risposta_strumento}).
Idealmente vorremmo che la risposta dello strumento fosse l'identità su tutto
l'intervallo operativo. In pratica essa non può essere misurata con
precisione infinita, ed il massimo che si può fare, attraverso la
calibrazione, è cercare di avvicinarsi il più possibile alla condizione
ideale e quantificare più o meno precisamente le deviazioni residue.
Si parla allora di \emph{errore di zero} quando la lettura di uno strumento è
traslata sistematicamente di una costante additiva (o \emph{offset}) rispetto
al valore del misurando. Si parla viceversa di un \emph{errore di scala}
quando il rapporto tra lettura e misurando è pari ad una costante
moltiplicativa diversa da $1$. Nel caso più generale la risposta di uno
strumento può essere affetta contemporaneamente da errori di zero ed errori
di scala, e questi ultimi possono variare in grandezza all'interno
dell'intervallo di misura dello strumento stesso.

Queste semplici considerazioni cambiano radicalmente le regole del gioco:
quando utilizziamo uno strumento non possiamo banalmente assumere che le
letture della grandezza fisica che esso misura siano accurate entro i limiti
della risoluzione strumentale. Uno dei compiti principali di un buon Fisico
sperimentale è proprio quello di valutare la presenza di possibili errori
sistematici nella misura e propagare l'effetto che essi hanno sulle conclusioni
che si vogliono trarre dalla misura stessa. La natura estremamente variegata
del problema rende complicata una discussione esaustiva in astratto, ma gli
esempi~\ref{exp:termometro_digitale_2} e~\ref{exp:dilatazione_termica}
dovrebbero servire da spunti di riflessione al proposito.

\begin{examplebox}
  \begin{example}\label{exp:termometro_digitale_2}
    Riesaminiamo brevemente l'esempio~\ref{exp:termometro_digitale} del
    nostro termometro digitale che, immerso in acqua, fornisce una lettura
    $T_1 = 34.5~^\circ$C (cui associamo un errore di $0.1~^\circ$C). Acquistiamo
    adesso un secondo termometro identico, lo immergiamo nello stesso
    contenitore, e notiamo che quest'ultimo segna $T_2 = 33.2^\circ$C. (Se
    l'esempio vi pare pretestuoso, comprate due termometri digitali a basso
    prezzo e provate voi stessi: non mancherete di rimanere sorpresi.)
    Aspettiamo qualche ora e notiamo che i due termometri segnano
    $T_1 = 26.5~^\circ$C e $T_2 = 25.2~^\circ$C---notate che la differenza è
    rimasta invariata ($1.3~^\circ$C), ed è molto più grande dell'incertezza
    di misura nominale.
    In queste condizioni è ragionevole concludere che i due termometri siano
    in grado di apprezzare variazioni di $0.1~^\circ$C (nel senso della
    risoluzione strumentale) ma che allo stesso tempo abbiano un errore di zero
    (incognito, perché a questo livello non possiamo dire quale dei due sia
    più vicino al misurando) dell'ordine $1.3~^\circ$C.
    (Se avessimo più termometri a disposizione potremmo costringere meglio
    l'entità di questo errore di zero.) Scriveremo allora l'ultima misura
    fornita (ad esempio) dal primo termometro come
    $T_1 = 26.5 \pm 0.1 \pm 1.3 (sys)~^\circ$C, indicando separatamente il
    contributo dell'errore sistematico di zero.
  \end{example}

  \begin{example}\label{exp:dilatazione_termica}
    Il coefficiente di dilatazione termica lineare dell'acciaio (materiale con
    cui sono tipicamente realizzati i metri a nastro metallici) è dell'ordine
    di $\lambda \approx 1.7 \times 10^{-5}~{}^\circ\text{C}^{-1}$. Questo significa
    che, passando da $T_1 = 0~^\circ$C a $T_2 = 30~^\circ$C, la lunghezza
    reale di un metro a nastro di lunghezza nominale $L = 2$~m aumenta
    di un fattore $\lambda L (T_2 - T_1) \approx 1$~mm---che è pari alla
    risoluzione del metro stesso. (Nel caso di un calibro cinquantesimale
    l'effetto può essere, in termini relativi, significativamente più
    importante.) Una contrazione o dilatazione di questo tipo introduce
    chiaramente un errore di scala.

    Questo dovrebbe darci alcuni spunti di riflessione: a che temperatura è
    stato calibrato il metro (o il calibro)? Qual è il coefficiente di
    dilatazione termica lineare dell'oggetto che stiamo misurando? E cosa
    significa misurare la lunghezza di un oggetto se essa non è una grandezza
    caratteristica dell'oggetto stesso, ma dipende dalla temperatura?
  \end{example}
\end{examplebox}


\subsection{Incertezze di modellizzazione}

Abbiamo già avuto modo di dire che l'utilizzo di un \emph{modello} inadeguato
può essere fonte di errori sistematici nella stima dei parametri del modello
stesso. Riesaminiamo per un attimo l'esempio~\ref{exp:misura_g_pendolo}---ossia
la stima del valore dell'accelerazione di gravità $g$ attraverso la misura
del periodo di un pendolo semplice.

Se misuriamo abbastanza accuratamente (nel senso dell'accuratezza, non solo
della precisione) $l$ e $T$ possiamo dire che misureremo altrettanto
accuratamente $g$? Sicuramente l'errore propagato secondo la prescrizione
dell'esempio~\ref{exp:misura_g_pendolo} sarà piccolo se $\sigma_l$ e
$\sigma_T$ sono piccoli, ma ad un certo punto entreremo nelle condizioni
in cui non sarà più lecito utilizzare la~\eqref{eq:periodo_pendolo}
per ricavare $g$ dalle misure di $l$ e $T$ (e, vale la pena ripeterlo ancora
una volta, non perché queste ultime non siano accurate).
Ci si aprono dunque due strade: (i) ci assicuriamo che il valore di
$\theta_0$ che scegliamo per eseguire la misura sia abbastanza piccolo da
garantire che i termini anarmonici dello sviluppo in serie del periodo siano
molto più piccoli dell'errore di misura; oppure (ii) si utilizzano a
posteriori tutti i termini della~\eqref{eq:periodo_pendolo_theta} che è
necessario usare per evitare errori sistematici%
\footnote{L'esempio appena fatto è in un certo senso fuorviante perché in
  una situazione in cui si ha a disposizione un modello arbitrariamente
  accurato~\eqref{eq:periodo_pendolo_theta}, utilizzare un modello inadeguato
  non costituisce un errore sistematico, ma semplicemente un
  \emph{errore}---non nel senso di incertezza, ma nel senso di sbaglio vero e
  proprio. Vi sono casi, tuttavia, in cui tale modello non è disponibile;
  ed in quei casi l'inadeguatezza del modello a disposizione costituisce il
  limite ultimo all'accuratezza della misura.}.

\begin{examplebox}
  \begin{example}
    Se misuriamo il periodo di un pendolo con una incertezza relativa
    dell'$1$\% (ad esempio $\nicefrac{1}{100}$~s su un periodo di $\sim 1$~s),
    allora seguendo la~\eqref{eq:condizione_piccole_oscillazioni}
    dobbiamo assicurarci che l'ampiezza massima di oscillazione
    soddisfi la condizione $\theta_0 \ll 0.4~\text{rad} \approx 23^\circ$
    per poter stimare correttamente l'accelerazione di gravità $g$ lavorando
    in approssimazione di piccole oscillazioni.
  \end{example}
\end{examplebox}


\subsection{Identificazione degli effetti sistematici}

Identificare e trattare opportunamente tutte le possibili sorgenti di errori
sistematici in una misura, specialmente in esperimenti complessi, è spesso
il compito più difficile con cui il Fisico sperimentale deve misurarsi.
Una trattazione esaustiva del problema a partire dai principi primi è
estremamente difficile, ma in linea generale alcune strategie possibili sono:
ripetere la misura con uno o più strumenti indipendenti---possibilmente
più accurati; verificare la calibrazione dello strumento utilizzato
attraverso la misura di una o più grandezze di riferimento note;
verificare la consistenza interna del proprio campione di dati (ad esempio
suddividendolo in sotto-campioni o intervalli temporali); verificare possibili
incongruenze con il modello utilizzato per interpretare i dati.

\begin{examplebox}
  \begin{example}
    Supponiamo di prendere i due termometri
    dell'esempio~\ref{exp:termometro_digitale_2} e di immergerli in un
    (abbondante) misto di acqua e ghiaccio---cioè in un bagno termico a
    $0^\circ$C. Il primo termometro segna $0.2^\circ$C ed il secondo $1.5^\circ$C.
    Data questa nuova informazione potremmo essere tentati di assumere che il
    primo termometro sia più accurato del secondo, in quanto si avvicina
    di più al nostro valore di riferimento. Potremmo addirittura essere
    tentati di usare questo procedimento per correggere lo zero
    di entrambi.
    (Prima di fare questo, dovremmo cercare di analizzare criticamente il
    nostro procedimento di taratura, però. Se immergiamo il termometro in
    un punto diverso del bagno termico, ad esempio, la lettura cambia? E in tal
    caso quali sono le implicazioni?)
  \end{example}

  \begin{example}
    Data una serie di oggetti dello stesso materiale, misuriamo il volume
    $V$ e la massa $m$ per ciascuno di essi. Mettiamo poi i valori
    ottenuti su di un grafico cartesiano (di $m$ in funzione di $V$) per
    stimare la densità $\rho$ del materiale dalla retta che meglio li
    approssima---secondo la relazione $m = \rho V$.
    Se non vincoliamo la nostra retta a passare per l'origine ci accorgiamo
    però che l'intercetta stimata è significativamente diversa da $0$.
    In questo caso abbiamo ottime ragioni fisiche per credere che (come
    previsto dal nostro modello) la massa tenda a $0$ quando il volume tende
    a $0$. Il fatto che i nostri dati non si dispongano su una retta passante
    per l'origine indica che, con ogni probabilità, siamo in presenza di
    un errore sistematico sulla misura delle masse o dei volumi---o tutti e due.
  \end{example}
\end{examplebox}


\subsection{Propagazione degli effetti sistematici}

Una caratteristica peculiare degli errori sistematici è costituita dal
fatto che, per loro natura, non seguono le regole di propagazione
dell'errore massimo (o statistico). La differenza di base è che gli errori
sistematici non sono indipendenti tra misura e misura: se un termometro, ad
esempio, ha un problema di zero, tutte le misure saranno troppo grandi o troppo
piccole della stessa quantità. Questo porta a situazioni interessanti---e a
volte sorprendenti---nella propagazione delle incertezze di questo tipo.

\begin{examplebox}
  \begin{example}
    Supponiamo di misurare le temperature $T_1$ e $T_2$ ai due estremi di una
    sbarretta conduttrice e di essere interessati \emph{solo} alla loro
    differenza---ad esempio per stimare la conducibilità termica del materiale
    di cui la sbarretta è costituita. Questo è un caso tipico in cui
    un possibile errore di zero del termometro (assumendo di utilizzare
    lo stesso termometro per le due misure) è rigorosamente ininfluente.
  \end{example}
\end{examplebox}


\summary

\begin{itemize}
\item Il risultato di una misura è composto da: (i) valore centrale; (ii)
  incertezza ad esso associata; (iii) unità di misura. Mai omettere alcuna
  delle tre per nessuna ragione.
\item L'incertezza di misura si scrive con 1 o, al massimo, 2 cifre
  significative. Il valore centrale si scrive, di conseguenza, con un numero di
  cifre decimali consistente con quelle dell'incertezza.
\item Controllate sempre le dimensioni fisiche delle equazioni che scrivete.
  Se i due membri di un'equazione non hanno le stesse dimensioni fisiche, allora
  l'equazione è sbagliata.
\item La propagazione delle incertezze per misure indipendenti si esegue in
  generale seguendo la~\eqref{eq:intro_err_stat_multidimensionale}:
  \begin{align*}
    \sigma^2_f \approx
    \left(\pd{f}{x}{\hat{x}, \hat{y}, \hat{z}\ldots}\right)^2 \sigma^2_x +
    \left(\pd{f}{y}{\hat{x}, \hat{y}, \hat{z}\ldots}\right)^2 \sigma^2_y +
    \left(\pd{f}{z}{\hat{x}, \hat{y}, \hat{z}\ldots}\right)^2 \sigma^2_z + \cdots
  \end{align*}
  Questo non vuol dire che dobbiate necessariamente riscoprire la ruota ogni
  volta che propagate l'errore su un'espressione: il fatto che nei prodotti e
  nei quozienti si sommino (in quadratura) gli errori relativi si può
  tenere a mente indipendentemente dalle derivate parziali.
\item Quando, per comodità o necessità, si fa un'approssimazione è
  strettamente necessario verificare che l'effetto che stiamo trascurando
  non sia misurabile con il nostro apparato. In caso contrario l'approssimazione
  non è lecita.
\item L'identificazione e la trattazione degli errori sistematici sono una
  parte fondamentale del processo di misura ed analisi dei dati.
\item Ricordate: sbagliare le cifre significative e/o le unità di misura sono
  gli \emph{errori di ortografia} del Laboratorio di Fisica!
\end{itemize}
