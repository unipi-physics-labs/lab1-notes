\chapter*{Colophon}

Queste dispense sono state realizzate utilizzando esclusivamente \foreign{software}
libero (se non sai cosa significa, è il momento di dare un'occhiata alla
pagina web della \href{http://www.fsf.org/}{Free Software Foundation}).

Il testo è stato battuto interamente con
\href{https://www.gnu.org/software/emacs/}{emacs} e/o \href{https://atom.io/}{atom},
ed impaginato utilizzando \href{https://www.latex-project.org/}{\LaTeX}---un programma
per la realizzazione di documenti che utilizza \TeX, di Donald Knuth, come motore
tipografico. La classe di stile utilizzata è una versione altamente
personalizzata (che sono certo l'autore disapproverebbe, ma mira essenzialmente
a risparmiare carta) del superbo pacchetto
\href{http://www.ctan.org/tex-archive/macros/latex/contrib/classicthesis/}{\texttt{classicthesis}} di André Miede.

Le figure che non contengono grafici sono state realizzate direttamente in
\LaTeX\ utilizzando il (niente meno che impressionante) pacchetto di grafica
vettoriale \href{https://www.ctan.org/pkg/pgf}{\texttt{pgf}}, insieme al suo
\foreign{layer} sintattico di alto livello
\href{http://www.texample.net/tikz/}{\texttt{tikz}}, di Till Tantau.
I grafici, invece, sono stati realizzati utilizzando il \foreign{backend}
\code{pgf} di \href{http://matplotlib.org/}{matplotlib}, insieme ad alcune
delle altre componenti dell'ecosistema di calcolo scientifico di
\python---in particolare \numpy\ e \scipy.
Il \foreign{word cloud} sulla pagina di copertina è stato realizzato con il pacchetto
\href{https://github.com/amueller/word_cloud}{\pymodule{wordcloud}} di \python,
utilizzando il testo delle dispense come input.

I frammenti di codice sono eseguiti automaticamente durante la compilazione
del documento da un semplice \emph{script} in \python\ che appende l'\foreign{output}
al frammento corrispondente. (In altre parole: sono garantiti funzionare,
almeno sul computer dell'autore.) I colori sono aggiunti utilizzando
l'ottimo evidenziatore di sintassi \href{http://pygments.org/}{Pygments}.

Le dispense sono state, per la maggior parte, sviluppate su macchine che
utilizzano la distribuzione \href{https://getfedora.org/}{Fedora} di GNU/Linux
come sistema operativo. Tutte le componenti del progetto sono sotto controllo
di configurazione ed il repositorio \href{https://git-scm.com/}{git} è
ospitato su \href{https://github.com}{github} a
\begin{center}
  \url{https://github.com/unipi-physics-labs/lab1-notes}
\end{center}


\vfill

\noindent\foreign{``Shut up, we know you can play! Jesus\ldots''}\\
---Steve Vai, \foreign{Juice} (1995)
