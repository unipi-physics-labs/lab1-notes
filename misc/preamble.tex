
% Get rid of the titlesec-related warnings, see
% https://bitbucket.org/amiede/classicthesis/issues/92/titlesec-deprecated
\RequirePackage{silence}
\WarningFilter{scrbook}{Usage of package `titlesec' together}
\WarningFilter{titlesec}{Non standard sectioning command}

% Preambolo.
\documentclass[10pt, a4paper, numbers=noenddot]{scrbook}

% Definizione dei pacchetti
\usepackage[italian]{babel}

% This is a workaround for using classic thesis without aligning equations to
% left (http://www.guit.sssup.it/phpbb/viewtopic.php?p=21797)
\usepackage{amsmath}
\let\origRequirePackage\RequirePackage
\renewcommand{\RequirePackage}[2][]{%
  \def\next{#2}\def\nxt{amsmath}%
  \def\rightRequirePackage{\origRequirePackage[#1]{#2}}%
  \ifx\next\nxt\else\expandafter\rightRequirePackage\fi}
% End of workaround.

\usepackage[linedheaders, eulerchapternumbers, eulermath, parts]{classicthesis}

% And another ugly workaround for a TeXlive bug, see
% http://tug.org/pipermail/tex-live/2018-May/041625.html
%\def\requestedLaTeXdate{20180412}
% End of workaround.

% This is a workaround for avoiding overfull hboxes for page numbers greater
% than 99.
\settowidth{\newnumberwidth}{999}
\setlength\cftpartnumwidth{\newnumberwidth}
% End of workaround.

% See http://stackoverflow.com/questions/30201310/use-of-hyphen-or-minus-sign-in-matplotlib-versus-compatibility-with-latex
\usepackage[utf8]{inputenc}
\DeclareUnicodeCharacter{2212}{$-$}

\makeatletter
% Small manual hack to avoid pygments-related warnings.
% See https://tex.stackexchange.com/questions/224209
\@namedef{T1/cmtt/b/n}{<->ssub*cmtt/bx/n}
\@namedef{T1/fvm/m/sc}{<->ssub*fvm/m/n}
\@namedef{T1/pplj/m/scit}{<->ssub*pplj/m/it}
\makeatother

% Suppress bogus pdfTeX warnings, see
% https://tex.stackexchange.com/questions/76273
\pdfsuppresswarningpagegroup=1

% This magically fixes \texttt{} not working. See
% https://tex.stackexchange.com/questions/664/why-should-i-use-usepackaget1fontenc
\usepackage[T1]{fontenc}

\usepackage{makeidx}
\usepackage{amsthm}
\usepackage{amsfonts}
\usepackage{amssymb}
\usepackage{verbatim}
\usepackage[framemethod=TikZ]{mdframed}
\usepackage{pgfplots}
\pgfplotsset{compat=1.16}
\usepgflibrary{arrows,arrows.meta}
\usepackage[nice]{nicefrac}
\usepackage{venndiagram}
\usepackage{manfnt}
\usepackage{tabularx}
%\usepackage{wrapfig}
\usepackage{eurosym}

\newcommand{\danger}{%
  \hangindent38pt\hangafter-3\noindent%
  \makebox[0pt]{\hspace{-45pt}\raisebox{-7pt}[0pt][0pt]{\Large\dbend}}%
}

\usepackage{pifont}
\newcommand{\cmark}{\text{\ding{51}}}%
\newcommand{\xmark}{\text{\ding{55}}}%


\usepackage{chngcntr}
\counterwithin{figure}{chapter}

\usepackage{fancyvrb}
\fvset{frame=single, numbers=left, numbersep=3pt, fontsize=\small,
       fontfamily=cmtt,
       baselinestretch=1.0, rulecolor=\color{lightgray}, framesep=5pt}

\usetikzlibrary{patterns}
\usetikzlibrary{positioning}
\graphicspath{{figures/}}

% Setup per le legende
\usepackage{caption}
\captionsetup{labelfont=sc, textfont=rm, size=normal, labelsep=period,
  format=plain}


% Setup per i mini-indici
\usepackage[tight, italian, nohints]{minitoc}
\renewcommand{\mtifont}{\normalsize\scshape\lsstyle}
\newcommand{\mt}{\minitoc\mtcskip}

% Newcommands.
\newcommand{\fixme}[1]{{\color{red}{Fixme: #1}}}
\newcommand{\weblink}[1]{Web link: \url{#1}}
\newcommand{\summary}{\section{In breve...}}

\newcounter{qcounter}
\renewenvironment{enumerate}{\begin{list}{\arabic{qcounter})}{%
      \usecounter{qcounter}%
      \setlength{\itemindent}{13pt}%
      \setlength{\labelwidth}{25pt}%
      \setlength{\leftmargin}{\rightmargin}}}
                 {\end{list}}

\usepackage{newfloat}
\usepackage[nomessages]{fp}

\usepackage{ascii}
\usepackage{MnSymbol,wasysym}
\usepackage{forest}

% https://tex.stackexchange.com/questions/5073/making-a-simple-directory-tree
\definecolor{fscol}{RGB}{180,180,180}
\newcommand\myfolder[2][fscol]{%
\begin{tikzpicture}[overlay]
\begin{scope}[xshift=20pt]
\filldraw[rounded corners=1pt,fill=#1,draw=white,double=black]
  (-23pt,10pt) -- ++(3pt,5pt) -- ++(18pt,0pt) -- ++(40:3pt) -- ++(9pt,0pt) -- ++(-40:3pt)
  -- (20pt,15pt) -- (23pt,10pt) -- cycle;
\filldraw[rounded corners,draw=white,double=black,top color=#1,bottom color=#1!30]
  (-22pt,-12pt) -- ++(44pt,0pt) -- (25pt,12pt) coordinate (topr) -- ++(-50pt,0pt) coordinate (topl) -- cycle;
\end{scope}
\end{tikzpicture}%
\makebox[35pt]{\raisebox{-3pt}{{\ttfamily/#2}}}%
}


% Additional settings.

\areaset[0.4in]{6.4in}{9.5in}
\setlength\topmargin{-35pt}

\pagestyle{scrheadings}

\cfoot{\scriptsize This work is licensed under a
       \href{http://creativecommons.org/licenses/by-sa/4.0/}%
       {Creative Commons Attribution-ShareAlike 4.0 International License}.}

\newcommand{\hstack}[3][0.65]{%
    \FPeval{\cw}{0.97 - #1}
    \begin{minipage}[t][][b]{#1\textwidth}#2\end{minipage}\hfill%
    \begin{minipage}[t][][b]{\cw\textwidth}\vspace*{-8pt}#3\end{minipage}%
}

\newsavebox{\hstackbox}
\newlength{\hstackboxwl}
\newlength{\hstackboxwr}

\newcommand{\autohstack}[3][-12pt]{%
    \sbox\hstackbox{#2}%
    \settowidth{\hstackboxwl}{\usebox{\hstackbox}}%
    \setlength{\hstackboxwr}{0.97\textwidth}%
    \addtolength{\hstackboxwr}{-\hstackboxwl}%
    \begin{minipage}[t][][b]{\hstackboxwl}\usebox\hstackbox\end{minipage}\hfill%
    \begin{minipage}[t][][b]{\hstackboxwr}\vspace*{#1}#3\end{minipage}%
}

\newcommand{\tablehstack}[2]{\autohstack[-12pt]{#1}{#2}}


\makeatletter
\def\PY@reset{\let\PY@it=\relax \let\PY@bf=\relax%
    \let\PY@ul=\relax \let\PY@tc=\relax%
    \let\PY@bc=\relax \let\PY@ff=\relax}
\def\PY@tok#1{\csname PY@tok@#1\endcsname}
\def\PY@toks#1+{\ifx\relax#1\empty\else%
    \PY@tok{#1}\expandafter\PY@toks\fi}
\def\PY@do#1{\PY@bc{\PY@tc{\PY@ul{%
    \PY@it{\PY@bf{\PY@ff{#1}}}}}}}
\def\PY#1#2{\PY@reset\PY@toks#1+\relax+\PY@do{#2}}

\@namedef{PY@tok@w}{\def\PY@tc##1{\textcolor[rgb]{0.73,0.73,0.73}{##1}}}
\@namedef{PY@tok@c}{\let\PY@it=\textit\def\PY@tc##1{\textcolor[rgb]{0.24,0.48,0.48}{##1}}}
\@namedef{PY@tok@cp}{\def\PY@tc##1{\textcolor[rgb]{0.61,0.40,0.00}{##1}}}
\@namedef{PY@tok@k}{\let\PY@bf=\textbf\def\PY@tc##1{\textcolor[rgb]{0.00,0.50,0.00}{##1}}}
\@namedef{PY@tok@kp}{\def\PY@tc##1{\textcolor[rgb]{0.00,0.50,0.00}{##1}}}
\@namedef{PY@tok@kt}{\def\PY@tc##1{\textcolor[rgb]{0.69,0.00,0.25}{##1}}}
\@namedef{PY@tok@o}{\def\PY@tc##1{\textcolor[rgb]{0.40,0.40,0.40}{##1}}}
\@namedef{PY@tok@ow}{\let\PY@bf=\textbf\def\PY@tc##1{\textcolor[rgb]{0.67,0.13,1.00}{##1}}}
\@namedef{PY@tok@nb}{\def\PY@tc##1{\textcolor[rgb]{0.00,0.50,0.00}{##1}}}
\@namedef{PY@tok@nf}{\def\PY@tc##1{\textcolor[rgb]{0.00,0.00,1.00}{##1}}}
\@namedef{PY@tok@nc}{\let\PY@bf=\textbf\def\PY@tc##1{\textcolor[rgb]{0.00,0.00,1.00}{##1}}}
\@namedef{PY@tok@nn}{\let\PY@bf=\textbf\def\PY@tc##1{\textcolor[rgb]{0.00,0.00,1.00}{##1}}}
\@namedef{PY@tok@ne}{\let\PY@bf=\textbf\def\PY@tc##1{\textcolor[rgb]{0.80,0.25,0.22}{##1}}}
\@namedef{PY@tok@nv}{\def\PY@tc##1{\textcolor[rgb]{0.10,0.09,0.49}{##1}}}
\@namedef{PY@tok@no}{\def\PY@tc##1{\textcolor[rgb]{0.53,0.00,0.00}{##1}}}
\@namedef{PY@tok@nl}{\def\PY@tc##1{\textcolor[rgb]{0.46,0.46,0.00}{##1}}}
\@namedef{PY@tok@ni}{\let\PY@bf=\textbf\def\PY@tc##1{\textcolor[rgb]{0.44,0.44,0.44}{##1}}}
\@namedef{PY@tok@na}{\def\PY@tc##1{\textcolor[rgb]{0.41,0.47,0.13}{##1}}}
\@namedef{PY@tok@nt}{\let\PY@bf=\textbf\def\PY@tc##1{\textcolor[rgb]{0.00,0.50,0.00}{##1}}}
\@namedef{PY@tok@nd}{\def\PY@tc##1{\textcolor[rgb]{0.67,0.13,1.00}{##1}}}
\@namedef{PY@tok@s}{\def\PY@tc##1{\textcolor[rgb]{0.73,0.13,0.13}{##1}}}
\@namedef{PY@tok@sd}{\let\PY@it=\textit\def\PY@tc##1{\textcolor[rgb]{0.73,0.13,0.13}{##1}}}
\@namedef{PY@tok@si}{\let\PY@bf=\textbf\def\PY@tc##1{\textcolor[rgb]{0.64,0.35,0.47}{##1}}}
\@namedef{PY@tok@se}{\let\PY@bf=\textbf\def\PY@tc##1{\textcolor[rgb]{0.67,0.36,0.12}{##1}}}
\@namedef{PY@tok@sr}{\def\PY@tc##1{\textcolor[rgb]{0.64,0.35,0.47}{##1}}}
\@namedef{PY@tok@ss}{\def\PY@tc##1{\textcolor[rgb]{0.10,0.09,0.49}{##1}}}
\@namedef{PY@tok@sx}{\def\PY@tc##1{\textcolor[rgb]{0.00,0.50,0.00}{##1}}}
\@namedef{PY@tok@m}{\def\PY@tc##1{\textcolor[rgb]{0.40,0.40,0.40}{##1}}}
\@namedef{PY@tok@gh}{\let\PY@bf=\textbf\def\PY@tc##1{\textcolor[rgb]{0.00,0.00,0.50}{##1}}}
\@namedef{PY@tok@gu}{\let\PY@bf=\textbf\def\PY@tc##1{\textcolor[rgb]{0.50,0.00,0.50}{##1}}}
\@namedef{PY@tok@gd}{\def\PY@tc##1{\textcolor[rgb]{0.63,0.00,0.00}{##1}}}
\@namedef{PY@tok@gi}{\def\PY@tc##1{\textcolor[rgb]{0.00,0.52,0.00}{##1}}}
\@namedef{PY@tok@gr}{\def\PY@tc##1{\textcolor[rgb]{0.89,0.00,0.00}{##1}}}
\@namedef{PY@tok@ge}{\let\PY@it=\textit}
\@namedef{PY@tok@gs}{\let\PY@bf=\textbf}
\@namedef{PY@tok@gp}{\let\PY@bf=\textbf\def\PY@tc##1{\textcolor[rgb]{0.00,0.00,0.50}{##1}}}
\@namedef{PY@tok@go}{\def\PY@tc##1{\textcolor[rgb]{0.44,0.44,0.44}{##1}}}
\@namedef{PY@tok@gt}{\def\PY@tc##1{\textcolor[rgb]{0.00,0.27,0.87}{##1}}}
\@namedef{PY@tok@err}{\def\PY@bc##1{{\setlength{\fboxsep}{\string -\fboxrule}\fcolorbox[rgb]{1.00,0.00,0.00}{1,1,1}{\strut ##1}}}}
\@namedef{PY@tok@kc}{\let\PY@bf=\textbf\def\PY@tc##1{\textcolor[rgb]{0.00,0.50,0.00}{##1}}}
\@namedef{PY@tok@kd}{\let\PY@bf=\textbf\def\PY@tc##1{\textcolor[rgb]{0.00,0.50,0.00}{##1}}}
\@namedef{PY@tok@kn}{\let\PY@bf=\textbf\def\PY@tc##1{\textcolor[rgb]{0.00,0.50,0.00}{##1}}}
\@namedef{PY@tok@kr}{\let\PY@bf=\textbf\def\PY@tc##1{\textcolor[rgb]{0.00,0.50,0.00}{##1}}}
\@namedef{PY@tok@bp}{\def\PY@tc##1{\textcolor[rgb]{0.00,0.50,0.00}{##1}}}
\@namedef{PY@tok@fm}{\def\PY@tc##1{\textcolor[rgb]{0.00,0.00,1.00}{##1}}}
\@namedef{PY@tok@vc}{\def\PY@tc##1{\textcolor[rgb]{0.10,0.09,0.49}{##1}}}
\@namedef{PY@tok@vg}{\def\PY@tc##1{\textcolor[rgb]{0.10,0.09,0.49}{##1}}}
\@namedef{PY@tok@vi}{\def\PY@tc##1{\textcolor[rgb]{0.10,0.09,0.49}{##1}}}
\@namedef{PY@tok@vm}{\def\PY@tc##1{\textcolor[rgb]{0.10,0.09,0.49}{##1}}}
\@namedef{PY@tok@sa}{\def\PY@tc##1{\textcolor[rgb]{0.73,0.13,0.13}{##1}}}
\@namedef{PY@tok@sb}{\def\PY@tc##1{\textcolor[rgb]{0.73,0.13,0.13}{##1}}}
\@namedef{PY@tok@sc}{\def\PY@tc##1{\textcolor[rgb]{0.73,0.13,0.13}{##1}}}
\@namedef{PY@tok@dl}{\def\PY@tc##1{\textcolor[rgb]{0.73,0.13,0.13}{##1}}}
\@namedef{PY@tok@s2}{\def\PY@tc##1{\textcolor[rgb]{0.73,0.13,0.13}{##1}}}
\@namedef{PY@tok@sh}{\def\PY@tc##1{\textcolor[rgb]{0.73,0.13,0.13}{##1}}}
\@namedef{PY@tok@s1}{\def\PY@tc##1{\textcolor[rgb]{0.73,0.13,0.13}{##1}}}
\@namedef{PY@tok@mb}{\def\PY@tc##1{\textcolor[rgb]{0.40,0.40,0.40}{##1}}}
\@namedef{PY@tok@mf}{\def\PY@tc##1{\textcolor[rgb]{0.40,0.40,0.40}{##1}}}
\@namedef{PY@tok@mh}{\def\PY@tc##1{\textcolor[rgb]{0.40,0.40,0.40}{##1}}}
\@namedef{PY@tok@mi}{\def\PY@tc##1{\textcolor[rgb]{0.40,0.40,0.40}{##1}}}
\@namedef{PY@tok@il}{\def\PY@tc##1{\textcolor[rgb]{0.40,0.40,0.40}{##1}}}
\@namedef{PY@tok@mo}{\def\PY@tc##1{\textcolor[rgb]{0.40,0.40,0.40}{##1}}}
\@namedef{PY@tok@ch}{\let\PY@it=\textit\def\PY@tc##1{\textcolor[rgb]{0.24,0.48,0.48}{##1}}}
\@namedef{PY@tok@cm}{\let\PY@it=\textit\def\PY@tc##1{\textcolor[rgb]{0.24,0.48,0.48}{##1}}}
\@namedef{PY@tok@cpf}{\let\PY@it=\textit\def\PY@tc##1{\textcolor[rgb]{0.24,0.48,0.48}{##1}}}
\@namedef{PY@tok@c1}{\let\PY@it=\textit\def\PY@tc##1{\textcolor[rgb]{0.24,0.48,0.48}{##1}}}
\@namedef{PY@tok@cs}{\let\PY@it=\textit\def\PY@tc##1{\textcolor[rgb]{0.24,0.48,0.48}{##1}}}

\def\PYZbs{\char`\\}
\def\PYZus{\char`\_}
\def\PYZob{\char`\{}
\def\PYZcb{\char`\}}
\def\PYZca{\char`\^}
\def\PYZam{\char`\&}
\def\PYZlt{\char`\<}
\def\PYZgt{\char`\>}
\def\PYZsh{\char`\#}
\def\PYZpc{\char`\%}
\def\PYZdl{\char`\$}
\def\PYZhy{\char`\-}
\def\PYZsq{\char`\'}
\def\PYZdq{\char`\"}
\def\PYZti{\char`\~}
% for compatibility with earlier versions
\def\PYZat{@}
\def\PYZlb{[}
\def\PYZrb{]}

\makeatother

% Basic stuff.
\newcommand{\code}[1]{\texttt{\detokenize{#1}}}
\newcommand{\ckey}[1]{{\texttt<\hbox{#1}>}}
\newcommand{\cchar}[1]{``{\texttt{\hbox{#1}}}''}
\newcommand{\ccmd}[1]{{\texttt{\hbox{#1}}}}

% Python and python modules.
\newcommand{\python}{\href{https://www.python.org/}{Python}}
\newcommand{\pymodule}[1]{\code{#1}}
\newcommand{\pyfunc}[1]{\code{#1()}}
\newcommand{\pyclass}[1]{\code{#1}}

\newcommand{\numpy}{\href{https://numpy.org/}{\pymodule{numpy}}}
\newcommand{\npmoduleurl}[1]{https://numpy.org/doc/stable/reference/#1}
\newcommand{\npfuncurl}[1]{https://numpy.org/doc/stable/reference/generated/numpy.#1.html}
\newcommand{\npmodule}[1]{\href{\npmoduleurl{#1}}{np.#1}}
\newcommand{\npfunc}[1]{\href{\npfuncurl{#1}}{np.\pyfunc{#1}}}
\newcommand{\nparray}{\emph{array} di \numpy}

\newcommand{\scipy}{\href{https://www.scipy.org/}{\pymodule{scipy}}}
\newcommand{\scipymoduleurl}[1]{https://docs.scipy.org/doc/scipy/reference/#1.html}
\newcommand{\scipyfuncurl}[1]{https://docs.scipy.org/doc/scipy/reference/generated/scipy.#1.html}
\newcommand{\scipymodule}[1]{\href{\scipymoduleurl{#1}}{scipy.\pymodule{#1}}}
\newcommand{\scipyfunc}[1]{\href{\scipyfuncurl{#1}}{scipy.\pyfunc{#1}}}

\newcommand{\matplotlib}{\href{https://matplotlib.org/}{\pymodule{matplotlib}}}

% Operating systems.
\newcommand{\windows}{Windows}
\newcommand{\dos}{DOS}
\newcommand{\linux}{GNU/Linux}
\newcommand{\macos}{Mac~OS}
\newcommand{\osx}{OS~X}
\newcommand{\freebsd}{FreeBSD}
\newcommand{\unix}{UNIX}

\newcommand{\sniptex}[1]{\medskip\input{sniptex/#1}}
\newcommand{\texdata}[1]{\medskip\input{texdata/#1}}

\newcommand{\emacs}{emacs}
\newcommand{\gedit}{gedit}

\DeclareFloatingEnvironment[name=Frammento]{snippet}

\newcommand{\snip}[3][0.62]{%
  \begin{snippet}[htb!]
    \bigskip
    \hstack[#1]{\input{sniptex/#2}}{
      \caption{#3}
      \label{snip:#2}
    }
  \end{snippet}
}

\newcommand{\npcmd}[3][0.4]{%
  \FPeval{\cw}{0.95 - #1}
  \noindent\begin{minipage}[t][][b]{#1\textwidth}
  \npfunc{#2}: #3
  \end{minipage}\hfill%
  \begin{minipage}[t][][b]{\cw\textwidth}
  \input{sniptex/np.#2.tex}
  \end{minipage}\\[8pt]
}

\newcommand{\scipycmd}[3][0.4]{%
  \FPeval{\cw}{0.95 - #1}
  \noindent\begin{minipage}[t][][b]{#1\textwidth}
  \scipyfunc{#2}: #3
  \end{minipage}\hfill%
  \begin{minipage}[t][][b]{\cw\textwidth}
  \input{sniptex/scipy.#2.tex}
  \end{minipage}\\[8pt]
}


% Logarithm with arbitrary base.
% -> log_10
\newcommand{\llog}[1][10]{\log_{#1}}

% Absolute value.
% -> |x|
\newcommand{\abs}[1]{\left| #1 \right|}

% Powers.
% -> x^a
\newcommand{\power}[2][2]{\left( #2 \right)^{#1}}

% Square.
% -> x^2
\newcommand{\sq}[1]{\power[2]{#1}}

% Expansion of the binomial coefficient.
% -> n1!/(n2!(n1 - n2)!)
\newcommand{\binomexpr}[2]{\frac{#1!}{#2!(#1 - #2)!}}

% Expression evaluation at a given point with square brackets.
% -> [x]_{a}
\newcommand{\at}[2]{\left[ #1\right]_{\makebox[-1pt][l]{${\scriptstyle#2}$}}}

% Expression evaluation in an interval.
% -> [x] _{a}^{b}
\newcommand{\eval}[3]{\left.#1%
  \right|_{\makebox[-1pt][l]{${\scriptstyle#2}$}}^{\makebox[-1pt][l]{${\scriptstyle#3}$}}}

% Upright d in math mode (for differentials).
% -> d
\newcommand{\ud}{\mathrm{d}}

% Differential.
% -> dx
\newcommand{\diff}[1][x]{\,\ud{#1}}

% Base command for defining derivatives.
% -> df/dx or d^kf/dx^k
\newcommand{\basederivative}[4][]{%
  \displaystyle%
  \ifx\\#1\\\frac{#4#2}{#4#3}%
  \else%
  \frac{#4^#1#2}{#4#3^#1}%
  \fi%
}

% Total derivative.
% -> df/dx(x) or d^kf/dx^k(x)
\newcommand{\td}[4][]{%
  \basederivative[#1]{#2}{#3}{\ud}%
  \ifx\\#4\\%
  \else%
  \mkern-4mu\left(#4\right)%
  \fi%
}

% Partial derivative.
% -> df/dx(x) or d^kf/dx^k(x)
\newcommand{\pd}[4][]{%
  \basederivative[#1]{#2}{#3}{\partial}%
  \ifx\\#4\\%
  \else%
  \mkern-4mu\left(#4\right)%
  \fi%
}

\newcommand{\intinf}{\int_{-\infty}^{\infty}\!\!\!}

\newcommand{\cinterval}[2]{\left[\, #1,~#2 \,\right]}

\newcommand{\linterval}[2]{\left[\, #1,~#2 \,\right)}

\newcommand{\rinterval}[2]{\left(\, #1,~#2 \,\right]}

\newcommand{\ointerval}[2]{\left(\, #1,~#2 \,\right)}




\newcommand{\prob}[1]{\displaystyle P\left(#1\right)}

\newcommand{\pvalue}{\emph{$p$-\emph{value}}}

\newcommand{\cond}{\,|\,}

\newcommand{\expect}[1]{\displaystyle E\left[#1\right]}

\newcommand{\mom}[2][]{\displaystyle {\cal M}_{#2}\ifx\\#1\\\else(#1)\fi}

\newcommand{\momalg}[1]{\displaystyle \lambda_{#1}}

\newcommand{\momcen}[1]{\displaystyle \mu_{#1}}

\newcommand{\skewness}{\displaystyle \gamma_1}

\newcommand{\kurtosis}{\displaystyle \gamma_2}

\newcommand{\charf}[1][x]{\phi_{#1}}

\newcommand{\momgenf}[1][x]{M_{#1}}

\newcommand{\fwhm}{{\scriptstyle \textsc{FWHM}}}

\newcommand{\hwhm}{{\scriptstyle \textsc{HWHM}}}

\newcommand{\median}{\mu_{\nicefrac{1}{2}}}

\newcommand{\var}[1]{\ensuremath{\text{Var}\left(#1\right)}}

\newcommand{\cov}[2]{\ensuremath{\text{Cov}\left(#1, #2\right)}}

\newcommand{\corr}[2]{\ensuremath{\text{Corr}\left(#1, #2\right)}}

\newcommand{\like}{\mathcal L}

\newcommand{\likelihood}[2][]{\like\ifx\\#2\\\else(#2\ifx\\#1\\\else;#1\fi)\fi}

\newcommand{\chisq}{\ensuremath{\chi^2}}

\newcommand{\chisquare}[2][]{\chisq\ifx\\#2\\\else(#2\ifx\\#1\\\else;#1\fi)\fi}

\newcommand{\loglikelihood}[2][]{\log\likelihood[#1]{#2}}

\newcommand{\pdf}[3][]{#2(#3\ifx\\#1\\\else;#1\fi)}

\newcommand{\binomialpdf}[2][]{\pdf[#1]{\mathcal B}{#2}}

\newcommand{\multinomialpdf}[2][]{\pdf[#1]{\mathcal M}{#2}}

\newcommand{\poissonpdf}[2][]{\pdf[#1]{\mathcal P}{#2}}

\newcommand{\uniformpdf}[2][]{\pdf[#1]{u}{#2}}

\newcommand{\exponentialpdf}[2][]{\pdf[#1]{\varepsilon}{#2}}

\newcommand{\gausspdf}[2][]{\pdf[#1]{N}{#2}}

\newcommand{\chisquarepdf}[2][]{\pdf[#1]{\wp}{#2}}

\newcommand{\cauchypdf}[2][]{\pdf[#1]{c}{#2}}

\newcommand{\erf}[1]{\ensuremath{\text{erf}\left(#1\right)}}

\newcommand{\dccases}[4][]{#2 \ifx\\#2\\\else=\fi %
  \begin{cases}
    \displaystyle #3 & \text{per variabili discrete}\\
    \displaystyle #4 & \text{per variabili continue}#1
  \end{cases}
}

\renewenvironment{enumerate}
{\begin{list}{\theenumi\rule{0.5pt}{0pt}.}{%
\usecounter{enumi}
\setlength{\itemindent}{12pt}%
\setlength{\labelwidth}{25pt}%
\setlength{\leftmargin}{\rightmargin}%
}}%
{\end{list}}

% 
\renewenvironment{itemize}
{\begin{list}{$\blacktriangleright$}{%
\setlength{\itemindent}{12pt}%
\setlength{\labelwidth}{25pt}%
\setlength{\leftmargin}{\rightmargin}%
}}
{\end{list}}

% Exercises.
% \exercise{text}
\newtheoremstyle{_exercise_}
{9pt}
{9pt}
{\rmfamily}
{}
{\scshape}
{.}
{ }
{$\blacktriangleright$ \thmname{#1} \thmnumber{#2}\thmnote{#3}}
\theoremstyle{_exercise_}
\newtheorem{_exercise_}{Esercizio}[chapter]

\newcommand{\exercise}[1]
{\begin{_exercise_} {\rmfamily #1} \end{_exercise_}}

\newenvironment{exercises}
{\clearpage
\section{Esercizi}}
{}

\newcommand{\hint}[1]
{\noindent \emph{Suggerimento: #1}}

\newcommand{\answer}[1]
{\hfill \emph{Risposta:} $\left[\, #1 \,\right]$}

% Examples.
% This used to be a \newcommand with the \begin{} and \end{} blocks of the
% newtheorem embedded, but that had the drawback that no \verbatim blocks can be
% used into command arguments. 
%
% First we define a \newtheorem...
%
\newtheoremstyle{example}
{\baselineskip}       % Space above
{\baselineskip}       % Space below
{\normalsize}         % The body font (use \slshape for slanted body)
{}                    % Indent amount (empty = no indent)
{\scshape}            % Thm head font
{.}                   % Punctuation after thm head
{ }                   % Thm head spec (can be left empty, meaning `normal')
{$\blacktriangleright$ \thmname{#1} \thmnumber{#2}%
                       \ifx\\#3\\%
                       \else~(\thmnote{#3})\fi}
\theoremstyle{example}
\newtheorem{example}{Esempio}[chapter]
%               
% ...then a new md environment...
%
\newmdenv[linecolor=lightgray,
          skipabove=0pt,
          skipbelow=0pt,
          innertopmargin=0.5\baselineskip,
          innerbottommargin=0.4\baselineskip,
          innerleftmargin=0.5\baselineskip,
          innerrightmargin=0.5\baselineskip,     
          roundcorner=4pt,
          nobreak=false]{exampleframe}
%
% Then a new environment to make the frames float...
%
\DeclareFloatingEnvironment[name=FloatExample]{floatexamplebox}

%
% ...finally we embed everything into a \newenvironment, which is the one we
% actually use.
%
\newenvironment{examplebox}[1][!htb]
{\begin{floatexamplebox}[#1]\begin{exampleframe}}
{\end{exampleframe}\end{floatexamplebox}}
%
% Done with the examples.


% Theorems
\newtheoremstyle{_theorem_}
{9pt}
{9pt}
{\rmfamily}
{}
{\scshape}
{.}
{ }
{\thmname{#1}\thmnumber{ #2}\ifx\\#3\\\else~(\thmnote{#3})\fi}
\theoremstyle{_theorem_}
\newtheorem{theorem}{Teorema}[chapter]

\newtheorem{corollary}{Corollario}[chapter]


\newcommand{\pgffigone}[3][!htb]{%
  \begin{figure}[#1]
    \hstack[0.5]{\input{figures/#2.pgf}}{\caption{#3}\label{fig:#2}}
  \end{figure}%
}


\newcommand{\pgffigtwo}[4][!htb]{%
  \begin{figure}[#1]
    \input{figures/#2.pgf}\hfill\input{figures/#3.pgf}
    \caption{#4}\label{fig:#2_#3}
  \end{figure}%
}


\newcommand{\pgffigthree}[5][!htb]{%
  \begin{figure}[#1]
    \hstack[0.5]{\input{figures/#2.pgf}}{\input{figures/#3.pgf}}
    \hstack[0.5]{\input{figures/#4.pgf}}{\caption{#5}\label{fig:#2_three}}
  \end{figure}%
}


\newcommand{\pgffigfour}[6][!htb]{%
  \begin{figure}[#1]
    \input{figures/#2.pgf}\hfill\input{figures/#3.pgf}
    \input{figures/#4.pgf}\hfill\input{figures/#5.pgf}
    \caption{#6}\label{fig:#2_four}
  \end{figure}%
}


\newcommand{\pgffigsix}[8][!htb]{%
  \begin{figure}[#1]
    \input{figures/#2.pgf}\hfill\input{figures/#3.pgf}
    \input{figures/#4.pgf}\hfill\input{figures/#5.pgf}
    \input{figures/#6.pgf}\hfill\input{figures/#7.pgf}
    \caption{#8}\label{fig:#2_six}
  \end{figure}%
}

\newcommand{\foreign}[1]{\emph{#1}}

\newcommand{\bias}{\foreign{bias}}
\newcommand{\errorfunc}{\foreign{error function}}
\newcommand{\fit}{\foreign{fit}}
\newcommand{\Fit}{\foreign{Fit}}
\newcommand{\fitting}{\foreign{fitting}}
\newcommand{\bestfit}{\foreign{best-fit}}
\newcommand{\software}{\foreign{software}}


\makeindex
