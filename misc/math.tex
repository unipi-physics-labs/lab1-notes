
% Logarithm with arbitrary base.
% -> log_10
\newcommand{\llog}[1][10]{\log_{#1}}

% Absolute value.
% -> |x|
\newcommand{\abs}[1]{\left| #1 \right|}

% Powers.
% -> x^a
\newcommand{\power}[2][2]{\left( #2 \right)^{#1}}

% Square.
% -> x^2
\newcommand{\sq}[1]{\power[2]{#1}}

% Expansion of the binomial coefficient.
% -> n1!/(n2!(n1 - n2)!)
\newcommand{\binomexpr}[2]{\frac{#1!}{#2!(#1 - #2)!}}

% Expression evaluation at a given point with square brackets.
% -> [x]_{a}
\newcommand{\at}[2]{\left[ #1\right]_{\makebox[-1pt][l]{${\scriptstyle#2}$}}}

% Expression evaluation in an interval.
% -> [x] _{a}^{b}
\newcommand{\eval}[3]{\left.#1%
  \right|_{\makebox[-1pt][l]{${\scriptstyle#2}$}}^{\makebox[-1pt][l]{${\scriptstyle#3}$}}}

% Upright d in math mode (for differentials).
% -> d
\newcommand{\ud}{\mathrm{d}}

% Differential.
% -> dx
\newcommand{\diff}[1][x]{\,\ud{#1}}

% Base command for defining derivatives.
% -> df/dx or d^kf/dx^k
\newcommand{\basederivative}[4][]{%
  \displaystyle%
  \ifx\\#1\\\frac{#4#2}{#4#3}%
  \else%
  \frac{#4^#1#2}{#4#3^#1}%
  \fi%
}

% Total derivative.
% -> df/dx(x) or d^kf/dx^k(x)
\newcommand{\td}[4][]{%
  \basederivative[#1]{#2}{#3}{\ud}%
  \ifx\\#4\\%
  \else%
  \mkern-4mu\left(#4\right)%
  \fi%
}

% Partial derivative.
% -> df/dx(x) or d^kf/dx^k(x)
\newcommand{\pd}[4][]{%
  \basederivative[#1]{#2}{#3}{\partial}%
  \ifx\\#4\\%
  \else%
  \mkern-4mu\left(#4\right)%
  \fi%
}

\newcommand{\intinf}{\int_{-\infty}^{\infty}\!\!\!}

\newcommand{\cinterval}[2]{\left[\, #1,~#2 \,\right]}

\newcommand{\linterval}[2]{\left[\, #1,~#2 \,\right)}

\newcommand{\rinterval}[2]{\left(\, #1,~#2 \,\right]}

\newcommand{\ointerval}[2]{\left(\, #1,~#2 \,\right)}


