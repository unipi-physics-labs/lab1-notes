\chapter{Glossario di \numpy}
\label{sec:numpy}

Questa appendice costituisce un breve glossario delle funzioni di \numpy\ che
sono usate più largamente in laboratorio.

Dato che il \software\ evolve sempre più velocemente della carta stampata,
non riportiamo per intero la segnatura delle funzioni (che è invece disponibile
nella documentazione online, \foreign{linkata} per convenienza), ma ci limitiamo,
ove utile, a menzionare gli argomenti più importanti. Notiamo che, ove le
funzioni operino su \foreign{array}, \numpy~è equipaggiato per convertire
opportunamente liste o tuple di numeri e, più in generale, qualsiasi oggetto
iterabile di \python~che possa essere trasformato in un \foreign{array}.


\section{Variabili campione}


\npcmd{average}{%
restituisce la media pesata dei valori di un \foreign{array}
\begin{align*}
  m = \frac{\sum_{i=1}^{n} w_i x_i}{\sum_{i=1}^{n} w_i}.
\end{align*}
L'argomento \code{weights} contiene i valori dei pesi nella forma
di un \foreign{array} delle stesse dimensioni di quello dei valori.
}

\npcmd{mean}{%
restituisce la media aritmetica dei valori di un \foreign{array}
\begin{align*}
  m = \frac{1}{n}\sum_{i=1}^{n} x_i.
\end{align*}
}

\npcmd{std}{%
restituisce la deviazione standard campione dei valori di un \foreign{array},
ovvero la radice quadrata della varianza campione
\begin{align*}
  s = \sqrt{\frac{1}{(n - \Delta_\text{dof})} \sum_{i=1}^{n} (x_i - m)^2}.
\end{align*}
L'argomento \code{ddof} contiene la differenza tra il numero di valori
ed il numero di gradi di libertà.
}

\npcmd{var}{%
restituisce la varianza campione dei valori di un \foreign{array}
\begin{align*}
  s^2 = \frac{1}{(n - \Delta_\text{dof})} \sum_{i=1}^{n} (x_i - m)^2.
\end{align*}
L'argomento \code{ddof} contiene la differenza tra il numero di valori
ed il numero di gradi di libertà. Per \code{ddof = 0} si ha la stima
non imparziale~\eqref{eq:varianza_campione}, mentre per \code{ddof = 1}
si ha la stima imparziale~\eqref{eq:varianza_campione_imparziale}.
}


\section{Generazione di numeri pseudo-casuali}

\npcmd{random.binomial}{%
restituisce un \foreign{array} di numeri della lunghezza richiesta (in questo caso~$15$),
estratti da una distribuzione binomiale con $n$ e $p$ fissati---vedi
sezione~\ref{sec:distribuzione_binomiale}}.

\npcmd{random.chisquare}{%
restituisce un \foreign{array} di numeri della lunghezza richiesta (in questo caso~$3$),
estratti da una distribuzione del $\chisq$ con un dato numero di gradi di
libertà---vedi sezione~\ref{sec:pdf_chi2}.}

\npcmd{random.exponential}{%
restituisce un \foreign{array} di numeri della lunghezza richiesta (in questo caso~$3$),
estratti da una distribuzione esponenziale con media fissata---vedi
sezione~\ref{sec:distribuzione_esponenziale}.}

\npcmd{random.normal}{%
restituisce un \foreign{array} di numeri della lunghezza richiesta (in questo caso~$3$),
estratti da una distribuzione di Gauss con media e deviazione standard
fissate---vedi sezione~\ref{sec:distribuzione_gauss}.}

\npcmd{random.poisson}{%
restituisce un \foreign{array} di numeri della lunghezza richiesta (in questo caso~$15$),
estratti da una distribuzione di Poisson con media fissata---vedi
sezione~\ref{sec:poisson_pdf}.}

\npcmd{random.uniform}{%
restituisce un \foreign{array} di numeri della lunghezza richiesta (in questo caso~$3$),
estratti da una distribuzione uniforme nell'intervallo
$[x_\text{min}, x_\text{max}]$---vedi sezione~\ref{sec:distribuzione_uniforme}.}


\section{Miscellanea}

\npcmd{full}{%
restituisce un \foreign{array} di lunghezza fissata i cui valori sono tutti
identici al numero passato come secondo argomento.
}

\npcmd{geomspace}{%
restituisce una griglia di numeri spaziati logaritmicamente tra un minimo ed
un massimo.
}

\npcmd{linspace}{%
restituisce una griglia equispaziata di numeri tra un minimo ed un massimo.
}
