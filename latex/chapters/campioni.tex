\chapter{Variabili campione e propagazione dell'errore statistico}
\label{sec:teoria_dei_campioni}

\`E giunto adesso il momento di fare tesoro dei concetti che abbiamo
introdotto nel capitolo precedente per rivedere criticamente la nozione di
errore di misura. Quando misuriamo una grandezza fisica---e specialmente quando
il valore della misura fluttua---possiamo pensare che il valore stesso sia
una variabile casuale con una certa distribuzione (tipicamente incognita a
priori) che chiamiamo \emph{distribuzione generatrice}. In questo schema
concettuale fare $n$ misure di una stessa grandezza fisica in condizioni di
ripetitività (che era il problema da cui eravamo partiti nel
capitolo~\ref{sec:primi_passi}) equivale a \emph{campionare} $n$ volte la
distribuzione generatrice corrispondente. \`E ovvio, allora, che in generale
non potremo conoscere mai perfettamente la forma della distribuzione
generatrice, né le sue caratteristiche (e.g., la media o la deviazione
standard); ma è altresì ovvio che, in questo processo di campionamento,
acquisiamo progressivamente informazioni sulla distribuzione stessa.
Se facciamo, ad esempio, un istogramma dei valori delle misure ottenute, ci
aspettiamo che per $n \rightarrow \infty$ la forma del nostro istogramma tenda
alla forma della distribuzione generatrice, e che, a partire dal nostro
campione, possiamo fornire stime via via più accurate delle sue
caratteristiche.

Se torniamo allora al problema iniziale di scrivere il risultato di una misura
come il dato di un valore centrale e di un'incertezza di misura ad esso
associata, cominciamo ad intravedere una possibilità operativa nel linguaggio
della teoria della probabilità che abbiamo imparato. Se è vero che il nostro
processo di misura equivale al campionamento ripetuto di una distribuzione,
allora potremo prendere come miglior stima della grandezza cui siamo
interessati una delle misure di tendenza centrale (tipicamente la media) della
distribuzione e come incertezza associata la sua deviazione standard---che,
come abbiamo visto, rappresenta l'entità delle fluttuazioni attorno al valor
medio e fornisce una misura conveniente delle fluttuazioni stesse.

Ripartiamo dunque dalla~\eqref{eq:errore_stat}, che costituisce il modo standard
in cui scriviamo il risultato di una misura%
\footnote{La scelta, onnipresente in letteratura, di utilizzare la lettera
  $\sigma$ per indicare sia la deviazione standard di una distribuzione che
  l'errore statistico è largamente infelice, perché le due cose sono
  concettualmente diverse. La prima è una proprietà matematica che si
  calcola dai principi primi, mentre il secondo, pur rappresentando una stima
  di una deviazione standard è, nella maggior parte dei casi, una quantità
  che si stima sperimentalmente. Purtuttavia ci adegueremo, sperando che il
  contesto sia sufficiente ad evitare ambiguità.}
\begin{align*}
  x = \hat{x} \pm \sigma_x~\text{[unità di misura]}.
\end{align*}
Come abbiamo avuto modo di dire, questa equazione definisce un intervallo
corrispondente ad una probabilità ben definita (che chiameremo
\emph{livello di confidenza}, o CL, dall'inglese \foreign{confidence level})
di contenere il valore del misurando.
Nella nostra prescrizione il livello di confidenza corrisponde all'integrale
della distribuzione generatrice entro una deviazione standard dalla media, e si
può stimare esplicitamente a partire della distribuzione generatrice stessa.
Il valore sarà in generale diverso caso per caso, ma sappiamo già che, nei
casi tipici, sarà grossolanamente dell'ordine del $60$--$70\%$.
Per completezza, e per evitare qualsiasi possibile ambiguità, uno può
(e dovrebbe) specificare esplicitamente il livello di confidenza, e.g.
\begin{align}\label{eq:errore_stat_cl}
  x = \hat{x} \pm \sigma_x~\text{[unità di misura]}~(68\%~\text{CL})
\end{align}
(o qualunque sia il livello di confidenza scelto.)


\section{Campionamenti singoli}
\label{sec:errore_campionamenti_singoli}

Se conosciamo a priori la deviazione standard $\sigma$ della distribuzione
generatrice del misurando cui siamo interessati, oppure ne abbiamo una stima,
allora una singola misura (cioè un singolo campionamento) è sufficiente per
definire tutte le componenti della~\eqref{eq:errore_stat}---prenderemo il (singolo)
valore misurato come valore centrale e $\sigma$ come incertezza associata.

Se conosciamo (esattamente o approssimativamente) la forma della
distribuzione generatrice, allora possiamo anche calcolare il livello di
confidenza associato ad una deviazione standard, oppure utilizzare come stima
dell'incertezza di misura un multiplo o sottomultiplo della deviazione standard
per ottenere un livello confidenza fissato a priori. Torneremo su questo
problema specifico nel seguito, e vedremo che utilizzare una deviazione
standard come incertezza (con un livello di confidenza che, di conseguenza,
dipende dai dettagli della distribuzione generatrice) è comodo in pratica.

\begin{examplebox}
  \begin{example}
    Supponiamo di misurare il peso $m$ di un oggetto con una bilancia digitale
    con una risoluzione di $1$~g. Se il valore indicato dal display è $58$~g
    possiamo assumere che, in assenza di errori sistematici, la distribuzione
    generatrice del misurando sia uniforme tra $57.5$~g e $58.5$~g, e possiamo
    utilizzare la deviazione standard di questa distribuzione (che
    dall'esercizio~\ref{exp:varianza_dist_uniforme} sappiamo essere
    $\sqrt{\nicefrac{1}{12}} \approx 0.29$~g) come incertezza di misura.
    Sappiamo inoltre che per una distribuzione uniforme la probabilità che
    la variabile disti meno di una deviazione standard dalla media è pari
    a circa il $58\%$, per cui possiamo scrivere
    \begin{align*}
      m = 58.00 \pm 0.29~g~(58\%~\text{CL}).
    \end{align*}
  \end{example}

  \begin{example}
    Il ragionamento dell'esempio precedente si applica banalmente alla misura di
    una lunghezza con un metro a nastro e, più in generale, a tutti gli
    strumenti digitali---nel caso in cui decidiamo di non interpolare tra le
    divisioni e assumendo di non commettere errori di lettura.
  \end{example}

  \begin{example}
    In generale gli strumenti si possono \emph{calibrare} attraverso misure
    ripetute di grandezze di riferimento (ad esempio misurate preliminarmente
    con uno strumento di misura più accurato o con un metodo indipendente).
    Questo permette, nel nostro linguaggio, di stimare la deviazione standard
    della distribuzione generatrice (e magari anche la sua forma), che può poi
    essere utilizzata come incertezza sulla singola misura.
  \end{example}
\end{examplebox}

Se la deviazione standard della distribuzione generatrice non è nota a priori,
possiamo stimarla a posteriori in base ad una serie di misure ripetute.
Sappiamo che, in presenza di fluttuazioni statistiche, ripetere la misura più
volte è anche un modo per aumentarne la precisione. Da un punto di vista
operativo il nostro problema fondamentale diventa: data una serie di $n$ misure
di una stessa grandezza fatte in condizioni di ripetitività, quali sono le
migliori stime che possiamo dare della media e della deviazione standard della
distribuzione generatrice? Si tratta di un problema che risolveremo in questo
capitolo ma, per cominciare, abbiamo bisogno di sapere come si "sommano" le
distribuzioni.


\section{Somma di variabili casuali indipendenti}
\label{sec:somma_variabili_indipendenti}

Date due variabili casuali indipendenti $x_1$ ed $x_2$ (continue o discrete)
caratterizzate dalle rispettive funzioni di distribuzione (in generale diverse),
la loro somma $x = x_1 + x_2$ è ovviamente ancora una variabile
casuale---ricordate l'esempio~\ref{exp:pdf_somma_due_dadi} del lancio di due
dadi. Allora è lecito chiedersi quale sia la funzione di distribuzione di
$x$ e come essa si possa calcolare a partire dalle funzioni di distribuzione
di $x_1$ ed $x_2$.

In generale si tratta di una domanda non banale, che richiede di identificare
e \emph{pesare} opportunamente le diverse configurazioni di $x_1$ ed $x_2$ che
danno origine allo stesso valore della somma $x$. Anche nel caso più semplice
possibile, quello in cui $x_1$ ed $x_2$ sono variabili casuali continue
distribuite entrambe uniformemente tra $0$ ed $1$, la risposta non è scontata.
Possiamo sicuramente dire che $x$ sarà compresa tra $0$ e $2$; ma cosa altro?
La densità di probabilità di $x$ sarà uniforme in questo intervallo?

\pgffigtwo{somma_pdf_uniforme}{pdf_somma_pdf_uniforme}{
  Rappresentazione geometrica della convoluzione di due funzioni di
  distribuzione uniformi (a sinistra). Se $x_1$ ed $x_2$ sono uniformemente
  distribuite tra $0$ ed $1$, allora le regioni del piano cartesiano a somma
  costante sono segmenti di retta con coefficiente angolare~$-1$ e se $x_1$ ed
  $x_2$ sono indipendenti, la probabilità che $x_1 + x_2$ assuma un valore
  fissato è proporzionale alla lunghezza del segmento corrispondente.
  I segmenti si riducono a punti per i casi limite $x_1 + x_2 = 0$ e
  $x_1 + x_2 = 2$, ove la densità di probabilità della somma (mostrata a
  destra) deve tendere a $0$.
}
In questo caso l'intuizione ci viene in aiuto nella forma del grafico mostrato
in~figura~\ref{fig:somma_pdf_uniforme_pdf_somma_pdf_uniforme}: le
configurazioni nel piano $x_1$--$x_2$ che danno origine ad un valore della somma
$x$ fissato sono linee diagonali con pendenza $-1$, e la probabilità
corrispondente sarà dunque proporzionale alla lunghezza delle linee
stesse---identicamente nulla in $x = 0$ ed $x = 2$ (ove le linee si riducono a
punti), massima in $x = 1$ e linearmente crescente/decrescente nelle due metà
dell'intervallo $[0,~2]$. La densità di probabilità cercata è dunque
triangolare
\begin{align*}
  p(x) =
  \begin{cases}
    x & \text{se}~x \leq 1\\
    (2 - x) &  \text{se}~x > 1,
  \end{cases}
\end{align*}
come mostrato in figura~\ref{fig:somma_pdf_uniforme_pdf_somma_pdf_uniforme}.
(Se questo vi sorprende, tornate per un attimo indietro a guardare la
figura~\ref{fig:pdf_un_dado_pdf_somma_due_dadi} e gli
esempi~\ref{exp:pdf_un_dado} e \ref{exp:pdf_somma_due_dadi}---si tratta
sostanzialmente dello stesso fenomeno, visto questa volta per una variabile
casuale continua.)


\subsection{Media e varianza della somma di variabili casuali indipendenti}
\label{sec:media_varianza_somma}

Torniamo al nostro problema iniziale, ovvero le proprietà della funzione
di distribuzione della somma $x = x_1 + x_2$ di due variabili casuali
indipendenti. Abbiamo visto che calcolare esplicitamente la funzione di
distribuzione, nel caso generale, non è banale. Fortunatamente il calcolo
della media e della varianza di $x$ sono problemi più facilmente
trattabili. La prima si scrive banalmente come
\begin{align}\label{eq:media_somma_ind}
  \mu = \expect{x} = \expect{x_1 + x_2} = \expect{x_1} + \expect{x_2} =
  \mu_1 + \mu_2.
\end{align}
E la varianza $\sigma^2$ è leggermente più complicata
\begin{align*}
  \sigma^2 & = \var{x} = \expect{(x - \mu)^2} = \expect{x^2} - \mu^2 =
  \expect{(x_1 + x_2)^2} - (\mu_1 + \mu_2)^2 =\\
  & = \expect{x_1^2} + \expect{x_2^2} + 2\expect{x_1x_2} -
  \mu_1^2 - \mu_2^2 - 2\mu_1\mu_2 =
  \sigma_1^2 + \sigma_2^2 + 2\expect{x_1x_2} - 2\mu_1\mu_2.
\end{align*}
(Nell'ultimo passaggio $\sigma_1^2$ e $\sigma_2^2$ sono le varianze di $x_1$ ed
$x_2$, rispettivamente.) Nell'espressione che abbiamo ottenuto riconosciamo
la covarianza delle due variabili di partenza, per cui nel caso generale
possiamo scrivere
\begin{align}\label{eq:varianza_somma}
  \sigma^2 =  \sigma_1^2 + \sigma_2^2 + 2\cov{x_1}{x_2}
\end{align}
e, nel caso in cui $x_1$ ed $x_2$ siano indipendenti, la loro covarianza è
per definizione nulla e la~\eqref{eq:varianza_somma} si semplifica in
\begin{align}\label{eq:varianza_somma_ind}
  \sigma^2 = \sigma_1^2 + \sigma_2^2 \quad \text{da cui} \quad
  \sigma = \sqrt{\sigma_1^2 + \sigma_2^2}
\end{align}
La scrittura nell'ultimo passaggio della~\eqref{eq:varianza_somma_ind} prende
generalmente il nome di \emph{somma in quadratura}. Notiamo che la somma in
quadratura è in generale più piccola della somma propriamente detta
\begin{align*}
  \sqrt{\sigma_1^2 + \sigma_2^2} \leq (\sigma_1 + \sigma_2),
\end{align*}
come si può dimostrare banalmente per via geometrica (pensate al teorema di
Pitagora).

A posteriori questo risultato non deve sorprendere---nel
capitolo~\ref{sec:probabilita} abbiamo visto, e.g., che la media della somma
delle uscite di due dadi equi è il doppio della media dell'uscita di un dado
equo (esempi~\ref{exp:media_dado} e~\ref{exp:media_due_dadi}) e che la varianza
della somma delle uscite di due dadi equi è il doppio della varianza
dell'uscita di un dado equo (esempi~\ref{exp:varianza_dado}
e~\ref{exp:varianza_due_dadi}).
L'esempio~\ref{exp:media_varianza_pdf_triangolare} mostra che la stessa cosa
è vera per la distribuzione triangolare in
figura~\ref{fig:somma_pdf_uniforme_pdf_somma_pdf_uniforme}.

\begin{examplebox}
  \begin{example}\label{exp:media_varianza_pdf_triangolare}
    Consideriamo la distribuzione triangolare in
    figura~\ref{fig:somma_pdf_uniforme_pdf_somma_pdf_uniforme}. Poiché essa
    è simmetrica rispetto a $x = 1$, la media sarà proprio $\mu = 1$.
    Per la varianza possiamo al solito utilizzare la~\eqref{eq:variance_alt}
    \begin{align*}
      \expect{x^2} & = \int_0^2 x^2 p(x) dx =
      \int_0^1 x^3 dx + \int_1^2 x^2(2 - x) dx =
      \eval{\frac{x^4}{4}}{0}{1} +
      \eval{\frac{2x^3}{3}}{1}{2} -
      \eval{\frac{x^4}{4}}{1}{2} =\\
      & = \left( \frac{1}{4} - 0 \right) +
      \left( \frac{16}{3} - \frac{2}{3} \right) -
      \left( \frac{16}{4} - \frac{1}{4} \right) =
      \frac{3 - 0 + 64 - 8 - 48 + 3}{12} = \frac{14}{12},
    \end{align*}
    da cui
    \begin{align*}
      \sigma^2 = \expect{x^2} - \mu^2 = \frac{14}{12} - 1 = \frac{2}{12} =
      \frac{1}{6}.
    \end{align*}
    La distribuzione in questione è la "somma" di due distribuzioni
    uniformi tra $0$ ed $1$ identiche, con media
    $\mu_1 = \mu_2 = \nicefrac{1}{2}$ e varianza
    $\sigma_1^2 = \sigma_2^2 = \nicefrac{1}{12}$, come illustrato negli
    esempi~\ref{exp:media_dist_uniforme} e~\ref{exp:varianza_dist_uniforme}.
  \end{example}
\end{examplebox}

La~\eqref{eq:media_somma_ind} e la~\eqref{eq:varianza_somma_ind} si
generalizzano banalmente alla somma di un numero arbitrario di variabili
casuali indipendenti $x = \sum_{i = 1}^{n} x_i$ con medie $\expect{x_i} = \mu_i$
e varianze $\var{x_i} = \sigma_i^2$. Dette $\mu = \expect{x}$ e
$\sigma^2 = \var{x}$ si ha
\begin{align}\label{eq:media_varianza_somma_ind}
  \mu = \sum_{i = 1}^{n} \mu_i \quad \text{e} \quad
  \sigma^2 = \sum_{i = 1}^{n} \sigma_i^2 \quad \text{ovvero} \quad
  \sigma  = \sqrt{\sum_{i = 1}^{n} \sigma_i^2}
\end{align}
In altre parole abbiamo il risultato fondamentale che
\emph{date $n$ variabili casuali indipendenti: la media della somma è uguale
  alla somma delle medie e la varianza della somma è uguale alla somma delle
  varianze---da cui la deviazione standard della somma è uguale alla somma
  in quadratura delle deviazioni standard.}
(Notiamo, per inciso, come quest'ultima proprietà sia uno dei motivi
fondamentali per definire la varianza come abbiamo fatto
nella~\eqref{eq:variance}. Se avessimo usato, ad esempio, il modulo al posto
del quadrato, non esisterebbe un equivalente altrettanto semplice di questa
legge di somma.)


\subsection{Una applicazione interessante: il \foreign{random walk} uni-dimensionale}
\label{sec:random_walk}

\danger%
Consideriamo un punto materiale vincolato a muoversi su una retta (per fissare
le idee consideriamo un sistema di coordinate diretto lungo la retta e la cui
origine coincida con la posizione del punto all'istante iniziale). Supponiamo
inoltre che il moto del nostro punto avvenga nella forma di una successione
(arbitrariamente lunga) di passi elementari di lunghezza prefissata
$\ell$, nella direzione delle coordinate positive o negative con uguale
probabilità. Siamo interessati a studiare la posizione $x_n$ del punto dopo
$n$ passi---che, va da sé, è una variabile casuale.

La variabile casuale $s$ che descrive il singolo passo può assumere i valori
$+\ell$ e $-\ell$ con funzione di distribuzione
\begin{align}
  \prob{s = +\ell} = \prob{s = -\ell} = \frac{1}{2},
\end{align}
ed è facile calcolarne media e varianza:
\begin{align*}
  \expect{s} = \ell \times \frac{1}{2} - \ell \times \frac{1}{2} = 0
  \quad \text{e} \quad
  \var{s} = \expect{s^2} - \mu_s^2 = \expect{s^2} =
  \ell^2 \times \frac{1}{2} + \ell^2 \times \frac{1}{2} = \ell^2.
\end{align*}
La variabile casuale $x_n$ (cioè la posizione cercata) sarà data dunque
dalla somma di $n$ variabili casuali identiche distribuite come $s$. Ma allora
sappiamo immediatamente che
\begin{align}\label{eq:media_varianza_random_walk}
  \expect{x_n} = n \expect{s} = 0 \quad \text{e} \quad
  \var{x_n} = \expect{x_n^2} - \mu_{x_n}^2 = \expect{x_n^2} =
  n \expect{s^2} = n\ell^2,
\end{align}
cioè la posizione media del punto dopo un numero arbitrario di passi $n$ è
$0$ (il che non dovrebbe sorprendere, dato che, ad ogni passo, ci muoviamo con
uguale probabilità verso destra o verso sinistra) e la varianza di $x_n$ è
pari ad $n\ell^2$---per cui la sua deviazione standard è proporzionale
a~$\sqrt{n}\ell$.

Il processo che abbiamo appena descritto prende generalmente il nome di
\foreign{random walk} uni-dimensionale. La figura~\ref{fig:random_walk} mostra, a
scopo illustrativo, 15 realizzazioni indipendenti dei primi $500$ passi di un
\foreign{random walk} uni-dimensionale con lunghezza unitaria $\ell = 1$.

\pgffigone{random_walk}{
  15 realizzazioni indipendenti dei primi $500$ passi di un \foreign{random walk}
  uni-dimensionale con lunghezza unitaria $\ell = 1$ (in unità di misura
  arbitrarie). La posizione $x_n$ è mostrata in funzione del numero di
  passo $n$. Le linee indicano i contorni ad $1$ e $2$ deviazioni standard,
  cioè $x_n = \pm \sqrt{n}$ e $x_n = \pm 2\sqrt{n}$.
}

La cosa interessante del \foreign{random walk} è che la distanza media $d_n$
dall'origine al passo $n$-esimo, che sarà proporzionale alla deviazione
standard di $x_n$, scala con la radice quadrata di $n$ per
la~\eqref{eq:media_varianza_random_walk}
\begin{align}\label{eq:distanza_random_walk}
  d_n \propto \ell\sqrt{n},
\end{align}
(mentre se uno viaggiasse in linea retta si avrebbe banalmente $d_n = \ell n$).
La differenza non è da poco, e produce spesso situazioni sorprendenti, come
quella descritta nell'esempio~\ref{exp:random_walk_ubriaco}

\begin{examplebox}
  \begin{example}\label{exp:random_walk_ubriaco}
    Un ubriaco esce dal bar e compie un \foreign{random walk} fino al suo hotel,
    che si trova sullo stesso viale a distanza $d = 1$~km. Assumendo che faccia
    un passo al secondo (verso destra o verso sinistra con probabilità
    $\nicefrac{1}{2}$) e che la lunghezza (fissa) di ciascun
    passo  sia $l = 1$~m, quanto tempo impiega (in media) per arrivare a
    destinazione? Trascurando il fattore di proporzionalità
    nella~\eqref{eq:distanza_random_walk}
    \begin{align*}
      n \sim (d/l)^2 = 10^6
    \end{align*}
    e dunque $t \sim 10^6$~s (ossia quasi due settimane). Per confronto, ad un
    sobrio servirebbero $n \sim d/l = 10^3$ passi, ossia circa un quarto
    d'ora.
  \end{example}
\end{examplebox}


\section{Misure ripetute: media e varianza campione}
\label{sec:media_varianza_campione}

Riprendiamo il filo della discussione. Abbiamo iniziato questo capitolo con la
domanda: data una serie di misure $x_i$ ($i = 1\ldots n$) indipendenti di una
stessa grandezza $x$ fatte in condizioni di ripetitività, quali sono le
migliori stime che possiamo dare della media $\mu$ e della varianza $\sigma^2$
della distribuzione generatrice?

Come stima $m$ della media possiamo prendere la media aritmetica delle misure,
che viene generalmente chiamata \emph{media campione}
\begin{align}\label{eq:media_campione}
  m = \frac{1}{n}\sum_{i = 1}^n x_i.
\end{align}
La media campione ha la buona proprietà che il suo valore di aspettazione
è uguale alla media della distribuzione generatrice
\begin{align*}
  \expect{m} = \expect{\frac{1}{n}\sum_{i = 1}^n x_i} =
  \frac{1}{n}\sum_{i = 1}^n \expect{x_i} = \frac{1}{n}\sum_{i = 1}^n \mu =
  \frac{n\mu}{n} = \mu.
\end{align*}
Per completezza, un estimatore che soddisfa questa proprietà (cioè il
cui valore di aspettazione coincida con la grandezza da stimare) si dice
\emph{imparziale}.

La questione della varianza è leggermente più complicata. Formalmente, se
conoscessimo a priori il valore $\mu$ della media della distribuzione
generatrice, l'analogo della~\eqref{eq:media_campione} sarebbe
\begin{align*}
  s^2 = \frac{1}{n}\sum_{i = 1}^n (x_i - \mu)^2,
\end{align*}
il cui valore di aspettazione, come prima, coincide con la quantità che
vogliamo stimare ($s$ così scritto sarebbe, cioè, imparziale):
\begin{align*}
  \expect{s^2} = \expect{\frac{1}{n}\sum_{i = 1}^n (x_i - \mu)^2} =
  \frac{1}{n}\sum_{i = 1}^n\expect{(x_i - \mu)^2} =
  \frac{1}{n}\sum_{i = 1}^n \sigma^2 = \frac{1}{n} \times n\sigma^2 = \sigma^2.
\end{align*}
Il problema è che in generale non conosciamo $\mu$---abbiamo solo la stima
$m$ data dalla~\eqref{eq:media_campione}. Come miglior stima della varianza
della distribuzione generatrice potremmo allora essere tentati di prendere
\begin{align}\label{eq:varianza_campione}
  s_n^2 = \frac{1}{n}\sum_{i = 1}^n (x_i - m)^2
\end{align}
E qui arriva la parte più interessante. La prima cosa che possiamo chiederci
è se il valore di aspettazione della~\eqref{eq:varianza_campione} sia ancora
pari a $\sigma^2$. Come vedremo tra un attimo la risposta è no, ed un modo
immediato per rendersene conto è che, detta $\xi$ una generica stima della
media, la derivata
\begin{align*}
  \td{s_n^2}{\xi}{} =
  \td{}{\xi}{} \left[\frac{1}{n}\sum_{i = 1}^n (x_i - \xi)^2\right] =
  -\frac{2}{n}\sum_{i = 1}^n (x_i - \xi) =
  -2 \left[ \frac{1}{n} \sum_{i = 1}^n x_i - \frac{1}{n} \times n\xi\right] =
  -2 (m - \xi)
\end{align*}
si annulla per $\xi = m$---cioè la~\eqref{eq:media_campione} è la stima
della media che \emph{minimizza} la stima della varianza
campione~\eqref{eq:varianza_campione}. Questo ci dice che, in media,
$s_n^2$ è una sottostima di $\sigma^2$.
Il valore di aspettazione di $s^2_n$ si calcola esplicitamente come
\begin{align*}
  \expect{s^2_n} = \expect{\frac{1}{n}\sum_{i = 1}^n (x_i - m)^2} =
  \frac{1}{n}\sum_{i = 1}^n \expect{x_i^2 + m^2 - 2mx_i}
\end{align*}
e, dato che i valori di aspettazione all'interno dell'ultima sommatoria sono
identici poiché gli $x_i$ hanno tutti la stessa distribuzione generatrice,
basta in effetti calcolarne uno solo.
Consideriamo dunque i pezzi uno alla volta; il primo si calcola facilmente come
\begin{align*}
  \expect{x_i^2} = \sigma^2 + \mu^2.
\end{align*}
Il secondo è più complicato
\begin{align*}
  \expect{m^2} & =
  \expect{\frac{1}{n^2}\sum_{j = 1}^n x_j\sum_{k = 1}^n x_k} =
  \frac{1}{n^2}\expect{\sum_{j,k = 1}^n x_jx_k} =
  \frac{1}{n^2}\expect{\sum_{j = 1}^nx_j^2 + \sum_{j = 1}^n\sum_{k \neq j}x_jx_k} =\\
  & = \frac{1}{n^2}\left( \sum_{j = 1}^n\expect{x_j^2} +
  \sum_{j = 1}^n\sum_{k \neq j} \expect{x_jx_k} \right) =
  \frac{1}{n^2}\left(n(\sigma^2 + \mu^2) +
  \sum_{j = 1}^n\sum_{k \neq j}\expect{x_j}\expect{x_k} \right) =\\
  & = \frac{1}{n^2}\left[ n(\sigma^2 + \mu^2) + n(n - 1)\mu^2 \right] =
  \frac{1}{n}\sigma^2 + \mu^2.
\end{align*}
(Abbiamo spezzato la somma doppia in due termini, a seconda del fatto che gli
indici $j$ e $k$ siano uguali o diversi. Nel primo caso si ha sostanzialmente
il valore di aspettazione di $x_j^2$, mentre nel secondo, essendo le misure
indipendenti, il valore di aspettazione di $x_jx_k$ si può fattorizzare come
prodotto di valori di aspettazione delle singole variabili.) L'ultimo pezzo
vale
\begin{align*}
  \expect{mx_i} = \expect{\frac{1}{n}\sum_{j = 1}^nx_jx_i} =
  \frac{1}{n} \expect{x_i^2 + \sum_{j \neq i} x_ix_j} =
  \frac{1}{n} \left( \sigma^2 + \mu^2 + (n - 1)\mu^2 \right) =
  \frac{1}{n}\sigma^2 + \mu^2.
\end{align*}
Vale la pena notare come nessuno dei tre valori di aspettazione che abbiamo
appena calcolato dipenda dall'indice $i$, per cui in effetti siamo di fronte
ad una sommatoria di $n$ termini identici---in altre parole: il segno di
sommatoria si elide con il termine moltiplicativo $\nicefrac{1}{n}$ e mettendo
tutto insieme si ottiene il risultato
\begin{align*}
  \expect{s^2_n} = \sigma^2 + \mu^2 + \frac{1}{n}\sigma^2 + \mu^2
  - \frac{2}{n}\sigma^2 - 2\mu^2 = \sigma^2 - \frac{1}{n}\sigma^2 =
  \frac{(n - 1)}{n}\sigma^2.
\end{align*}
La varianza campione, scritta come la~\eqref{eq:varianza_campione}, non
è imparziale se non asintoticamente, nel senso che il suo valore di
aspettazione, in generale, non è uguale alla varianza della distribuzione
generatrice se non per $n \rightarrow \infty$. Fisicamente questo dipende dal
fatto che le $n$ quantità nella somma~\eqref{eq:varianza_campione} non sono
tutte indipendenti---sono legate dalla stima della media $m$ e, se conosco
i primi $n - 1$ termini ed $m$, posso banalmente calcolare il termine
$n$-esimo. (Tecnicamente si dice che il fatto di utilizzare $m$ al posto di
$\mu$ causa la perdita di un grado di libertà.)

Si può ottenere una stima imparziale della varianza della distribuzione
generatrice semplicemente moltiplicando per il fattore correttivo
$\nicefrac{n}{(n - 1)}$
\begin{align}\label{eq:varianza_campione_imparziale}
  s^2_{n - 1} = \frac{1}{(n - 1)}\sum_{i = 1}^n (x_i - m)^2
  \quad \text{che soddisfa banalmente} \quad
  \expect{s^2_{n - 1}} = \sigma^2.
\end{align}
Per $n$ grande le due definizioni coincidono, ma in alcuni testi la proprietà
di imparzialità è considerata così importante che la varianza del campione
è \emph{definita} secondo la~\eqref{eq:varianza_campione_imparziale}. Noi
utilizzeremo questa definizione, ma mantenendo una posizione pragmatica e non
ideologica, come discusso brevemente nella prossima sezione.


\subsection{Digressione: l'imparzialità è davvero irrinunciabile?}
\label{sec:imparzialita_var_campione}

\danger%
A questo punto la domanda è d'obbligo: è più "giusta"
la~\eqref{eq:varianza_campione} o la~\eqref{eq:varianza_campione_imparziale}?
Nella maggioranza dei casi si tratta di un problema più formale che
sostanziale---il rapporto tra le due tende a $1$ per $n$ grande
\begin{align*}
  \lim_{n \rightarrow \infty} \frac{s_n^2}{s_{n-1}^2} =
  \lim_{n \rightarrow \infty} \frac{(n - 1)}{n} = 1
\end{align*}
e la differenza numerica tra $s_n^2$ e $s_{n-1}^2$ è entro il $10\%$ per
$n > 10$. Per piccoli campioni, però, la domanda è rilevante, e la
diversa distribuzione delle due stime è mostrata in
figura~\ref{fig:bias_varianza_campione} per $100\,000$ campioni di prova con
$n = 8$ elementi estratti da una distribuzione generatrice di varianza (nota)
$\sigma^2 = 1$.

\pgffigone{bias_varianza_campione}{
  Varianza campione, calcolata secondo le~\eqref{eq:varianza_campione}
  e~\eqref{eq:varianza_campione_imparziale} per $100\,000$ campioni di $n = 8$
  elementi estratti da una distribuzione generatrice di varianza $\sigma^2 = 1$
  (indicata della linea verticale tratteggiata).
  La stima imparziale è mostrata in grigio, mentre quella "non corretta"
  è in nero. Entrambe le distribuzioni sono evidentemente asimmetriche;
  la media di $s_{n-1}^2$ coincide con la varianza $\sigma^2$ della distribuzione
  generatrice, ma allo stesso tempo la coda a destra è pi\'u pronunciata.
}

Notiamo innanzitutto che le due distribuzioni sono asimmetriche ed in entrambi i
casi la moda (cioè il valore più probabile) è significativamente diversa
dal valore vero di $\sigma^2$ indicato dalla linea verticale tratteggiata.
La stima imparziale $s_{n-1}^2$ ha (come abbiamo visto) media $\sigma^2$, ma il
fattore correttivo che garantisce l'imparzialità ha l'effetto collaterale di
enfatizzare la coda a destra della distribuzione---cioè di aumentare la
frequenza relativa dei valori più distanti da $\sigma^2$. Il risultato è
che, in questo caso particolare, tra le $100\,000$ realizzazioni
mostrate in figura~\ref{fig:bias_varianza_campione} la stima non corretta
$s_n^2$ della varianza è più vicina al valore vero $\sigma^2$ rispetto
alla stima imparziale $s_{n-1}^2$ in circa il $62\%$ dei casi. Siamo cioè nella
situazione interessante in cui una stima è \emph{giusta in media} e
l'altra è più vicina al valore vero nella maggior parte dei casi.
(Aggiungiamo, per completezza, che il fatto che $s_{n-1}^2$ sia una stima
imparziale di $\sigma^2$ non implica di per sé che la sua radice $s_{n-1}$ sia
una stima imparziale di $\sigma$.)

\`E tempo di passare oltre---abbiamo discusso il problema abbastanza a lungo e,
come detto, si tratta tutto sommato di una questione inessenziale. Utilizziamo
pure la~\eqref{eq:varianza_campione_imparziale} ma teniamo a mente il fatto
che essa non ha niente di fondamentalmente più giusto
della~\eqref{eq:varianza_campione}.


\subsection{La deviazione standard della media}
\label{sec:deviazione_standard_media}

Ricapitoliamo ciò che abbiamo imparato fino a questo momento. Abbiamo
modellizzato il problema della misura di una grandezza fisica in condizioni di
ripetitività come il campionamento di una variabile casuale caratterizzata da
una funzione di distribuzione (in generale ignota) che chiamiamo distribuzione
generatrice, e abbiamo derivato le relazioni fondamentali per la migliore stima
della media e della varianza di questa distribuzione generatrice a partire da
un campione finito $x_1 \ldots x_n$ di misure. La domanda ovvia adesso è:
qual è la relazione tra queste stime ed il modo in cui scriviamo il risultato
di una misura?

La media campione~\eqref{eq:media_campione} è il candidato ovvio per il
valore centrale della misura---questa è la parte semplice. Per l'incertezza
associata potremmo essere tentati di prendere la varianza
campione~\eqref{eq:varianza_campione_imparziale}, ma qui le cose si fanno più
complicate, perché la varianza campione è rappresentativa delle fluttuazioni
della \emph{singola misura}, mentre ciò a cui siamo interessati sono le
fluttuazioni del valore centrale, cioè della media. E, intuitivamente, è
lecito aspettarsi che le fluttuazioni della media di $n$ misure siano più
piccole di quelle sulla singola misura---e tanto più piccole quanto più
grande è $n$. (In fondo, se così non fosse, perché le persone si
affannerebbero a prendere dati il più a lungo possibile?)

Torniamo dunque per un attimo alla media campione~\eqref{eq:media_campione}.
Essendo una media di variabili casuali, $m$ è essa stessa una variabile
casuale. Se siamo interessati alle fluttuazioni di $m$, la domanda giusta, nel
nostro linguaggio, è il valore della deviazione standard di $m$:
\begin{align}
  \var{m} = \var{\frac{1}{n}\sum_{i = 1}^n x_i} =
  \frac{1}{n^2}\sum_{i = 1}^n \var{x_i} = \frac{n\sigma^2}{n^2} =
  \frac{\sigma^2}{n},
\end{align}
da cui la deviazione standard della media $\sigma_m$ è banalmente
\begin{align}\label{eq:stdev_media}
  \sigma_m = \sqrt{\var{m}} = \frac{\sigma}{\sqrt{n}}.
\end{align}

La~\eqref{eq:stdev_media} è un risultato di fondamentale importanza, che ci
dice nel dettaglio come le fluttuazioni della media di $n$ misure si riducono,
al variare di $n$, rispetto alle fluttuazioni sulla singola misura.
Possiamo utilizzare questo risultato, insieme alla stima della varianza
campione, per la stima della varianza e della deviazione standard della media
$s_m$:
\begin{align}\label{eq:std_media_imparziale}
  s^2_m = \frac{1}{n(n - 1)}\sum_{i = 1}^n (x_i - m)^2 \quad \text{e} \quad
  s_m = \sqrt{\frac{1}{n(n - 1)}\sum_{i = 1}^n (x_i - m)^2}.
\end{align}
Questo risponde alla nostra domanda iniziale, e con il nostro nuovo bagaglio di
conoscenze~\eqref{eq:media_campione} e \eqref{eq:varianza_campione_imparziale}
possiamo riscrivere la~\eqref{eq:errore_stat} nella forma che utilizzeremo in
pratica
\begin{align}
  x = m \pm s_m~\text{[unità di misura].}
\end{align}
Vedremo più avanti che per $n$ abbastanza grande il teorema centrale del limite
ci garantisce anche la possibilità di calcolare il livello di confidenza
corrispondente.


\subsection{Un esempio concreto: la statura dei coscritti}

Illustriamo la differenza tra la deviazione standard del campione e la
deviazione standard della media con un esempio concreto.
La tabella~\ref{tab:statura_leva} riporta alcuni dati (anno di nascita, numero
di iscritti, media campione e suddivisione in classi) relativi alla statura
degli iscritti di leva nati negli anni 1972--1980, che l'Istituto Nazionale di
Statistica mette a disposizione insieme a decine di altre serie storiche
interessanti.

\begin{table}[htb!]
  \begin{center}
  \begin{tabular}{lllllllllllll}
    \hline
    Anno di & Iscritti & Statura &
    \multicolumn{8}{c}{Ripartizione percentuale per classi di statura}\\
    \cline{4-11}%\\[0pt]
    nascita & & media & \small$<150$ & \small$150$--$154$ &
    \small$155$--$159$ & \small$160$--$164$ & \small$165$--$169$ &
    \small$170$--$174$ & \small$175$--$179$ & \small$\geq 180$\\
    \hline
    \hline
    1972&$445451$&$173.96$&$0.0$&$0.1$&$1.6$&$6.8$&$18.6$&$29.1$&$25.2$&$18.6$\\
    1973&$440115$&$174.12$&$0.1$&$0.2$&$1.2$&$6.5$&$18.1$&$29.3$&$25.8$&$18.8$\\
    1974&$429159$&$174.18$&$0.1$&$0.2$&$1.1$&$6.1$&$17.8$&$29.1$&$26.0$&$19.6$\\
    1975&$382065$&$174.29$&$0.1$&$0.2$&$1.1$&$6.0$&$17.4$&$29.5$&$26.0$&$19.7$\\
    1976&$366357$&$174.42$&$0.1$&$0.4$&$1.1$&$6.2$&$17.8$&$28.4$&$26.0$&$20.0$\\
    1977&$324080$&$174.44$&$0.1$&$0.3$&$1.0$&$6.3$&$17.9$&$28.4$&$26.0$&$20.0$\\
    1978&$366804$&$174.45$&$0.1$&$0.3$&$1.0$&$6.1$&$18.0$&$28.5$&$26.0$&$20.0$\\
    1979&$368336$&$174.49$&$0.1$&$0.3$&$1.0$&$6.1$&$18.0$&$28.3$&$26.1$&$20.1$\\
    1980&$369310$&$174.58$&$0.1$&$0.3$&$1.0$&$6.1$&$18.0$&$28.3$&$26.1$&$20.1$\\
    \hline
  \end{tabular}
  \end{center}
  \caption{Statura media e ripartizione percentuale per classi di statura
    degli iscritti di leva nati negli anni 1972--1980 (adattato dalle
    \href{http://seriestoriche.istat.it/}{serie storiche} dell'Istituto
    Nazionale di Statistica). Tutte le stature sono misurate in~cm.}
  \label{tab:statura_leva}
\end{table}

\pgffigone{stature_leva_classi_1980}{
  Ripartizione percentuale per classi di statura dei nati nel 1980, in base ai
  dati della tabella~\ref{tab:statura_leva}. La classe più a destra include
  tutti i coscritti più alti di $180$~cm, il che rende difficile stimare
  la varianza del campione. La linea grigia tratteggiata rappresenta, a scopo
  puramente illustrativo, una possibile distribuzione generatrice.
}

La ripartizione percentuale è interessante perché ci dà un'idea diretta
della forma della distribuzione generatrice delle stature. Purtroppo la
definizione degli intervalli per la ripartizione in classi è rimasta
invariata dal 1854 e, a causa dell'aumento della statura media avvenuta
negli ultimi $120$~anni, non permette di campionare adeguatamente per anni
recenti la distribuzione generatrice al di sopra di $180$~cm (ove abbiamo
una sola classe). Questo rende difficile valutarne con precisione la
deviazione standard, ma sulla base del fatto che la maggior parte delle
misure cadono entro quattro intervalli (di $5$~cm) attorno al valor medio
possiamo dire rozzamente che essa sarà dell'ordine di $\sim 10$~cm.
\emph{Questo è il numero rappresentativo delle fluttuazioni intorno
  al valor medio per la singola misura.} Se prendiamo una persona a caso
nata negli anni settanta ci aspettiamo che la sua statura sia $174 \pm 10$~cm
(a patto che la nostra stima ad occhio della deviazione standard possa essere
considerata affidabile).

Tutt'altro discorso vale per la media delle stature su un campione di
persone. La terza colonna della tabella indica che le stature medie
fluttuano negli anni molto meno di questi $10$~cm. Con circa $400\,000$
misure ci aspettiamo che la deviazione standard della media sia circa
$630$~volte più piccola di quella del campione---nel nostro caso
specifico dell'ordine di $\sim 0.2$~mm.
\emph{Questo è il numero rappresentativo delle fluttuazioni intorno al
  valor medio per la media di circa $400\,000$ misure.}
In effetti, a guardare bene la tabella, la media del campione cresce
sistematicamente con il tempo---e di circa $1$~mm l'anno, come mostrato in
figura~\ref{fig:stature_leva_trend}. Il fatto che questa crescita si veda così
chiaramente implica che le fluttuazioni statistiche sul singolo punto
debbono essere $\ll 1$~mm, che è compatibile con la nostra stima grossolana
di $0.2$~mm.

\pgffigone{stature_leva_trend}{
  Andamento temporale della statura media dei coscritti in funzione dell'anno
  di nascita, in base ai dati della tabella~\ref{tab:statura_leva}. Le barre
  d'errore corrispondono alla deviazione standard della media per un campione
  di $400\,000$ unità, che abbiamo stimato rozzamente essere dell'ordine di
  $0.2$~mm. Si vede chiaramente un aumento significativo della statura in
  funzione del tempo dell'ordine di $\sim 1$~mm l'anno, il che implica che le
  fluttuazioni statistiche dei singoli punto debbono essere $\ll 1$~mm.
}


\section{Calcolo della media e varianza campione in pratica}

Nella vita reale, ogni volta che vi trovate a dover calcolare la media e la
deviazione standard di un campione (o la deviazione standard della media),
tenete a mente questa semplice regola: non cedete alla tentazione di re-inventare
la ruota, ed usate invece quello che il vostro \foreign{software} di analisi
preferito mette a disposizione---ad esempio le funzioni \npfunc{mean} e
\npfunc{std} di \numpy\ descritte in appendice~\ref{sec:numpy}.
Il frammento di codice~\ref{snip:sample_stat_numpy} mostra un semplice esempio
di calcolo della statistica di base di un piccolo campione, che potete usare
come ispirazione ed adattare alle vostre esigenze nei casi più semplici.

\snip{sample_stat_numpy}{
  Calcolo della media e della deviazione standard campione utilizzando le
  funzioni di \numpy\ descritte in appendice~\ref{sec:numpy}. Si noti l'uso
  dell'argomento \texttt{ddof=1} per ottenere l'estimatore imparziale $s_{n-1}$,
  in assenza del quale la funzione restituirebbe $s_n$.}

Un aspetto interessante delle funzioni statistiche di \numpy, su cui non
insisteremo troppo, ma che illustriamo brevemente data la sua rilevanza pratica,
è il supporto che esse offrono alla \emph{vettorizzazione} dei problemi.
Spesso, in laboratorio, ci troviamo a dover eseguire un certo numero di serie di
misure prese in condizioni diverse---ad esempio misure ripetute del periodo di
un pendolo in corrispondenza di diversi valori della lunghezza---e dover calcolare
media e deviazione standard separatamente per ciascuna serie. In situazioni di
questo tipo processare le serie di misure una alla volta utilizzando l'approccio
illustrato nel frammento~\ref{snip:sample_stat_numpy} diventa macchinoso molto
velocemente, ed è invece più comodo e naturale organizzare i nostri dati in
forma matriciale, utilizzando \numpy\ per calcolare le grandezze statistiche
desiderate per righe o per colonne, come illustrato nel
frammento~\ref{snip:sample_stat_numpy_vec}

\snip{sample_stat_numpy_vec}{
  Calcolo della media e della deviazione standard campione per quattro serie
  di misure distinte---che potrebbero rappresentare, ad esempio, misure ripetute
  del periodo di un pendolo per 4~valori diversi della lunghezza del pendolo
  stesso. Organizzando i dati in modo opportuno, ovverosia come una matrice
  $4 \times 8$ in cui ciascuna riga corrisponde ad un valore specifico di lunghezza,
  si possono calcolare le medie e le deviazioni standard riga per riga utilizzando
  l'argomento \texttt{axis=1} delle funzioni statistiche di \numpy.}


\subsection{Algoritmi di calcolo a ciclo singolo}

\danger%
Il modo più immediato per calcolare la media e la deviazione standard di un
campione di dati è quello di eseguire due cicli separati sui dati stessi---il
primo per calcolare la media campione $m$ secondo la~\eqref{eq:media_campione}
ed il secondo per calcolare la varianza campione $s_{n-1}^2$, una volta nota $m$.
Anche nei due frammenti~\ref{snip:sample_stat_numpy} e \ref{snip:sample_stat_numpy_vec}
che abbiamo appena visto, sebbene i cicli siano \emph{nascosti} dentro le funzioni
di \numpy\, la chiamata a \npfunc{std} richiede internamente il calcolo della media
campione, da utilizzare nella~\eqref{eq:varianza_campione_imparziale}---in
pratica stiamo calcolando la media campione due volte!

Anche se, in situazioni semplici come quelle che abbiamo analizzato, la ripetizione
del calcolo della media campione non ha tipicamente conseguenze pratiche misurabili,
esiste una classe di algoritmi che permette di calcolare media e deviazione
standard di un campione all'interno dello stesso ciclo---cioè in un passo.
Questi algoritmi sono utili non solo perché computazionalmente più efficienti,
ma anche (e soprattutto) perché si prestano naturalmente ad aggiornare la stima
delle nostre statistiche ogni volta che aggiungiamo un elemento al campione, senza
dovere riprocessare tutti i valori precedenti, e senza dover conservare in
memoria il campione intero---cosa che è chiaramente desiderabile, e.g., nelle
applicazioni in tempo reale.

Con qualche semplice manipolazione algebrica la formula per la varianza
campione~\eqref{eq:varianza_campione_imparziale} può essere trasformata in
\begin{align*}
  s_{n - 1}^2 & = \frac{1}{(n - 1)}\sum_{i = 1}^n (x_i - m)^2 =
  \frac{1}{(n - 1)}\sum_{i = 1}^n(x_i^2 + m^2 - 2mx_i) =
  \frac{1}{(n - 1)}\left( \sum_{i = 1}^nx_i^2 + nm^2 - 2m\sum_{i = 1}^nx_i\right)=\\
  & = \frac{1}{(n - 1)}\left( \sum_{i = 1}^nx_i^2 + nm^2 - 2nm^2 \right) =
  \frac{1}{(n - 1)}\left( \sum_{i = 1}^nx_i^2 - nm^2 \right) =
  \frac{1}{(n - 1)}\sum_{i = 1}^nx_i^2 - \frac{n}{(n - 1)}m^2,
\end{align*}
che è l'equivalente della~\eqref{eq:variance_alt} nel contesto della teoria
dei campioni. Questa formulazione alternativa ci permette, come abbiamo detto,
di eseguire il calcolo della media e della deviazione standard con un solo
ciclo in cui accumuliamo in parallelo le somme parziali di $x_i$ e $x_i^2$.

\snip{sample_stat_single}{
  Calcolo della media e della deviazione standard del campione con un ciclo
  singlo, in cui accumuliamo le somme di $x_i$ e $x_i^2$. Se applicata sullo
  stesso campione del frammento~\ref{snip:sample_stat_numpy}, la funzione
  \pyfunc{sample_stats} resitusce lo stesso risultato, entro l'accuratezza
  intrinseca dell'aritmetica in virgola mobile. Se però la media del campione
  in ingresso è molto più grande della sua deviazione standard, in generale
  questo algoritmo richiede la sottrazione di numeri potenzialmente molto grandi
  e vicini tra loro e diventa numericamente instabile---quando gli $x_i$ sono
  molto grandi l'accumulazione di $x_i^2$ è problematica.}

Il frammento~\ref{snip:sample_stat_single} mostra la più semplice implementazione
possibile di questa idea, nella forma di un semplice ciclo \emph{for} in \python,
ed effettivamente fornisce la risposta corretta se applicato sul campione
di test utilizzato nel frammento~\ref{snip:sample_stat_numpy}. Se però adesso
aggiungiamo a tutti gli elementi del nostro campione un numero grande ($10^9$
nel caso in questione) succede una cosa inattesa e sorprendente: la deviazione
standard, che dovrebbe rimanere rigorosamente invariata, diventa molto più
grande---e dunque, chiaramente, il risultato è sbagliato.
Come è possibile? Il problema è una combinazione dei due fatti: (i) poiché gli
$x_i$ sono grandi, la somma parziale $\sum x_i^2$ tende a crescere velocemente;
e (ii) poiché la deviazione standard del campione è piccola, $\sum x_i^2$ e $(\sum x_i)^2$
non sono molto diversi tra di loro---e quando sottraiamo due numeri molto grandi
vicini tra loro, l'errore numerico può diventare arbitrariamente grande.
Questa fenomeno mette in luce come questo tipo di algoritmi sia potenzialmente
suscettibile a instabilità numerica, e richieda una certa attenzione in fase di
implementazione. (In altre parole abbiamo imparato che, a causa della natura
intrinsecamente inesatta dell'aritmetica in virgola mobile, due espressioni
esattamente equivalenti sulla carta, possono fornire risultati diversi se calcolate
su un computer. Il lettore interessato è incoraggiato a procedere con la lettura dell'appendice~\ref{sec:sistema_binario}.)

\snip{sample_stat_welford}{
  Calcolo della media e della deviazione standard campione secondo l'algoritmo
  di Welford~\cite{welford}, che permette di utilizzare un ciclo singolo e,
  allo stesso tempo, è numericamente stabile.}

A questo punto possiamo chiederci se ci sia una soluzione al problema che abbiamo
appena incontrato. La risposta è in un algoritmo dovuto a Welford~\cite{welford}
e descritto in dettaglio in~\cite{taocp2}, di cui mostriamo una semplice implementazione nel
frammento~\ref{snip:sample_stat_welford}. (Si tratta di un'idea tanto semplice
quanto ingegnosa, ed il lettore è senz'altro incoraggiato a consultare l'articolo
originale.)
Rimandiamo il lettore alle referenze~\cite{welford,taocp2} per una dimostrazione
formale della correttezza dell'algoritmo e ci limitiamo a notare come
esso funzioni (a parte l'usuale rumore numerico) anche nel caso in cui
quello mostrato nel frammento~\ref{snip:sample_stat_single} fallisce miseramente.


\section{Covarianza e correlazione campione}
\label{sec:covarianza_campione}

Supponiamo di avere una serie di campionamenti $x_i$ ed $y_i$, con
$i = 1\ldots n$ di due variabili casuali $x$ ed $y$, ovverosia una serie di
$n$ coppie ordinate $(x_i, y_i)$. Se diciamo $m_x$ ed $m_y$ le rispettive
medie campione
\begin{align*}
  m_x = \frac{1}{n}\sum_{i = 1}^n x_i \quad \text{e} \quad
  m_y = \frac{1}{n}\sum_{i = 1}^n y_i
\end{align*}
possiamo, analogamente a quanto abbiamo fatto in precedenza con la varianza,
stimare la covarianza $\cov{x}{y}$ delle variabili di partenza attraverso la
covarianza del campione
\begin{align}\label{eq:covarianza_campione}
  q_{xy} = \frac{1}{(n - 1)} \sum_{i = 1}^n (x_i - m_x)(y_i - m_y).
\end{align}
Abbiamo sviscerato ampiamente la questione del $(n - 1)$ nella
sezione~\ref{sec:imparzialita_var_campione} e non insisteremo oltre, così come
non insisteremo oltre sull'efficienza e la stabilità degli algoritmi
a disposizione per il calcolo della~\eqref{eq:covarianza_campione} in pratica.

La stima della correlazione $r_{xy}$ (o semplicemente $r$) del campione si
scrive banalmente in termini delle stime $s_x$ ed $s_y$ delle deviazioni
standard di $x$ ed $y$
\begin{align*}
  s^2_x = \frac{1}{(n - 1)}\sum_{i = 1}^n (x_i - m_x)^2 \quad \text{e} \quad
  s^2_y = \frac{1}{(n - 1)}\sum_{i = 1}^n (y_i - m_y)^2
\end{align*}
come
\begin{align}\label{eq:coeff_correlazione}
  r_{xy} = \frac{q_{xy}}{s_x s_y} =
  \frac{\sum_{i = 1}^n (x_i - m_x)(y_i - m_y)}%
  {\sqrt{\sum_{i = 1}^n (x_i - m_x)^2 \sum_{i = 1}^n (y_i - m_y)^2}}
\end{align}
che è talvolta chiamato anche \emph{coefficiente di correlazione lineare} o
\emph{indice di correlazione di Pearson} ed è molto utilizzato in pratica.

\pgffigone{chiodini_scatter}{
  Grafico di dispersione della massa $m$ in funzione della lunghezza $l$ per un
  campione di $70$ chiodini (le misure sono state realmente effettuate il $5$
  agosto 2005 da uno degli autori). Le misure di lunghezza sono state effettuate
  con un calibro Palmer (risoluzione $0.01$~mm) e quelle di massa con una
  bilancia di precisione (risoluzione $1$~mg), per cui le incertezze di
  misura sono molto più piccole della dispersione delle due quantità.
  \`E chiaro che a valori piccoli (grandi) della lunghezza tendono ad essere
  associati valori piccoli (grandi) della massa, per cui le due quantità sono
  correlate. L'indice di correlazione di Pearson, calcolato secondo
  la~\eqref{eq:coeff_correlazione}, è $0.85$.
}

Uno dei modi più semplici per individuare una possibile correlazione tra due
variabili in un campione è quello di realizzare un grafico di dispersione,
come mostrato in figura~\ref{fig:chiodini_scatter}. Se valori piccoli di una
variabile tendono ad essere associati a valori piccoli (o grandi) dell'altra, e
viceversa, questo è indice di una correlazione positiva (o negativa). Se le
fluttuazioni attorno al valor medio delle due quantità appaiono indipendenti,
cioè la distribuzione dei punti nel diagramma di dispersione non è elongata
in una direzione privilegiata, allora le quantità sono \emph{scorrelate}.
L'indice di Pearson indica l'entità della correlazione ed è $\pm1$ se e solo
se le due grandezze nel grafico di dispersione stanno esattamente su una retta.


\subsection{Correlazioni e rapporti di causa effetto}

Per quanto la tentazione di fare il collegamento possa essere forte, non si
può sottolineare abbastanza il fatto che \emph{una correlazione tra due
  grandezze non implica necessariamente un rapporto di causa-effetto tra le
  grandezze stesse.} In altre parole, se $x$ ed $y$ sono correlate non è
detto che $x$ causi $y$ o $y$ causi $x$. Lo abbiamo visto in concreto nel caso
dei chiodini di figura~\ref{fig:chiodini_scatter}, in cui non diremmo
che la massa causa la lunghezza o viceversa, e la correlazione, più
prosaicamente, è determinata da un fattore fisico (il fatto che i chiodi siano
fatti tutti dello stesso materiale) che influenza entrambe le grandezze sotto
studio. Negli studi clinici e sociologici, tanto per fare un esempio, spesso
non è semplice fattorizzare l'effetto che determinate concause ambientali
possono avere sulle grandezze oggetto di studio, e questo costituisce una delle
difficoltà maggiori nel passaggio delicato dall'osservazione di una
correlazione allo stabilire una relazione di causa ed effetto.

\pgffigone{miss_america}{
  Grafico di dispersione del numero di omicidi per anno commessi negli Stati
  Uniti utilizzando vapore o oggetti caldi in funzione dell'età di Miss
  America (nello stesso anno)---per gli anni 1999--2009 (adattato da
  \url{http://tylervigen.com/spurious-correlations}). L'indice di correlazione
  di Pearson è $0.87$ ma in questo caso è ovvio che la correlazione è
  puramente casuale e nessuno direbbe che l'età di Miss America sia
  legata causalmente con il numero di omicidi.
}

La materia è complessa e non abbiamo il tempo di sviscerarla in dettaglio.
La figura~\ref{fig:miss_america} è un buon modo per convincersi del punto
principale di questa sezione---correlazione e causalità non sono la stessa
cosa---e che, a cercare bene è sorprendentemente facile trovare
correlazioni tra cose che non hanno niente a che vedere l'una con l'altra.
Se non sapevate che il numero di persone morte per affogamento in piscina per
anno negli Stati Uniti è positivamente correlato con il numero di film
(dello stesso anno) in cui Nicolas Cage appare,provate a dare un'occhiata a
\url{http://tylervigen.com/spurious-correlations}.


\section{Media e varianza di una funzione di variabili casuali}
\label{sec:media_var_funzioni}

Riprendiamo adesso il filo del discorso. Nella sezione~\ref{sec:somma_variabili_indipendenti}
abbiamo imparato a \emph{sommare} le distribuzioni. Prima di tornare di nuovo
sulla propagazione degli errori ci rimane ancora un ulteriore piccolo passo da
fare. La domanda che ci facciamo in questa sezione è la seguente: come possiamo
stimare, anche approssimativamente, la media e la varianza di una funzione generica
di variabili casuali?


\subsection{Il caso uni-dimensionale}

Consideriamo dunque una variabile casuale $x$, di cui assumiamo di sapere la
funzione di distribuzione---e sia $\mu$ il valor medio di $x$ e $\sigma$ la
sua deviazione standard. Come detto all'inizio della sezione, il nostro scopo
è quello di stimare la media $\mu_f$ e la varianza $\sigma^2_f$ di una
generica funzione $f(x)$.
Poiché sappiamo che i valori di $x$ tendono a \emph{concentrarsi} attorno al
valor medio, possiamo partire dallo sviluppo in serie di Taylor di $f(x)$
attorno a~$\mu$ troncato al prim'ordine:
\begin{align*}
  f(x) \approx f(\mu) + \td{f}{x}{\mu}(x - \mu)
  \quad \text{ovvero} \quad
  f(x) - f(\mu) \approx \td{f}{x}{\mu}(x - \mu).
\end{align*}

La media si calcola formalmente come il valore di aspettazione $\expect{f(x)}$,
che sviluppato al prim'ordine diviene
\begin{align*}
  \mu_f = \expect{f(x)} \approx \expect{f(\mu) +
    \td{f}{x}{\mu}(x - \mu)} =
  \expect{f(\mu)} + \td{f}{x}{\mu}\expect{x - \mu}.
\end{align*}
(Ricordate: $f(\mu)$ è una costante che può essere portata fuori dal
valore di aspettazione, e così pure la derivata---sono entrambe calcolate in
un punto e dunque non dipendono da $x$.) A questo punto, notando che
$\expect{x - \mu}$ si annulla banalmente per la linearità del valore di
aspettazione, si ottiene
\begin{align}\label{eq:media_f_unidim}
  \mu_f \approx f(\mu).
\end{align}
Il valor medio di $f(x)$, cioè, può essere approssimato con il valore della
funzione $f(x)$ calcolata nel valor medio di $x$, a meno di termini del
second'ordine in $(x - \mu)$. Che il termine al prim'ordine dello sviluppo si
annulli è un fatto importante, che ci dice che la~\eqref{eq:media_f_unidim}
vale \emph{esattamente} se $f(x)$ è una funzione lineare di $x$.

Per la varianza possiamo procedere analogamente, utilizzando il nostro
sviluppo in serie di $f(x)$ ed il risultato che abbiamo appena ottenuto per la
media
\begin{align*}
  \sigma^2_f = \expect{\left(f(x) - \mu_f\right)^2} \approx
  \expect{\left(f(x) - f(\mu)\right)^2} \approx
  \expect{\left(\td{f}{x}{\mu}(x - \mu)\right)^2} =
  \left(\td{f}{x}{\mu}\right)^2\expect{(x - \mu)^2},
\end{align*}
da cui
\begin{align}\label{eq:sigma_f_unidim}
  \sigma^2_f \approx \left(\td{f}{x}{\mu}\right)^2\sigma^2.
\end{align}


\subsection{Il caso generale}

Nel caso multidimensionale (quando, cioè, si ha una funzione
$f(x_1, x_2, x_3\ldots)$ di un certo numero di variabili casuali $x_1$,  $x_2$,
$x_3$\ldots con medie $\mu_1$, $\mu_2$, $\mu_3$\dots e varianze
$\sigma^2_1$, $\sigma^2_2$, $\sigma^2_3$\dots)
il linguaggio rimane sostanzialmente lo stesso, ma la situazione diventa più
interessante---e rientra in gioco la questione dell'indipendenza statistica tra
le variabili che abbiamo già affrontato nella
sezione~\ref{sec:somma_variabili_indipendenti} a proposito della somma.

Andiamo per gradi e cominciamo dal caso relativamente semplice di una funzione
$f(x_1, x_2)$ di due variabili casuali. Lo sviluppo in serie di Taylor al
prim'ordine è in questo caso
\begin{align*}
  f(x_1, x_2) \approx f(\mu_1, \mu_2) +
  \pd{f}{x_1}{\mu_1, \mu_2}(x_1 - \mu_1) +
  \pd{f}{x_2}{\mu_1, \mu_2}(x_2 - \mu_2).
\end{align*}
La media di $f$ (sviluppata al prim'ordine) sarà, esattamente come prima
\begin{align}\label{eq:media_f2}
  \mu_f = \expect{f(x_1, x_2)} \approx
  f(\mu_1, \mu_2) +
  \pd{f}{x_1}{\mu_1, \mu_2}\expect{x_1 - \mu_1} +
  \pd{f}{x_2}{\mu_1, \mu_2}\expect{x_2 - \mu_2} = f(\mu_1, \mu_2).
\end{align}
Per la varianza il calcolo è leggermente più complicato. Partendo dalla
definizione si ha, ricordando che come nel caso uni-dimensionale le derivate
sono valutate nei valori medi, per cui sono \emph{numeri} che passano al di
fuori dei valori di aspettazione
\begin{align*}
  \sigma^2_f & =
  \expect{\left(f(x_1, x_2) - f(\mu_1, \mu_2)\right)^2} \approx
  \expect{\left(
    \pd{f}{x_1}{\mu_1, \mu_2}(x_1 - \mu_1) +
    \pd{f}{x_2}{\mu_1, \mu_2}(x_2 - \mu_2)
    \right)^2} =\\
  %& =
  %\left(\pd{f}{x_1}{\mu_1, \mu_2}\right)^2 \expect{(x_1 - \mu_1)^2} +
  %\left(\pd{f}{x_2}{\mu_1, \mu_2}\right)^2 \expect{(x_2 - \mu_2)^2} +\\
  %& + 2\pd{f}{x_1}{\mu_1, \mu_2}\pd{f}{x_2}{\mu_1, \mu_2}
  %\expect{(x_1 - \mu_1)(x_2 - \mu_2)} = \\
  & =
  \left(\pd{f}{x_1}{\mu_1, \mu_2}\right)^2 \sigma^2_1 +
  \left(\pd{f}{x_2}{\mu_1, \mu_2}\right)^2 \sigma^2_2 +
  2\pd{f}{x_1}{\mu_1, \mu_2}\pd{f}{x_2}{\mu_1, \mu_2}\,\cov{x_1}{x_2}.
\end{align*}
L'espressione che abbiamo appena ricavato si può riscrivere più
convenientemente utilizzando la definizione di coefficiente di correlazione
lineare come
\begin{align}\label{eq:varianza_f2}
  \sigma^2_f & \approx
  \left(\pd{f}{x_1}{\mu_1, \mu_2}\right)^2 \sigma^2_1 +
  \left(\pd{f}{x_2}{\mu_1, \mu_2}\right)^2 \sigma^2_2 +
  2\pd{f}{x_1}{\mu_1, \mu_2}\pd{f}{x_2}{\mu_1, \mu_2}\,
  \corr{x_1}{x_2} \sigma_1 \sigma_2.
\end{align}

Le~\eqref{eq:media_f2} e~\eqref{eq:varianza_f2} si possono generalizzare al
caso di un numero arbitrario di variabili nella forma compatta
\begin{align}\label{eq:media_varianza_func}
  \mu_f \approx f(\mu_1,\ldots,\mu_n) \quad \text{e} \quad
  \sigma^2_f \approx \sum_{i = 1}^n \sum_{j = 1}^n
  \pd{f}{x_i}{\mu_1, \ldots, \mu_n}\pd{f}{x_j}{\mu_1, \ldots, \mu_n}
  \,\corr{x_i}{x_j} \sigma_i \sigma_j,
\end{align}
che costituisce la formula generale linearizzata per il calcolo della
varianza di una funzione di variabili casuali.

La cosa interessante è che se le variabili casuali sono mutuamente
indipendenti, tutte le correlazioni $\corr{x_i}{x_j}$ con $i \neq j$ si
annullano e la formula per la varianza si semplifica notevolmente: ricordando
che $\corr{x_i}{x_i} = 1$
\begin{align}\label{eq:varianza_func_indipendente}
  \sigma^2_f \approx
  \sum_{i = 1}^{n} \left(\pd{f}{x_i}{\mu_1,\ldots,\mu_n}\right)^2 \sigma^2_i.
\end{align}

Vale la pena sottolineare come tutte le relazioni che abbiamo ricavato in
questa sezione valgono approssimativamente quando i termini di ordine
superiore al primo sono trascurabili ed esattamente nel caso lineare, come
illustrato nell'esempio~\ref{exp:varianza_somma}.

\begin{examplebox}
  \begin{example}\label{exp:varianza_somma}
    Possiamo utilizzare il formalismo sviluppato in questa sezione per ricavare
    le formule per la media e la deviazione standard della somme di due
    variabili casuali indipendenti $x_1$ ed $x_2$---che abbiamo calcolato
    direttamente nella sezione~\ref{sec:media_varianza_somma}.
    Sia dunque $f(x_1, x_2) = x_1 + x_2$. La funzione è lineare in entrambe
    le variabili, nel senso che le derivate parziali di ordine superiore al
    primo rispetto ad entrambe le variabili si annullano---ragion per cui
    la~\eqref{eq:media_varianza_func} è in questo caso esatta. Abbiamo
    \begin{align*}
      \pd{f}{x_1}{x_1, x_2} = \pd{f}{x_2}{x_1, x_2} = 1,
    \end{align*}
    da cui segue direttamente
    \begin{align*}
      \mu_f = f(\mu_1, \mu_2)
      \quad \text{e} \quad
      \sigma_f = \sqrt{\sigma_1^2 + \sigma_2^2},
    \end{align*}
    che sono gli stessi risultati della~\eqref{eq:media_somma_ind} e
    \eqref{eq:varianza_somma_ind}.
  \end{example}
\end{examplebox}


\section{La propagazione dell'errore statistico}
\label{sec:prop_errore_stat}

Siamo pronti per chiudere il cerchio---riprendiamo il filo della nostra
discussione e riesaminiamo, nel linguaggio della probabilità e delle funzioni
di distribuzione, la propagazione dell'errore che abbiamo discusso, in un
contesto diverso, nella sezione~\ref{sec:propagazione_errrore_max}.
Abbiamo visto che, nella teoria dei campioni, l'incertezza di misura ha il
significato di una stima della deviazione standard della distribuzione
generatrice. Visto che sappiamo calcolare, almeno approssimativamente, la
deviazione standard di una funzione arbitraria di variabili casuali, date le
deviazioni standard delle variabili stesse, di fatto abbiamo tutti gli
ingredienti per propagare gli errori in modo statisticamente corretto.

Supponiamo dunque di avere una serie di $n$ grandezze misurate
$x_i = \hat{x}_i \pm \sigma_i$ ed una generica funzione $f(x_1, \ldots, x_n)$,
e partiamo dal caso più semplice---quello cioè in cui le grandezze di
partenza sono tutte indipendenti. Come già sappiamo, la formula generale di
propagazione dell'errore è:
\begin{align}\label{eq:propagazione_errore_stat_ind}
  \sigma^2_f \approx
  \sum_{i = 1}^n
  \left(\pd{f}{x_i}{\hat{x}_1,\ldots,\hat{x}_n}\right)^2 \sigma^2_i
  \quad \text{ovvero} \quad
  \sigma_f \approx \sqrt{\sum_{i = 1}^n
    \left(\pd{f}{x_i}{\hat{x}_1,\ldots,\hat{x}_n}\right)^2 \sigma^2_i.
  }
\end{align}
La~\eqref{eq:propagazione_errore_stat_ind} è formalmente identica alla
\eqref{eq:varianza_func_indipendente} ma le due scritture sono logicamente
diverse, perché in questo contesto i $\sigma_i$ rappresentano in generale le
nostre miglior stime delle deviazioni standard delle distribuzioni generatrici e
non i valori calcolati a partire dalla forma analitica delle funzioni di
distribuzioni stesse. La differenza è sottile ma importante. Il contenuto
fisico fondamentale della~\eqref{eq:propagazione_errore_stat_ind} è che nel
caso di grandezze indipendenti (che è tipico in laboratorio) \emph{gli errori
  statistici non si sommano linearmente come gli errori massimi, ma in
  quadratura}.

Possiamo andare oltre. Se le grandezze non sono indipendenti ed abbiamo delle
stime $r_{ij}$ dei coefficienti di correlazione $\corr{x_i}{x_j}$,
la~\ref{eq:propagazione_errore_stat_ind} si generalizza come
\begin{align}\label{eq:propagazione_errore_stat}
  \sigma^2_f \approx
  \sum_{i = 1}^n \sum_{j = 1}^n
  \pd{f}{x_i}{\hat{x}_1,\ldots,\hat{x}_n}
  \pd{f}{x_j}{\hat{x}_1,\ldots,\hat{x}_n} \, r_{ij}\sigma_i\sigma_j.
\end{align}

\begin{examplebox}
  \begin{example}\label{eq:prop_errore_densita}
    Si misura la massa $m = \hat{m} \pm \sigma_m$ il volume
    $V = \hat{V} \pm \sigma_V$ di un oggetto e si vuole stimare la densità
    $\rho$ dell'oggetto stesso. La propagazione dell'errore nel caso generale si
    scrive come
    \begin{align*}
      \sigma^2_\rho & =
      \left( \pd{\rho}{m}{\hat{m}, \hat{V}} \right)^2 \sigma_m^2 +
      \left( \pd{\rho}{V}{\hat{m}, \hat{V}} \right)^2 \sigma_V^2 +
      2r_{mV} \pd{\rho}{m}{\hat{m}, \hat{V}}
      \pd{\rho}{V}{\hat{m}, \hat{V}}\sigma_m \sigma_V =\\
      & = \frac{\hat{m}^2 \sigma_V^2 + \hat{V}^2 \sigma_m^2 -
        2r_{mV}\hat{m}\hat{V}\sigma_m \sigma_V}{\hat{V}^4}
    \end{align*}
    Se le due misure sono indipendenti $r_{mV}$ è nullo e la formula si
    semplifica in
    \begin{align*}
      \sigma^2_\rho =
      \frac{\hat{m}^2 \sigma_V^2 + \hat{V}^2 \sigma_m^2}{\hat{V}^4}.
    \end{align*}
  \end{example}
\end{examplebox}


\subsection{Un esempio di propagazione degli errori per variabili correlate}

Concludiamo il capitolo con un esempio di propagazione degli errori nel caso di
variabili non indipendenti---e a questo scopo ripartiamo dall'esempio dei chiodini
illustrato in figura~\ref{fig:chiodini_scatter}. La figura~\ref{fig:chiodini_scatter_hist} è
una rappresentazione alternativa del grafico di dispersione che comprende le
proiezioni sui due assi ortogonali---ovvero gli istogrammi uni-dimensionali delle
due variabili in gioco. Questo tipo di rappresentazione riassume tutte le informazioni
su una data coppia di quantità e si trova spesso utilizzata in pratica.

\begin{figure}[!htb]
  \centering%% Creator: Matplotlib, PGF backend
%%
%% To include the figure in your LaTeX document, write
%%   \input{<filename>.pgf}
%%
%% Make sure the required packages are loaded in your preamble
%%   \usepackage{pgf}
%%
%% Also ensure that all the required font packages are loaded; for instance,
%% the lmodern package is sometimes necessary when using math font.
%%   \usepackage{lmodern}
%%
%% Figures using additional raster images can only be included by \input if
%% they are in the same directory as the main LaTeX file. For loading figures
%% from other directories you can use the `import` package
%%   \usepackage{import}
%%
%% and then include the figures with
%%   \import{<path to file>}{<filename>.pgf}
%%
%% Matplotlib used the following preamble
%%   \usepackage[nice]{nicefrac}
%%   \usepackage{amsmath}
%%   \usepackage[utf8]{inputenc}
%%   \DeclareUnicodeCharacter{2212}{\ensuremath{-}}
%%
\begingroup%
\makeatletter%
\begin{pgfpicture}%
\pgfpathrectangle{\pgfpointorigin}{\pgfqpoint{5.104000in}{4.240000in}}%
\pgfusepath{use as bounding box, clip}%
\begin{pgfscope}%
\pgfsetbuttcap%
\pgfsetmiterjoin%
\definecolor{currentfill}{rgb}{1.000000,1.000000,1.000000}%
\pgfsetfillcolor{currentfill}%
\pgfsetlinewidth{0.000000pt}%
\definecolor{currentstroke}{rgb}{1.000000,1.000000,1.000000}%
\pgfsetstrokecolor{currentstroke}%
\pgfsetdash{}{0pt}%
\pgfpathmoveto{\pgfqpoint{0.000000in}{0.000000in}}%
\pgfpathlineto{\pgfqpoint{5.104000in}{0.000000in}}%
\pgfpathlineto{\pgfqpoint{5.104000in}{4.240000in}}%
\pgfpathlineto{\pgfqpoint{0.000000in}{4.240000in}}%
\pgfpathlineto{\pgfqpoint{0.000000in}{0.000000in}}%
\pgfpathclose%
\pgfusepath{fill}%
\end{pgfscope}%
\begin{pgfscope}%
\pgfsetbuttcap%
\pgfsetmiterjoin%
\definecolor{currentfill}{rgb}{1.000000,1.000000,1.000000}%
\pgfsetfillcolor{currentfill}%
\pgfsetlinewidth{0.000000pt}%
\definecolor{currentstroke}{rgb}{0.000000,0.000000,0.000000}%
\pgfsetstrokecolor{currentstroke}%
\pgfsetstrokeopacity{0.000000}%
\pgfsetdash{}{0pt}%
\pgfpathmoveto{\pgfqpoint{0.622050in}{0.424000in}}%
\pgfpathlineto{\pgfqpoint{3.030500in}{0.424000in}}%
\pgfpathlineto{\pgfqpoint{3.030500in}{2.544000in}}%
\pgfpathlineto{\pgfqpoint{0.622050in}{2.544000in}}%
\pgfpathlineto{\pgfqpoint{0.622050in}{0.424000in}}%
\pgfpathclose%
\pgfusepath{fill}%
\end{pgfscope}%
\begin{pgfscope}%
\pgfsetbuttcap%
\pgfsetroundjoin%
\definecolor{currentfill}{rgb}{0.000000,0.000000,0.000000}%
\pgfsetfillcolor{currentfill}%
\pgfsetlinewidth{0.803000pt}%
\definecolor{currentstroke}{rgb}{0.000000,0.000000,0.000000}%
\pgfsetstrokecolor{currentstroke}%
\pgfsetdash{}{0pt}%
\pgfsys@defobject{currentmarker}{\pgfqpoint{0.000000in}{-0.048611in}}{\pgfqpoint{0.000000in}{0.000000in}}{%
\pgfpathmoveto{\pgfqpoint{0.000000in}{0.000000in}}%
\pgfpathlineto{\pgfqpoint{0.000000in}{-0.048611in}}%
\pgfusepath{stroke,fill}%
}%
\begin{pgfscope}%
\pgfsys@transformshift{1.073634in}{0.424000in}%
\pgfsys@useobject{currentmarker}{}%
\end{pgfscope}%
\end{pgfscope}%
\begin{pgfscope}%
\definecolor{textcolor}{rgb}{0.000000,0.000000,0.000000}%
\pgfsetstrokecolor{textcolor}%
\pgfsetfillcolor{textcolor}%
\pgftext[x=1.073634in,y=0.326778in,,top]{\color{textcolor}\rmfamily\fontsize{9.000000}{10.800000}\selectfont \(\displaystyle {16.5}\)}%
\end{pgfscope}%
\begin{pgfscope}%
\pgfsetbuttcap%
\pgfsetroundjoin%
\definecolor{currentfill}{rgb}{0.000000,0.000000,0.000000}%
\pgfsetfillcolor{currentfill}%
\pgfsetlinewidth{0.803000pt}%
\definecolor{currentstroke}{rgb}{0.000000,0.000000,0.000000}%
\pgfsetstrokecolor{currentstroke}%
\pgfsetdash{}{0pt}%
\pgfsys@defobject{currentmarker}{\pgfqpoint{0.000000in}{-0.048611in}}{\pgfqpoint{0.000000in}{0.000000in}}{%
\pgfpathmoveto{\pgfqpoint{0.000000in}{0.000000in}}%
\pgfpathlineto{\pgfqpoint{0.000000in}{-0.048611in}}%
\pgfusepath{stroke,fill}%
}%
\begin{pgfscope}%
\pgfsys@transformshift{1.826275in}{0.424000in}%
\pgfsys@useobject{currentmarker}{}%
\end{pgfscope}%
\end{pgfscope}%
\begin{pgfscope}%
\definecolor{textcolor}{rgb}{0.000000,0.000000,0.000000}%
\pgfsetstrokecolor{textcolor}%
\pgfsetfillcolor{textcolor}%
\pgftext[x=1.826275in,y=0.326778in,,top]{\color{textcolor}\rmfamily\fontsize{9.000000}{10.800000}\selectfont \(\displaystyle {17.0}\)}%
\end{pgfscope}%
\begin{pgfscope}%
\pgfsetbuttcap%
\pgfsetroundjoin%
\definecolor{currentfill}{rgb}{0.000000,0.000000,0.000000}%
\pgfsetfillcolor{currentfill}%
\pgfsetlinewidth{0.803000pt}%
\definecolor{currentstroke}{rgb}{0.000000,0.000000,0.000000}%
\pgfsetstrokecolor{currentstroke}%
\pgfsetdash{}{0pt}%
\pgfsys@defobject{currentmarker}{\pgfqpoint{0.000000in}{-0.048611in}}{\pgfqpoint{0.000000in}{0.000000in}}{%
\pgfpathmoveto{\pgfqpoint{0.000000in}{0.000000in}}%
\pgfpathlineto{\pgfqpoint{0.000000in}{-0.048611in}}%
\pgfusepath{stroke,fill}%
}%
\begin{pgfscope}%
\pgfsys@transformshift{2.578916in}{0.424000in}%
\pgfsys@useobject{currentmarker}{}%
\end{pgfscope}%
\end{pgfscope}%
\begin{pgfscope}%
\definecolor{textcolor}{rgb}{0.000000,0.000000,0.000000}%
\pgfsetstrokecolor{textcolor}%
\pgfsetfillcolor{textcolor}%
\pgftext[x=2.578916in,y=0.326778in,,top]{\color{textcolor}\rmfamily\fontsize{9.000000}{10.800000}\selectfont \(\displaystyle {17.5}\)}%
\end{pgfscope}%
\begin{pgfscope}%
\definecolor{textcolor}{rgb}{0.000000,0.000000,0.000000}%
\pgfsetstrokecolor{textcolor}%
\pgfsetfillcolor{textcolor}%
\pgftext[x=1.826275in,y=0.160111in,,top]{\color{textcolor}\rmfamily\fontsize{9.000000}{10.800000}\selectfont \(\displaystyle l\)~[mm]}%
\end{pgfscope}%
\begin{pgfscope}%
\pgfsetbuttcap%
\pgfsetroundjoin%
\definecolor{currentfill}{rgb}{0.000000,0.000000,0.000000}%
\pgfsetfillcolor{currentfill}%
\pgfsetlinewidth{0.803000pt}%
\definecolor{currentstroke}{rgb}{0.000000,0.000000,0.000000}%
\pgfsetstrokecolor{currentstroke}%
\pgfsetdash{}{0pt}%
\pgfsys@defobject{currentmarker}{\pgfqpoint{-0.048611in}{0.000000in}}{\pgfqpoint{-0.000000in}{0.000000in}}{%
\pgfpathmoveto{\pgfqpoint{-0.000000in}{0.000000in}}%
\pgfpathlineto{\pgfqpoint{-0.048611in}{0.000000in}}%
\pgfusepath{stroke,fill}%
}%
\begin{pgfscope}%
\pgfsys@transformshift{0.622050in}{0.424000in}%
\pgfsys@useobject{currentmarker}{}%
\end{pgfscope}%
\end{pgfscope}%
\begin{pgfscope}%
\definecolor{textcolor}{rgb}{0.000000,0.000000,0.000000}%
\pgfsetstrokecolor{textcolor}%
\pgfsetfillcolor{textcolor}%
\pgftext[x=0.332121in, y=0.380597in, left, base]{\color{textcolor}\rmfamily\fontsize{9.000000}{10.800000}\selectfont \(\displaystyle {260}\)}%
\end{pgfscope}%
\begin{pgfscope}%
\pgfsetbuttcap%
\pgfsetroundjoin%
\definecolor{currentfill}{rgb}{0.000000,0.000000,0.000000}%
\pgfsetfillcolor{currentfill}%
\pgfsetlinewidth{0.803000pt}%
\definecolor{currentstroke}{rgb}{0.000000,0.000000,0.000000}%
\pgfsetstrokecolor{currentstroke}%
\pgfsetdash{}{0pt}%
\pgfsys@defobject{currentmarker}{\pgfqpoint{-0.048611in}{0.000000in}}{\pgfqpoint{-0.000000in}{0.000000in}}{%
\pgfpathmoveto{\pgfqpoint{-0.000000in}{0.000000in}}%
\pgfpathlineto{\pgfqpoint{-0.048611in}{0.000000in}}%
\pgfusepath{stroke,fill}%
}%
\begin{pgfscope}%
\pgfsys@transformshift{0.622050in}{1.130667in}%
\pgfsys@useobject{currentmarker}{}%
\end{pgfscope}%
\end{pgfscope}%
\begin{pgfscope}%
\definecolor{textcolor}{rgb}{0.000000,0.000000,0.000000}%
\pgfsetstrokecolor{textcolor}%
\pgfsetfillcolor{textcolor}%
\pgftext[x=0.332121in, y=1.087264in, left, base]{\color{textcolor}\rmfamily\fontsize{9.000000}{10.800000}\selectfont \(\displaystyle {270}\)}%
\end{pgfscope}%
\begin{pgfscope}%
\pgfsetbuttcap%
\pgfsetroundjoin%
\definecolor{currentfill}{rgb}{0.000000,0.000000,0.000000}%
\pgfsetfillcolor{currentfill}%
\pgfsetlinewidth{0.803000pt}%
\definecolor{currentstroke}{rgb}{0.000000,0.000000,0.000000}%
\pgfsetstrokecolor{currentstroke}%
\pgfsetdash{}{0pt}%
\pgfsys@defobject{currentmarker}{\pgfqpoint{-0.048611in}{0.000000in}}{\pgfqpoint{-0.000000in}{0.000000in}}{%
\pgfpathmoveto{\pgfqpoint{-0.000000in}{0.000000in}}%
\pgfpathlineto{\pgfqpoint{-0.048611in}{0.000000in}}%
\pgfusepath{stroke,fill}%
}%
\begin{pgfscope}%
\pgfsys@transformshift{0.622050in}{1.837333in}%
\pgfsys@useobject{currentmarker}{}%
\end{pgfscope}%
\end{pgfscope}%
\begin{pgfscope}%
\definecolor{textcolor}{rgb}{0.000000,0.000000,0.000000}%
\pgfsetstrokecolor{textcolor}%
\pgfsetfillcolor{textcolor}%
\pgftext[x=0.332121in, y=1.793931in, left, base]{\color{textcolor}\rmfamily\fontsize{9.000000}{10.800000}\selectfont \(\displaystyle {280}\)}%
\end{pgfscope}%
\begin{pgfscope}%
\pgfsetbuttcap%
\pgfsetroundjoin%
\definecolor{currentfill}{rgb}{0.000000,0.000000,0.000000}%
\pgfsetfillcolor{currentfill}%
\pgfsetlinewidth{0.803000pt}%
\definecolor{currentstroke}{rgb}{0.000000,0.000000,0.000000}%
\pgfsetstrokecolor{currentstroke}%
\pgfsetdash{}{0pt}%
\pgfsys@defobject{currentmarker}{\pgfqpoint{-0.048611in}{0.000000in}}{\pgfqpoint{-0.000000in}{0.000000in}}{%
\pgfpathmoveto{\pgfqpoint{-0.000000in}{0.000000in}}%
\pgfpathlineto{\pgfqpoint{-0.048611in}{0.000000in}}%
\pgfusepath{stroke,fill}%
}%
\begin{pgfscope}%
\pgfsys@transformshift{0.622050in}{2.544000in}%
\pgfsys@useobject{currentmarker}{}%
\end{pgfscope}%
\end{pgfscope}%
\begin{pgfscope}%
\definecolor{textcolor}{rgb}{0.000000,0.000000,0.000000}%
\pgfsetstrokecolor{textcolor}%
\pgfsetfillcolor{textcolor}%
\pgftext[x=0.332121in, y=2.500597in, left, base]{\color{textcolor}\rmfamily\fontsize{9.000000}{10.800000}\selectfont \(\displaystyle {290}\)}%
\end{pgfscope}%
\begin{pgfscope}%
\definecolor{textcolor}{rgb}{0.000000,0.000000,0.000000}%
\pgfsetstrokecolor{textcolor}%
\pgfsetfillcolor{textcolor}%
\pgftext[x=0.276565in,y=1.484000in,,bottom,rotate=90.000000]{\color{textcolor}\rmfamily\fontsize{9.000000}{10.800000}\selectfont \(\displaystyle m\)~[mg]}%
\end{pgfscope}%
\begin{pgfscope}%
\pgfpathrectangle{\pgfqpoint{0.622050in}{0.424000in}}{\pgfqpoint{2.408450in}{2.120000in}}%
\pgfusepath{clip}%
\pgfsetbuttcap%
\pgfsetroundjoin%
\pgfsetlinewidth{1.254687pt}%
\definecolor{currentstroke}{rgb}{0.000000,0.000000,0.000000}%
\pgfsetstrokecolor{currentstroke}%
\pgfsetdash{{3.750000pt}{3.750000pt}}{0.000000pt}%
\pgfpathmoveto{\pgfqpoint{0.622050in}{1.577886in}}%
\pgfpathlineto{\pgfqpoint{3.030500in}{1.577886in}}%
\pgfusepath{stroke}%
\end{pgfscope}%
\begin{pgfscope}%
\pgfpathrectangle{\pgfqpoint{0.622050in}{0.424000in}}{\pgfqpoint{2.408450in}{2.120000in}}%
\pgfusepath{clip}%
\pgfsetbuttcap%
\pgfsetroundjoin%
\pgfsetlinewidth{1.254687pt}%
\definecolor{currentstroke}{rgb}{0.000000,0.000000,0.000000}%
\pgfsetstrokecolor{currentstroke}%
\pgfsetdash{{3.750000pt}{3.750000pt}}{0.000000pt}%
\pgfpathmoveto{\pgfqpoint{1.993361in}{0.424000in}}%
\pgfpathlineto{\pgfqpoint{1.993361in}{2.544000in}}%
\pgfusepath{stroke}%
\end{pgfscope}%
\begin{pgfscope}%
\pgfpathrectangle{\pgfqpoint{0.622050in}{0.424000in}}{\pgfqpoint{2.408450in}{2.120000in}}%
\pgfusepath{clip}%
\pgfsetbuttcap%
\pgfsetroundjoin%
\definecolor{currentfill}{rgb}{0.000000,0.000000,0.000000}%
\pgfsetfillcolor{currentfill}%
\pgfsetlinewidth{1.003750pt}%
\definecolor{currentstroke}{rgb}{0.000000,0.000000,0.000000}%
\pgfsetstrokecolor{currentstroke}%
\pgfsetdash{}{0pt}%
\pgfsys@defobject{currentmarker}{\pgfqpoint{-0.027778in}{-0.027778in}}{\pgfqpoint{0.027778in}{0.027778in}}{%
\pgfpathmoveto{\pgfqpoint{0.000000in}{-0.027778in}}%
\pgfpathcurveto{\pgfqpoint{0.007367in}{-0.027778in}}{\pgfqpoint{0.014433in}{-0.024851in}}{\pgfqpoint{0.019642in}{-0.019642in}}%
\pgfpathcurveto{\pgfqpoint{0.024851in}{-0.014433in}}{\pgfqpoint{0.027778in}{-0.007367in}}{\pgfqpoint{0.027778in}{0.000000in}}%
\pgfpathcurveto{\pgfqpoint{0.027778in}{0.007367in}}{\pgfqpoint{0.024851in}{0.014433in}}{\pgfqpoint{0.019642in}{0.019642in}}%
\pgfpathcurveto{\pgfqpoint{0.014433in}{0.024851in}}{\pgfqpoint{0.007367in}{0.027778in}}{\pgfqpoint{0.000000in}{0.027778in}}%
\pgfpathcurveto{\pgfqpoint{-0.007367in}{0.027778in}}{\pgfqpoint{-0.014433in}{0.024851in}}{\pgfqpoint{-0.019642in}{0.019642in}}%
\pgfpathcurveto{\pgfqpoint{-0.024851in}{0.014433in}}{\pgfqpoint{-0.027778in}{0.007367in}}{\pgfqpoint{-0.027778in}{0.000000in}}%
\pgfpathcurveto{\pgfqpoint{-0.027778in}{-0.007367in}}{\pgfqpoint{-0.024851in}{-0.014433in}}{\pgfqpoint{-0.019642in}{-0.019642in}}%
\pgfpathcurveto{\pgfqpoint{-0.014433in}{-0.024851in}}{\pgfqpoint{-0.007367in}{-0.027778in}}{\pgfqpoint{0.000000in}{-0.027778in}}%
\pgfpathlineto{\pgfqpoint{0.000000in}{-0.027778in}}%
\pgfpathclose%
\pgfusepath{stroke,fill}%
}%
\begin{pgfscope}%
\pgfsys@transformshift{2.292912in}{1.554667in}%
\pgfsys@useobject{currentmarker}{}%
\end{pgfscope}%
\begin{pgfscope}%
\pgfsys@transformshift{2.247754in}{1.908000in}%
\pgfsys@useobject{currentmarker}{}%
\end{pgfscope}%
\begin{pgfscope}%
\pgfsys@transformshift{2.187542in}{1.625333in}%
\pgfsys@useobject{currentmarker}{}%
\end{pgfscope}%
\begin{pgfscope}%
\pgfsys@transformshift{1.510166in}{1.413333in}%
\pgfsys@useobject{currentmarker}{}%
\end{pgfscope}%
\begin{pgfscope}%
\pgfsys@transformshift{2.458493in}{1.696000in}%
\pgfsys@useobject{currentmarker}{}%
\end{pgfscope}%
\begin{pgfscope}%
\pgfsys@transformshift{1.540272in}{1.413333in}%
\pgfsys@useobject{currentmarker}{}%
\end{pgfscope}%
\begin{pgfscope}%
\pgfsys@transformshift{2.383229in}{1.837333in}%
\pgfsys@useobject{currentmarker}{}%
\end{pgfscope}%
\begin{pgfscope}%
\pgfsys@transformshift{2.187542in}{1.766667in}%
\pgfsys@useobject{currentmarker}{}%
\end{pgfscope}%
\begin{pgfscope}%
\pgfsys@transformshift{2.006909in}{1.413333in}%
\pgfsys@useobject{currentmarker}{}%
\end{pgfscope}%
\begin{pgfscope}%
\pgfsys@transformshift{1.239215in}{0.848000in}%
\pgfsys@useobject{currentmarker}{}%
\end{pgfscope}%
\begin{pgfscope}%
\pgfsys@transformshift{1.901539in}{1.413333in}%
\pgfsys@useobject{currentmarker}{}%
\end{pgfscope}%
\begin{pgfscope}%
\pgfsys@transformshift{2.157437in}{1.837333in}%
\pgfsys@useobject{currentmarker}{}%
\end{pgfscope}%
\begin{pgfscope}%
\pgfsys@transformshift{1.931645in}{1.413333in}%
\pgfsys@useobject{currentmarker}{}%
\end{pgfscope}%
\begin{pgfscope}%
\pgfsys@transformshift{1.916592in}{1.484000in}%
\pgfsys@useobject{currentmarker}{}%
\end{pgfscope}%
\begin{pgfscope}%
\pgfsys@transformshift{2.413335in}{1.908000in}%
\pgfsys@useobject{currentmarker}{}%
\end{pgfscope}%
\begin{pgfscope}%
\pgfsys@transformshift{1.766064in}{1.484000in}%
\pgfsys@useobject{currentmarker}{}%
\end{pgfscope}%
\begin{pgfscope}%
\pgfsys@transformshift{2.262807in}{1.766667in}%
\pgfsys@useobject{currentmarker}{}%
\end{pgfscope}%
\begin{pgfscope}%
\pgfsys@transformshift{2.669232in}{1.978667in}%
\pgfsys@useobject{currentmarker}{}%
\end{pgfscope}%
\begin{pgfscope}%
\pgfsys@transformshift{2.172490in}{1.978667in}%
\pgfsys@useobject{currentmarker}{}%
\end{pgfscope}%
\begin{pgfscope}%
\pgfsys@transformshift{1.179004in}{0.918667in}%
\pgfsys@useobject{currentmarker}{}%
\end{pgfscope}%
\begin{pgfscope}%
\pgfsys@transformshift{2.082173in}{1.696000in}%
\pgfsys@useobject{currentmarker}{}%
\end{pgfscope}%
\begin{pgfscope}%
\pgfsys@transformshift{1.916592in}{1.554667in}%
\pgfsys@useobject{currentmarker}{}%
\end{pgfscope}%
\begin{pgfscope}%
\pgfsys@transformshift{2.292912in}{1.837333in}%
\pgfsys@useobject{currentmarker}{}%
\end{pgfscope}%
\begin{pgfscope}%
\pgfsys@transformshift{1.600483in}{1.413333in}%
\pgfsys@useobject{currentmarker}{}%
\end{pgfscope}%
\begin{pgfscope}%
\pgfsys@transformshift{2.277859in}{1.696000in}%
\pgfsys@useobject{currentmarker}{}%
\end{pgfscope}%
\begin{pgfscope}%
\pgfsys@transformshift{2.187542in}{1.554667in}%
\pgfsys@useobject{currentmarker}{}%
\end{pgfscope}%
\begin{pgfscope}%
\pgfsys@transformshift{2.021962in}{1.696000in}%
\pgfsys@useobject{currentmarker}{}%
\end{pgfscope}%
\begin{pgfscope}%
\pgfsys@transformshift{2.217648in}{1.766667in}%
\pgfsys@useobject{currentmarker}{}%
\end{pgfscope}%
\begin{pgfscope}%
\pgfsys@transformshift{1.796169in}{1.484000in}%
\pgfsys@useobject{currentmarker}{}%
\end{pgfscope}%
\begin{pgfscope}%
\pgfsys@transformshift{2.037014in}{1.484000in}%
\pgfsys@useobject{currentmarker}{}%
\end{pgfscope}%
\begin{pgfscope}%
\pgfsys@transformshift{2.518704in}{2.120000in}%
\pgfsys@useobject{currentmarker}{}%
\end{pgfscope}%
\begin{pgfscope}%
\pgfsys@transformshift{1.660694in}{1.342667in}%
\pgfsys@useobject{currentmarker}{}%
\end{pgfscope}%
\begin{pgfscope}%
\pgfsys@transformshift{2.006909in}{1.837333in}%
\pgfsys@useobject{currentmarker}{}%
\end{pgfscope}%
\begin{pgfscope}%
\pgfsys@transformshift{1.269321in}{0.848000in}%
\pgfsys@useobject{currentmarker}{}%
\end{pgfscope}%
\begin{pgfscope}%
\pgfsys@transformshift{2.368176in}{1.837333in}%
\pgfsys@useobject{currentmarker}{}%
\end{pgfscope}%
\begin{pgfscope}%
\pgfsys@transformshift{1.811222in}{1.484000in}%
\pgfsys@useobject{currentmarker}{}%
\end{pgfscope}%
\begin{pgfscope}%
\pgfsys@transformshift{2.262807in}{1.484000in}%
\pgfsys@useobject{currentmarker}{}%
\end{pgfscope}%
\begin{pgfscope}%
\pgfsys@transformshift{1.751011in}{1.272000in}%
\pgfsys@useobject{currentmarker}{}%
\end{pgfscope}%
\begin{pgfscope}%
\pgfsys@transformshift{2.172490in}{1.696000in}%
\pgfsys@useobject{currentmarker}{}%
\end{pgfscope}%
\begin{pgfscope}%
\pgfsys@transformshift{2.262807in}{1.908000in}%
\pgfsys@useobject{currentmarker}{}%
\end{pgfscope}%
\begin{pgfscope}%
\pgfsys@transformshift{2.157437in}{1.908000in}%
\pgfsys@useobject{currentmarker}{}%
\end{pgfscope}%
\begin{pgfscope}%
\pgfsys@transformshift{2.307965in}{1.766667in}%
\pgfsys@useobject{currentmarker}{}%
\end{pgfscope}%
\begin{pgfscope}%
\pgfsys@transformshift{1.826275in}{1.625333in}%
\pgfsys@useobject{currentmarker}{}%
\end{pgfscope}%
\begin{pgfscope}%
\pgfsys@transformshift{1.781117in}{1.484000in}%
\pgfsys@useobject{currentmarker}{}%
\end{pgfscope}%
\begin{pgfscope}%
\pgfsys@transformshift{1.871433in}{1.696000in}%
\pgfsys@useobject{currentmarker}{}%
\end{pgfscope}%
\begin{pgfscope}%
\pgfsys@transformshift{2.473546in}{1.696000in}%
\pgfsys@useobject{currentmarker}{}%
\end{pgfscope}%
\begin{pgfscope}%
\pgfsys@transformshift{2.262807in}{1.837333in}%
\pgfsys@useobject{currentmarker}{}%
\end{pgfscope}%
\begin{pgfscope}%
\pgfsys@transformshift{1.735958in}{1.272000in}%
\pgfsys@useobject{currentmarker}{}%
\end{pgfscope}%
\begin{pgfscope}%
\pgfsys@transformshift{1.961750in}{1.554667in}%
\pgfsys@useobject{currentmarker}{}%
\end{pgfscope}%
\begin{pgfscope}%
\pgfsys@transformshift{1.841328in}{1.625333in}%
\pgfsys@useobject{currentmarker}{}%
\end{pgfscope}%
\begin{pgfscope}%
\pgfsys@transformshift{1.871433in}{1.554667in}%
\pgfsys@useobject{currentmarker}{}%
\end{pgfscope}%
\begin{pgfscope}%
\pgfsys@transformshift{1.871433in}{1.908000in}%
\pgfsys@useobject{currentmarker}{}%
\end{pgfscope}%
\begin{pgfscope}%
\pgfsys@transformshift{2.067120in}{1.554667in}%
\pgfsys@useobject{currentmarker}{}%
\end{pgfscope}%
\begin{pgfscope}%
\pgfsys@transformshift{1.751011in}{1.554667in}%
\pgfsys@useobject{currentmarker}{}%
\end{pgfscope}%
\begin{pgfscope}%
\pgfsys@transformshift{1.751011in}{1.625333in}%
\pgfsys@useobject{currentmarker}{}%
\end{pgfscope}%
\begin{pgfscope}%
\pgfsys@transformshift{1.871433in}{1.484000in}%
\pgfsys@useobject{currentmarker}{}%
\end{pgfscope}%
\begin{pgfscope}%
\pgfsys@transformshift{1.781117in}{1.272000in}%
\pgfsys@useobject{currentmarker}{}%
\end{pgfscope}%
\begin{pgfscope}%
\pgfsys@transformshift{1.781117in}{1.413333in}%
\pgfsys@useobject{currentmarker}{}%
\end{pgfscope}%
\begin{pgfscope}%
\pgfsys@transformshift{1.766064in}{1.413333in}%
\pgfsys@useobject{currentmarker}{}%
\end{pgfscope}%
\begin{pgfscope}%
\pgfsys@transformshift{2.533757in}{1.837333in}%
\pgfsys@useobject{currentmarker}{}%
\end{pgfscope}%
\begin{pgfscope}%
\pgfsys@transformshift{2.413335in}{1.625333in}%
\pgfsys@useobject{currentmarker}{}%
\end{pgfscope}%
\begin{pgfscope}%
\pgfsys@transformshift{1.720905in}{1.272000in}%
\pgfsys@useobject{currentmarker}{}%
\end{pgfscope}%
\begin{pgfscope}%
\pgfsys@transformshift{1.901539in}{1.413333in}%
\pgfsys@useobject{currentmarker}{}%
\end{pgfscope}%
\begin{pgfscope}%
\pgfsys@transformshift{1.946698in}{1.625333in}%
\pgfsys@useobject{currentmarker}{}%
\end{pgfscope}%
\begin{pgfscope}%
\pgfsys@transformshift{1.796169in}{1.413333in}%
\pgfsys@useobject{currentmarker}{}%
\end{pgfscope}%
\begin{pgfscope}%
\pgfsys@transformshift{2.428387in}{1.837333in}%
\pgfsys@useobject{currentmarker}{}%
\end{pgfscope}%
\begin{pgfscope}%
\pgfsys@transformshift{1.991856in}{1.554667in}%
\pgfsys@useobject{currentmarker}{}%
\end{pgfscope}%
\begin{pgfscope}%
\pgfsys@transformshift{2.142384in}{1.625333in}%
\pgfsys@useobject{currentmarker}{}%
\end{pgfscope}%
\begin{pgfscope}%
\pgfsys@transformshift{1.058582in}{0.706667in}%
\pgfsys@useobject{currentmarker}{}%
\end{pgfscope}%
\begin{pgfscope}%
\pgfsys@transformshift{1.931645in}{1.625333in}%
\pgfsys@useobject{currentmarker}{}%
\end{pgfscope}%
\end{pgfscope}%
\begin{pgfscope}%
\pgfsetrectcap%
\pgfsetmiterjoin%
\pgfsetlinewidth{0.803000pt}%
\definecolor{currentstroke}{rgb}{0.000000,0.000000,0.000000}%
\pgfsetstrokecolor{currentstroke}%
\pgfsetdash{}{0pt}%
\pgfpathmoveto{\pgfqpoint{0.622050in}{0.424000in}}%
\pgfpathlineto{\pgfqpoint{0.622050in}{2.544000in}}%
\pgfusepath{stroke}%
\end{pgfscope}%
\begin{pgfscope}%
\pgfsetrectcap%
\pgfsetmiterjoin%
\pgfsetlinewidth{0.803000pt}%
\definecolor{currentstroke}{rgb}{0.000000,0.000000,0.000000}%
\pgfsetstrokecolor{currentstroke}%
\pgfsetdash{}{0pt}%
\pgfpathmoveto{\pgfqpoint{3.030500in}{0.424000in}}%
\pgfpathlineto{\pgfqpoint{3.030500in}{2.544000in}}%
\pgfusepath{stroke}%
\end{pgfscope}%
\begin{pgfscope}%
\pgfsetrectcap%
\pgfsetmiterjoin%
\pgfsetlinewidth{0.803000pt}%
\definecolor{currentstroke}{rgb}{0.000000,0.000000,0.000000}%
\pgfsetstrokecolor{currentstroke}%
\pgfsetdash{}{0pt}%
\pgfpathmoveto{\pgfqpoint{0.622050in}{0.424000in}}%
\pgfpathlineto{\pgfqpoint{3.030500in}{0.424000in}}%
\pgfusepath{stroke}%
\end{pgfscope}%
\begin{pgfscope}%
\pgfsetrectcap%
\pgfsetmiterjoin%
\pgfsetlinewidth{0.803000pt}%
\definecolor{currentstroke}{rgb}{0.000000,0.000000,0.000000}%
\pgfsetstrokecolor{currentstroke}%
\pgfsetdash{}{0pt}%
\pgfpathmoveto{\pgfqpoint{0.622050in}{2.544000in}}%
\pgfpathlineto{\pgfqpoint{3.030500in}{2.544000in}}%
\pgfusepath{stroke}%
\end{pgfscope}%
\begin{pgfscope}%
\definecolor{textcolor}{rgb}{0.000000,0.000000,0.000000}%
\pgfsetstrokecolor{textcolor}%
\pgfsetfillcolor{textcolor}%
\pgftext[x=0.862895in,y=2.226000in,left,base]{\color{textcolor}\rmfamily\fontsize{9.000000}{10.800000}\selectfont \(\displaystyle r = 0.85\)}%
\end{pgfscope}%
\begin{pgfscope}%
\pgfsetbuttcap%
\pgfsetmiterjoin%
\definecolor{currentfill}{rgb}{1.000000,1.000000,1.000000}%
\pgfsetfillcolor{currentfill}%
\pgfsetlinewidth{0.000000pt}%
\definecolor{currentstroke}{rgb}{0.000000,0.000000,0.000000}%
\pgfsetstrokecolor{currentstroke}%
\pgfsetstrokeopacity{0.000000}%
\pgfsetdash{}{0pt}%
\pgfpathmoveto{\pgfqpoint{0.622050in}{2.654240in}}%
\pgfpathlineto{\pgfqpoint{3.030500in}{2.654240in}}%
\pgfpathlineto{\pgfqpoint{3.030500in}{4.134000in}}%
\pgfpathlineto{\pgfqpoint{0.622050in}{4.134000in}}%
\pgfpathlineto{\pgfqpoint{0.622050in}{2.654240in}}%
\pgfpathclose%
\pgfusepath{fill}%
\end{pgfscope}%
\begin{pgfscope}%
\pgfsetbuttcap%
\pgfsetroundjoin%
\definecolor{currentfill}{rgb}{0.000000,0.000000,0.000000}%
\pgfsetfillcolor{currentfill}%
\pgfsetlinewidth{0.803000pt}%
\definecolor{currentstroke}{rgb}{0.000000,0.000000,0.000000}%
\pgfsetstrokecolor{currentstroke}%
\pgfsetdash{}{0pt}%
\pgfsys@defobject{currentmarker}{\pgfqpoint{0.000000in}{-0.048611in}}{\pgfqpoint{0.000000in}{0.000000in}}{%
\pgfpathmoveto{\pgfqpoint{0.000000in}{0.000000in}}%
\pgfpathlineto{\pgfqpoint{0.000000in}{-0.048611in}}%
\pgfusepath{stroke,fill}%
}%
\begin{pgfscope}%
\pgfsys@transformshift{1.073634in}{2.654240in}%
\pgfsys@useobject{currentmarker}{}%
\end{pgfscope}%
\end{pgfscope}%
\begin{pgfscope}%
\pgfsetbuttcap%
\pgfsetroundjoin%
\definecolor{currentfill}{rgb}{0.000000,0.000000,0.000000}%
\pgfsetfillcolor{currentfill}%
\pgfsetlinewidth{0.803000pt}%
\definecolor{currentstroke}{rgb}{0.000000,0.000000,0.000000}%
\pgfsetstrokecolor{currentstroke}%
\pgfsetdash{}{0pt}%
\pgfsys@defobject{currentmarker}{\pgfqpoint{0.000000in}{-0.048611in}}{\pgfqpoint{0.000000in}{0.000000in}}{%
\pgfpathmoveto{\pgfqpoint{0.000000in}{0.000000in}}%
\pgfpathlineto{\pgfqpoint{0.000000in}{-0.048611in}}%
\pgfusepath{stroke,fill}%
}%
\begin{pgfscope}%
\pgfsys@transformshift{1.826275in}{2.654240in}%
\pgfsys@useobject{currentmarker}{}%
\end{pgfscope}%
\end{pgfscope}%
\begin{pgfscope}%
\pgfsetbuttcap%
\pgfsetroundjoin%
\definecolor{currentfill}{rgb}{0.000000,0.000000,0.000000}%
\pgfsetfillcolor{currentfill}%
\pgfsetlinewidth{0.803000pt}%
\definecolor{currentstroke}{rgb}{0.000000,0.000000,0.000000}%
\pgfsetstrokecolor{currentstroke}%
\pgfsetdash{}{0pt}%
\pgfsys@defobject{currentmarker}{\pgfqpoint{0.000000in}{-0.048611in}}{\pgfqpoint{0.000000in}{0.000000in}}{%
\pgfpathmoveto{\pgfqpoint{0.000000in}{0.000000in}}%
\pgfpathlineto{\pgfqpoint{0.000000in}{-0.048611in}}%
\pgfusepath{stroke,fill}%
}%
\begin{pgfscope}%
\pgfsys@transformshift{2.578916in}{2.654240in}%
\pgfsys@useobject{currentmarker}{}%
\end{pgfscope}%
\end{pgfscope}%
\begin{pgfscope}%
\pgfsetbuttcap%
\pgfsetroundjoin%
\definecolor{currentfill}{rgb}{0.000000,0.000000,0.000000}%
\pgfsetfillcolor{currentfill}%
\pgfsetlinewidth{0.803000pt}%
\definecolor{currentstroke}{rgb}{0.000000,0.000000,0.000000}%
\pgfsetstrokecolor{currentstroke}%
\pgfsetdash{}{0pt}%
\pgfsys@defobject{currentmarker}{\pgfqpoint{-0.048611in}{0.000000in}}{\pgfqpoint{-0.000000in}{0.000000in}}{%
\pgfpathmoveto{\pgfqpoint{-0.000000in}{0.000000in}}%
\pgfpathlineto{\pgfqpoint{-0.048611in}{0.000000in}}%
\pgfusepath{stroke,fill}%
}%
\begin{pgfscope}%
\pgfsys@transformshift{0.622050in}{2.654240in}%
\pgfsys@useobject{currentmarker}{}%
\end{pgfscope}%
\end{pgfscope}%
\begin{pgfscope}%
\definecolor{textcolor}{rgb}{0.000000,0.000000,0.000000}%
\pgfsetstrokecolor{textcolor}%
\pgfsetfillcolor{textcolor}%
\pgftext[x=0.460592in, y=2.610837in, left, base]{\color{textcolor}\rmfamily\fontsize{9.000000}{10.800000}\selectfont \(\displaystyle {0}\)}%
\end{pgfscope}%
\begin{pgfscope}%
\pgfsetbuttcap%
\pgfsetroundjoin%
\definecolor{currentfill}{rgb}{0.000000,0.000000,0.000000}%
\pgfsetfillcolor{currentfill}%
\pgfsetlinewidth{0.803000pt}%
\definecolor{currentstroke}{rgb}{0.000000,0.000000,0.000000}%
\pgfsetstrokecolor{currentstroke}%
\pgfsetdash{}{0pt}%
\pgfsys@defobject{currentmarker}{\pgfqpoint{-0.048611in}{0.000000in}}{\pgfqpoint{-0.000000in}{0.000000in}}{%
\pgfpathmoveto{\pgfqpoint{-0.000000in}{0.000000in}}%
\pgfpathlineto{\pgfqpoint{-0.048611in}{0.000000in}}%
\pgfusepath{stroke,fill}%
}%
\begin{pgfscope}%
\pgfsys@transformshift{0.622050in}{3.147493in}%
\pgfsys@useobject{currentmarker}{}%
\end{pgfscope}%
\end{pgfscope}%
\begin{pgfscope}%
\definecolor{textcolor}{rgb}{0.000000,0.000000,0.000000}%
\pgfsetstrokecolor{textcolor}%
\pgfsetfillcolor{textcolor}%
\pgftext[x=0.460592in, y=3.104091in, left, base]{\color{textcolor}\rmfamily\fontsize{9.000000}{10.800000}\selectfont \(\displaystyle {5}\)}%
\end{pgfscope}%
\begin{pgfscope}%
\pgfsetbuttcap%
\pgfsetroundjoin%
\definecolor{currentfill}{rgb}{0.000000,0.000000,0.000000}%
\pgfsetfillcolor{currentfill}%
\pgfsetlinewidth{0.803000pt}%
\definecolor{currentstroke}{rgb}{0.000000,0.000000,0.000000}%
\pgfsetstrokecolor{currentstroke}%
\pgfsetdash{}{0pt}%
\pgfsys@defobject{currentmarker}{\pgfqpoint{-0.048611in}{0.000000in}}{\pgfqpoint{-0.000000in}{0.000000in}}{%
\pgfpathmoveto{\pgfqpoint{-0.000000in}{0.000000in}}%
\pgfpathlineto{\pgfqpoint{-0.048611in}{0.000000in}}%
\pgfusepath{stroke,fill}%
}%
\begin{pgfscope}%
\pgfsys@transformshift{0.622050in}{3.640747in}%
\pgfsys@useobject{currentmarker}{}%
\end{pgfscope}%
\end{pgfscope}%
\begin{pgfscope}%
\definecolor{textcolor}{rgb}{0.000000,0.000000,0.000000}%
\pgfsetstrokecolor{textcolor}%
\pgfsetfillcolor{textcolor}%
\pgftext[x=0.396356in, y=3.597344in, left, base]{\color{textcolor}\rmfamily\fontsize{9.000000}{10.800000}\selectfont \(\displaystyle {10}\)}%
\end{pgfscope}%
\begin{pgfscope}%
\pgfsetbuttcap%
\pgfsetroundjoin%
\definecolor{currentfill}{rgb}{0.000000,0.000000,0.000000}%
\pgfsetfillcolor{currentfill}%
\pgfsetlinewidth{0.803000pt}%
\definecolor{currentstroke}{rgb}{0.000000,0.000000,0.000000}%
\pgfsetstrokecolor{currentstroke}%
\pgfsetdash{}{0pt}%
\pgfsys@defobject{currentmarker}{\pgfqpoint{-0.048611in}{0.000000in}}{\pgfqpoint{-0.000000in}{0.000000in}}{%
\pgfpathmoveto{\pgfqpoint{-0.000000in}{0.000000in}}%
\pgfpathlineto{\pgfqpoint{-0.048611in}{0.000000in}}%
\pgfusepath{stroke,fill}%
}%
\begin{pgfscope}%
\pgfsys@transformshift{0.622050in}{4.134000in}%
\pgfsys@useobject{currentmarker}{}%
\end{pgfscope}%
\end{pgfscope}%
\begin{pgfscope}%
\definecolor{textcolor}{rgb}{0.000000,0.000000,0.000000}%
\pgfsetstrokecolor{textcolor}%
\pgfsetfillcolor{textcolor}%
\pgftext[x=0.396356in, y=4.090597in, left, base]{\color{textcolor}\rmfamily\fontsize{9.000000}{10.800000}\selectfont \(\displaystyle {15}\)}%
\end{pgfscope}%
\begin{pgfscope}%
\pgfpathrectangle{\pgfqpoint{0.622050in}{2.654240in}}{\pgfqpoint{2.408450in}{1.479760in}}%
\pgfusepath{clip}%
\pgfsetbuttcap%
\pgfsetmiterjoin%
\pgfsetlinewidth{0.752812pt}%
\definecolor{currentstroke}{rgb}{0.000000,0.000000,0.000000}%
\pgfsetstrokecolor{currentstroke}%
\pgfsetdash{}{0pt}%
\pgfpathmoveto{\pgfqpoint{1.058582in}{2.654240in}}%
\pgfpathlineto{\pgfqpoint{1.058582in}{2.752891in}}%
\pgfpathlineto{\pgfqpoint{1.143353in}{2.752891in}}%
\pgfpathlineto{\pgfqpoint{1.143353in}{2.752891in}}%
\pgfpathlineto{\pgfqpoint{1.228124in}{2.752891in}}%
\pgfpathlineto{\pgfqpoint{1.228124in}{2.851541in}}%
\pgfpathlineto{\pgfqpoint{1.312895in}{2.851541in}}%
\pgfpathlineto{\pgfqpoint{1.312895in}{2.654240in}}%
\pgfpathlineto{\pgfqpoint{1.397666in}{2.654240in}}%
\pgfpathlineto{\pgfqpoint{1.397666in}{2.654240in}}%
\pgfpathlineto{\pgfqpoint{1.482437in}{2.654240in}}%
\pgfpathlineto{\pgfqpoint{1.482437in}{2.851541in}}%
\pgfpathlineto{\pgfqpoint{1.567208in}{2.851541in}}%
\pgfpathlineto{\pgfqpoint{1.567208in}{2.752891in}}%
\pgfpathlineto{\pgfqpoint{1.651979in}{2.752891in}}%
\pgfpathlineto{\pgfqpoint{1.651979in}{2.950192in}}%
\pgfpathlineto{\pgfqpoint{1.736750in}{2.950192in}}%
\pgfpathlineto{\pgfqpoint{1.736750in}{3.739397in}}%
\pgfpathlineto{\pgfqpoint{1.821521in}{3.739397in}}%
\pgfpathlineto{\pgfqpoint{1.821521in}{3.443445in}}%
\pgfpathlineto{\pgfqpoint{1.906293in}{3.443445in}}%
\pgfpathlineto{\pgfqpoint{1.906293in}{3.246144in}}%
\pgfpathlineto{\pgfqpoint{1.991064in}{3.246144in}}%
\pgfpathlineto{\pgfqpoint{1.991064in}{3.246144in}}%
\pgfpathlineto{\pgfqpoint{2.075835in}{3.246144in}}%
\pgfpathlineto{\pgfqpoint{2.075835in}{3.048843in}}%
\pgfpathlineto{\pgfqpoint{2.160606in}{3.048843in}}%
\pgfpathlineto{\pgfqpoint{2.160606in}{3.246144in}}%
\pgfpathlineto{\pgfqpoint{2.245377in}{3.246144in}}%
\pgfpathlineto{\pgfqpoint{2.245377in}{3.542096in}}%
\pgfpathlineto{\pgfqpoint{2.330148in}{3.542096in}}%
\pgfpathlineto{\pgfqpoint{2.330148in}{3.048843in}}%
\pgfpathlineto{\pgfqpoint{2.414919in}{3.048843in}}%
\pgfpathlineto{\pgfqpoint{2.414919in}{2.950192in}}%
\pgfpathlineto{\pgfqpoint{2.499690in}{2.950192in}}%
\pgfpathlineto{\pgfqpoint{2.499690in}{2.851541in}}%
\pgfpathlineto{\pgfqpoint{2.584461in}{2.851541in}}%
\pgfpathlineto{\pgfqpoint{2.584461in}{2.752891in}}%
\pgfpathlineto{\pgfqpoint{2.669232in}{2.752891in}}%
\pgfpathlineto{\pgfqpoint{2.669232in}{2.654240in}}%
\pgfusepath{stroke}%
\end{pgfscope}%
\begin{pgfscope}%
\pgfpathrectangle{\pgfqpoint{0.622050in}{2.654240in}}{\pgfqpoint{2.408450in}{1.479760in}}%
\pgfusepath{clip}%
\pgfsetbuttcap%
\pgfsetroundjoin%
\pgfsetlinewidth{1.254687pt}%
\definecolor{currentstroke}{rgb}{0.000000,0.000000,0.000000}%
\pgfsetstrokecolor{currentstroke}%
\pgfsetdash{{3.750000pt}{3.750000pt}}{0.000000pt}%
\pgfpathmoveto{\pgfqpoint{1.993361in}{2.654240in}}%
\pgfpathlineto{\pgfqpoint{1.993361in}{4.134000in}}%
\pgfusepath{stroke}%
\end{pgfscope}%
\begin{pgfscope}%
\pgfsetrectcap%
\pgfsetmiterjoin%
\pgfsetlinewidth{0.803000pt}%
\definecolor{currentstroke}{rgb}{0.000000,0.000000,0.000000}%
\pgfsetstrokecolor{currentstroke}%
\pgfsetdash{}{0pt}%
\pgfpathmoveto{\pgfqpoint{0.622050in}{2.654240in}}%
\pgfpathlineto{\pgfqpoint{0.622050in}{4.134000in}}%
\pgfusepath{stroke}%
\end{pgfscope}%
\begin{pgfscope}%
\pgfsetrectcap%
\pgfsetmiterjoin%
\pgfsetlinewidth{0.803000pt}%
\definecolor{currentstroke}{rgb}{0.000000,0.000000,0.000000}%
\pgfsetstrokecolor{currentstroke}%
\pgfsetdash{}{0pt}%
\pgfpathmoveto{\pgfqpoint{3.030500in}{2.654240in}}%
\pgfpathlineto{\pgfqpoint{3.030500in}{4.134000in}}%
\pgfusepath{stroke}%
\end{pgfscope}%
\begin{pgfscope}%
\pgfsetrectcap%
\pgfsetmiterjoin%
\pgfsetlinewidth{0.803000pt}%
\definecolor{currentstroke}{rgb}{0.000000,0.000000,0.000000}%
\pgfsetstrokecolor{currentstroke}%
\pgfsetdash{}{0pt}%
\pgfpathmoveto{\pgfqpoint{0.622050in}{2.654240in}}%
\pgfpathlineto{\pgfqpoint{3.030500in}{2.654240in}}%
\pgfusepath{stroke}%
\end{pgfscope}%
\begin{pgfscope}%
\pgfsetrectcap%
\pgfsetmiterjoin%
\pgfsetlinewidth{0.803000pt}%
\definecolor{currentstroke}{rgb}{0.000000,0.000000,0.000000}%
\pgfsetstrokecolor{currentstroke}%
\pgfsetdash{}{0pt}%
\pgfpathmoveto{\pgfqpoint{0.622050in}{4.134000in}}%
\pgfpathlineto{\pgfqpoint{3.030500in}{4.134000in}}%
\pgfusepath{stroke}%
\end{pgfscope}%
\begin{pgfscope}%
\definecolor{textcolor}{rgb}{0.000000,0.000000,0.000000}%
\pgfsetstrokecolor{textcolor}%
\pgfsetfillcolor{textcolor}%
\pgftext[x=0.790642in,y=3.912036in,left,base]{\color{textcolor}\rmfamily\fontsize{9.000000}{10.800000}\selectfont \(\displaystyle m = 17.1f\)~mm}%
\end{pgfscope}%
\begin{pgfscope}%
\definecolor{textcolor}{rgb}{0.000000,0.000000,0.000000}%
\pgfsetstrokecolor{textcolor}%
\pgfsetfillcolor{textcolor}%
\pgftext[x=0.790642in,y=3.764060in,left,base]{\color{textcolor}\rmfamily\fontsize{9.000000}{10.800000}\selectfont \(\displaystyle s_{n-1} = 0.219\)~mm}%
\end{pgfscope}%
\begin{pgfscope}%
\pgfsetbuttcap%
\pgfsetmiterjoin%
\definecolor{currentfill}{rgb}{1.000000,1.000000,1.000000}%
\pgfsetfillcolor{currentfill}%
\pgfsetlinewidth{0.000000pt}%
\definecolor{currentstroke}{rgb}{0.000000,0.000000,0.000000}%
\pgfsetstrokecolor{currentstroke}%
\pgfsetstrokeopacity{0.000000}%
\pgfsetdash{}{0pt}%
\pgfpathmoveto{\pgfqpoint{3.195104in}{0.424000in}}%
\pgfpathlineto{\pgfqpoint{4.944500in}{0.424000in}}%
\pgfpathlineto{\pgfqpoint{4.944500in}{2.544000in}}%
\pgfpathlineto{\pgfqpoint{3.195104in}{2.544000in}}%
\pgfpathlineto{\pgfqpoint{3.195104in}{0.424000in}}%
\pgfpathclose%
\pgfusepath{fill}%
\end{pgfscope}%
\begin{pgfscope}%
\pgfsetbuttcap%
\pgfsetroundjoin%
\definecolor{currentfill}{rgb}{0.000000,0.000000,0.000000}%
\pgfsetfillcolor{currentfill}%
\pgfsetlinewidth{0.803000pt}%
\definecolor{currentstroke}{rgb}{0.000000,0.000000,0.000000}%
\pgfsetstrokecolor{currentstroke}%
\pgfsetdash{}{0pt}%
\pgfsys@defobject{currentmarker}{\pgfqpoint{0.000000in}{-0.048611in}}{\pgfqpoint{0.000000in}{0.000000in}}{%
\pgfpathmoveto{\pgfqpoint{0.000000in}{0.000000in}}%
\pgfpathlineto{\pgfqpoint{0.000000in}{-0.048611in}}%
\pgfusepath{stroke,fill}%
}%
\begin{pgfscope}%
\pgfsys@transformshift{3.195104in}{0.424000in}%
\pgfsys@useobject{currentmarker}{}%
\end{pgfscope}%
\end{pgfscope}%
\begin{pgfscope}%
\definecolor{textcolor}{rgb}{0.000000,0.000000,0.000000}%
\pgfsetstrokecolor{textcolor}%
\pgfsetfillcolor{textcolor}%
\pgftext[x=3.195104in,y=0.326778in,,top]{\color{textcolor}\rmfamily\fontsize{9.000000}{10.800000}\selectfont \(\displaystyle {0}\)}%
\end{pgfscope}%
\begin{pgfscope}%
\pgfsetbuttcap%
\pgfsetroundjoin%
\definecolor{currentfill}{rgb}{0.000000,0.000000,0.000000}%
\pgfsetfillcolor{currentfill}%
\pgfsetlinewidth{0.803000pt}%
\definecolor{currentstroke}{rgb}{0.000000,0.000000,0.000000}%
\pgfsetstrokecolor{currentstroke}%
\pgfsetdash{}{0pt}%
\pgfsys@defobject{currentmarker}{\pgfqpoint{0.000000in}{-0.048611in}}{\pgfqpoint{0.000000in}{0.000000in}}{%
\pgfpathmoveto{\pgfqpoint{0.000000in}{0.000000in}}%
\pgfpathlineto{\pgfqpoint{0.000000in}{-0.048611in}}%
\pgfusepath{stroke,fill}%
}%
\begin{pgfscope}%
\pgfsys@transformshift{3.778236in}{0.424000in}%
\pgfsys@useobject{currentmarker}{}%
\end{pgfscope}%
\end{pgfscope}%
\begin{pgfscope}%
\definecolor{textcolor}{rgb}{0.000000,0.000000,0.000000}%
\pgfsetstrokecolor{textcolor}%
\pgfsetfillcolor{textcolor}%
\pgftext[x=3.778236in,y=0.326778in,,top]{\color{textcolor}\rmfamily\fontsize{9.000000}{10.800000}\selectfont \(\displaystyle {5}\)}%
\end{pgfscope}%
\begin{pgfscope}%
\pgfsetbuttcap%
\pgfsetroundjoin%
\definecolor{currentfill}{rgb}{0.000000,0.000000,0.000000}%
\pgfsetfillcolor{currentfill}%
\pgfsetlinewidth{0.803000pt}%
\definecolor{currentstroke}{rgb}{0.000000,0.000000,0.000000}%
\pgfsetstrokecolor{currentstroke}%
\pgfsetdash{}{0pt}%
\pgfsys@defobject{currentmarker}{\pgfqpoint{0.000000in}{-0.048611in}}{\pgfqpoint{0.000000in}{0.000000in}}{%
\pgfpathmoveto{\pgfqpoint{0.000000in}{0.000000in}}%
\pgfpathlineto{\pgfqpoint{0.000000in}{-0.048611in}}%
\pgfusepath{stroke,fill}%
}%
\begin{pgfscope}%
\pgfsys@transformshift{4.361368in}{0.424000in}%
\pgfsys@useobject{currentmarker}{}%
\end{pgfscope}%
\end{pgfscope}%
\begin{pgfscope}%
\definecolor{textcolor}{rgb}{0.000000,0.000000,0.000000}%
\pgfsetstrokecolor{textcolor}%
\pgfsetfillcolor{textcolor}%
\pgftext[x=4.361368in,y=0.326778in,,top]{\color{textcolor}\rmfamily\fontsize{9.000000}{10.800000}\selectfont \(\displaystyle {10}\)}%
\end{pgfscope}%
\begin{pgfscope}%
\pgfsetbuttcap%
\pgfsetroundjoin%
\definecolor{currentfill}{rgb}{0.000000,0.000000,0.000000}%
\pgfsetfillcolor{currentfill}%
\pgfsetlinewidth{0.803000pt}%
\definecolor{currentstroke}{rgb}{0.000000,0.000000,0.000000}%
\pgfsetstrokecolor{currentstroke}%
\pgfsetdash{}{0pt}%
\pgfsys@defobject{currentmarker}{\pgfqpoint{0.000000in}{-0.048611in}}{\pgfqpoint{0.000000in}{0.000000in}}{%
\pgfpathmoveto{\pgfqpoint{0.000000in}{0.000000in}}%
\pgfpathlineto{\pgfqpoint{0.000000in}{-0.048611in}}%
\pgfusepath{stroke,fill}%
}%
\begin{pgfscope}%
\pgfsys@transformshift{4.944500in}{0.424000in}%
\pgfsys@useobject{currentmarker}{}%
\end{pgfscope}%
\end{pgfscope}%
\begin{pgfscope}%
\definecolor{textcolor}{rgb}{0.000000,0.000000,0.000000}%
\pgfsetstrokecolor{textcolor}%
\pgfsetfillcolor{textcolor}%
\pgftext[x=4.944500in,y=0.326778in,,top]{\color{textcolor}\rmfamily\fontsize{9.000000}{10.800000}\selectfont \(\displaystyle {15}\)}%
\end{pgfscope}%
\begin{pgfscope}%
\pgfsetbuttcap%
\pgfsetroundjoin%
\definecolor{currentfill}{rgb}{0.000000,0.000000,0.000000}%
\pgfsetfillcolor{currentfill}%
\pgfsetlinewidth{0.803000pt}%
\definecolor{currentstroke}{rgb}{0.000000,0.000000,0.000000}%
\pgfsetstrokecolor{currentstroke}%
\pgfsetdash{}{0pt}%
\pgfsys@defobject{currentmarker}{\pgfqpoint{-0.048611in}{0.000000in}}{\pgfqpoint{-0.000000in}{0.000000in}}{%
\pgfpathmoveto{\pgfqpoint{-0.000000in}{0.000000in}}%
\pgfpathlineto{\pgfqpoint{-0.048611in}{0.000000in}}%
\pgfusepath{stroke,fill}%
}%
\begin{pgfscope}%
\pgfsys@transformshift{3.195104in}{0.424000in}%
\pgfsys@useobject{currentmarker}{}%
\end{pgfscope}%
\end{pgfscope}%
\begin{pgfscope}%
\pgfsetbuttcap%
\pgfsetroundjoin%
\definecolor{currentfill}{rgb}{0.000000,0.000000,0.000000}%
\pgfsetfillcolor{currentfill}%
\pgfsetlinewidth{0.803000pt}%
\definecolor{currentstroke}{rgb}{0.000000,0.000000,0.000000}%
\pgfsetstrokecolor{currentstroke}%
\pgfsetdash{}{0pt}%
\pgfsys@defobject{currentmarker}{\pgfqpoint{-0.048611in}{0.000000in}}{\pgfqpoint{-0.000000in}{0.000000in}}{%
\pgfpathmoveto{\pgfqpoint{-0.000000in}{0.000000in}}%
\pgfpathlineto{\pgfqpoint{-0.048611in}{0.000000in}}%
\pgfusepath{stroke,fill}%
}%
\begin{pgfscope}%
\pgfsys@transformshift{3.195104in}{1.130667in}%
\pgfsys@useobject{currentmarker}{}%
\end{pgfscope}%
\end{pgfscope}%
\begin{pgfscope}%
\pgfsetbuttcap%
\pgfsetroundjoin%
\definecolor{currentfill}{rgb}{0.000000,0.000000,0.000000}%
\pgfsetfillcolor{currentfill}%
\pgfsetlinewidth{0.803000pt}%
\definecolor{currentstroke}{rgb}{0.000000,0.000000,0.000000}%
\pgfsetstrokecolor{currentstroke}%
\pgfsetdash{}{0pt}%
\pgfsys@defobject{currentmarker}{\pgfqpoint{-0.048611in}{0.000000in}}{\pgfqpoint{-0.000000in}{0.000000in}}{%
\pgfpathmoveto{\pgfqpoint{-0.000000in}{0.000000in}}%
\pgfpathlineto{\pgfqpoint{-0.048611in}{0.000000in}}%
\pgfusepath{stroke,fill}%
}%
\begin{pgfscope}%
\pgfsys@transformshift{3.195104in}{1.837333in}%
\pgfsys@useobject{currentmarker}{}%
\end{pgfscope}%
\end{pgfscope}%
\begin{pgfscope}%
\pgfsetbuttcap%
\pgfsetroundjoin%
\definecolor{currentfill}{rgb}{0.000000,0.000000,0.000000}%
\pgfsetfillcolor{currentfill}%
\pgfsetlinewidth{0.803000pt}%
\definecolor{currentstroke}{rgb}{0.000000,0.000000,0.000000}%
\pgfsetstrokecolor{currentstroke}%
\pgfsetdash{}{0pt}%
\pgfsys@defobject{currentmarker}{\pgfqpoint{-0.048611in}{0.000000in}}{\pgfqpoint{-0.000000in}{0.000000in}}{%
\pgfpathmoveto{\pgfqpoint{-0.000000in}{0.000000in}}%
\pgfpathlineto{\pgfqpoint{-0.048611in}{0.000000in}}%
\pgfusepath{stroke,fill}%
}%
\begin{pgfscope}%
\pgfsys@transformshift{3.195104in}{2.544000in}%
\pgfsys@useobject{currentmarker}{}%
\end{pgfscope}%
\end{pgfscope}%
\begin{pgfscope}%
\pgfpathrectangle{\pgfqpoint{3.195104in}{0.424000in}}{\pgfqpoint{1.749396in}{2.120000in}}%
\pgfusepath{clip}%
\pgfsetbuttcap%
\pgfsetmiterjoin%
\pgfsetlinewidth{0.752812pt}%
\definecolor{currentstroke}{rgb}{0.000000,0.000000,0.000000}%
\pgfsetstrokecolor{currentstroke}%
\pgfsetdash{}{0pt}%
\pgfpathmoveto{\pgfqpoint{3.195104in}{0.706667in}}%
\pgfpathlineto{\pgfqpoint{3.311730in}{0.706667in}}%
\pgfpathlineto{\pgfqpoint{3.311730in}{0.781053in}}%
\pgfpathlineto{\pgfqpoint{3.428357in}{0.781053in}}%
\pgfpathlineto{\pgfqpoint{3.428357in}{0.855439in}}%
\pgfpathlineto{\pgfqpoint{3.311730in}{0.855439in}}%
\pgfpathlineto{\pgfqpoint{3.311730in}{0.929825in}}%
\pgfpathlineto{\pgfqpoint{3.195104in}{0.929825in}}%
\pgfpathlineto{\pgfqpoint{3.195104in}{1.004211in}}%
\pgfpathlineto{\pgfqpoint{3.195104in}{1.004211in}}%
\pgfpathlineto{\pgfqpoint{3.195104in}{1.078596in}}%
\pgfpathlineto{\pgfqpoint{3.195104in}{1.078596in}}%
\pgfpathlineto{\pgfqpoint{3.195104in}{1.152982in}}%
\pgfpathlineto{\pgfqpoint{3.195104in}{1.152982in}}%
\pgfpathlineto{\pgfqpoint{3.195104in}{1.227368in}}%
\pgfpathlineto{\pgfqpoint{3.661610in}{1.227368in}}%
\pgfpathlineto{\pgfqpoint{3.661610in}{1.301754in}}%
\pgfpathlineto{\pgfqpoint{3.311730in}{1.301754in}}%
\pgfpathlineto{\pgfqpoint{3.311730in}{1.376140in}}%
\pgfpathlineto{\pgfqpoint{4.361368in}{1.376140in}}%
\pgfpathlineto{\pgfqpoint{4.361368in}{1.450526in}}%
\pgfpathlineto{\pgfqpoint{4.128115in}{1.450526in}}%
\pgfpathlineto{\pgfqpoint{4.128115in}{1.524912in}}%
\pgfpathlineto{\pgfqpoint{4.128115in}{1.524912in}}%
\pgfpathlineto{\pgfqpoint{4.128115in}{1.599298in}}%
\pgfpathlineto{\pgfqpoint{4.128115in}{1.599298in}}%
\pgfpathlineto{\pgfqpoint{4.128115in}{1.673684in}}%
\pgfpathlineto{\pgfqpoint{4.011489in}{1.673684in}}%
\pgfpathlineto{\pgfqpoint{4.011489in}{1.748070in}}%
\pgfpathlineto{\pgfqpoint{3.661610in}{1.748070in}}%
\pgfpathlineto{\pgfqpoint{3.661610in}{1.822456in}}%
\pgfpathlineto{\pgfqpoint{4.128115in}{1.822456in}}%
\pgfpathlineto{\pgfqpoint{4.128115in}{1.896842in}}%
\pgfpathlineto{\pgfqpoint{3.778236in}{1.896842in}}%
\pgfpathlineto{\pgfqpoint{3.778236in}{1.971228in}}%
\pgfpathlineto{\pgfqpoint{3.428357in}{1.971228in}}%
\pgfpathlineto{\pgfqpoint{3.428357in}{2.045614in}}%
\pgfpathlineto{\pgfqpoint{3.311730in}{2.045614in}}%
\pgfpathlineto{\pgfqpoint{3.311730in}{2.120000in}}%
\pgfpathlineto{\pgfqpoint{3.195104in}{2.120000in}}%
\pgfusepath{stroke}%
\end{pgfscope}%
\begin{pgfscope}%
\pgfpathrectangle{\pgfqpoint{3.195104in}{0.424000in}}{\pgfqpoint{1.749396in}{2.120000in}}%
\pgfusepath{clip}%
\pgfsetbuttcap%
\pgfsetroundjoin%
\pgfsetlinewidth{1.254687pt}%
\definecolor{currentstroke}{rgb}{0.000000,0.000000,0.000000}%
\pgfsetstrokecolor{currentstroke}%
\pgfsetdash{{3.750000pt}{3.750000pt}}{0.000000pt}%
\pgfpathmoveto{\pgfqpoint{3.195104in}{1.577886in}}%
\pgfpathlineto{\pgfqpoint{4.944500in}{1.577886in}}%
\pgfusepath{stroke}%
\end{pgfscope}%
\begin{pgfscope}%
\pgfsetrectcap%
\pgfsetmiterjoin%
\pgfsetlinewidth{0.803000pt}%
\definecolor{currentstroke}{rgb}{0.000000,0.000000,0.000000}%
\pgfsetstrokecolor{currentstroke}%
\pgfsetdash{}{0pt}%
\pgfpathmoveto{\pgfqpoint{3.195104in}{0.424000in}}%
\pgfpathlineto{\pgfqpoint{3.195104in}{2.544000in}}%
\pgfusepath{stroke}%
\end{pgfscope}%
\begin{pgfscope}%
\pgfsetrectcap%
\pgfsetmiterjoin%
\pgfsetlinewidth{0.803000pt}%
\definecolor{currentstroke}{rgb}{0.000000,0.000000,0.000000}%
\pgfsetstrokecolor{currentstroke}%
\pgfsetdash{}{0pt}%
\pgfpathmoveto{\pgfqpoint{4.944500in}{0.424000in}}%
\pgfpathlineto{\pgfqpoint{4.944500in}{2.544000in}}%
\pgfusepath{stroke}%
\end{pgfscope}%
\begin{pgfscope}%
\pgfsetrectcap%
\pgfsetmiterjoin%
\pgfsetlinewidth{0.803000pt}%
\definecolor{currentstroke}{rgb}{0.000000,0.000000,0.000000}%
\pgfsetstrokecolor{currentstroke}%
\pgfsetdash{}{0pt}%
\pgfpathmoveto{\pgfqpoint{3.195104in}{0.424000in}}%
\pgfpathlineto{\pgfqpoint{4.944500in}{0.424000in}}%
\pgfusepath{stroke}%
\end{pgfscope}%
\begin{pgfscope}%
\pgfsetrectcap%
\pgfsetmiterjoin%
\pgfsetlinewidth{0.803000pt}%
\definecolor{currentstroke}{rgb}{0.000000,0.000000,0.000000}%
\pgfsetstrokecolor{currentstroke}%
\pgfsetdash{}{0pt}%
\pgfpathmoveto{\pgfqpoint{3.195104in}{2.544000in}}%
\pgfpathlineto{\pgfqpoint{4.944500in}{2.544000in}}%
\pgfusepath{stroke}%
\end{pgfscope}%
\begin{pgfscope}%
\definecolor{textcolor}{rgb}{0.000000,0.000000,0.000000}%
\pgfsetstrokecolor{textcolor}%
\pgfsetfillcolor{textcolor}%
\pgftext[x=4.706396in, y=2.484758in, left, base,rotate=270.000000]{\color{textcolor}\rmfamily\fontsize{9.000000}{10.800000}\selectfont \(\displaystyle m = 276.33\)~mg}%
\end{pgfscope}%
\begin{pgfscope}%
\definecolor{textcolor}{rgb}{0.000000,0.000000,0.000000}%
\pgfsetstrokecolor{textcolor}%
\pgfsetfillcolor{textcolor}%
\pgftext[x=4.558854in, y=2.504047in, left, base,rotate=270.000000]{\color{textcolor}\rmfamily\fontsize{9.000000}{10.800000}\selectfont \(\displaystyle s_{n-1} = 3.81\)~mg}%
\end{pgfscope}%
\end{pgfpicture}%
\makeatother%
\endgroup%

  \caption{Grafico di dispersione dei dati mostrati in
    figura~\ref{fig:chiodini_scatter} con l'aggiunta dei due istogrammi
    uni-dimensionali delle due quantità in gioco. Le linee tratteggiate
    indicano le medie campione delle due variabili e, per completezza, i
    valori numerici delle medie stesse sono riportate nei due istogrammi insieme
    a quelli delle stime imparziali delle rispettive deviazioni standard.}
  \label{fig:chiodini_scatter_hist}
\end{figure}

Supponiamo adesso di voler stimare, a partire dai dati a nostra disposizione,
la densità lineare dei nostri chiodini
\begin{align*}
  \lambda = \frac{m}{l}.
\end{align*}
(Se avessimo misurato il diametro dei chiodini, e assumendo i chiodini stessi
perfettamente simmetrici, avremmo potuto anche stimare la densità del
materiale propriamente detta, ma per i nostri fini si tratta di un dettaglio.)
Al solito prenderemo la media campione e la stima della deviazione
standard della media come valore centrale ed incertezza associata alla misura
delle due grandezze di partenza e, in base alle informazioni in
figura~\ref{fig:chiodini_scatter_hist}, scriveremo
\begin{align*}
  m = \hat{m} \pm s_m = 276.33 \pm 0.45~\text{mg} \quad \text{e} \quad
  l = \hat{l} \pm s_l = 17.111 \pm 0.026~\text{mm}
\end{align*}
(abbiamo diviso le stime delle deviazioni standard del campione per
$\sqrt{70}$ per passare alla deviazione standard della media). Ora, se
utilizzassimo acriticamente le formule per la propagazione degli errori per
variabili indipendenti scriveremmo
\begin{align*}
  \hat\lambda = \frac{\hat{m}}{\hat{l}}
  \quad \text{e} \quad
  \sigma^2_\lambda = \frac{\hat{m}^2\sigma_l^2 + \hat{l}^2\sigma_m^2}{\hat{l}^4}
  \quad \text{ovverosia} \quad
  \lambda = 16.149 \pm 0.036~\text{g~m}^{-1}.
\end{align*}
Se invece dividiamo, chiodino per chiodino, il valore della massa per il
valore corrispondente della lunghezza, la deviazione standard della media
per il campione risultante, che è rappresentato nell'istogramma in
figura~\ref{fig:chiodini_lambda} è
\begin{align*}
  s_\lambda = 0.014,
\end{align*}
vale a dire circa due volte e mezzo più piccolo dell'errore propagato. Da
dove viene, allora, la discrepanza?

\pgffigone{chiodini_lambda}{
  Istogramma dei valori di densità lineare $\lambda_i = \nicefrac{m_i}{l_i}$
  ottenuti dividendo, chiodino per chiodino, i valori misurata di massa e
  lunghezza---per l'intero campione mostrato nel diagramma di dispersione in
  figura~\ref{fig:chiodini_scatter_hist}. Per completezza i valori della media
  campione e della stima imparziale della deviazione standard sono
  riportate esplicitamente.
}

La risposta è semplice: $m$ e $l$ hanno un coefficiente di correlazione
diverso da zero, per cui non sono indipendenti, e
la~\eqref{eq:propagazione_errore_stat_ind} non si applica. Dobbiamo invece
utilizzare la formula completa~\eqref{eq:propagazione_errore_stat}
\begin{align*}
  \sigma^2_\lambda =
  %\left( \pd{\lambda}{m}{\hat{m}, \hat{l}} \right)^2 \sigma_m^2 +
  %\left( \pd{\lambda}{l}{\hat{m}, \hat{l}} \right)^2 \sigma_l^2 +
  %2r_{ml} \pd{\lambda}{m}{\hat{m}, \hat{l}}
  %\pd{\lambda}{l}{\hat{m}, \hat{l}} \sigma_m \sigma_l =
  \frac{\hat{m}^2 \sigma_l^2 + \hat{l}^2 \sigma_m^2 -
    2r_{ml}\hat{m}\hat{l}\sigma_m \sigma_l}{\hat{l}^4}
\end{align*}
che, non per caso, fornisce un valore
\begin{align*}
  \sigma_\lambda = 0.014
\end{align*}
consistente con la stima $s_\lambda$ fatta direttamente sul campione.


\section{Il disegno degli esperimenti}

Una delle applicazioni naturali della propagazione degli errori è la
progettazione di esperimenti, che introduciamo qui in modo informale.


\subsection{Un esempio storico}

Nei suoi \emph{"Discorsi e dimostrazioni matematiche intorno a due nuove scienze
attenenti alla meccanica e i movimenti locali"}, pubblicato nel 1638, Galileo
propone uno schema di esperimento per la misura della velocità della luce:

\begin{quotation}
  La poca concludenza di queste e di altre simili osservazioni mi fece
  una volta pensare a qualche modo di poterci senza errore accertar, se
  l'illuminazione, cioè se l'espansion del lume, fusse veramente instantanea;
  poiché il moto assai veloce del suono ci assicura, quella della luce
  non poter esser se non velocissima: e l'esperienza che mi sovvenne, fu tale.
  Voglio che due piglino un lume per uno, il quale, tenendolo dentro lanterna
  o altro ricetto, possino andar coprendo e scoprendo, con l'interposizion
  della mano, alla vista del compagno, e che, ponendosi l'uno incontro
  all'altro in distanza di poche braccia, vadano addestrandosi nello scoprire
  ed occultare il lor lume alla vista del compagno, sì che quando l'uno
  vede il lume dell'altro, immediatamente scuopra il suo; la qual corrispondenza,
  dopo alcune risposte fattesi scambievolmente, verrà loro talmente aggiustata,
  che, senza sensibile svario, alla scoperta dell'uno risponderà immediatamente
  la scoperta dell'altro, sì che quando l'uno scuopre il suo lume, vedrà
  nell'istesso tempo comparire alla sua vista il lume dell'altro. Aggiustata cotal
  pratica in questa piccolissima distanza, pongansi i due medesimi compagni con due simili
  lumi in lontananza di due o tre miglia, e tornando di notte a far l'istessa esperienza,
  vadano osservando attentamente se le risposte delle loro scoperte ed
  occultazioni seguono secondo l'istesso tenore che facevano da vicino; che
  seguendo, si potrà assai sicuramente concludere, l'espansion del lume essere
  instantanea: ché quando ella ricercasse tempo, in una lontananza di tre
  miglia, che importano sei per l'andata d'un lume e venuta dell'altro,
  la dimora dovrebbe esser assai osservabile. E quando si volesse far tal osservazione
  in distanze maggiori, cioè di otto o dieci miglia, potremmo servirci del telescopio,
  aggiustandone un per uno gli osservatori al luogo dove la notte si hanno a mettere
  in pratica i lumi; li quali, ancor che non molto grandi, e per ciò
  invisibili in tanta lontananza all'occhio libero, ma ben facili a coprirsi
  e scoprirsi, con l'aiuto de i telescopii già aggiustati e fermati potranno
  esser commodamente veduti.
\end{quotation}

Sebbene questo passaggio sia lontano dagli standard moderni, è interessante
notare come siano presenti, spiegati chiaramente ed in modo sintetico, tutti gli
elementi qualificanti del metodo scientifico: le motivazioni ed il principio della
misura, la \emph{calibrazione} dell'apparato, e, soprattutto, la verifica della
consistenza interna---non basta fare la misura una volta, ma è necessario
ripeterla ad una seconda distanza, più lunga, per assicurarsi che l'effetto
scali correttamente.

Parafrasando in notazione moderna il capoverso di Galileo, la proposta è
essenzialmente quella di misurare (manualmente) il tempo di transito $\tau$
che la luce impiega a percorrere (avanti e indietro) una distanza nota $d$
\begin{align*}
  c = \frac{2d}{\tau}.
\end{align*}
A posteriori sappiamo che, essendo la velocità della luce $c \approx 300000$~km/s,
il tempo di transito sul doppio di una distanza di $5$~km sarebbe dell'ordine di
$\tau \sim 3~\mu$s---Galileo non aveva alcuna speranza di misurare l'effetto con
gli strumenti a sua disposizione. (Oggi la misura si potrebbe fare comodamente
su un tavolo, con poca elettronica a basso costo, ma questo è un altro discorso.)

Se esaminiamo la situazione dal punto di vista di Galileo---ovvero di chi
deve pianificare la misura di una grandezza fisica di cui non si conosce il valore,
nemmeno approssimativamente---il discorso è completamente diverso. Ed è qui che
entra in gioco la propagazione degli errori. Assumendo che l'incertezza relativa
sulla misura di $d$ sia molto più piccola di quella su $\tau$ (che è più o
meno garantito in questo caso), la situazione è molto semplice:
\begin{align}\label{eq:c_err}
  \frac{\sigma^2_c}{\hat{c}^2} =
  \frac{\sigma^2_d}{\hat{d}^2} + \frac{\sigma^2_\tau}{\hat{\tau}^2}
  \approx \frac{\sigma^2_\tau}{\hat{\tau}^2} \quad \text{ovvero} \quad
  \frac{\sigma_c}{\hat{c}} \approx \frac{\sigma_\tau}{\hat{\tau}} =
  \frac{\hat{c}}{2\hat{d}} \sigma_\tau.
\end{align}
Parafrasando: l'errore relativo su $c$ è direttamente proporzionale all'incertezza
su $\tau$ ed alla velocità (incognita) della luce, ed inversamente proporzionale
alla distanza $d$. (Siccome su $c$ non abbiamo nessun controllo, quello che
possiamo cercare di fare è di massimizzare la precisione nella misura dei tempi
e la lunghezza di propagazione---ma questo lo sapevamo anche senza scrivere
equazioni.)

\pgffigone{misura_c}{Dipendenza dell'errore relativo sulla misura della
velocità della luce $c$ dall'incertezza di misura su $\tau$, nell'esperimento
di Galileo, per diversi valori ipotetici di $c$ e per una distanza $d=5$~km.
La linea grigia tratteggiata rappresenta il valore esatto della velocità della
luce $c = 299792458$~m~s$^{-1}$, che ovviamente Galileo non poteva sapere.
L'estremo superiore dell'intervallo dinamico sull'asse $y$ è fissato a $1$
(corrispondente ad un'incertezza relativa del $100\%$) perché oltre questo
valore la misura è sostanzialmente impossibile.}

Con un minimo di fantasia si può rappresentare questa informazione in un
grafico come quello mostrato in figura~\ref{fig:misura_c}, in cui abbiamo
l'incertezza assoluta sulle misure di tempo $\sigma_\tau$ sull'asse delle $x$,
e l'incertezza relativa sulla misura di $c$ su quello delle $y$. Fissata la
distanza $d$ (in questo caso $5$~km), per ogni valore di $c$ le due grandezze
sono legate da una relazione di proporzionalità diretta, secondo
la~\eqref{eq:c_err}. Un modo ovvio di leggere questo grafico (che è sostanzialmente
il modo in cui lo presenteremmo ad un'agenzia finanziatrice) è quello di
tracciare una linea verticale in corrispondenza della risoluzione temporale del
nostro apparato di misura, e vedere come questa interseca le linee a $c$ fissata.
Assumendo (ottimisticamente) che con il metodo di Galileo si possa raggiungere
una precisione di un decimo di secondo ($\sigma_\tau = 0.1$~s), potremmo
riformulare la sua proposta in termini moderni dicendo che l'esperimento
permette di misurare $c$ all'$1\%$ se $c \sim 1000$~m~s$^{-1}$; nel caso
$c \sim 10000$~m~s$^{-1}$ probabilmente possiamo ancora fare una misura
ragionevole, ma per $c \gg 10000$~m~s$^{-1}$ la nostra risoluzione temporale
semplicemente non è sufficiente. In quel caso, se non misuriamo niente,
possiamo mettere un \emph{limite inferiore} su $c$.


\subsection{Un esempio più interessante}

Supponiamo di avere un apparato sperimentale, come quello illustrato in
figura~\ref{fig:tracciatore}, composto da una sorgente infinitamente sottile
centrata nell'origine del nostro sistema di coordinate e due piani di rivelazione
posti a distanza $x_1$ ed $x_2$. La sorgente emette particelle cariche---che, per
semplicità, assumeremo viaggiare su una linea retta---ed i piani di
rivelazione misurano le coordinate di passaggio $y_1$ ed $y_2$ con incertezza
di misura $\sigma_{y_1}$ e $\sigma_{y_2}$. Lo scopo è quello di ricostruire la coordinata
di partenza $y_0$ in cui è stata emessa la particella.

\begin{figure}[!htbp]
  \center\begin{tikzpicture}[main node/.style={rectangle,draw,inner sep=0.,outer sep=0.}]%
  \draw (0., -0.75)--(0., 0.75);
  \draw (2., -2.)--(2., 2.);
  \draw (12., -2.)--(12., 2.);
  \draw[->] (-1., 0.)--(15., 0.);
  \node[anchor=north west] at (0., 0.) {$0$};
  \node[anchor=north west] at (2., 0.) {$x_1$};
  \node[anchor=north west] at (12., 0.) {$x_2$};
  \draw[dashed] (0., 0.25)--(15., 1.5);
  \node at (0., 2.5) {Sorgente};
  \node at (2., 2.5) {Piano 1};
  \node at (12., 2.5) {Piano 2};
  \node[anchor=south east] at (0., 0.25) {$(0, y_0)$};
  \node[anchor=south east] at (2., 0.5) {$(x_1, y_1)$};
  \node[anchor=south east] at (12., 1.25) {$(x_2, y_2)$};
\end{tikzpicture}

  \caption{Illustrazione schematica di un \emph{tracciatore} di particelle a due
  piani per la determinazione della coordinata $y_0$ di partenza della particella.}
  \label{fig:tracciatore}
\end{figure}

Non dovrebbe essere difficile convincersi che l'intercetta della retta passante
per i due punti misurati, che coincide con coordinata cercata, si scrive come
\begin{align*}
  y_0 = y_1 - x_1 \frac{y_2 - y_1}{x_2 - x_1} = \frac{y_1x_2 - y_2x_1}{x_2 - x_1},
\end{align*}
e l'incertezza associata è
\begin{align}\label{eq:errore_tracciatore}
  \sigma^2_{y_0} = \left( \pd{y_0}{y_1}{\hat{y}_1, \hat{y}_2}\right)^2 \sigma^2_{y_1} +
  \left( \pd{y_0}{y_2}{\hat{y}_1, \hat{y}_2}\right)^2 \sigma^2_{y_2} =
  %\left( \frac{x_2}{x_2 - x_1} \right)^2 \sigma^2_{y_1} +
  %\left( \frac{x_1}{x_2 - x_1} \right)^2 \sigma^2_{y_2} =
  \frac{\hat{x}_2^2 \sigma^2_{y_1} + \hat{x}_1^2 \sigma^2_{y_2}}{(\hat{x}_2 - \hat{x}_1)^2}.
\end{align}
La~\eqref{eq:errore_tracciatore} è \emph{rivelatrice}: non solo ci permette
di stimare l'incertezza sulla quantità misurata, data una generica configurazione
sperimentale---ci dice esattamente come questa incertezza dipende dalle grandezze
in gioco, e ci guida nella progettazione dell'apparato. Si tratta, formalmente,
di una funzione di 4 variabili ($x_1$, $x_2$, $\sigma_{y_1}$ e $\sigma_{y_2}$),
che non è banale da visualizzare in tutta la sua ricchezza. Vi è, tuttavia,
un certo numero di cose che possiamo dire immediatamente, senza fare ulteriori assunzioni:
\begin{itemize}
  \item dato che $x_2 > x_1$, la risoluzione spaziale del piano più vicino alla
  sorgente $\sigma_{y_1}$ è più critica di quella del rivelatore lontano e
  al limite, quando $x_2 \gg x_1$, quest'ultima diviene irrilevante
  \begin{align*}
    \lim_{x_2 \gg x_1} \sigma_{y_0} = \sigma_{y_1}
  \end{align*}
  (in altre parole: \emph{il rivelatore vicino alla sorgente deve essere più preciso
  di quello lontano}---e molto più preciso se quello lontano è molto più
  lontano);
  \item la~\eqref{eq:errore_tracciatore} è una funzione monotona crescente di
  $\hat{x}_1$, come si può verificare calcolando la derivata
  \begin{align*}
    \pd{\sigma^2_{y_0}}{x_1}{} =
    2\hat{x}_2 \frac{\left( \hat{x}_1\sigma^2_{y_2} + \hat{x}_2\sigma^2_{y_1} \right)}%
    {(\hat{x}_2 - \hat{x}_1)^3} > 0,
  \end{align*}
  per cui \emph{il rivelatore vicino alla sorgente deve essere il più vicino
  possibile alla sorgente};
  \item analogamente, la~\eqref{eq:errore_tracciatore} è una funzione monotona
  decrescente di $\hat{x}_2$
  \begin{align*}
    \pd{\sigma^2_{y_0}}{x_2}{} =
    - 2\hat{x}_1 \frac{\left( \hat{x}_1\sigma^2_{y_2} + \hat{x}_2\sigma^2_{y_1} \right)}%
    {(\hat{x}_2 - \hat{x}_1)^3} < 0,
  \end{align*}
  ovvero \emph{il rivelatore lontano dalla sorgente deve essere il più lontano
  possibile dalla sorgente}.
\end{itemize}
Ovviamente, nella vita reale, queste considerazioni generali si scontrano
con i problemi pratici del caso: migliorare la risoluzione dei piani di rivelazione
è potenzialmente costoso, ha un impatto in termini di massa e di potenza assorbita e,
oltre un certo livello, è tecnologicamente impossibile; e le distanze minime e
massime dei piani dalla sorgente sono soggette ai vincoli logistici del resto
dell'apparato sperimentale. \`E proprio questo che rende la progettazione di
esperimenti una materia complessa e stimolante.


\summary

\begin{itemize}
\item Possiamo modellizzare il processo di misura di una grandezza fisica in
  condizioni di ripetitività come il campionamento di una distribuzione
  di probabilità---che chiamiamo distribuzione generatrice. In questo schema
  il valore centrale e l'incertezza di misura coincidono con la nostra miglior
  stima della media e della deviazione standard della distribuzione generatrice.
  Ne segue che l'intervallo determinato dall'incertezza di misura non è
  garantito contenere il misurando, ma ha una probabilità data di contenerlo.
\item Nella somma di variabili casuali indipendenti la media della somma è
  uguale alla somma delle medie e la varianza della somma è uguale alla somma
  delle varianze. Ne segue che le deviazioni standard si sommano in
  quadratura.
\item Nel caso di $n$ misure ripetute di una stessa grandezza in condizioni di
  dispersione non nulle la media e la varianza campione
  \begin{align*}
    m = \frac{1}{n}\sum_{i = 1}^n x_i
    \quad\text{e}\quad
    s^2_{n - 1} = \frac{1}{(n - 1)}\sum_{i = 1}^n (x_i - m)^2
  \end{align*}
  sono le nostre migliori stime della media e della varianza della distribuzione
  generatrice.
\item La deviazione standard della media di $n$ campionamenti di una
  distribuzione generatrice data è $\sqrt{n}$ volte più piccola della
  deviazione standard della distribuzione stessa. Questa è la legge
  fondamentale con cui scalano gli errori statistici in funzione della
  dimensione del campione.
\item Gli errori statistici su grandezze indipendenti si sommano in
  quadratura---e non linearmente come gli errori massimi
  \begin{align*}
    s_f = \sqrt{\sum_{i = 1}^n
      \left(\pd{f}{x_i}{\hat{x}_1,\ldots,\hat{x}_n}\right)^2 s^2_i
    }.
  \end{align*}
  Questo deriva essenzialmente dal fatto che se le grandezze sono indipendenti
  le fluttuazioni statistiche tendono in media a compensarsi ed è
  relativamente poco probabile che esse tendano a contribuire nello stesso
  verso per generare configurazioni estreme. Ne segue che l'errore massimo
  tende sistematicamente ad essere una sovrastima dell'incertezza di misura.
\end{itemize}
