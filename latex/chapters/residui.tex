\chapter{Il grafico dei residui perfetto}

Come abbiamo visto nella sezione~\ref{sec:grafico_residui}, il grafico dei residui
è uno strumento grafico fondamentale per verificare la bontà di un \fit.
Dato che la creazione di un buon grafico dei residui richiede la padronanza
di un certo numero di comandi di \matplotlib, in questa appendice forniamo un
esempio completo, auto-contenuto ed ampiamente commentato.

Benché, strettamente parlando, un grafico di dispersione ed il corrispondente
grafico dei residui possano \emph{vivere} indipendentemente, ciascuno nella propria
figura, il massimo dell'effetto estetico ed informativo si ottiene combinando i
due, sovrapposti verticalmente, come mostrato ad esempio
in figura~\ref{fig:grafico_dei_residui}.
Gran parte della complessità nella realizzazione di questo tipo di figura deriva
dalla necessità di allineare propriamente i grafici, le divisioni e le etichette
degli assi, ed eventualmente le griglie.

\pgffigone{grafico_dei_residui}{
  Esempio di grafico dei residui (adattato dalla figura~\ref{fig:chi2_test_residui}).
  Questa figura illustra la maggior parte degli elementi stilistici propri di
  un buon grafico dei residui, ed il frammento di codice in
  figura~\ref{fig:grafico_residui} mostra i comandi essenziali
  di \matplotlib\ per produrlo.
}

\begin{figure}[htbp!]
  \begin{Verbatim}[label=\makebox{\href{https://bitbucket.org/lbaldini/statnotes/src/master/snippets/residual\_plot.py}{https://bitbucket.org/.../residual\_plot.py}},commandchars=\\\{\}]
\PY{k+kn}{import} \PY{n+nn}{numpy} \PY{k}{as} \PY{n+nn}{np}
\PY{k+kn}{import} \PY{n+nn}{matplotlib}\PY{n+nn}{.}\PY{n+nn}{pyplot} \PY{k}{as} \PY{n+nn}{plt}
\PY{k+kn}{from} \PY{n+nn}{scipy}\PY{n+nn}{.}\PY{n+nn}{optimize} \PY{k+kn}{import} \PY{n}{curve\PYZus{}fit}

\PY{c+c1}{\PYZsh{} Sample input data for the plot and the fit\PYZhy{}\PYZhy{}\PYZhy{}use your data points and errors :\PYZhy{})}
\PY{n}{x} \PY{o}{=} \PY{n}{np}\PY{o}{.}\PY{n}{array}\PY{p}{(}\PY{p}{[}\PY{l+m+mf}{0.5}\PY{p}{,} \PY{l+m+mf}{1.1}\PY{p}{,} \PY{l+m+mf}{1.7}\PY{p}{,} \PY{l+m+mf}{2.3}\PY{p}{,} \PY{l+m+mf}{2.9}\PY{p}{,} \PY{l+m+mf}{3.5}\PY{p}{,} \PY{l+m+mf}{4.1}\PY{p}{,} \PY{l+m+mf}{4.7}\PY{p}{,} \PY{l+m+mf}{5.3}\PY{p}{,} \PY{l+m+mf}{5.9}\PY{p}{,} \PY{l+m+mf}{6.5}\PY{p}{,} \PY{l+m+mf}{7.1}\PY{p}{,} \PY{l+m+mf}{7.7}\PY{p}{,} \PY{l+m+mf}{8.3}\PY{p}{,} \PY{l+m+mf}{8.9}\PY{p}{,} \PY{l+m+mf}{9.5}\PY{p}{]}\PY{p}{)}
\PY{n}{y} \PY{o}{=} \PY{n}{np}\PY{o}{.}\PY{n}{array}\PY{p}{(}\PY{p}{[}\PY{l+m+mi}{35}\PY{p}{,} \PY{l+m+mi}{61}\PY{p}{,} \PY{l+m+mi}{76}\PY{p}{,} \PY{l+m+mi}{70}\PY{p}{,} \PY{l+m+mi}{92}\PY{p}{,} \PY{l+m+mi}{100}\PY{p}{,} \PY{l+m+mi}{111}\PY{p}{,} \PY{l+m+mi}{113}\PY{p}{,} \PY{l+m+mi}{140}\PY{p}{,} \PY{l+m+mi}{163}\PY{p}{,} \PY{l+m+mi}{177}\PY{p}{,} \PY{l+m+mi}{209}\PY{p}{,} \PY{l+m+mi}{223}\PY{p}{,} \PY{l+m+mi}{264}\PY{p}{,} \PY{l+m+mi}{290}\PY{p}{,} \PY{l+m+mi}{312}\PY{p}{]}\PY{p}{)}
\PY{n}{sigma} \PY{o}{=} \PY{n}{np}\PY{o}{.}\PY{n}{full}\PY{p}{(}\PY{n}{y}\PY{o}{.}\PY{n}{shape}\PY{p}{,} \PY{l+m+mf}{10.0}\PY{p}{)}

\PY{c+c1}{\PYZsh{} Definition of the model for fitting the data.}
\PY{k}{def} \PY{n+nf}{model}\PY{p}{(}\PY{n}{x}\PY{p}{,} \PY{n}{a}\PY{p}{,} \PY{n}{b}\PY{p}{,} \PY{n}{c}\PY{p}{)}\PY{p}{:}
    \PY{l+s+sd}{\PYZdq{}\PYZdq{}\PYZdq{}Simple quadratic fitting model.}
\PY{l+s+sd}{    \PYZdq{}\PYZdq{}\PYZdq{}}
    \PY{k}{return} \PY{n}{a} \PY{o}{+} \PY{n}{b} \PY{o}{*} \PY{n}{x} \PY{o}{+} \PY{n}{c} \PY{o}{*} \PY{n}{x}\PY{o}{*}\PY{o}{*}\PY{l+m+mf}{2.}

\PY{c+c1}{\PYZsh{} Perform the fit...}
\PY{n}{popt}\PY{p}{,} \PY{n}{pcov} \PY{o}{=} \PY{n}{curve\PYZus{}fit}\PY{p}{(}\PY{n}{model}\PY{p}{,} \PY{n}{x}\PY{p}{,} \PY{n}{y}\PY{p}{,} \PY{n}{sigma}\PY{o}{=}\PY{n}{sigma}\PY{p}{)}
\PY{c+c1}{\PYZsh{} ...and calculate the residuals with respect to the best\PYZhy{}fit model.}
\PY{n}{res} \PY{o}{=} \PY{n}{y} \PY{o}{\PYZhy{}} \PY{n}{model}\PY{p}{(}\PY{n}{x}\PY{p}{,} \PY{o}{*}\PY{n}{popt}\PY{p}{)}

\PY{c+c1}{\PYZsh{} Create the main figure...}
\PY{n}{fig} \PY{o}{=} \PY{n}{plt}\PY{o}{.}\PY{n}{figure}\PY{p}{(}\PY{l+s+s1}{\PYZsq{}}\PY{l+s+s1}{Un grafico dei residui}\PY{l+s+s1}{\PYZsq{}}\PY{p}{)}
\PY{c+c1}{\PYZsh{} ...and make space for the two plots. Note that `gridspec\PYZus{}kw` and `hspace`}
\PY{c+c1}{\PYZsh{} control the arrangements of the two sub\PYZhy{}panels within the figure, see}
\PY{c+c1}{\PYZsh{} https://matplotlib.org/stable/api/\PYZus{}as\PYZus{}gen/matplotlib.gridspec.GridSpec.html}
\PY{n}{ax1}\PY{p}{,} \PY{n}{ax2} \PY{o}{=} \PY{n}{fig}\PY{o}{.}\PY{n}{subplots}\PY{p}{(}\PY{l+m+mi}{2}\PY{p}{,} \PY{l+m+mi}{1}\PY{p}{,} \PY{n}{sharex}\PY{o}{=}\PY{k+kc}{True}\PY{p}{,} \PY{n}{gridspec\PYZus{}kw}\PY{o}{=}\PY{n+nb}{dict}\PY{p}{(}\PY{n}{height\PYZus{}ratios}\PY{o}{=}\PY{p}{[}\PY{l+m+mi}{2}\PY{p}{,} \PY{l+m+mi}{1}\PY{p}{]}\PY{p}{,} \PY{n}{hspace}\PY{o}{=}\PY{l+m+mf}{0.05}\PY{p}{)}\PY{p}{)}

\PY{c+c1}{\PYZsh{} Main plot: the scatter plot of x vs. y, on the top panel.}
\PY{n}{ax1}\PY{o}{.}\PY{n}{errorbar}\PY{p}{(}\PY{n}{x}\PY{p}{,} \PY{n}{y}\PY{p}{,} \PY{n}{sigma}\PY{p}{,} \PY{n}{fmt}\PY{o}{=}\PY{l+s+s1}{\PYZsq{}}\PY{l+s+s1}{o}\PY{l+s+s1}{\PYZsq{}}\PY{p}{,} \PY{n}{label}\PY{o}{=}\PY{l+s+s1}{\PYZsq{}}\PY{l+s+s1}{Dati}\PY{l+s+s1}{\PYZsq{}}\PY{p}{)}
\PY{c+c1}{\PYZsh{} Plot the best\PYZhy{}fit model on a dense grid.}
\PY{n}{xgrid} \PY{o}{=} \PY{n}{np}\PY{o}{.}\PY{n}{linspace}\PY{p}{(}\PY{l+m+mf}{0.0}\PY{p}{,} \PY{l+m+mf}{10.0}\PY{p}{,} \PY{l+m+mi}{100}\PY{p}{)}
\PY{n}{ax1}\PY{o}{.}\PY{n}{plot}\PY{p}{(}\PY{n}{xgrid}\PY{p}{,} \PY{n}{model}\PY{p}{(}\PY{n}{xgrid}\PY{p}{,} \PY{o}{*}\PY{n}{popt}\PY{p}{)}\PY{p}{,} \PY{n}{label}\PY{o}{=}\PY{l+s+s1}{\PYZsq{}}\PY{l+s+s1}{Modello di best\PYZhy{}fit}\PY{l+s+s1}{\PYZsq{}}\PY{p}{)}
\PY{c+c1}{\PYZsh{} Setup the axes, grids and legend.}
\PY{n}{ax1}\PY{o}{.}\PY{n}{set\PYZus{}ylabel}\PY{p}{(}\PY{l+s+s1}{\PYZsq{}}\PY{l+s+s1}{y [a. u.]}\PY{l+s+s1}{\PYZsq{}}\PY{p}{)}
\PY{n}{ax1}\PY{o}{.}\PY{n}{grid}\PY{p}{(}\PY{n}{color}\PY{o}{=}\PY{l+s+s1}{\PYZsq{}}\PY{l+s+s1}{lightgray}\PY{l+s+s1}{\PYZsq{}}\PY{p}{,} \PY{n}{ls}\PY{o}{=}\PY{l+s+s1}{\PYZsq{}}\PY{l+s+s1}{dashed}\PY{l+s+s1}{\PYZsq{}}\PY{p}{)}
\PY{n}{ax1}\PY{o}{.}\PY{n}{legend}\PY{p}{(}\PY{p}{)}

\PY{c+c1}{\PYZsh{} And now the residual plot, on the bottom panel.}
\PY{n}{ax2}\PY{o}{.}\PY{n}{errorbar}\PY{p}{(}\PY{n}{x}\PY{p}{,} \PY{n}{res}\PY{p}{,} \PY{n}{sigma}\PY{p}{,} \PY{n}{fmt}\PY{o}{=}\PY{l+s+s1}{\PYZsq{}}\PY{l+s+s1}{o}\PY{l+s+s1}{\PYZsq{}}\PY{p}{)}
\PY{c+c1}{\PYZsh{} This will draw a horizontal line at y=0, which is the equivalent of the best\PYZhy{}fit}
\PY{c+c1}{\PYZsh{} model in the residual representation.}
\PY{n}{ax2}\PY{o}{.}\PY{n}{plot}\PY{p}{(}\PY{n}{xgrid}\PY{p}{,} \PY{n}{np}\PY{o}{.}\PY{n}{full}\PY{p}{(}\PY{n}{xgrid}\PY{o}{.}\PY{n}{shape}\PY{p}{,} \PY{l+m+mf}{0.0}\PY{p}{)}\PY{p}{)}
\PY{c+c1}{\PYZsh{} Setup the axes, grids and legend.}
\PY{n}{ax2}\PY{o}{.}\PY{n}{set\PYZus{}xlabel}\PY{p}{(}\PY{l+s+s1}{\PYZsq{}}\PY{l+s+s1}{x [a. u.]}\PY{l+s+s1}{\PYZsq{}}\PY{p}{)}
\PY{n}{ax2}\PY{o}{.}\PY{n}{set\PYZus{}ylabel}\PY{p}{(}\PY{l+s+s1}{\PYZsq{}}\PY{l+s+s1}{Residuals [a. u.]}\PY{l+s+s1}{\PYZsq{}}\PY{p}{)}
\PY{n}{ax2}\PY{o}{.}\PY{n}{grid}\PY{p}{(}\PY{n}{color}\PY{o}{=}\PY{l+s+s1}{\PYZsq{}}\PY{l+s+s1}{lightgray}\PY{l+s+s1}{\PYZsq{}}\PY{p}{,} \PY{n}{ls}\PY{o}{=}\PY{l+s+s1}{\PYZsq{}}\PY{l+s+s1}{dashed}\PY{l+s+s1}{\PYZsq{}}\PY{p}{)}

\PY{c+c1}{\PYZsh{} The final touch to main canvas :\PYZhy{})}
\PY{n}{plt}\PY{o}{.}\PY{n}{xlim}\PY{p}{(}\PY{l+m+mf}{0.0}\PY{p}{,} \PY{l+m+mf}{10.0}\PY{p}{)}
\PY{n}{fig}\PY{o}{.}\PY{n}{align\PYZus{}ylabels}\PY{p}{(}\PY{p}{(}\PY{n}{ax1}\PY{p}{,} \PY{n}{ax2}\PY{p}{)}\PY{p}{)}

\PY{n}{plt}\PY{o}{.}\PY{n}{show}\PY{p}{(}\PY{p}{)}
\end{Verbatim}
  \caption{Frammento di codice per la generazione di una figura che combina un
  grafico di dispersione con il relativo grafico dei residui, allineati verticalmente.
  Se avete necessità di copiare ed incollare il codice sul vostro \foreign{editor}
  di testo preferito, evitate di farlo dal pdf (correte il rischio che alcuni
  caratteri speciali non siano copiati correttamente in un ambiente ASCII), ed usate il
  \href{https://bitbucket.org/lbaldini/statnotes/src/master/snippets/residual_plot.py}{link}
  nella parte superiore della cornicetta.}
  \label{fig:grafico_residui}
\end{figure}

Ma andiamo con ordine, ed esaminiamo con attenzione il frammento di codice in
figura~\ref{fig:grafico_residui}. Nella parte iniziale definiamo i punti sperimentali
ed il modello di fit, eseguiamo il \fit\ vero e proprio e calcoliamo i residui
rispetto al modello ottimale---le linee 5--16 servono essenzialmente a rendere
il frammento auto-contenuto e, a questo punto, non dovrebbero avere bisogno di
spiegazione.
Le novità cominciano dopo la chiamata alla funzione \pyfunc{plt.figure}, ed in
particolare con il comando:
\begin{align*}
  \text{\foreign{ax1, ax2 = fig.subplots(2, 1, sharex=True,
    gridspec\_kw=dict(height\_ratios=[2, 1], hspace=0.05))},}
\end{align*}
in cui creiamo nella nostra figura una griglia $2 \times 1$ (due righe ed una colonna)
per ospitare il grafico di dispersione ed il grafico dei residui.
La disposizione geometrica della griglia è dettata dal valore dell'argomento
\code{gridspec_kw} ed in particolare:
\begin{itemize}
  \item il valore di \code{height_ratios} controlla il rapporto tra le altezze
    degli spazi riservati ai due grafici---in questo caso stiamo dicendo che
    il grafico di dispersione sarà alto il doppio rispetto a quello dei
    residui;
  \item il valore di \code {hspace} controlla lo spazio tra i due grafici---in
    questo caso il 5\% dell'altezza totale della figura.
\end{itemize}
L'argomento \code{sharex=True}, nel nostro caso, è fondamentale, perché fa sì che
i due grafici \emph{condividano lo stesso asse delle ascisse}: come vedete la scala
graduata e l'etichetta di testo che indica la quantità rappresentata appare solo
nell'asse $x$ del grafico dei residui, e non in quello del grafico di dispersione.
Inoltre, da questo punto in poi, è garantito che qualsiasi cambiamento dell'intervallo
dinamico di uno dei due assi delle $x$ si propagherà correttamente anche sull'altro.
In altre parole: le griglie ed i punti saranno per definizione sempre allineati
verticalmente. Notiamo, per completezza, che la chiamata alla funzione \pyfunc{subplot}
restituisce le due istanze di tipo \href{https://matplotlib.org/stable/api/axes_api.html}{Axes},
che noi abbiamo chiamato \code{ax1} ed \code{ax2}, e che utilizzeremo nel seguito
per riempire di contenuto le porzioni della figura riservate rispettivamente al
grafico di dispersione ed al grafico dei residui.

Il resto del frammento dovrebbe essere più o meno auto-esplicativo, ma vale la
pena fare alcuni commenti specifici:
\begin{itemize}
  \item abbiamo fatto in modo che le griglie sul grafico, che tipicamente aumentano
    la leggibilità, non fossero troppo prominenti, scegliendo un colore grigio
    chiaro ed uno stile tratteggiato per le linee;
  \item abbiamo aggiunto una legenda al pannello superiore per rendere
    immediatamente ovvio il significato dei punti e della linea (notate che
    abbiamo deciso deliberatemente di non fare lo stesso per il plot dei residui,
    confindando che la corrispondenza degli elementi grafici sia sufficiente);
  \item il comando \code{fig.align_ylabels((ax1, ax2))} fa in modo che
    le etichette di testo sui due assi delle $y$ siano allineate orizzontalmente,
    evitando asimmetrie che sono in genere visualmente spiacevoli.
\end{itemize}

\noindent Che altro dire? Sperimentate e buon divertimento!
