\chapter{Elementi di teoria delle probabilità}
\label{sec:probabilita}

Cominciamo questo capitolo con una domanda e due problemi. La domanda è la
seguente: qual è la differenza (o le differenze) tra la
\emph{teoria delle probabilità} e la \emph{statistica}? Ed i due problemi,
connessi con la nostra domanda, sono:
\begin{itemize}\label{item:due problemi}
\item \emph{Problema~1}: abbiamo un'urna contenente $5$ palline, di cui $3$
  rosse e $2$ blu. Se estraiamo (bendati) una pallina qual è la probabilità
  che essa sia blu?
\item \emph{Problema~2}: abbiamo un'urna contenente $5$ palline---alcune rosse
  ed alcune blu (ma non conosciamo la proporzione tra i colori). Estraiamo una
  pallina, che risulta essere blu, e la mettiamo da parte. Estraiamo una
  seconda pallina, che stavolta è rossa. Qual è il numero di palline rosse
  nell'urna?
\end{itemize}

La prima domanda è facile, e la risposta è univocamente determinata dalle
condizioni del problema (e, per completezza, è $\nicefrac{2}{5}$). La seconda,
invece, sembra apparentemente senza speranza. Ma, a pensarci meglio, abbiamo
sicuramente informazione per dire \emph{qualcosa}; possiamo dire, ad esempio,
che le palline non sono tutte blu e non sono tutte rosse. E il fatto di aver
estratto una pallina rossa ed una pallina blu potrebbe farci pensare che
l'eventualità che il numero di palline blu sia $1$ o $4$ sia relativamente
poco probabile, anche se non impossibile.

Proviamo a riformulare il tutto. Il primo è un tipico problema di probabilità,
in cui abbiamo un sistema descritto da un modello ben preciso e possiamo trarre
conclusioni in modo deduttivo. Il secondo, che è più vicino al mestiere
del Fisico, è invece un problema di statistica, in cui a partire da un numero
finito di osservazioni, cerchiamo (induttivamente) di trarre conclusioni sul
modello che regola il nostro sistema. \`E chiaro che i due problemi sono legati
tra loro e, in un certo senso, il secondo è l'inverso del primo---in effetti
potremmo chiamarlo un problema di \emph{probabilità inversa}.

Benché non universalmente accettata (si veda~\cite{janes} come esempio di
uno schema logico alternativo in cui le due cose sono unificate), questa
distinzione di fondo tra teoria delle probabilità, come branca della
matematica pura, e statistica, come mezzo di indagine dei fenomeni fisici,
ci pare utile per inquadrare, almeno in una fase iniziale, i problemi che
incontreremo.


\section{Definizione assiomatica della probabilità}%
\label{sec:definizione_assiomtica_prob}

Si deve a Kolmogorov~\cite{kolmogorov} la prima costruzione rigorosa della
teoria della probabilità, in una struttura che sostanzialmente sopravvive
nei manuali moderni. Quella che segue non è una esposizione rigorosa della
teoria assiomatica della probabilità, ma solo un accenno superficiale ad
alcune idee di base ad essa connesse, in cui daremo per noti i concetti di
insieme ed operazioni tra insiemi.

La struttura di base su cui si fonda la teoria assiomatica della probabilità
\emph{a la} Kolmogorov è data dal cosiddetto \emph{spazio campionario}
$\Omega$, cioè l'insieme (che per semplicità assumeremo numerabile) di
tutte le possibili realizzazioni elementari di un dato fenomeno. Sulla base di
questo si definisce poi lo spazio degli eventi $\mathcal F$ come
\emph{l'insieme di tutti i sottoinsiemi} di $\Omega$, che si dice anche
l'insieme di potenza di $\Omega$. L'idea di base è che la probabilità è
definita proprio sullo spazio degli eventi---cioè si può assegnare una
probabilità non solo ad un qualsiasi elemento dello spazio campionario, ma
anche ad uno qualsiasi dei suoi sottoinsiemi. Siccome l'unione e l'intersezione
di sottoinsiemi dello spazio campionario fanno ancora parte dello spazio degli
eventi, avrà senso parlare anche della probabilità dell'unione e
dell'intersezione di eventi.
Anziché addentrarci troppo nei dettagli tecnici cercheremo di chiarire il
concetto con i due esempi~\ref{exp:spazio_eventi_moneta}
e~\ref{exp:spazio_eventi_dado}.

\begin{examplebox}
  \begin{example}\label{exp:spazio_eventi_moneta}
    Nel lancio di una moneta lo spazio campionario $\Omega$ è dato
    semplicemente dalle due possibili uscite $\{T,~C\}$ (testa o croce).
    L'insieme di potenza di $\Omega$ contiene, oltre all'insieme nullo
    $\emptyset$ ed allo spazio campionario stesso, i due sottoinsiemi propri di
    $\Omega$ $\{T\}$ e $\{C\}$. Le quattro domande che possiamo farci a
    proposito del lancio di una moneta sono dunque a proposito degli eventi
    \begin{align*}
      \{T\}:~& \text{qual è la probabilità che esca testa?}\\
      \{C\}:~& \text{qual è la probabilità che esca croce?}\\
      \{T,~C\}:~& \text{qual è la probabilità che esca testa oppure
        croce?}\\
      \{\}:~& \text{qual è la probabilità che non esca né testa né
        croce?}
    \end{align*}
  \end{example}

  \begin{example}\label{exp:spazio_eventi_dado}
    Nel lancio di un dado a sei facce lo spazio campionario $\Omega$ è dato
    dai sei valori possibili delle uscite $\{1,~2,~3,~4,~5,~6\}$.
    L'insieme di potenza di $\Omega$ contiene tutti gli eventi di cui sia
    lecito chiedersi quale sia la probabilità---ad esempio:
    \begin{align*}
      \{3\}:~& \text{qual è la probabilità che esca il numero 3?}\\
      \{2,~4,~6\}:~&  \text{qual è la probabilità che l'uscita sia pari?}\\
      \{1,~2,~3\}:~& \text{qual è la probabilità che l'uscita sia~$< 4$?}\\
      \{1,~2,~3,~4,~5,~6\}:~& \text{qual è la probabilità che esca una
        faccia qualsiasi?}\\
      \{\}:~& \text{qual è la probabilità che non esca nessuna faccia?}
    \end{align*}
    In entrambi gli esempi gli ultimi due casi corrispondono all'intero spazio
    campionario $\Omega$ ed all'insieme vuoto $\emptyset$, e le risposte alle
    domande corrispondenti sembrano semplici.
  \end{example}
\end{examplebox}

Si definisce probabilità una misura $P$ su $\mathcal F$ che associ
univocamente ad ogni elemento $E$ di $\mathcal F$ un numero reale $\prob{E}$
che soddisfa le seguenti tre proprietà (o assiomi di Kolmogorov):
\begin{enumerate}
\item $0 \leq \prob{E} \leq 1~\forall~E \in \mathcal F$;
\item $\prob{\Omega} = 1$;
\item $\prob{E_1 \cup E_2} = \prob{E_1} + \prob{E_2}~\text{se}~E_1 \cap E_2 =
  \emptyset$.
\end{enumerate}
Notiamo che il terzo assioma si estende all'unione numerabile di
eventi disgiunti. Per completezza, due eventi che abbiano intersezione nulla
si dicono disgiunti o \emph{incompatibili}---nel senso che la probabilità che
essi accadano contemporaneamente è nulla. Gli assiomi di Kolmogorov
costituiscono una sorta di \emph{grammatica} di base su cui sviluppare il
linguaggio delle probabilità.

\begin{examplebox}
  \begin{example}
    Nel caso del lancio di una moneta possiamo associare una probabilità allo
    spazio campionario definendo, ad esempio,
    $\prob{T} = \prob{C} = \nicefrac{1}{2}$ (con questa prescrizione la moneta
    è per definizione equa). Per gli assiomi di Kolmogorov la
    probabilità che non esca né testa né croce è $\prob{\emptyset} = 0$
    e la probabilità che esca testa oppure croce è $\prob{\Omega} = 1$.
  \end{example}

  \begin{example}\label{example:prob_die}
    Nel lancio di un dado a sei facce la misura che associa $\nicefrac{1}{6}$
    alle~$6$ possibili realizzazioni elementari $\{1\}$\ldots$\{6\}$ è una
    probabilità, come si può verificare banalmente. Si noti in particolare
    che
    $$
    \Omega = \{1\} \cup \{2\} \cup \{3\} \cup \{4\} \cup \{5\} \cup \{6\}
    $$
    e $\prob{\Omega} = 1$ per l'assioma 3. Va da sé che, con questa
    prescrizione, il dado è equo.
  \end{example}
\end{examplebox}


\subsection{Leggi elementari della probabilità}

Gli assiomi di Kolmogorov possono essere utilizzati per dimostrare un certo
numero di risultati elementari.

\begin{theorem}[della probabilità complementare]
  Dato un evento $E$, e detto $\overline{E}$ il suo complementare in $\Omega$,
  si ha
  \begin{align}\label{eq:probabilita_complementare}
    \prob{\overline{E}} = 1 - \prob{E}.
  \end{align}
\end{theorem}
\begin{proof}
  Segue banalmente dal fatto che $\overline{E} \cup E = \Omega$ e
  $\overline{E} \cap E = \emptyset$ per cui, utilizzando gli assiomi 2. e
  3. si ha
  \begin{align*}
    1 = \prob{\Omega} = \prob{E \cup \overline{E}} =
    \prob{E} + \prob{\overline{E}},
  \end{align*}
  da cui la tesi.
\end{proof}

\begin{examplebox}
  \begin{example}
    Nel lancio di un dado equo a sei facce la probabilità che l'uscita non
    sia un $3$, cioè dell'evento $\{1,~2,~4,~5,~6\}$, è data dalla somma
    $\prob{1} + \prob{2} + \prob{4} + \prob{5} + \prob{6} = \nicefrac{5}{6}$,
    che si può scrivere più facilmente come
    $$
    \prob{\overline{3}} = 1 - \prob{3} = \nicefrac{5}{6}.
    $$
    Per completezza, la sezione~\ref{sec:paradosso_compleanni} illustra
    un'applicazione più interessante del teorema della probabilità
    complementare.
  \end{example}
\end{examplebox}


\begin{corollary}
  $\prob{\emptyset} = 0$.
\end{corollary}

\begin{corollary}
  Se $E_1 \subset E_2$, allora $\prob{E_1} \leq \prob{E_2}$.
\end{corollary}

\begin{figure}[!htbp]
  \autohstack{\begin{venndiagram2sets}%
  [tikzoptions={xscale=1.55,yscale=1.15}, overlap=1.cm,
    labelA=$E_1$, labelB=$E_2$, labelAB=$E_1 \cap E_2$, labelNotAB=$\Omega$]
  \fillACapB
\end{venndiagram2sets}
}{
    \caption{Diagramma di Venn per il teorema di addizione delle
      probabilità~\ref{theorem:prob_addition}.
      Il diagramma rende intuitivo il contenuto dell'ultimo termine della
      formula: sommando semplicemente $\prob{E_1}$ e $\prob{E_2}$ conteremmo
      due volte l'intersezione. Il diagramma illustra anche i due fatti
      elementari utilizzati nella dimostrazione:
      $E_2 = (E_2 \cap E_1) \cup (E_2 \cap \overline{E_1})$ e
      $E_1 \cup E_2 = E_1 \cup (\overline{E_1} \cap E_2)$.}
    \label{fig:probabilita_venn}
  }
\end{figure}

\begin{theorem}[di addizione delle probabilità]\label{theorem:prob_addition}
  Dati due eventi $E_1$ ed $E_2$, si ha
   \begin{align}\label{eq:prob_addition}
     \prob{E_1 \cup E_2} = \prob{E_1} + \prob{E_2} - \prob{E_1 \cap E_2}.
  \end{align}
\end{theorem}
\begin{proof}
  Possiamo scrivere $E_2$ come la somma di due insiemi disgiunti
  notando semplicemente che, come illustrato nel diagramma in
  figura~\ref{fig:probabilita_venn}
  \begin{align*}
    E_2 = E_2 \cap \Omega = E_2 \cap (E_1 \cup \overline{E_1}) =
    (E_2 \cap E_1) \cup (E_2 \cap \overline{E_1}),
  \end{align*}
  che, nel nostro linguaggio, equivale a dire
  \begin{align*}
    \prob{E_2} = \prob{(E_2 \cap E_1) \cup (E_2 \cap \overline{E_1})} =
    \prob{E_2 \cap E_1} + \prob{E_2 \cap \overline{E_1}}.
  \end{align*}
  D'altra parte si ha anche
  \begin{align*}
    E_1 \cup E_2 & = (E_1 \cup E_2) \cap \Omega =
    (E_1 \cup E_2) \cap (E_1 \cup \overline{E_1}) =
    (E_1 \cap E_1) \cup (E_1 \cap \overline{E_1}) \cup
    (E_2 \cap E_1) \cup (E_2 \cap \overline{E_1}) =\\
    & = E_1 \cup \emptyset \cup (E_2 \cap E_1) \cup (E_2 \cap \overline{E_1}) =
    E_1 \cup (E_2 \cap \overline{E_1}),
  \end{align*}
  ovverosia
  \begin{align*}
    \prob{E_1 \cup E_2} = \prob{E_1} + \prob{E_2 \cap \overline{E_1}}.
  \end{align*}
  Mettendo insieme le due equazioni sopra si ottiene la tesi.
\end{proof}



Il diagramma di Venn in figura~\ref{fig:probabilita_venn} illustra in modo
intuitivo il contenuto della dimostrazione. Se gli eventi $E_1$ e $E_2$ sono
disgiunti (cioè $E_1 \cap E_2 = \emptyset$) allora la formula generale di
addizione delle probabilità~\eqref{eq:prob_addition} si riduce banalmente al
terzo assioma di Kolmogorov.

\begin{examplebox}
  \begin{example}
    Nel lancio di un dado equo a 6 facce la probabilità che esca il $2$
    oppure il $3$ è data da
    $\prob{2~\text{o}~3} = \prob{2} + \prob{3} =
    \nicefrac{1}{6} + \nicefrac{1}{6} = \nicefrac{1}{3}$.
  \end{example}

  \begin{example}
    La probabilità di estrarre un re oppure una carta di cuori da un mazzo
    di $52$ carte è data da
    $\prob{\text{K} \cup \heartsuit} =
    \prob{\text{K}} + \prob{\heartsuit} - \prob{\text{K} \cap \heartsuit} =
    \nicefrac{4}{52} + \nicefrac{13}{52} - \nicefrac{1}{52} = \nicefrac{16}{52}$
    (dove l'ultimo termine della somma serve a non contare due volte il
    re di cuori).
  \end{example}
\end{examplebox}


\section{Definizioni operative di probabilità}

Come vedremo nel seguito, la definizione assiomatica di probabilità che
abbiamo appena enunciato permette di derivare in modo rigoroso un certo
numero di proprietà praticamente rilevanti di cui la probabilità stessa
gode. Essa non dice niente, però, su come si possa \emph{calcolare} la
probabilità di un dato evento in casi concreti. (Non a caso,
nell'esempio~\ref{example:prob_die} abbiamo detto che la nostra misura era
\emph{una} probabilità.) In questa sezione esaminiamo brevemente alcune
definizioni \emph{operative} di probabilità che ci saranno utili nel seguito.


\subsection{Definizione combinatoriale}

Nella sua definizione \emph{combinatoriale} la probabilità di un evento $E$
coincide con il rapporto tra il numero di casi favorevoli $n$ ed il numero di
casi possibili $N$, a condizione che questi ultimi siano tutti ugualmente
probabili:
\begin{align}\label{eq:prob_comb}
  \prob{E} = \frac{n}{N}.
\end{align}

Notiamo che questa definizione soddisfa gli assiomi di Kolmogorov: il primo
assioma discende dalla ovvia condizione $0 \leq n \leq N$ ed il secondo dal
fatto che se $n = N$, allora $\prob{E} = 1$. Si ha inoltre che, se $E_1$ ed
$E_2$ sono due eventi disgiunti con un numero di casi favorevoli pari a
$n_1$ ed $n_2$, rispettivamente, si ha
\begin{align*}
  \prob{E_1 \cup E_2} = \frac{n_1 + n_2}{N} = \frac{n_1}{N} + \frac{n_2}{N} =
  \prob{E_1} + \prob{E_2}.
\end{align*}

Il lettore più attento si sarà accorto che si tratta di una definizione
circolare, nel senso che richiediamo l'equiprobabilità dei casi nella
definizione stessa di probabilità. \`E altresì chiaro che il campo
di applicabilità della~\eqref{eq:prob_comb} è limitato ai problemi più
elementari---ad una domanda come "qual è la probabilità che un certo
dispositivo elettronico subisca un guasto nel primo anno di funzionamento?"
è estremamente difficile rispondere nel quadro di questo schema logico
poiché non è affatto ovvio come si potrebbero definire i casi favorevoli
ed i casi possibili. E, purtuttavia, si tratta di una nozione utile che
utilizzeremo occasionalmente in seguito---negli
esempi~\ref{example:prob_comb_one_die}--\ref{example:prob_comb_two_dice} e,
soprattutto, nella derivazione della distribuzione binomiale
(sezione~\ref{sec:distribuzione_binomiale}).

\begin{examplebox}
  \begin{example}\label{example:prob_comb_one_die}
    Il lancio di un dado equo a sei facce ha sei possibili esiti
    (equiprobabili). La probabilità che esca un numero fissato, ad esempio
    il $3$, è dunque $\prob{3} = \nicefrac{1}{6}$.
  \end{example}

  \begin{example}
    Supponiamo di avere un mazzo di 52 carte. La probabilità di estrarre
    un re è $\prob{\text{K}} = \nicefrac{4}{52} = \nicefrac{1}{13}$;
    la probabilità di estrarre una carta di cuori è
    $\prob{\heartsuit} = \nicefrac{13}{52} = \nicefrac{1}{4}$.
  \end{example}

  \begin{example}\label{example:prob_comb_two_dice}
    Supponiamo di lanciare due dadi equi a sei facce ed essere interessati alla
    somma delle uscite. Gli esiti possibili sono in questo caso $11$ (i numeri
    interi da $2$ a $12$) ma essi non sono equiprobabili, per cui
    $\prob{3} \neq \nicefrac{1}{11}$.
    Il modo corretto di affrontare il problema dalla prospettiva della
    definizione combinatoriale di probabilità è il seguente: vi sono
    esattamente $6 \times 6 = 36$ configurazioni (equiprobabili) distinte in
    cui i dadi possono atterrare e, tra queste, esattamente due danno come
    somma $3$---per cui $\prob{3} = \nicefrac{2}{36} = \nicefrac{1}{18}$.
    Torneremo sulla questione nell'esempio~\ref{exp:pdf_somma_due_dadi}.
  \end{example}
\end{examplebox}


\subsection{Definizione frequentista}
\label{sec:prop_definizione_freq}

Quando è possibile ripetere un esperimento in condizioni controllate,
possiamo definire la probabilità di un evento $E$ come il limite
della frequenza relativa dell'evento stesso quando il numero di ripetizioni
dell'esperimento è molto grande. In altre parole possiamo pensare di
eseguire l'esperimento un numero (arbitrariamente grande) di volte $N$, contare
il numero di volte $n$ in cui $E$ si verifica, e definire $\prob{E}$ come
il limite del rapporto $\nicefrac{n}{N}$ per $N\rightarrow \infty$:
\begin{align}\label{eq:prob_freq}
  \prob{E} = \lim_{N \rightarrow \infty}\frac{n}{N}.
\end{align}
(Vale la pena sottolineare come, benché formalmente simili, la
\eqref{eq:prob_comb} e \eqref{eq:prob_freq} siano completamente diverse dal
punto di vista concettuale.)

La \eqref{eq:prob_freq} si dice generalmente definizione \emph{frequentista} di
probabilità e, seppure non particolarmente adatta ad una costruzione
matematica rigorosa della teoria della probabilità, può essere
operativamente utile---specialmente in Fisica, ove la ripetizione di un
esperimento controllato è tipica. In particolare lo schema frequentista è
alla base del metodo Monte Carlo che discuteremo (molto brevemente) in seguito.

Ovviamente non si ha nessuna garanzia che un numero fissato di ripetizioni sia
in generale sufficiente perché la frequenza relativa sia una stima
\emph{sufficientemente buona} della probabilità cercata. Il limite non è
cioè da intendersi nel senso usuale dell'analisi matematica
\begin{align*}
  \forall \epsilon > 0 \ \exists \ \tilde{N} > 0 : N \geq \tilde{N} \rightarrow
  \abs{\frac{n}{N} - \prob{E}} \leq \epsilon
\end{align*}
ma piuttosto in quello (più debole) della convergenza statistica, che possiamo
definire formalmente come
\begin{align*}
  \forall \epsilon, \delta > 0 \ \exists \ \tilde{N} > 0 :
  N > \tilde{N} \rightarrow
  \prob{\abs{\frac{n}{N} - \prob{E}} \geq \delta } \leq \epsilon.
\end{align*}
In pratica il punto (sottile) è che non esiste un numero $N$ di ripetizioni
del nostro esperimento che mi permetta di affermare con certezza che la
differenza tra la frequenza registrata $\nicefrac{n}{N}$ e la probabilità
$\prob{E}$ cui essa tende sia al di sotto di un valore prefissato $\epsilon$.
In generale, se eseguo due serie di $N$ ripetizioni dello stesso esperimento
otterrò frequenze relative $\nicefrac{n}{N}$ diverse anche se $\prob{E}$ è
la stessa, come illustrato in figura~\ref{fig:frequenza_dado}. Quello che posso
dire è che se $N$ è abbastanza grande, allora posso rendere piccola a
piacere la probabilità che $\nicefrac{n}{N}$ si discosti da $\prob{E}$ di
più di un valore prefissato $\delta$.

\begin{examplebox}
  \begin{example}\label{example:prob_freq_one_die}
    Se lanciamo $N$ volte un dado equo a sei facce e registriamo (al crescere
    di $N$) il numero $n$ di volte in cui esce, ad esempio, il numero $3$, per
    $N$ molto grande il rapporto $n/N$ tenderà a $\prob{3} = \nicefrac{1}{6}$
    (nel senso della convergenza statistica), come mostrato in
    figura~\ref{fig:frequenza_dado}.
  \end{example}
\end{examplebox}

\pgffigone{frequenza_dado}{
  $10$ diverse realizzazioni di $10\,000$ lanci di un dado equo a sei
  facce, in cui si registra, al variare del numero parziale $N$ di lanci,
  la frazione $\nicefrac{n}{N}$ delle volte in cui esce il numero $3$.
  La linea orizzontale tratteggiata indica la probabilità
  $p = \nicefrac{1}{6}$ dell'evento e, come si vede, la frequenza tende
  a questa probabilità per $N$ grande---nel senso che le fluttuazioni attorno
  a questo valore si riducono al crescere di $N$.
}


\subsection{Definizione soggettiva}

Nella sua definizione \emph{soggettiva} la probabilità di un evento $E$ si
identifica con la
\emph{misura del grado di fiducia che un individuo attribuisce al verificarsi
  di $E$, sulla base dell'informazione a sua disposizione}. (La seconda parte
della definizione è importante, perché persone diverse, con informazioni
diverse, in generale assoceranno una probabilità diversa allo stesso
evento---da cui il termine "soggettivo" nel titolo di questa sezione.)

Una volta superato lo sbigottimento iniziale---causato dall'attaccamento dei
Fisici al concetto di oggettività---questa definizione è tutto sommato
naturale e vicina al modo in cui operiamo nella vita di tutti i giorni. Ogni
volta che facciamo una scelta tra due o più possibilità diverse, di fatto
associamo implicitamente delle probabilità agli eventi che ne conseguono---e
persone diverse, in base alle informazioni a loro disposizione, fanno scelte
diverse. \`E altresì chiaro che la definizione soggettiva di probabilità ha
un campo di applicabilità estremamente più vasto di quelle combinatoriale o
frequentista: all'interno di questo schema concettuale possiamo parlare della
probabilità di qualsiasi evento o proposizione.

All'interno della scuola soggettivista vi sono diversi approcci distinti per
derivare le regole fondamentali della probabilità in un modo logicamente
consistente, il più popolare dei quali è probabilmente l'idea del principio
della scommessa coerente: \emph{una volta assegnata la probabilità ad un
  evento dovremmo essere disposti ad accettare scommesse sul verificarsi
  dell'evento stesso con un rapporto tra puntata e vincita determinato
  dalla probabilità stessa}. In altre parole, se diciamo che due eventi sono
equiprobabili, allora dobbiamo essere pronti ad accettare scommesse 1:1; in
caso contrario la nostra assegnazione di probabilità non è coerente.
(Riflettete: si tratta di una formulazione meno banale di quanto non possa
sembrare a prima vista.) La cosa interessante è che, una volta derivate,
le regole della probabilità soggettiva sono le stesse di quella combinatoriale
e frequentista---essenzialmente gli assiomi di Kolmogorov che abbiamo visto
nella sezione~\ref{sec:definizione_assiomtica_prob} (con la differenza che,
in questo, schema, esse non sono assiomi ma, appunto, regole che si derivano
formalmente da un principio più fondamentale).

La definizione soggettiva di probabilità ed il teorema di Bayes, che vedremo
nella sezione~\ref{sec:teorema_bayes}, sono alla base della cosiddetta scuola
Bayesiana, che tradizionalmente si contrappone (e spesso in modo veemente) alla
scuola classica o frequentista. Il livello della nostra esposizione sarà
abbastanza rudimentale da poter evitare quasi completamente questa
contrapposizione, ma vale la pena sottolineare come l'impostazione Bayesiana
permetta di derivare entro un quadro logico e coerente, e sotto condizioni ben
definite, tutte le applicazioni elementari della probabilità e della
statistica che deriveremo nel seguito. Per una introduzione pedagogica
all'argomento si rimanda a~\cite{dagostini_review}.


\subsection{Uno spunto di riflessione: esistono monete e dadi equi?}

Fino a questo momento abbiamo parlato disinvoltamente di monete eque e dadi
equi come se la probabilità di uscita di una data faccia (ad esempio, testa
per una moneta o il numero $3$ per un dado) fosse una proprietà intrinseca
dell'oggetto. A pensarci meglio si tratta di una assunzione non banale che
merita una riflessione più approfondita.

Trattandosi di sistemi fisici macroscopici, il lancio di una moneta o di un dado
sono fenomeni regolati da leggi deterministiche: se conoscessimo esattamente
le condizioni iniziali del lancio (posizione, velocità iniziale del centro
di massa, asse istantaneo di rotazione e momento angolare) e tutti i fattori che
determinano l'evoluzione temporale (forza di gravità, attrito dell'aria,
vincoli fisici) potremmo in linea di principio predire con esattezza l'esito di
ciascun lancio. Nella realtà quando lanciamo una moneta o un dado non
misuriamo nessuna di queste quantità---e anche se volessimo, sarebbe
estremamente complicato farlo con l'accuratezza necessaria. \`E proprio questo
il motivo per cui trattiamo con le leggi della probabilità un fenomeno che
è intrinsecamente deterministico.

Detto questo: che cosa vuol dire che una moneta è equa? Per definizione
vuol dire che $\prob{T} = \prob{C} = \nicefrac{1}{2}$. Ma su quale base
assumiamo che questa sia una proprietà della moneta e non dipenda, ad esempio,
da \emph{come} la lanciamo? Siamo sicuri che la stessa moneta, assumendo che
sia equa se lanciata da una certa persona, continui ad essere equa anche
quando lanciata da una persona diversa? Siamo sicuri che la stessa moneta,
assumendo che sia equa quando lasciata cadere da $1$~m di altezza, continui
ad essere equa se lasciata cadere (dalla stessa persona) da $10$~cm di altezza?

Se ci pensiamo per un secondo, la risposta più logica a tutte queste domande
è: no. \`E noto che esistono bari ai tavoli da gioco, per cui è chiaro che,
con un po' di pratica, si può imparare a lanciare un dado in modo che le
probabilità di uscita delle singole facce siano significativamente differenti
da~$\nicefrac{1}{6}$. Nel seguito, dunque, quando parleremo di monete e dadi
equi, avremo bene in mente che si tratta di astrazioni (sia pure non
irragionevoli) che facciamo a scopo illustrativo, ma che una definizione
operativa rigorosa di questi concetti richiederebbe un'analisi estremamente
più approfondita di quella che possiamo fare in questo contesto.


\section{Digressione: elementi di calcolo combinatorio}

Il calcolo combinatorio è quella branca della matematica che si occupa del
\emph{contare}. Si tratta di un argomento ovviamente connesso con la
definizione combinatoriale di probabilità e in questa sezione introdurremo
brevemente alcuni dei concetti fondamentali.


\subsection{Permutazioni e funzione fattoriale}

Dato un numero intero non negativo $n$ definiamo il suo fattoriale (che
scriveremo come $n!$) come il prodotto di tutti gli interi positivi minori o
uguali ad $n$
\begin{align}
  n! = \prod_{k=1}^{n} k = n (n - 1)\ldots 1,
\end{align}
con l'ulteriore convenzione che $0! = 1$. Equivalentemente, la funzione
fattoriale può essere definita per ricorrenza come
\begin{align}
  \begin{cases}
    0! = 1\\
    n! = n (n - 1)!.
  \end{cases}
\end{align}

Il fattoriale $n!$ rappresenta il numero di permutazioni di un insieme di
$n$ elementi, cioè il numero di modi in cui si possono disporre $n$
elementi. La convenzione $0! = 1$ è consistente con il fatto che vi è
un solo modo di permutare zero oggetti.

\begin{examplebox}
  \begin{example}
    In quanti modi diversi possiamo disporre $6$ cd in un caricatore a $6$ posti
    di un autoradio? Possiamo ragionare come segue: siamo liberi di mettere un
    cd qualsiasi tra i $6$ da caricare nella posizione $1$, dopodiché ci
    rimangono $5$ cd tra cui scegliere per la posizione $2$, $4$ per la
    posizione $3$ e così via. Giunti all'ultimo cd, non abbiamo più scelta
    perché l'unica posizione libera è la sesta.
    La risposta è dunque $6! = 720$.
  \end{example}
\end{examplebox}

\pgffigone{fattoriale}{
  Grafico della funzione $n!$ per $n = 0 \ldots 50$. Per confronto, la linea
  grigia tratteggiata rappresenta la funzione $y = e^n$, che in scala
  semilogaritmica è una retta. Come è ovvio, la funzione fattoriale cresce
  più che esponenzialmente al crescere di $n$.
}

La funzione fattoriale cresce molto velocemente al crescere di $n$---molto
più velocemente, e.g., di un esponenziale, come illustrato in
figura~\ref{fig:fattoriale}.
Tanto per fare un esempio, $50! \approx 3 \times 10^{64}$.
Per $n$ grande, il fattoriale può essere approssimato con la formula di
Stirling
\begin{align}\label{eq:formula_di_stirling}
  n! \approx \sqrt{2\pi n}\left( \frac{n}{e} \right)^n,
\end{align}
o, equivalentemente
\begin{align}\label{eq:formula_di_stirling_log}
  \ln n! \approx \frac{1}{2}\ln(2\pi n) + n\ln n - n.
\end{align}
Per fissare le idee, l'approssimazione~\eqref{eq:formula_di_stirling} è
accurata a meglio dell'$1\%$ per $n \geq 10$, e l'errore relativo diminuisce
al crescere di $n$.
Dato che il logaritmo è una funzione che per grandi valori del suo argomento
varia piuttosto lentamente, lo sviluppo del $\ln n!$ può, a seconda della
situazione, essere utilizzato trascurando il primo termine, visto che
$\ln n \ll n$---cioè
\begin{align}\label{eq:formula_di_stirling_log_troncata}
  \ln n! \approx n\ln n - n.
\end{align}
(Più precisamente: nello sviluppo di
$\ln n!$~\eqref{eq:formula_di_stirling_log} il primo termine è una
costante additiva che diventa irrilevante nel limite $n \rightarrow \infty$,
mentre nello sviluppo di $n!$~\eqref{eq:formula_di_stirling} esso corrisponde
alla costante moltiplicativa $\sqrt{2\pi n}$ che, ovviamente, non può
essere omessa.)

Il fattoriale è, come è noto, utilizzato nello sviluppo in serie di Taylor.
In particolare, sviluppando in serie la funzione $e^x$ si ottiene la
relazione interessante
\begin{align}
  e^x = \sum_{n=0}^{\infty} \frac{x^n}{n!}
\end{align}
da cui anche
\begin{align}
  e = \sum_{n=0}^{\infty} \frac{1}{n!} = 2 + \frac{1}{2} + \frac{1}{6} +
  \frac{1}{24} \ldots
\end{align}


\subsection{Coefficienti binomiali}

Il numero di modi possibili per scegliere $k$ elementi non ordinati da un
insieme di $n$ elementi è dato dal \emph{coefficiente binomiale $n$ su $k$}
\begin{align}
  \binom{n}{k} = \frac{n!}{k!(n - k)!} =
  \frac{n(n - 1)(n - 2)\ldots (n - k + 1)}{k(k - 1)(k - 2)\ldots 1}.
\end{align}
In breve, il ragionamento procede nel modo seguente: per la scelta del primo
dei $k$ elementi abbiamo $n$ possibilità, per il secondo $n - 1$, per il
terzo $n - 2$ e così via fino a $n - (k - 1)$; il $k!$ al denominatore
serve per non contare più di una volta scelte che differiscono solo per
l'ordine.

\begin{examplebox}
  \begin{example}
    Dato un insieme di tre elementi $\{A,~B,~C\}$, vi sono esattamente tre
    sottoinsiemi non ordinati di $2$ elementi: $\{A,~B\}$, $\{A,~C\}$ e
    $\{B,~C\}$ ($\{A,~B\}$ e $\{B,~A\}$ sono lo stesso insieme).
    La risposta si può scrivere anche come
    \begin{align*}
      \binom{3}{2} = \frac{3!}{2!1!} = \frac{6}{2} = 3.
    \end{align*}
  \end{example}

  \begin{example}
    Quanti \foreign{cin-cin} ci sono in un brindisi fra $10$ amici? Possiamo
    ragionare come segue: date $n$ persone, ognuna brinda con le altre
    $n - 1$, ma dobbiamo poi dividere per $2$ per non contare doppio ogni
    singolo \foreign{cin-cin}. La risposta è dunque $n(n - 1)/2$ o, nel nostro
    caso, $45$. In termini dei coefficienti binomiali, la risposta è
    semplicemente
    \begin{align*}
      \binom{10}{2} = \frac{10!}{2!8!} = \frac{10 \times 9}{2} = 45.
    \end{align*}
  \end{example}
\end{examplebox}

Il coefficiente binomiale è strettamente connesso al triangolo di Pascal e
può anche essere caratterizzato attraverso la formula della potenza di
binomio (o teorema binomiale), che utilizzeremo nel seguito:
\begin{align}\label{eq:potenza_binomio}
  (x_1 + x_2)^n = \sum_{k=0}^{n}\binom{n}{k} x_1^k x_2^{n-k}.
\end{align}
Nel caso in cui $x_1 = x_2 = 1$, la~\eqref{eq:potenza_binomio} si riduce alla
formula notevole
\begin{align}
  \sum_{k=0}^{n}\binom{n}{k} = 2^n
\end{align}
in cui, nel membro a sinistra, si riconosce il numero di elementi dell'insieme
di potenza di un generico insieme di $n$ elementi---scritto come la somma del
numero di sottoinsiemi di $0, 1\ldots n$ elementi. Così, come abbiamo visto
negli esempi~\ref{exp:spazio_eventi_moneta} e~\ref{exp:spazio_eventi_dado},
lo spazio degli eventi corrispondenti al lancio di una moneta ha $2^2 = 4$
elementi e quello corrispondente al lancio di un dado ha $2^6 = 64$ elementi.


\subsection{Una applicazione elementare: il gioco del lotto}
\label{sec:lotto_probabilita}

Come applicazione elementare dei coefficienti binomiali ci chiediamo quale
sia la probabilità di vincere giocando un ambo secco al lotto su una singola
ruota---ad esempio quella di Napoli.
Il calcolo è leggermente più complicato di quelli che abbiamo visto sino
ad ora, ma gli ingredienti sono essenzialmente due: il numero di ambi
(cioè di coppie non ordinate) possibili con $90$ numeri
\begin{align*}
  \binom{90}{2} = \frac{90 \times 89}{2} = 4005
\end{align*}
(che sono \emph{tutti} gli ambi possibili) ed il numero di ambi generato
dai $5$ numeri estratti
\begin{align*}
  \binom{5}{2} = \frac{5 \times 4}{2} = 10.
\end{align*}
La probabilità che il nostro ambo sia estratto è dunque data da
\begin{align*}
  \binom{5}{2}/\binom{90}{2} = \frac{10}{4005} \approx 0.25\%,
\end{align*}
a fronte della quale il lotto paga un ambo secco $250:1$, cioè poco più
della metà del valore $400.5:1$ che renderebbe il gioco equo.
Il calcolo si può generalizzare facilmente a terne, quaterne e
cinquine, come mostrato nella tabella che segue.

\begin{table}[htbp]
  \tablehstack{
    \begin{tabular}{lll}
      \hline
      Giocata & Probabilità di vincita & Premio \\
      \hline
      \hline
      Ambo & $1:401$ & $250:1$\\
      Terno & $1:11\,748$ & $4\,500:1$\\
      Quaterna & $1:511\,038$ & $120\,000:1$\\
      Cinquina & $1:43\,949\,268$& $6\,000\,000:1$\\
      \hline
  \end{tabular}}{
    \caption{Probabilità di vincita e premio corrisposto per alcune possibili
      giocate al lotto (da intendersi su una singola ruota). Come è ovvio,
      il rapporto tra questi due numeri è sempre sfavorevole per il
      giocatore.}
    \label{tab:lotto_probabilita}
  }
\end{table}

Il gioco è sempre sfavorevole per il giocatore, con un rapporto che va
da meno di $2$ per l'ambo a più di $7$ per la cinquina.


\subsection{Coefficienti multinomiali}
\label{sec:coefficienti_multinomiali}

\danger%
Il concetto di coefficiente binomiale si può generalizzare in modo naturale
all'idea della suddivisione di un insieme di $n$ elementi in $m$ sottoinsiemi
disgiunti, ciascuno con un numero $k_i$ di elementi, la cui unione
costituisca l'insieme di partenza. \`E ovvio che, sotto queste condizioni
i $k_i$ soddisfano la relazione
\begin{align*}
  \sum_{i = 1}^m k_i = n.
\end{align*}
Il numero di modi in cui si può effettuare tale suddivisione prende il nome
di \emph{coefficiente multinomiale} $n$ su $k_1, \ldots, k_m$ e non è
difficile da calcolare---essenzialmente si hanno $n$ su $k_1$ (binomiale) modi
per scegliere il primo sottoinsieme, dopo di che si hanno $n-k_1$ su $k_2$
(binomiale) modi di scegliere il secondo e così via fino all'ultimo
\begin{align}
  \binom{n}{k_1,\ldots,k_m} & =
  \binom{n}{k_1} \binom{n-k_1}{k_2} \ldots \binom{n-k_1-\ldots-k_{m-1}}{k_m} =
  \nonumber\\
  & = \frac{n!}{k_1!(n-k_1!)} \, \frac{(n-k_1)!}{k_2!(n-k_1-k_2)!} \cdots
  \frac{(n-k_1-\ldots-k_{m-1})!}{k_m!0!} = \frac{n!}{k_1!k_2! \cdots k_m!}.
\end{align}

\begin{examplebox}
  \begin{example}
    In quanti modi si possono disporre $20$ invitati ad un banchetto di
    matrimonio su tre tavoli, rispettivamente da $4$, $6$ e $10$ persone?
    La risposta è
    \begin{align*}
      \binom{20}{4,6,10} = \frac{20!}{4!6!10!} = 38798760
    \end{align*}
    ossia più di $38$ milioni di modi distinti.
  \end{example}
\end{examplebox}

Per inciso, notiamo che i coefficienti multinomiali permettono di scrivere in
modo compatto la formula per la potenza della somma di un numero arbitrario
di monomi attraverso il cosiddetto teorema multinomiale, che è la
generalizzazione del teorema binomiale~\eqref{eq:potenza_binomio}
\begin{align}
  (x_1 + x_2 + \cdots x_m)^n =
  \sum \binom{n}{k_1,\ldots,k_m} x_1^{k_1} x_2^{k_2} \cdots x_m^{k_m},
\end{align}
in cui la somma è estesa a tutte le sequenze $(k_1, k_1,\ldots,k_m)$ di interi
non negativi la cui somma sia $n$.


\section{Probabilità condizionata}

Dati due eventi $E_1$ ed $E_2$, con $\prob{E_2} \neq 0$, definiamo la
probabilità \emph{condizionata} $\prob{E_1 \cond E_2}$ di $E_1$ dato $E_2$
(cioè la probabilità che si verifichi l'evento $E_1$ nel caso in cui sappiamo
già che si è verificato l'evento $E_2$) come
\begin{align}\label{eq:cond_prob_definition}
  \prob{E_1 \cond E_2} = \frac{\prob{E_1 \cap E_2}}{\prob{E_2}}.
\end{align}
Intuitivamente, come si vede in figura~\ref{fig:probabilita_venn}, la
probabilità condizionata $\prob{E_1 \cond E_2}$ rappresenta la misura
dell'intersezione $E_1 \cap E_2$ \emph{relativamente} ad $E_2$. Si può
dimostrare banalmente che la probabilità condizionata soddisfa gli assiomi
della probabilità---e di fatto ogni probabilità $\prob{E}$ può essere
vista come una probabilità condizionata $\prob{E \cond \Omega}$.

\begin{examplebox}
  \begin{example}
    Qual è la probabilità che in un lancio di un dado equo a sei facce esca
    il numero $2$, condizionata al fatto che l'uscita sia pari? Intuitivamente
    risponderemmo $\nicefrac{1}{3}$, poiché ci sono tre numeri pari tra
    $1$ e $6$---ossia $\{2,~4,~6\}$. Formalmente dobbiamo applicare
    la~\eqref{eq:cond_prob_definition}; $\{2\} \cap \{2,~4,~6\} = \{2\}$ per
    cui il numeratore è $\nicefrac{1}{6}$, mentre il denominatore è
    $\prob{\text{pari}} = \nicefrac{1}{2}$, da cui segue che la nostra
    probabilità condizionata è effettivamente $\nicefrac{1}{3}$.
  \end{example}

  \begin{example}
    In un mazzo da $52$ carte la probabilità di estrarre un re condizionata
    ad aver estratto una carta di cuori è
    $\prob{\text{K} \cond \heartsuit} = \nicefrac{1}{13}$. Di
    converso la probabilità di estrarre una carta di cuori condizionata
    all'aver estratto un re è
    $\prob{\heartsuit \cond \text{K}} = \nicefrac{1}{4}$.
  \end{example}
\end{examplebox}

Il concetto di probabilità condizionata è utile, operativamente, per
calcolare la probabilità che due eventi si verifichino contemporaneamente
(cioè per calcolare la misura della loro intersezione):
la~\eqref{eq:cond_prob_definition} si riscrive banalmente come
\begin{align}\label{eq:prob_intersection}
  \prob{E_1 \cap E_2} = \prob{E_1 \cond E_2}\prob{E_2} =
  \prob{E_2 \cond E_1}\prob{E_1}
\end{align}
(dove la doppia uguaglianza deriva banalmente dal fatto che
$E_1 \cap E_2 = E_2 \cap E_1$.)


\subsection{Eventi indipendenti: un esempio elementare}

Si dice che due eventi $E_1$ ed $E_2$ sono \emph{indipendenti} se il fatto
che si sia verificato $E_2$ non influenza la probabilità che si verifichi
$E_1$ (e viceversa), cioè se
\begin{align}\label{eq:indipendenza_eventi}
  \prob{E_1 \cond E_2} = \prob{E_1} \quad \text{e} \quad
  \prob{E_2 \cond E_1} = \prob{E_2}.
\end{align}
Per la definizione di probabilità condizionata~\eqref{eq:prob_intersection}
la condizione di indipendenza tra eventi si può riscrivere nella forma
più espressiva
\begin{align}\label{eq:prob_mult_ind}
  \prob{E_1 \cap E_2} = \prob{E_1}\prob{E_2}.
\end{align}
(Notiamo, per inciso, che il lettore non deve confondere la nozione di eventi
indipendenti con quella di eventi disgiunti---eventi, cioè, che hanno
intersezione nulla in $\Omega$. In quel caso gli eventi si dicono, come abbiamo
già avuto modo di dire, incompatibili.)

Il concetto di indipendenza di eventi è più sottile di quanto non possa
sembrare in apparenza---cosa che illustriamo con un semplice esempio.
Consideriamo il nostro solito mazzo di 52 carte. Come sappiamo la probabilità
di estrarre un re è $\prob{\text{K}} = \nicefrac{1}{13}$ e la probabilità
di estrarre una carta di cuori è $\prob{\heartsuit} = \nicefrac{1}{4}$.
Qual è la probabilità di estrarre il re di cuori? Sappiamo già la
risposta, perché esiste un solo re di cuori nel mazzo per cui la probabilità
di estrarlo è $\prob{\text{K} \cap \heartsuit} = \nicefrac{1}{52}$. Il fatto
che
\begin{align*}
  \prob{\text{K}} \prob{\heartsuit} =
  \frac{1}{13}\times\frac{1}{4} =
  \frac{1}{52} = \prob{\text{K} \cap \heartsuit}
\end{align*}
dimostra che i due eventi ("la carta estratta è un re" e "la carta
estratta è di cuori") sono indipendenti.

Proviamo adesso a complicare appena il nostro problema ed aggiungiamo un jolly
al nostro mazzo (che diventa così di 53 carte). La probabilità di estrarre
un re diviene $\prob{\text{K}} = \nicefrac{4}{53}$ mentre quella di estrarre
una carta di cuori è adesso $\prob{\heartsuit} = \nicefrac{13}{53}$. La
probabilità di estrarre il re di cuori è banalmente
$\prob{\text{K} \cap \heartsuit} = \nicefrac{1}{53}$ per cui
\begin{align*}
  \prob{\text{K}} \prob{\heartsuit} =
  \frac{4}{53}\times\frac{13}{53} =
  \frac{52}{2809} \neq \prob{\text{K} \cap \heartsuit},
\end{align*}
cioè i nostri due eventi non sono più indipendenti!

\begin{figure}
  \begin{venndiagram2sets}%
  [tikzoptions={xscale=1.55,yscale=1.15}, overlap=1.2cm,
    labelA=$\text{K}~(4)$,
    labelB=$\heartsuit~(13)$,
    labelAB=$\text{K} \cap \heartsuit~(1)$,
    labelNotAB=$\Omega~(52)$]
  \fillACapB
\end{venndiagram2sets}\hfill
\begin{venndiagram2sets}%
  [tikzoptions={xscale=1.55,yscale=1.15}, overlap=1.2cm,
    labelA=$\text{K}~(4)$,
    labelB=$\heartsuit~(13)$,
    labelAB=$\text{K} \cap \heartsuit~(1)$,
    labelNotAB=$\Omega~(53)$]
  \fillACapB
\end{venndiagram2sets}

  \caption{Diagrammi di Venn relativi agli esempi che illustrano l'indipendenza
    tra eventi. I numeri tra parentesi indicano il numero di elementi che
    compongono gli insiemi corrispondenti e consentono di calcolare in modo
    immediato le probabilità condizionate necessarie.
    I diagrammi illustrano come la vera differenza causata dall'aggiunta del
    jolly sia l'aumentare il numero di elementi dello spazio campionario da
    $52$ a $53$.}
  \label{fig:probabilita_cond_carte_venn}
\end{figure}

Che cosa è successo? Se analizziamo i due eventi più da vicino notiamo
che nel primo caso: se estraiamo una carta di cuori la probabilità che
essa sia un re è~$\nicefrac{1}{13}$; se la carta estratta \emph{non} è
di cuori, allora la probabilità che essa sia un re è~$\nicefrac{3}{39}$
(ci sono $13 \times 3 = 39$ carte non di cuori e tra queste $3$ sono re),
cioè di nuovo~$\nicefrac{1}{13}$---i due eventi sono indipendenti.
Nel secondo caso: se estraiamo una carta di cuori la probabilità che
essa sia un re è, come prima, $\nicefrac{1}{13}$; se però la carta
estratta \emph{non} è di cuori, allora la probabilità che essa sia un re
è~$\nicefrac{3}{40} \neq \nicefrac{1}{13}$. In altre parole, se sappiamo che
la carta estratta è di fiori, allora sappiamo che non può essere un jolly,
e questo cambia la probabilità che essa sia un re---i due eventi non sono
indipendenti. Ma se riesaminiamo brevemente l'esempio del mazzo di $53$ carte
alla luce di quanto sappiamo sulla probabilità condizionata, tutto diventa
perfettamente consistente:
\begin{align*}
  \prob{\text{K} \cap \heartsuit} = \left \{ \begin{array}{ll}
    \prob{\text{K} \cond \heartsuit} \prob{\heartsuit} =
    \frac{1}{13}\times\frac{13}{53} = \frac{1}{53}\\[7pt]
    \prob{\heartsuit \cond \text{K}} \prob{\text{K}} =
    \frac{1}{4}\times\frac{4}{53} = \frac{1}{53}\end{array}\right.
\end{align*}


\section{Il teorema di Bayes}
\label{sec:teorema_bayes}

La~\eqref{eq:prob_intersection} ha come conseguenza immediata e banale il
cosiddetto teorema di Bayes, che lega tra loro le probabilità condizionate
$\prob{E_1 \cond E_2}$ e $\prob{E_2 \cond E_1}$
\begin{align}\label{eq:teorema_bayes}
  \prob{E_1 \cond E_2} = \frac{\prob{E_2 \cond E_1}\prob{E_1}}{\prob{E_2}}.
\end{align}
Sfruttando il fatto che $E_1 \cup \overline{E_1} = \Omega$,
la~\eqref{eq:teorema_bayes} si può scrivere anche nella forma esplicita
(ma leggermente meno compatta)
\begin{align}
  \prob{E_1 \cond E_2} =
  \frac{\prob{E_2 \cond E_1}\prob{E_1}}{\prob{E_2 \cond E_1}\prob{E_1} +
  \prob{E_2 \cond \overline{E_1}}\prob{\overline{E_1}}}.
\end{align}
Più in generale, se si dispone di una partizione dello spazio campionario
in un insieme $\{E_i\}$ di insiemi disgiunti ($E_i \cap E_j = \emptyset$) la cui
unione coincida con lo spazio stesso ($\cup_i E_i = \Omega$), allora il teorema
di Bayes si può anche esprimere nella forma
\begin{align}
  \prob{E_1 \cond E_2} =
  \frac{\prob{E_2 \cond E_1}\prob{E_1}}{\sum_i \prob{E_2 \cond E_i}\prob{E_i}}.
\end{align}

La~\eqref{eq:teorema_bayes} è di fondamentale importanza perché collega un
problema di probabilità diretta al problema corrispondente di probabilità
inversa. Torniamo per un attimo alla nostra moneta ideale. Se sappiamo
esattamente la probabilità $p$ che esca testa (ad esempio $p = 0.5$) in un
singolo lancio è facile calcolare la probabilità di ottenere esattamente
$n$ teste in $N$ lanci---si tratta della distribuzione binomiale che vedremo
nella sezione~\ref{sec:distribuzione_binomiale}. Ma non è questo che fa il
Fisico, nella sua professione. Tutto il contrario: il Fisico lancia la
moneta $N$ volte e, a partire dal numero di volte $n$ in cui esce testa,
cerca di inferire la probabilità (incognita) $p$. \`E chiaro che i due
problemi sono legati tra di loro, ed in questa sezione cominceremo a sviscerare
questo collegamento.


\subsection{Un semplice problema di probabilità inversa}
\label{sec:rilevatore_banconote}

A questo punto in un qualsiasi articolo o libro di statistica trovereste
il famoso problema del test per l'HIV~\cite{dagostini_review}. Noi preferiamo
invece una variante meno cruenta---quella del rilevatore automatico di
banconote false. Le specifiche fornite dalla casa produttrice dicono che
l'oggetto ha un'efficienza del $100\%$ nel segnalare banconote contraffatte
ma, allo stesso tempo, ha un tasso di falsi positivi del $5\%$---cioè nel
$5\%$ dei casi segnala come contraffatta una banconota che invece è
autentica. (In fondo non esistono dispositivi perfetti.) Prendiamo dunque una
banconota e la passiamo dal rilevatore, che la segnala come contraffatta.
Qual è la probabilità che la banconota sia effettivamente falsa?

Prima di rispondere $95\%$ (come uno potrebbe ingenuamente essere tentato di
fare), fermiamoci per un attimo a pensare e convinciamoci che la domanda,
così come è scritta, è mal posta. Manca un ingrediente fondamentale,
ossia la probabilità \emph{a priori} che, presa una banconota a caso, essa
sia contraffatta. Dopo tutto in un mondo ideale in cui la contraffazione non
esistesse (e quindi non esistessero banconote false) se il nostro dispositivo
scattasse non potremmo far altro che imputare il fatto al $5\%$ dei falsi
positivi, non vi pare? (E in quel caso la probabilità che cerchiamo sarebbe
identicamente nulla.)

Torniamo dunque un attimo indietro e cerchiamo di formalizzare meglio il
problema. Per snellire la notazione definiamo i seguenti eventi:
\begin{align*}
A & {\text{: la banconota è autentica}}\\
C & {\text{: la banconota è contraffatta}}\\
\cmark & {\text{: il rilevatore segnala la banconota come autentica}}\\
\xmark & {\text{: il rilevatore segnala la banconota come contraffatta}}
\end{align*}
Notiamo che questi eventi sono a due a due disgiunti e partizionano in due modi
diversi il nostro spazio degli eventi $\mathcal{F}$, che possiamo vedere come il
prodotto cartesiano dei due insiemi $\{A,~C\}$ e
$\{\cmark,~\xmark\}$
\begin{align*}
  \mathcal{F} = \{(A,~\cmark),~(A,~\xmark),~(C,~\cmark),~(C,~\xmark)\}.
\end{align*}
Le specifiche del costruttore si possono allora riscrivere (e precisare) come
\begin{align*}
  \prob{\xmark \cond C} &= 1\\
  \prob{\xmark \cond A} &= 0.05
\end{align*}
Assumeremo inoltre che la probabilità a priori che una banconota sia
contraffatta sia $\prob{C} = 10^{-3}$ (cioè che una banconota su mille
in circolazione sia, in media, falsa).

Torniamo dunque al nostro problema iniziale, cioè la stima della
probabilità che una banconota segnalata come contraffatta dal nostro
rilevatore sia effettivamente contraffatta. Per la~\eqref{eq:teorema_bayes}
possiamo scrivere
\begin{align*}
  \prob{C \cond \xmark} =
  \frac{\prob{\xmark \cond C}\prob{C}}{\prob{\xmark}}.
\end{align*}
L'unica cosa che ci manca per calcolare la probabilità cercata è
$\prob{\xmark}$, che non abbiamo direttamente, ma che possiamo calcolare
sfruttando il fatto che $A \cup C = \mathcal{F}$
\begin{align*}
  \prob{\xmark} =
  \prob{\xmark \cap A} + \prob{\xmark \cap C} =
  \prob{\xmark \cond A}\prob{A} +
  \prob{\xmark \cond C}\prob{C}.
\end{align*}
Questo ci permette di scrivere la risposta cercata nella forma meno compatta
ma esplicita (che è anche la forma in cui si trova spesso espresso in
letteratura il teorema di Bayes)
\begin{align*}
  \prob{C \cond \xmark} =
  \frac{\prob{\xmark \cond C}\prob{C}}%
       {\prob{\xmark \cond A}\prob{A} +
  \prob{\xmark \cond C}\prob{C}} =
  \frac{1 \times 0.001}{0.05 \times 0.999 + 1 \times 0.001} \approx 2\%
\end{align*}
Sorprendentemente (?) la probabilità cercata è dell'ordine del $2\%$,
nonostante l'efficienza del nostro rilevatore sia il $100\%$ ed il tasso di
falsi positivi sia solo del $5\%$. Il commento sopra a proposito del mondo
ideale (senza contraffazione) può essere qui precisato notando che se
$\prob{C} = 0$, allora $\prob{C \cond \xmark} = 0$
come anticipato (vedi figura~\ref{fig:rilevatore_banconote}).

\pgffigone{rilevatore_banconote}{
  Probabilità condizionata $\prob{C \cond \xmark}$ che la
  banconota segnalata dal nostro rilevatore sia effettivamente contraffatta
  in funzione della probabilità a priori che una banconota scelta a caso tra
  quelle in circolazione sia contraffatta $\prob{C}$, dati
  $\prob{\xmark \cond C} = 1$ e
  $\prob{\xmark \cond A} = 0.05$. Per $\prob{C} = 10^{-3}$
  si ha il $\approx 2\%$ calcolato sopra.
}

(Pensate un attimo alla vostra esperienza di tutti i giorni: perché il
cassiere del supermercato non chiama immediatamente la sicurezza ogni volta
che una banconota viene segnalata come contraffatta dal rilevatore? Quante
volte vi è capitato di vedere la stessa banconota segnalata come autentica
al secondo o terzo passaggio?)


\subsection{Ancora sul teorema di Bayes: un po' di nomenclatura}
\label{sec:bayes_nomenclatura}

Torniamo per un attimo sul teorema di Bayes in un contesto più vicino al
mestiere del Fisico. Supponiamo di aver raccolto un campione di dati $d$
e di avere un modello del nostro sistema fisico dipendente da un parametro
$\theta$, che vogliamo stimare a partire dai dati stessi. Possiamo riscrivere
la~\eqref{eq:teorema_bayes} come
\begin{align}\label{eq:teorema_bayes_alt}
  \prob{\theta \cond d} = \frac{\prob{d \cond \theta} \prob{\theta}}{\prob{d}}.
\end{align}
Il primo termine dell'equazione $\prob{\theta \cond d}$ è detto
\emph{probabilità a posteriori}, o \foreign{posterior}, ed è esattamente ciò
a cui siamo interessati da Fisici: la probabilità (inversa) che il parametro
del nostro modello assuma un certo valore, date le nostre misure.
$\prob{d \cond \theta}$, che si dice \emph{verosimiglianza} o \foreign{likelihood},
è invece la probabilità (diretta) di misurare $d$, dato un generico valore
di $\theta$, ed in pratica è specificata completamente dal nostro modello.
$\prob{\theta}$ prende il nome di \emph{probabilità a priori}, o \foreign{prior},
ed è il termine del teorema di Bayes su cui le diverse scuole statistiche
discutono più veementemente.

E $\prob{d}$? A prima vista potrebbe sembrare assurdo parlare della
probabilità che le nostre misure $d$ assumano una serie di valori dati---in
fondo una volta che le abbiamo fatte le misure sono fissate, no?
(Ma ricordatevi: nello schema soggettivista possiamo assegnare una probabilità
a qualsiasi evento o proposizione, per cui questa non è una difficoltà
insuperabile.) Ma se guardiamo meglio la~\eqref{eq:teorema_bayes_alt} la cosa
fondamentale è che stiamo facendo una affermazione (probabilistica) su
$\theta$, mentre il denominatore del membro di destra dell'equazione dipende
solo da $d$ per cui, dal nostro punto di vista, esso è solo una costante
di normalizzazione. Allora la~\eqref{eq:teorema_bayes_alt} può essere
riscritta in modo espressivo nella forma
\begin{align}
  \text{probabilità a posteriori} \propto
  \text{verosimiglianza} \times \text{probabilità a priori},
\end{align}
che si trova frequentemente in letteratura e che utilizzeremo, sia pure solo
occasionalmente, nel seguito.


\section{Variabili casuali e funzioni di distribuzione}

Si dice \emph{variabile casuale} o \emph{variabile aleatoria} una variabile
che rappresenta la realizzazione numerica di un processo casuale, per cui
il suo valore è soggetto a fluttuazioni casuali e non è noto a priori.
Esistono casi di variabili casuali discrete, cioè variabili che possono
assumere un numero finito o numerabile di valori, e casi di variabili
continue, ed entrambe possono essere definite su intervalli finiti o infiniti.

\begin{examplebox}
  \begin{example}
    L'uscita del lancio di un dado a sei facce (ossia la faccia del dado
    rivolta verso l'alto) è una variabile casuale discreta che può assumere
    esattamente sei valori: $1 \ldots 6$.
  \end{example}

  \begin{example}
    Il tempo necessario per arrivare da casa al luogo di lavoro è un esempio
    di variabile casuale continua.
  \end{example}
\end{examplebox}


\subsection{Variabili casuali discrete}

Consideriamo una variabile discreta $x$ che può assumere $n$ valori distinti
possibili $x_1 \ldots x_n$. Indicando con $\prob{x_k}$ la probabilità che
$x$ assuma il valore $x_k$, definiamo \emph{funzione di distribuzione}%
\footnote{Il termine "funzione di distribuzione" è talvolta usato in
  letteratura per indicare la funzione cumulativa di una distribuzione, di cui
  ci occuperemo nella sezione~\ref{sec:funzione_cumulativa}. Un altro termine
  molto usato per indicare quello che noi chiamiamo funzione di distribuzione
  nel caso continuo è l'inglese \foreign{probability density function},
  che si trova anche abbreviato come \foreign{pdf}.}
di $x$ (o semplicemente \emph{distribuzione} di $x$) la funzione che associa ad
ogni valore $x_k$ della variabile $x$ la sua probabilità $\prob{x_k}$.
La funzione di distribuzione per una variabile discreta può essere
rappresentata in forma tabellare o, più efficacemente, con un grafico a barre,
come mostrato in figura~\ref{fig:pdf_un_dado_pdf_somma_due_dadi}.

\begin{examplebox}
  \begin{example}\label{exp:pdf_un_dado}
    Sia $x$ l'uscita del lancio di un dado a sei facce. Se il dado è equo,
    ciascuno dei valori $x_k = 1\ldots6$ ha probabilità $1/6$ di uscire.
    La funzione di distribuzione corrispondente è mostrata in
    figura~\ref{fig:pdf_un_dado_pdf_somma_due_dadi} (a sinistra).
  \end{example}

  \begin{example}\label{exp:pdf_somma_due_dadi}
    Consideriamo il lancio di due dadi e sia $x$ la somma delle due uscite.
    $x$ può assumere tutti i valori interi tra $2$ e $12$, ma stavolta gli
    $x_k$ non sono equiprobabili anche se i dadi sono equi. I casi possibili
    sono in totale $6 \times 6 = 36$ ed ognuno può essere univocamente
    identificato con la coppia $(u_1, u_2)$, in cui $u_1$ è l'uscita del primo
    dado e $u_2$ quella del secondo. Mentre il $2$ ha solamente un modo per
    realizzarsi, $(1, 1)$, il $4$, ad esempio, può realizzarsi in tre modi
    diversi: $(1, 3)$, $(2, 2)$ o $(3, 1)$, come illustrato nella
    tabella~\ref{tab:pdf_due_dadi}. La probabilità di uscita può allora
    essere calcolata come rapporto tra il numero di casi favorevoli ed il
    numero totale di casi possibili, ed ha la forma triangolare mostrata in
    figura~\ref{fig:pdf_un_dado_pdf_somma_due_dadi} (a destra).
  \end{example}
\end{examplebox}


\begin{table}[!hbt]
  \tablehstack{
    \begin{tabular}{lll}
      \hline
      $x$ & Combinazioni possibili & $\prob{x}$\\
      \hline
      \hline
      $2$ & $(1,~1)$ & $\nicefrac{1}{36}$\\
      $3$ & $(1,~2),~(2,~1)$ & $\nicefrac{2}{36}$\\
      $4$ & $(1,~3),~(3,~1),~(2,~2)$ & $\nicefrac{3}{36}$\\
      $5$ & $(1,~4),~(4,~1),~(2,~3),~(3,~2)$ & $\nicefrac{4}{36}$\\
      $6$ & $(1,~5),~(5,~1),~(2,~4),~(4,~2),~(3,~3)$ & $\nicefrac{5}{36}$\\
      $7$ & $(1,~6),~(6,~1),~(2,~5),~(5,~2),~(4,~3),~(3,~4)$ &
      $\nicefrac{6}{36}$\\
      $8$ & $(2,~6),~(6,~2),~(3,~5),~(5,~3),~(4,~4)$ & $\nicefrac{5}{36}$\\
      $9$ & $(3,~6),~(6,~3),~(4,~5),~(5,~4)$ & $\nicefrac{4}{36}$\\
      $10$ & $(4,~6),~(6,~4),~(5,~5)$ & $\nicefrac{3}{36}$\\
      $11$ & $(5,~6),~(6,~5)$ & $\nicefrac{2}{36}$\\
      $12$ & $(6,~6)$ & $\nicefrac{1}{36}$\\
      \hline
    \end{tabular}
  }{
    \caption{Funzione di distribuzione per la somme delle uscite di due dadi
      equi. Per completezza tutte le combinazioni di uscite che danno luogo
      ad un dato valore della somma sono elencate esplicitamente.}
    \label{tab:pdf_due_dadi}
  }
\end{table}


\pgffigtwo{pdf_un_dado}{pdf_somma_due_dadi}{
  Funzione di distribuzione per l'uscita di un dado equo a sei facce (a
  sinistra) e per la somma delle uscite di due dadi equi a sei facce
  (a destra)---vedi gli esempi~\ref{exp:pdf_un_dado} e
  \ref{exp:pdf_somma_due_dadi}.
}

In questo contesto il secondo assioma di Kolmogorov si scrive nella forma di
una \emph{condizione di normalizzazione}, che tutte le funzioni di
distribuzione debbono soddisfare
\begin{align}
  \sum_k \prob{x_k} = 1
\end{align}
(la sommatoria è estesa a tutti i valori $x_k$ che la variabile $x$ può
assumere). \`E banale verificare direttamente che le funzioni di distribuzione
illustrate negli esempi~\ref{exp:pdf_un_dado} e \ref{exp:pdf_somma_due_dadi}
ed in figura~\ref{fig:pdf_un_dado_pdf_somma_due_dadi} sono normalizzate.


\subsection{Variabili casuali continue}

Nel caso di una variabile continua la definizione di funzione di distribuzione
data per una variabile discreta non è più applicabile poiché la
probabilità che la variabile $x$ assuma un valore \emph{esattamente} definito
è identicamente zero (vedremo tra un attimo che si tratta essenzialmente
di un integrale su un dominio di misura nulla).
\`E invece sensato chiedersi quale sia la probabilità che la $x$ assuma
un valore compreso in un generico intervallo compreso tra $x_0$ ed $x_0 + dx$
\begin{align*}
  \prob{x_0, dx} = \prob{x_0 \leq x < x_0 + dx}.
\end{align*}
A questo punto, se dividiamo per la larghezza dell'intervallo e prendiamo il
limite per $dx \rightarrow 0$, otteniamo una sorta di probabilità
specifica o probabilità per unità di intervallo che chiamiamo
\emph{densità di probabilità}
\begin{align}\label{eq:definizione_densita_prob}
  p(x_0) = \lim_{dx \rightarrow 0} \frac{\prob{x_0, dx}}{dx}
\end{align}
La~\eqref{eq:definizione_densita_prob} è una sorta di rapporto incrementale,
per cui la densità di probabilità è, in un certo senso, la derivata
della funzione probabilità. La probabilità (infinitesima) che la variabile
casuale assuma un valore compreso nell'intervallo $dx$ centrato attorno al
valore $x_0$ si scrive come
\begin{align*}
  \prob{x_0, dx} = p(x_0) dx
\end{align*}
e, corrispondentemente, la probabilità che essa assuma un valore compreso
nell'intervallo \emph{finito} $[x_1,~x_2]$ è data da:
\begin{align}
  \prob{x_1 \leq x \leq x_2} = \int_{x_1}^{x_2} p(x)dx.
\end{align}
(Ne segue, per inciso, che la densità di probabilità di una variabile
casuale continua $x$ ha le dimensioni fisiche dell'inverso di $x$.)
La condizione di normalizzazione per una funzione di distribuzione di
variabile continua si scrive allora come
\begin{align}\label{eq:normalizzazione_pdf}
  \int_{-\infty}^{\infty} p(x)dx = 1.
\end{align}
(Formalmente l'integrale andrebbe calcolato sul supporto della funzione di
distribuzione ma, se assumiamo che la funzione di distribuzione sia
identicamente nulla al di fuori del supporto stesso, possiamo integrare su
tutta la retta reale senza perdere di generalità.)

\begin{examplebox}
  \begin{example}
    Il più semplice esempio di funzione di distribuzione di variabile
    continua è quello in cui la densità di probabilità è costante
    su un dato intervallo $[a,~b]$ (vedi figura~\ref{fig:pdf_esempio_uniforme} e
    sezione~\ref{sec:distribuzione_uniforme}). La normalizzazione è fissata
    dalla~\eqref{eq:normalizzazione_pdf} e vale $1/(b - a)$.
  \end{example}
\end{examplebox}

\pgffigone{pdf_esempio_uniforme}{
  Esempio di funzione di distribuzione di variabile continua, in cui la densità
  di probabilità è costante tra $0$ e $1$. (Per ovvi motivi questa
  distribuzione prende il nome di distribuzione uniforme.) La distribuzione
  è correttamente normalizzata e la regione ombreggiata illustra il concetto
  di probabilità come integrale della densità di probabilità su un
  intervallo finito.
}


\section{Valore di aspettazione}

Sia data una funzione $f(x)$ di una variabile casuale $x$ (continua o discreta).
Definiamo il \emph{valore di aspettazione}, o \emph{speranza matematica},
o semplicemente \emph{speranza} di $f(x)$ come
\begin{align}
  \dccases[.]{\expect{f(x)}}%
          {\sum_k f(x_k)\prob{x_k}}%
          {\intinf f(x)p(x)\diff}
\end{align}
Notiamo esplicitamente che il valore di aspettazione trasforma una funzione
di $x$ in un \emph{numero}. Si tratta di una sorta di procedura di media in
cui i valori di $f(x)$ sono pesati con il valore della funzione di distribuzione
di $x$. \`E uno strumento fondamentale che, come vedremo in seguito, permette
di definire molte delle proprietà di base comuni alle distribuzioni.

\begin{examplebox}
  \begin{example}
    Torniamo per un attimo alla sezione~\ref{sec:lotto_probabilita} e supponiamo
    di giocare 1~euro su un ambo secco sulla ruota di Napoli. Consideriamo
    la nostra vincita come una variabile casuale e calcoliamo il suo valore di
    aspettazione. La probabilità di vincere è $10/4005$, ed in tal caso
    il guadagno netto è $250 - 1 = 249$~euro. In caso contrario semplicemente
    perdiamo 1~euro (che possiamo considerare una vincita negativa di
    $-1$~euro). Il valore di aspettazione cercato è dunque
    \begin{align*}
      249 \times \frac{10}{4005} - 1 \times \frac{3995}{4005} \approx
      -0.38~\text{euro},
    \end{align*}
    che è come dire che in media perdiamo circa 38~centesimi per ogni
    euro giocato. (Come è ovvio dalla tabella nella
    sezione~\ref{sec:lotto_probabilita} la situazione è anche peggio
    per il terno, la quaterna e la cinquina.)
  \end{example}
\end{examplebox}

Per la linearità di integrale e sommatoria, il valore di aspettazione è un
\emph{operatore lineare}, nel senso che
\begin{align}
  \expect{c_1 f(x) + c_2 g(x)} = c_1 \expect{f(x)} + c_2 \expect{g(x)},
\end{align}
come si verifica facilmente per sostituzione. Notiamo anche che, per la
condizione di normalizzazione, il valore di aspettazione di una \emph{costante}
(cioè di una espressione che non dipende dalla variabile casuale $x$)
è uguale alla costante stessa
\begin{align}
  \expect{c} = c.
\end{align}


\section{Tendenza centrale e dispersione intorno alla media}

Nelle applicazioni pratiche è comune condensare l'informazione
contenuta nella funzione di distribuzione di una variabile casuale in pochi
parametri significativi. Tipicamente siamo interessati a sapere quale valore
ci aspettiamo che la variabile assuma \emph{in media} e quanto la variabile
stessa si discosti \emph{in media} da questo valore.


\subsection{Media, mediana e moda}

Definiamo il \emph{valor medio} (o semplicemente la \emph{media}) di una
variabile casuale $x$ (continua o discreta) come il valore di aspettazione
di $x$
\begin{align}\label{eq:mean}
  \dccases[.]{\mu = \expect{x}}%
          {\sum_k x_k\prob{x_k}}%
          {\intinf xp(x)\diff}
\end{align}
Nel seguito indicheremo il valor medio di $x$ con $\mu$, $\mu_x$ o
$\expect{x}$ a seconda del contesto, cercando di conciliare sintesi e
chiarezza. Notiamo esplicitamente che se $c$ è una costante
\begin{align}
  \expect{cx} = c\expect{x}.
\end{align}

\begin{examplebox}
  \begin{example}\label{exp:media_dado}
    Sia la variabile casuale $x$ l'uscita di un dado equo a sei facce. La media
    vale
    \begin{align*}
      \mu = \sum_{k=1}^{6} x_k \prob{x_k} =
      \frac{1}{6} \sum_{k = 1}^{6} k =
      \frac{1}{6} (1 + 2 + 3 + 4 + 5 + 6) = \frac{7}{2}.
    \end{align*}
    Notiamo esplicitamente che in questo caso il valor medio non coincide con
    nessuno dei valori che $x$ può assumere.
  \end{example}

  \begin{example}\label{exp:media_due_dadi}
    Ripetiamo l'esercizio precedente nel caso in cui $x$ sia la somma delle
    uscite nel lancio di due dadi.
    \begin{eqnarray*}
      \mu &=& \displaystyle \sum_{k=2}^{12} x_k \prob{x_k} =
      2 \times \frac{1}{36} + 3 \times \frac{2}{36} +
      4 \times \frac{3}{36} + 5 \times \frac{4}{36} + 6 \times \frac{5}{36} +\\
      &+& \displaystyle 7 \times \frac{6}{36} + 8 \times \frac{5}{36} +
      9 \times \frac{4}{36} + 10 \times \frac{3}{36} + 11 \times \frac{2}{36} +
      12 \times \frac{1}{36} = \frac{252}{36} = 7
    \end{eqnarray*}
    che è esattamente il doppio di quanto calcolato poco fa per un
    singolo dado. (Vedremo nel seguito che non si tratta di un caso.)
  \end{example}

  \begin{example}\label{exp:media_dist_uniforme}
  Sia $x$ una variabile casuale continua distribuita uniformemente tra $0$ ed
  $1$ (cioè $p(x) = 1$ in $[0,~1]$ e $p(x) = 0$ fuori). La media vale:
  \begin{align*}
    \mu = \intinf x p(x)\diff = \int_{0}^{1} x \diff =
    \eval{\frac{x^2}{2}}{0}{1} = \frac{1}{2} - 0 = \frac{1}{2},
  \end{align*}
  che coincide con il punto medio dell'intervallo $[0,~1]$.
  \end{example}
\end{examplebox}

Nel caso particolare di una variabile casuale e discreta per cui si abbiano
$n$ uscite possibili $x_k$ equiprobabili (cioè
$\prob{x_1} = \cdots = \prob{x_n} = 1/n$) si ha
\begin{align*}
  \mu = \sum_{k=1}^n x_k\prob{x_k} = \frac{1}{n}\sum_{k=1}^n x_k,
\end{align*}
cioè la nostra definizione \eqref{eq:mean} di media coincide con la
media aritmetica delle uscite possibili, come è ragionevole aspettarsi.
Nel caso generale in cui i  valori $x_k$ non siano equiprobabili,
essi vengono correttamente \emph{pesati} a seconda della probabilità
corrispondente in modo che la media si sposti verso i valori più probabili.

Il valor medio non è l'unica stima possibile di tendenza centrale. Definiamo
{\itshape mediana} di una distribuzione quel valore $\median$ della variabile
casuale tale che
\begin{align}\label{eq:median}
  \prob{x \le \median} = \prob{x \ge \median}.
\end{align}
Per una variabile casuale continua la mediana è dunque definita dalla
condizione
\begin{align*}
  \int_{-\infty}^{\median} p(x) \diff =
  \int_{\median}^{\infty} p(x) \diff = \frac{1}{2}.
\end{align*}
Per una variabile discreta non è detto che questo valore esista e sia
univocamente determinato (motivo per il quale la mediana è rilevante
soprattutto per distribuzioni continue). Se una funzione di distribuzione è
simmetrica rispetto al valor medio, media e mediana coincidono.

%\begin{examplebox}
%  \begin{example}\label{example:die_median}
%    Nel caso del dado a sei facce qualsiasi numero nell'intervallo
%    aperto $]3,4[$ soddisfa la~\eqref{eq:median}, per cui è necessario
%    fissare una convenzione per definire la mediana (possiamo ad esempio
%    prendere il valore centrale $\nicefrac{7}{2}$ di tale intervallo).
%  \end{example}
%\end{examplebox}

La \emph{moda} (o \emph{valore più probabile}) è semplicemente il valore
della variabile casuale (se esiste ed è unico) in corrispondenza del quale la
funzione di distribuzione ha un massimo.


\subsection{Varianza e deviazione standard}

Caratterizzare la \emph{dispersione} attorno alla media di una variabile
casuale significa definire una funzione il cui valore di aspettazione sia una
misura di quanto la variabile $x$ tenda a discostarsi dal proprio valor medio.
Potremmo essere tentati di utilizzare la funzione $f(x) = x - \mu$, ma è
facile verificare che
\begin{align*}
  \expect{x - \mu} = \expect{x} - \mu = \mu - \mu = 0.
\end{align*}
Questo valore di aspettazione non fornisce alcuna informazione utile in quanto
fluttuazioni per eccesso e fluttuazione per difetto tendono, statisticamente,
a compensarsi. Per ovviare a questo inconveniente si usa $f(x) = \sq{x - \mu}$
e si definisce la \emph{varianza} come%
\footnote{La quantità $f(x) = \abs{x - \mu}$ potrebbe servire altrettanto
  bene allo scopo, ma come vedremo tra breve, $\sq{x - \mu}$ ha alcune
  proprietà che la rendono più comoda da utilizzare nella pratica.}
\begin{align}\label{eq:variance}
  \dccases{\sigma^2 = \expect{\sq{x - \mu}}}%
          {\sum_k \sq{x_k - \mu}\prob{x_k}}%
          {\intinf \sq{x - \mu}p(x)\diff}
\end{align}
(nel seguito indicheremo la varianza di $x$ con $\sigma^2$, $\sigma^2_x$ o
$\var{x}$ a seconda del contesto). Definiamo inoltre la
\emph{deviazione standard} come radice quadrata della varianza
\begin{align}\label{eq:sigma}
  \sigma = \sqrt{\sigma^2}.
\end{align}
\`E proprio la deviazione standard (che, al contrario della varianza, ha le
stesse dimensioni fisiche della variabile casuale di partenza) a rappresentare
la misura della dispersione attorno alla media cercata. Esamineremo
dettagliatamente in seguito la connessione con l'errore statistico.
Notiamo, per inciso, che se $c$ è una costante
\begin{align}\label{eq:varianza_cost_molt}
  \var{cx} & = \expect{\left(cx - \expect{cx}\right)^2} =
  \expect{(cx - c\mu)^2} =
  \expect{c^2(x - \mu)^2} = c^2\expect{(x - \mu)^2} = c^2\var{x}.
\end{align}
Questa è una relazione molto importante: una costante passa fuori
dall'operatore varianza \emph{al quadrato}---da cui, se facciamo la radice
quadrata di entrambi i membri, si vede che la deviazione standard di $cx$
è, come uno si aspetterebbe, $c$~volte la deviazione standard di $x$.

\begin{examplebox}
  \begin{example}\label{exp:varianza_dado}
    Torniamo al nostro dado a sei facce. Utilizzando il valor medio calcolato
    nell'esempio~\ref{exp:media_dado} la varianza si scrive come
    \begin{align*}
      \sigma^2 & = \sum_{k = 1}^6(x_k - \mu)^2 \prob{x_k} =
      \frac{1}{6}\sum_{k = 1}^6\left( k - \frac{7}{2} \right)^2 =
      \frac{1}{6} \left(\frac{25}{4} + \frac{9}{4} + \frac{1}{4} +
      \frac{1}{4} + \frac{9}{4} + \frac{25}{4} \right) = \frac{35}{12},
    \end{align*}
    e la deviazione standard è banalmente $\sigma = \sqrt{\nicefrac{35}{12}}$.
  \end{example}

  \begin{example}\label{exp:varianza_due_dadi}
    Calcoliamo adesso la varianza per la somma delle uscite di due dadi
    equi
    \begin{eqnarray*}
      \sigma^2 &=& \sum_{k = 2}^{12}(x_k - \mu)^2 \prob{x_k}  =
      \sum_{k = 2}^{12} (x_k - 7)^2 \prob{x_k} =
      25 \times \frac{1}{36} + 16 \times \frac{2}{36} +
      9 \times \frac{3}{36} + 4 \times \frac{4}{36} + 1 \times \frac{5}{36} +\\
      &+& 0 \times \frac{6}{36} + 1 \times \frac{5}{36} +
      4 \times \frac{4}{36} + 9 \times \frac{3}{36} + 16 \times \frac{2}{36} +
      25 \times \frac{1}{36} = \frac{210}{36} = \frac{35}{6}.
    \end{eqnarray*}
    (Che è esattamente il doppio della varianza per l'uscita di un dado a sei
    facce calcolata nell'esercizio precedente.)
  \end{example}

  \begin{example}\label{exp:varianza_dist_uniforme}
    Consideriamo la variabile casuale $x$ distribuita uniformemente tra $0$ ed
    $1$ introdotta nell'esempio~\ref{exp:media_dist_uniforme}. La varianza
    si calcola come
    \begin{align*}
      \sigma^2 & = \intinf (x - \mu)^2p(x)\diff =
      \int_{0}^{1} \left(x - \frac{1}{2}\right)^2 \diff =
      \int_{0}^{1} \left(x^2 - x + \frac{1}{4}\right) \diff =
      \eval{\left(\frac{x^3}{3} - \frac{x^2}{2} + \frac{x}{4}\right)}{0}{1} =\\
      &= \frac{1}{3} - \frac{1}{2} + \frac{1}{4} = \frac{1}{12},
    \end{align*}
    e la deviazione standard è $\sigma = \sqrt{\nicefrac{1}{12}}$.
    Vedremo nel seguito che questo esempio non è solo di interesse accademico.
  \end{example}
\end{examplebox}

Partendo dalla definizione di varianza~\eqref{eq:variance}, attraverso una
semplice manipolazione algebrica
\begin{align*}
  \sigma^2 & = \expect{(x - \mu)^2} = \expect{(x^2 - 2x\mu + \mu^2)} =
  \expect{x^2} - \expect{2x\mu} + \expect{\mu^2} =\\
  & =\expect{x^2} - 2\mu \expect{x} + \mu^2 =
  \expect{x^2} - 2\mu^2 + \mu^2
\end{align*}
si arriva alla relazione utile che utilizzeremo spesso nel seguito
\begin{align}\label{eq:variance_alt}
  \sigma^2 =  \expect{x^2} - \mu^2.
\end{align}


\subsection{La larghezza a metà altezza}

Un concetto utile che si applica alle funzioni di distribuzione di variabile
continua è quello di \emph{semilarghezza a metà altezza}, illustrato in
figura~\ref{fig:hwhm}. Se assumiamo che la distribuzione sia unimodale---cioè
che abbia un solo massimo---la retta orizzontale che interseca l'asse delle
ordinate in corrispondenza della metà del valore del punto di massimo della
distribuzione intersecherà la distribuzione stessa esattamente in due punti
$x_a$ ed $x_b$. Definiamo allora la larghezza a metà altezza $\fwhm$
(\foreign{full width at half maximum} in inglese) come
\begin{align}
  \fwhm = \abs{x_b - x_a},
\end{align}
e la semilarghezza a metà altezza $\hwhm$ (\foreign{half width at half maximum}
in inglese) come
\begin{align}
  \hwhm = \frac{\abs{x_b - x_a}}{2}.
\end{align}

\pgffigone{hwhm}{
  Significato geometrico della larghezza a metà altezza. Notiamo
  esplicitamente che, come in questo caso, non è necessario che la
  distribuzione sia simmetrica rispetto al valor medio.
}

Nella maggior parte delle distribuzioni continue di interesse pratico la
semilarghezza a metà altezza è una stima ragionevole della deviazione
standard, nel senso che
\begin{align}
  \hwhm = c\sigma
\end{align}
con $c$ dell'ordine dell'unità---e ne vedremo molti esempi concreti nel
seguito. Questo offre un'ulteriore interpretazione geometrica del concetto di
deviazione standard come \emph{larghezza} di una distribuzione
(vedi figura~\ref{fig:hwhm}).


\subsection{La disuguaglianza di Chebyshev ed il significato della deviazione
  standard}

Il significato profondo della deviazione standard è sostanzialmente che,
se misuriamo le deviazioni dal valor medio in unità di $\sigma$, è
\emph{poco probabile} che esse siano molto grandi. Si tratta di una
affermazione piuttosto generica che si può precisare in un certo numero
di modi alternativi, uno dei quali è il teorema di Chebyshev.

\begin{theorem}[di Chebyshev]
  Sia $x$ una variabile casuale tale che esistano finiti la media $\mu$
  e la varianza $\sigma^2$; preso un numero reale positivo $c > 0$ si ha:
  \begin{align}\label{eq:teorema_chebyshev}
    \prob{\abs{x - \mu} \ge c\sigma} \le \frac{1}{c^2}
  \end{align}
\end{theorem}
\begin{proof}
  Per comodità illustriamo la dimostrazione nel caso di una variabile
  continua (il caso discreto è del tutto analogo):
  \begin{align*}
    \sigma^2 & = \intinf(x - \mu)^2p(x)\diff \ge
    \int_{\abs{x - \mu} \ge c\sigma} \!\!\!\!\!\!\!\!\!\!\!\!\!\!\!\!\!\!\!\!\!
    (x - \mu)^2p(x) \diff \ge
    \int_{\abs{x - \mu} \ge c\sigma} \!\!\!\!\!\!\!\!\!\!\!\!\!\!\!\!\!\!\!\!\!
    c^2\sigma^2 p(x) \diff =
    c^2\sigma^2 \int_{\abs{x - \mu} \ge c\sigma}
    \!\!\!\!\!\!\!\!\!\!\!\!\!\!\!\!\!\!\!\!\! p(x) \diff =
    c^2\sigma^2 \prob{\abs{x - \mu} \ge c\sigma},
  \end{align*}
  da cui segue banalmente la tesi.
\end{proof}

Il teorema di Chebyshev vale sotto ipotesi molto generali (il che è la
ragione principale della sua importanza), per cui il limite
superiore fornito è tipicamente piuttosto blando. In tutti i casi di
interesse pratico, se conosciamo la forma della distribuzione, o anche solo
alcune delle sue proprietà, la probabilità in~\eqref{eq:teorema_chebyshev}
si può calcolare esplicitamente oppure limitare in modo significativamente
più stringente.

\begin{examplebox}
  \begin{example}
    Nel caso di una variabile casuale $x$ distribuita uniformemente tra $0$ ed
    $1$ i punti a $\mu - 2\sigma$ e $\mu + 2\sigma$ cadono al di fuori
    dell'intervallo di variabilità di $x$ per cui
    $\prob{\abs{x-\mu} \ge 2\sigma} = 0$. Il teorema di Chebyshev fornisce il
    limite $\prob{\abs{x-\mu} \ge 2\sigma} \le \nicefrac{1}{4}$.
  \end{example}
\end{examplebox}

Se poniamo $c = 1$ nella~\eqref{eq:teorema_chebyshev}, il teorema di Chebyshev
che abbiamo appena dimostrato si traduce nella condizione banale
\begin{align*}
  \prob{\abs{x - \mu} \ge \sigma} \le 1 \quad \text{ovvero} \quad
  \prob{\abs{x - \mu} \le \sigma} \ge 0.
\end{align*}
Cioè la probabilità che la variabile casuale assuma un valore
nell'intervallo $[\mu-\sigma,~\mu+\sigma]$ è maggiore di zero. Il che è
come non dire niente. Se però ci fermiamo un attimo a guardare i pochi
esempi di distribuzione che abbiamo incontrato sino a questo momento ci
rendiamo conto che in realtà la probabilità
$\prob{\abs{x - \mu} \le \sigma}$ è sostanziale---tipicamente più grande
del $60\%$, come mostrato in tabella~\ref{tab:prob_media_stdev} e come
vedremo più in dettaglio nel seguito.

\begin{table}[htbp]
  \tablehstack{
    \centering\begin{tabular}{lll}
      \hline
      Distribuzione & Esempi & $\prob{\abs{x - \mu} \le \sigma}$\\
      \hline
      \hline
      Un dado & \ref{exp:varianza_dado} &
      $\nicefrac{4}{6} \approx 67\%$\\
      Somma di due dadi & \ref{exp:varianza_due_dadi} &
      $\nicefrac{24}{36} \approx 67\%$ \\
      Distribuzione uniforme & \ref{exp:varianza_dist_uniforme} &
      $\nicefrac{2}{\sqrt{12}} \approx 58\%$\\
      \hline
    \end{tabular}
  }{
    \caption{Valore della probabilità $\prob{\abs{x - \mu} \le \sigma}$
      che una variabile casuale disti di meno di una deviazione standard
      dalla media per tre distribuzioni particolari.}
    \label{tab:prob_media_stdev}
  }
\end{table}

In un certo senso potremmo dire che, nella maggior parte dei casi di interesse
pratico, e a meno di non scegliere distribuzioni multi-modali a scopo di
controesempio, l'intervallo $[\mu-\sigma,~\mu+\sigma]$ \emph{racchiude} la
maggior parte della distribuzione, nel senso che
$\prob{\abs{x - \mu} \le \sigma}$ è dell'ordine del $60$--$70\%$.


\section{Momenti di una distribuzione}

Definiamo il \emph{momento di ordine $n$} di una variabile casuale $x$ attorno
ad un punto $x_0$ come:
\begin{align}
  \dccases[.]{\mom[x_0]{n} = \expect{\power[n]{x - x_0}}}%
          {\sum_k \power[n]{x_k - x_0}\prob{x_k}}%
          {\intinf \power[n]{x - x_0}p(x)\diff}
\end{align}
Hanno particolare rilevanza i \emph{momenti algebrici} $\momalg{n}$, ossia i
momenti di ordine generico attorno al punto $x_0 = 0$
\begin{align}
  \momalg{n} = \mom[0]{n}
\end{align}
ed i \emph{momenti centrali} $\momcen{n}$, ossia i momenti di ordine generico
attorno al valor medio $\mu$ di $x$:
\begin{align}
  \momcen{n} = \mom[\mu]{n}
\end{align}
In particolare la media è il momento algebrico di ordine $1$ e la varianza
è il momento centrale di ordine $2$ (e, cosa meno interessante, tutti i
momenti di ordine $0$ valgono $1$ per la condizione di normalizzazione).

Quello di ordine tre è il primo momento centrale di ordine dispari che
non si annulla ed è importante perché, pesando con il segno le code a destra
e sinistra della media, misura l'eventuale asimmetria della funzione di
distribuzione. Sfruttando la linearità del valore di aspettazione, il
momento centrale di ordine tre si può scrivere come
\begin{align*}
  \momcen{3} & = \expect{(x - \mu)^3} =
  \expect{x^3 - 3\mu x^2 + 3\mu^2 x - \mu^3} =
  \expect{x^3} - 3\mu\expect{x^2} + 3\mu^2\expect{x} -
  \expect{\mu^3} =\\\nonumber
  & = \expect{x^3} - 3\mu(\sigma^2 + \mu^2) + 3\mu^3 - \mu^3.
\end{align*}
(Nell'ultimo passaggio abbiamo sfruttato il fatto che
$\sigma^2 = \expect{x^2} - \mu^2$.) Si ha dunque la relazione notevole
\begin{align}\label{eq:momcen3}
  \momcen{3} = \expect{x^3} - 3\mu\sigma^2 - \mu^3,
\end{align}
che è in un qualche senso l'equivalente della~\eqref{eq:variance_alt}.
A partire dal momento centrale di ordine tre si può definire il
\emph{coefficiente di asimmetria} o \foreign{skewness} $\skewness$
\begin{align}
  \skewness = \frac{\momcen{3}}{\sigma^3}.
\end{align}
La \foreign{skewness} è una quantità adimensionale che vale zero per le
distribuzioni simmetriche rispetto al valor medio (non è in generale vero
il viceversa) e che è diversa da zero se la funzione di distribuzione
presenta una coda più \emph{lunga} dell'altra---nel qual caso una
\foreign{skewness} positiva (negativa) indica che la coda a destra (sinistra)
è più lunga.


\section{La funzione cumulativa}
\label{sec:funzione_cumulativa}

Data una variabile casuale $x$ la \emph{funzione cumulativa} o
\foreign{cumulative density function} (cdf) è definita come
\begin{align}
  F(x') = \prob{x \leq x'}.
\end{align}
Dato che la funzione cumulativa ha lo stesso dominio della funzione di
distribuzione, il suo argomento si indica generalmente con lo stesso nome
della variabile casuale. Operativamente:
\begin{align}
  \dccases{F(x)}{\sum_{x_k \leq x}\prob{x_k}}{\int_{-\infty}^x p(t) dt}
\end{align}

Poiché la probabilità (o la densità di probabilità) è non negativa,
la funzione cumulativa è monotona crescente, e, per la condizione di
normalizzazione, assume valori nell'intervallo $[0, 1]$:
\begin{align}
  \lim_{x \rightarrow -\infty}F(x) = 0 \quad \text{e} \quad
  \lim_{x \rightarrow +\infty}F(x) = 1
\end{align}
(nel caso la densità di probabilità abbia supporto finito, la condizione
implica che la funzione cumulativa assume i valori $0$ e $1$ negli estremi del
supporto stesso). Si ha inoltre l'uguaglianza ovvia
\begin{align}
  \prob{x_1 < x \leq x_2} = F(x_2) - F(x_1).
\end{align}

Se la funzione cumulativa è strettamente crescente e continua, allora
per ogni numero reale $0 \leq q \leq 1$ esiste uno ed un solo valore di
$x$ per cui $F(x) = q$. L'inverso della funzione cumulativa nel senso della
composizione delle funzioni
\begin{align}
  F^{-1}(q)
\end{align}
si dice \emph{funzione di distribuzione inversa} o
\foreign{percent point function}.

\begin{examplebox}
  \begin{example}
    Nel caso della variabile casuale $x$ distribuita uniformemente tra $0$ ed
    $1$ introdotta nell'esempio~\ref{exp:media_dist_uniforme} la distribuzione
    cumulativa è semplicemente
    \begin{align*}
      F(x) = \int_0^x p(t) dt = \eval{t\rule{0pt}{10pt}}{0}{x} = (x - 0) = x.
    \end{align*}
    Notiamo che $F(0) = 0$ e $F(1) = 1$, come deve essere. La funzione di
    distribuzione inversa e` invece
    \begin{align*}
      F^{-1}(q) = q.
    \end{align*}
  \end{example}
\end{examplebox}


\section{Breve riepilogo: un semplice problema di calcolo delle probabilità}

Fermiamoci per un attimo e passiamo in rassegna alcune delle cose che abbiamo
visto in questa sezione nel contesto di un (semplice) problema concreto.
Supponiamo di avere due variabili casuali continue indipendenti $x_1$ ed $x_2$,
entrambe distribuite uniformemente tra $0$ ed $1$. Vogliamo calcolare la
funzione di distribuzione della variabile
\begin{align*}
  x = \max(x_1, x_2)
\end{align*}
e stimare i parametri fondamentali (e.g., media e deviazione standard)
caratteristici della distribuzione stessa.
(Prima di proseguire intendiamoci sulle regole del gioco: la realizzazione
elementare del nostro ipotetico esperimento consiste nel campionare due volte
una distribuzione uniforme e prendere il massimo tra i due campionamenti.
Un possibile equivalente discreto sarebbe lanciare due dadi e considerare
l'uscita più alta come variabile causale.)


\subsection{Calcolo della funzione di distribuzione}

Cominciamo dal calcolo della funzione di distribuzione di $x$. La prima cosa che
possiamo dire senza alcun dubbio è che, se $x_1$ ed $x_2$ sono comprese tra
$0$ e $1$, allora anche $x = \max(x_1, x_2)$ sarà compresa tra $0$ ed $1$.
Intuitivamente ci aspettiamo anche che, se prendiamo il massimo tra due
campionamenti di una variabile distribuita uniformemente tra $0$ ed $1$, valori
relativamente più grandi (cioè vicino ad $1$) saranno più probabili
di valori relativamente più piccoli (cioè vicino a $0$).

La figura~\ref{fig:max_pdf_uniforme_pdf_max_pdf_uniforme} mostra una possibile
costruzione geometrica utile per il calcolo della funzione di distribuzione che
stiamo cercando. Se $x_1$ ed $x_2$ sono indipendenti, allora la realizzazione
elementare del nostro esperimento consiste essenzialmente nello scegliere un
punto distribuito uniformemente nel piano $x_1$-$x_2$. Prendere il massimo tra
$x_1$ ed $x_2$ è come dire che scegliamo $x_1$ se il punto scelto è sotto la
diagonale, mentre scegliamo $x_2$ se il punto è sopra la diagonale.
\`E chiaro allora che il luogo geometrico dei punti, nel piano $x_1$-$x_2$,
che corrisponde ad un valore fissato di $x = \max(x_1, x_2)$ è una linea
costituita da due segmenti di uguale lunghezza, paralleli agli assi cartesiani,
che si incontrano sulla diagonale, come mostrato in
figura~\ref{fig:max_pdf_uniforme_pdf_max_pdf_uniforme}.

\pgffigtwo{max_pdf_uniforme}{pdf_max_pdf_uniforme}{
  Costruzione geometrica per il calcolo della funzione di distribuzione del
  massimo $x = \max(x_1, x_2)$ di due variabili casuali continue distribuite
  uniformemente tra $0$ e $1$ (a sinistra). Il luogo geometrico dei punti che
  corrisponde ad un valore fissato di $x = \max(x_1, x_2)$ è dato da due
  segmenti di uguale lunghezza, paralleli agli assi cartesiani, che si
  incontrano sulla diagonale. Corrispondentemente, il valore della densità di
  probabilità $p(x)$ è proporzionale alla lunghezza di tali segmenti, e
  cresce linearmente da $0$ ad $1$, come mostrato a destra.
}

Tornando alla funzione di distribuzione, possiamo dire che la densità di
probabilità $p(x)$ in un generico punto $0 \leq x \leq 1$ sarà proporzionale
alla lunghezza del segmento corrispondente nella nostra costruzione
geometrica. Essa sarà nulla in $x = 0$ (dove i due segmenti degenerano in un
punto) e aumenterà linearmente con $x$ fino a raggiungere il massimo in
$x = 1$. La forma analitica sarà cioè $p(x) = cx$---con la
costante moltiplicativa $c$ fissata a $2$ dalla condizione di normalizzazione
\begin{align}\label{eq:pdf_triangolare_max}
  p(x) = 2x,
\end{align}
come si può verificare facilmente per calcolo diretto. Adesso che abbiamo
la forma analitica delle funzione di distribuzione possiamo calcolare
facilmente tutte le proprietà derivate.


\subsection{Media, mediana e moda}

Cominciamo dunque dalle stime di tendenza centrale. La media si calcola al
solito secondo la definizione
\begin{align*}
  \mu = \expect{x} = \int_{-\infty}^{\infty} xp(x)dx = \int_{0}^{1}2x^2dx =
  \eval{\frac{2x^3}{3}}{0}{1} = \frac{2}{3}.
\end{align*}
Non dovrebbe sorprendere che $\mu > \nicefrac{1}{2}$, poiché la densità di
probabilità è più alta nella parte destra dell'intervallo di definizione
della variabile rispetto alla parte sinistra. La mediana è definita
dall'uguaglianza
\begin{align*}
  \frac{1}{2} = \int_{0}^{\median}p(x)dx = \int_{0}^{\median}2xdx =
  \eval{x^2}{0}{\median} \;\;\;\; = \median^2,
\end{align*}
da cui $\median = \nicefrac{1}{\sqrt{2}} \approx 0.707$. La moda, infine,
coincide con il massimo della funzione di distribuzione $x = 1$. Il fatto che
media, mediana e moda non coincidano, come illustrato in
figura~\ref{fig:pdf_max_pdf_uniforme_media}, non dovrebbe sorprendere poiché
la funzione di distribuzione non è simmetrica.

\pgffigone{pdf_max_pdf_uniforme_media}{
  Illustrazione delle posizioni della media ($\mu = \nicefrac{2}{3}$),
  della mediana ($\median = \nicefrac{1}{\sqrt{2}}$) e della moda ($1$) per la
  funzione di distribuzione triangolare~\eqref{eq:pdf_triangolare_max}.
  Non essendo la distribuzione simmetrica, non sorprende il fatto che
  media, mediana e moda non coincidano. (Incidentalmente, questo è anche
  un esempio di distribuzione per la quale, a causa della peculiarità della
  forma, il valore più probabile non è molto interessante come stima di
  tendenza centrale.)
}


\subsection{Varianza, deviazione standard e larghezza a metà altezza}

Procediamo con il calcolo delle metriche di dispersione attorno alla media per
la nostra distribuzione. La varianza vale
\begin{align*}
  \sigma^2 = \expect{(x - \mu)^2} = \expect{x^2} - \mu^2 =
  \int_{-\infty}^{\infty}x^2p(x)dx - \mu^2 =
  \int_{0}^{1} 2x^3dx - \frac{4}{9} = \eval{\frac{x^4}{2}}{0}{1} - \frac{4}{9} =
  \frac{1}{2} - \frac{4}{9} = \frac{1}{18},
\end{align*}
da cui $\sigma = \nicefrac{1}{\sqrt{18}} \approx 0.236$. La retta $p(x) = 1$
interseca il grafico della funzione di distribuzione nei punti
$x = \nicefrac{1}{2}$ e $x = 1$, come mostrato in
figura~\ref{fig:pdf_max_pdf_uniforme_stdev}. La larghezza intera a metà
altezza è dunque $\fwhm = \nicefrac{1}{2}$ e la semilarghezza a metà
altezza $\hwhm = \nicefrac{1}{4}$. Il rapporto tra semilarghezza a metà
altezza e deviazione standard vale dunque
\begin{align*}
  \frac{\hwhm}{\sigma} = \frac{\sqrt{18}}{4} \approx 1.061
\end{align*}
che, come ci aspettiamo, è dell'ordine dell'unità. La frazione della
funzione di distribuzione contenuta entro una deviazione standard della media
è infine
\begin{align*}
  \prob{\abs{x - \mu} \leq \sigma} = \int_{\mu - \sigma}^{\mu + \sigma} p(x)dx =
  \int_{\mu - \sigma}^{\mu + \sigma} 2x dx = \eval{x^2}{\mu - \sigma}{\mu + \sigma}
  \;\;\;\; = (\mu + \sigma)^2 - (\mu - \sigma)^2 = 4\mu\sigma =
  \frac{8}{9\sqrt{2}} \approx 0.629,
\end{align*}
in linea con i valori riportati nella tabella~\ref{tab:prob_media_stdev}.

\pgffigone{pdf_max_pdf_uniforme_stdev}{
  Illustrazione della deviazione standard $\sigma = \nicefrac{1}{\sqrt{18}}$ e
  della larghezza a metà altezza $\fwhm = \nicefrac{1}{2}$ per la funzione di
  distribuzione triangolare~\eqref{eq:pdf_triangolare_max}. Come abbiamo avuto
  occasione di dire la deviazione standard e la semilarghezza a metà altezza
  sono vicine tra loro (nel senso che il loro rapporto è dell'ordine di $1$.
  Inoltre l'intervallo $[\mu - \sigma,~\mu + \sigma]$ racchiude una frazione
  sostanziale (circa il $63\%$) della funzione di distribuzione.)
}


\subsection{Coefficiente di asimmetria}

La funzione di distribuzione~\eqref{eq:pdf_triangolare_max} non è simmetrica
rispetto al valor medio, per cui ci aspettiamo che il momento centrale di
ordine tre $\momcen{3}$ e, di conseguenza, il coefficiente di asimmetria
$\skewness$ siano non nulli. Per la precisione, dato che la coda più
pronunciata è quella a sinistra della media, ci aspettiamo che la
\foreign{skewness} sia negativa. Procediamo allora per gradi e calcoliamo il
momento terzo utilizzando la~\eqref{eq:momcen3}
\begin{align*}
  \momcen{3} = \expect{x^3} - 3\mu\sigma^2 - \mu^3 =
  \int_{-\infty}^{\infty}x^3 p(x) dx - \frac{1}{9} - \frac{8}{27} =
  \int_{0}^{1}2x^4dx - \frac{11}{27} =
  \eval{\frac{2x^5}{5}}{0}{1} - \frac{11}{27} = \frac{2}{5} - \frac{11}{27} =
  -\frac{1}{135},
\end{align*}
da cui, come ci aspettavamo qualitativamente
\begin{align*}
  \skewness = -\frac{\sqrt{18^3}}{135} = -\frac{2\sqrt{2}}{5} \approx -0.566.
\end{align*}


\subsection{Distribuzione cumulativa e distribuzione inversa}

Completiamo la nostra panoramica con il calcolo esplicito della distribuzione
cumulativa
\begin{align*}
  F(x) = \int_{-\infty}^x p(t)dt = \int_0^x 2t dt = \eval{t^2}{0}{x} \;\; = x^2,
\end{align*}
da cui, tra le altre cose, possiamo derivare in modo leggermente più semplice
il risultato trovato prima
\begin{align*}
  \prob{\abs{x - \mu} \leq \sigma} = F(\mu + \sigma) - F(\mu - \sigma) =
  (\mu + \sigma)^2 - (\mu - \sigma)^2.
\end{align*}
La funzione di distribuzione inversa si scrive infine come
\begin{align*}
  F^{-1}(q) = \sqrt{q}.
\end{align*}


\section{Digressione: alcuni paradossi probabilistici}

Per \emph{paradosso} si intende letteralmente una proposizione logicamente
corretta ma in apparente contraddizione con l'esperienza comune. Se inserite
in un motore di ricerca le due parole chiave "paradosso" e "probabilità"
(in qualsiasi lingua) non rimarrete delusi dalla quantità di informazione
sull'argomento disponibile su web. Non c'è che l'imbarazzo della scelta.

I paradossi probabilistici si possono rozzamente suddividere in due
macro-categorie: (i) conseguenze dirette (e logicamente corrette) delle leggi
della probabilità che, applicate ad una specifica situazione, sono
sorprendenti perché anti-intuitive e (ii) problemi in cui il paradosso in
effetti ha origine da una formulazione incompleta o volutamente ambigua del
problema. Questi ultimi hanno spesso più a che vedere con la semantica che
con la probabilità, per cui non sono molto interessanti in questo contesto
(ma se siete interessati potete dare un'occhiata, tra le altre cose,
a~\cite{dagostini_bertrand} e~\cite{dagostini_boy_girl}). I veri
paradossi, invece, possono essere utili ad illustrare i fondamenti del
ragionamento probabilistico, per cui ne esamineremo alcuni tra i più famosi.


\subsection{Quanti compleanni lo stesso giorno?}
\label{sec:paradosso_compleanni}

Dato un insieme di $n$ persone scelte casualmente, qual è la probabilità che
almeno due di esse abbiano lo stesso compleanno? (Per semplicità trascureremo
gli anni bisestili ed assumeremo che le nascite siano uniformemente
distribuite sui 365 giorni dell'anno---escluso il 29~febbraio. Assumeremo anche
che nel gruppo di persone non vi siano gemelli.)

Una breve nota a margine: ogni volta che la parola \emph{almeno} compare in un
problema di probabilità dovete riflettere attentamente su come essa si traduca
effettivamente in termini quantitativi. In questo caso il nostro evento
si può realizzare con una singola coppia di persone con lo stesso compleanno,
oppure due coppie, oppure con 3 o 4 persone nate lo stesso giorno e così
via---addirittura con la (improbabile) situazione in cui tutte le $n$ persone
sono nate lo stesso giorno. Al crescere di $n$ il numero di possibilità
(che si dice anche genericamente il \emph{combinatoriale} del problema) cresce
rapidamente e spesso è più semplice calcolare la probabilità dell'evento
complementare all'evento di partenza ed usare il teorema della probabilità
complementare~\eqref{eq:probabilita_complementare}.

Attacchiamo allora il nostro paradosso chiedendoci quale sia la probabilità
che i compleanni delle $n$ persone che abbiamo scelto siano tutti diversi.
La prima persona avrà il compleanno in un determinato giorno dell'anno (tra i
$365$). Per la seconda persona rimangono $364/365$ disponibili diversi dal
compleanno della prima, per cui la probabilità che il compleanno non cada lo
stesso giorno è proprio $364/365$. Per la terza persona la probabilità che
il compleanno cada in un giorno diverso dalle prime due è $363/365$ e così
via---fino all'ultima, per cui questa probabilità è $(365 - n + 1)/365$.
Visto che le date di nascita sono tutte indipendenti tra loro, la probabilità
che esse siano \emph{tutte} diverse sarà
\begin{align}
  \prob{\text{compleanni tutti diversi}} =
  1 \times \frac{364}{365} \times \frac{363}{365} \times \cdots \times
  \frac{365 - (n - 1)}{365} = \frac{365!}{365^n(365 - n)!}
\end{align}
e la probabilità che almeno due persone abbiano il compleanno lo stesso giorno
è, per il teorema~\eqref{eq:probabilita_complementare}
\begin{align}\label{eq:paradosso_compleanni}
  \prob{\text{almeno due compleanni nello stesso giorno}} =
  1 - \frac{365!}{365^n(365 - n)!}.
\end{align}

\pgffigone{paradosso_compleanni}{
  Andamento della~\eqref{eq:paradosso_compleanni} per $n \leq 100$.
  La probabilità che in un gruppo di $n$ persone almeno due festeggino il
  compleanno lo stesso giorno supera il $50\%$ per $n = 23$.
}

La~\eqref{eq:paradosso_compleanni} non è molto espressiva, ma la
figura~\ref{fig:paradosso_compleanni} illustra l'andamento
della~\eqref{eq:paradosso_compleanni} per $n \leq 100$. \`E interessante notare
come la funzione superi il $50\%$ per $n = 23$ ed il $99\%$ per $n = 60$.
In un gruppo di $23$ persone, dunque, la probabilità che ve ne siano
almeno due che festeggino il compleanno lo stesso giorno è maggiore del
$50\%$---che è proprio il contenuto del nostro paradosso.
(Cosa suggeriva il vostro senso comune prima di fare il conto?)


\subsection{Breve digressione: somma della serie geometrica}

Dato un generico numero reale $r$, con $\abs{r} < 1$ si dice serie geometrica
la somma (infinita)
\begin{align}\label{eq:serie_geometrica}
  \sum_{k = 0}^{\infty} r^k
\end{align}
(è ovvio che per $\abs{r} \geq 1$ i termini della serie crescono o al più
rimangono costanti al crescere di $k$ per cui la serie diverge---il caso
$r = -1$ è interessante, e di nuovo la~\eqref{eq:serie_geometrica} non ha
somma definita).

Si tratta di uno dei casi in cui la somma parziale dei primi $n$ termini
si può scrivere esplicitamente in forma chiusa
\begin{align*}
  & \sum_{k = 0}^n r^k = 1 + r + r^2 + \cdots + r^n =
  (1 + r + r^2 + \cdots + r^n) \times \frac{1 - r}{1 - r} =\\
  & = \frac{1 + r + r^2 + \cdots + r^n - r + -r^2 - r^3 - \cdots - r^{n+1}}{1 - r}
  = \frac{1 - r^{n+1}}{1-r}
\end{align*}
a partire dalla quale la somma della serie si calcola banalmente come
limite per $n\rightarrow\infty$
\begin{align}\label{eq:somma_serie_geometrica}
  \sum_{k = 0}^{\infty} r^k = \lim_{n\rightarrow\infty} \frac{1 - r^{n+1}}{1 - r} =
  \frac{1}{1 - r}.
\end{align}

Con lo stesso metodo si può calcolare anche la somma parziale
(che è diversa, ma correlata alla precedente)
\begin{align*}
  & \sum_{k = 0}^{n} kr^k = 0 + r + 2r^2 + \cdots + nr^n =
  (r + 2r^2 + \cdots + nr^n) \times \frac{1 - r}{1 - r} =\\
  & = \frac{r + 2r^2 + \cdots + nr^n - r^2 - 2r^3 - (n -1)r^n - nr^{n+1}}{1 - r} =
  \frac{r + r^2 + \cdots r^n - nr^{n+1}}{1 - r} =\\
  & = \frac{\sum_{k = 1}^n r^k - nr^{n+1}}{1 - r} =
  \frac{\sum_{k = 0}^n r^k - 1 - nr^{n+1}}{1 - r} =
  \frac{(1 - r^{n+1})/(1 - r) - 1 - nr^{n+1}}{1 - r} =\\
  & = \frac{1 - r^{n+1} - 1 - nr^{n+1} + r + nr^{n+2}}{(1 - r)^2} =
  \frac{r - (n + 1)r^{n+1} + nr^{n+2}}{(1 - r)^2},
\end{align*}
da cui, prendendo come prima il limite per $n\rightarrow\infty$ si ottiene
\begin{align}
  \sum_{k = 0}^{\infty} kr^k =
  \lim_{n\rightarrow\infty} \frac{r - (n + 1)r^{n+1} + nr^{n+2}}{(1 - r)^2} =
  \frac{r}{(1 - r)^2}.
\end{align}
Cogliamo l'occasione per notare che lo stesso risultato si poteva ottenere,
più semplicemente, derivando la~\eqref{eq:somma_serie_geometrica} rispetto ad
$r$
\begin{align*}
  \sum_{k = 0}^{\infty} kr^k = r\td{}{r}{} \left[ \sum_{k = 0}^{\infty} r^k \right] =
  \frac{r}{(1 - r)^2},
\end{align*}
che illustra come spesso, con un minimo di creatività, lo stesso calcolo può
diventare da laborioso a banale.


\subsection{Maschio o femmina?}

Riprendiamo il filo della nostra discussione sui paradossi della probabilità.
Una coppia di genitori decide di avere figli fino alla nascita della prima
femmina---avvenuta la quale semplicemente smetterà di procreare. Quanti
maschi (e quanti figli in totale) avrà \emph{in media} la coppia?

Assumeremo per semplicità che la probabilità che un figlio sia maschio
(o femmina) sia esattamente il $50\%$---il che è notoriamente non vero, ma
è di fatto irrilevante per la nostra discussione. Si tratta banalmente di un
processo binomiale in cui, detti $F$ ed $M$ la nascita di un maschio e di una
femmina, rispettivamente, siamo interessati alle sequenze $F$, $MF$, $MMF$,
$MMMF$ e così via. La probabilità di avere $k$ figli, cioè $n_M = (k - 1)$
maschi seguite da $n_F = 1$ femmina, è ovviamente $\nicefrac{1}{2^k}$, come
illustrato nella tabella~\ref{tab:paradosso_maschi_femmine}.

\begin{table}[htbp]
  \tablehstack{
    \centering\begin{tabular}{lllll}
      \hline
      Sequenza & $n$ & $n_F$ & $n_M$ & Probabilità\\
      \hline
      \hline
      $F$ & $1$ & $1$ & $0$ & $\nicefrac{1}{2}$\\
      $MF$ & $2$ & $1$ & $1$ & $\nicefrac{1}{4}$\\
      $MMF$ & $3$ & $1$ & $2$ & $\nicefrac{1}{8}$\\
      $MMMF$ & $4$ & $1$ & $3$ & $\nicefrac{1}{16}$\\
      \multicolumn{5}{c}{\ldots}\\
      $(k - 1)MF$ & $k$ & $1$ & $k - 1$ & $\nicefrac{1}{2^k}$\\
      $kMF$ & $k + 1$ & $1$ & $k$ & $\nicefrac{1}{2^{k+1}}$\\
      \multicolumn{5}{c}{\ldots}\\
      \hline
    \end{tabular}
  }{
    \caption{Illustrazioni delle sequenze possibili e delle probabilità
      corrispondenti per il paradosso delle nascite (maschio o femmina?).}
    \label{tab:paradosso_maschi_femmine}
  }
\end{table}

Nelle nostre ipotesi il numero di femmine è fisso a $1$---anche se, in
effetti, può essere considerato come una variabile casuale che può assumere
un solo valore ($1$, appunto) la cui probabilità di uscita è correttamente
normalizzata
\begin{align*}
  \prob{n_F = 1} = \sum_{k = 0}^{\infty} \frac{1}{2^{k+1}} =
  \frac{1}{2}\sum_{k = 0}^{\infty} \frac{1}{2^k} = \frac{1}{2} \times 2 = 1,
\end{align*}
ed il cui valore di aspettazione è, ovviamente, $1$. Il valore di
aspettazione per il numero di maschi $n_M$ si scrive invece come
\begin{align}
  \expect{n_M} = \sum_{k = 0}^{\infty} \frac{k}{2^{k+1}} =
  \frac{1}{2} \sum_{k = 0}^{\infty} \frac{k}{2^k} = \frac{1}{2} \times 2 = 1
\end{align}
ed è anch'esso pari ad $1$, nonostante $n_M$ possa essere arbitrariamente
grande. In altre parole il numero medio (totale) atteso di figli è
esattamente due---un maschio ed una femmina.

\`E interessante chiedersi come cambi il tutto se introduciamo un limite
$n_\text{max}$ al numero totale di figli---cioè se diciamo che, nell'ipotesi
che una femmina non sia nata nei primi $n_\text{max}$ tentativi, la coppia decida
di darsi pace e non riprovare.
In questo nuovo schema la tabella~\ref{tab:paradosso_maschi_femmine} cessa di
essere di lunghezza infinita e termina con la sequenza $n_\text{max}M$, per la
quale $n = n_\text{max}$, $n_F = 0$ e $n_M = n_\text{max}$, con probabilità
$1/2^{n_\text{max}}$. (Notate che le ultime due righe di questa tabella
\emph{finita} sono equiprobabili, perché corrispondono a due sequenze di
lunghezza identica in cui l'unica differenza è l'ultimo elemento---$M$ o $F$.)

\begin{table}[htbp]
  \tablehstack{
    \centering\begin{tabular}{lllll}
      \hline
      Sequenza & $n$ & $n_F$ & $n_M$ & Probabilità\\
      \hline
      \hline
      $F$ & $1$ & $1$ & $0$ & $\nicefrac{1}{2}$\\
      $MF$ & $2$ & $1$ & $1$ & $\nicefrac{1}{4}$\\
      $MMF$ & $3$ & $1$ & $2$ & $\nicefrac{1}{8}$\\
      $MMMF$ & $4$ & $1$ & $3$ & $\nicefrac{1}{16}$\\
      $MMMM$ & $4$ & $0$ & $4$ & $\nicefrac{1}{16}$\\
      \hline
    \end{tabular}
  }{
    \caption{Illustrazioni delle sequenze possibili e delle probabilità
      corrispondenti per il paradosso delle nascite modificato, nel caso
      particolare $n_\text{max} = 4$.}
    \label{tab:paradosso_maschi_femmine_finita}
  }
\end{table}

La prima conseguenza è che il numero di femmine $n_F$ è adesso a tutti gli
effetti una variabile casuale (che può assumere più di un valore), il cui
valore di aspettazione è
\begin{align*}
  \expect{n_F} = 1 \times \left(1 - \frac{1}{2^{n_\text{max}}} \right) +
  0 \times \frac{1}{2^{n_\text{max}}} = 1 - \frac{1}{2^{n_\text{max}}},
\end{align*}
mentre il valore di aspettazione del numero di maschi è questa volta
\begin{align}
  \expect{n_M} = \sum_{k = 0}^{n_\text{max} - 1} \frac{k}{2^{k+1}} +
  \frac{n_\text{max}}{2^{n_\text{max}}} =
  \frac{1}{2} \sum_{k = 0}^{n_\text{max} - 1} \frac{k}{2^k} +
  \frac{n_\text{max}}{2^{n_\text{max}}} =
  \frac{1}{2} \times \frac{\frac{1}{2} -
    \frac{n_\text{max}}{2^{n_\text{max}}} +
    \frac{(n_\text{max} - 1)}{2^{n_\text{max} + 1}}}{\frac{1}{4}} +
  \frac{n_\text{max}}{2^{n_\text{max}}} = 1 - \frac{1}{2^{n_\text{max}}}.
\end{align}
Quindi, indipendentemente dal numero massimo di figli $n_\text{max}$, i valori
di aspettazione per il numero di femmine e di maschi sono esattamente uguali
(e $\leq 1$). Per completezza, nel caso $n_\text{max} = 4$ illustrato in
tabella~\ref{tab:paradosso_maschi_femmine_finita} questi valori di aspettazione
sono $\nicefrac{15}{16}$.


\subsection{Una strategia di gioco interessante}

Un casinò offre un tavolo di gioco di testa ($T$) o croce ($C$) con una moneta
equa, in cui banalmente si riceve il doppio della puntata in caso di vittoria
(e, ovviamente, si perde la posta in caso contrario).
Un giocatore decide di adottare la seguente strategia di gioco. Inizialmente
scommette una certa somma $s$ su $T$; se vince si ritira con il corrispettivo
$2s$ (ed una vincita netta pari ad $s$), mentre se perde continua raddoppiando
la puntata ogni volta fino alla prima uscita di $T$---momento nel quale
interrompe il gioco.

La prima cosa che ci chiediamo è la somma netta vinta (o perduta) al termine
del gioco. Se $T$ esce per la prima volta al secondo lancio la somma totale
giocata è $\mathcal{S}_2 = s + 2s = 3s$ ed il corrispettivo pagato dal
casinò è $4s$, per una vincita netta di $s$. Se $T$ esce per la prima volta
al terzo lancio la somma giocata è $\mathcal{S}_3 = s + 2s + 4s = 7s$ ed il
corrispettivo è $8s$, per una vincita netta che è di nuovo $s$. \`E allora
facile convincersi che se $T$ esce per la prima volta dopo $n$ lanci la somma
totale giocata è
\begin{align*}
  \mathcal{S}_n = \sum_{k = 0}^{n - 1} s 2^k = s(2^n - 1).
\end{align*}
e, come detto, la vincita netta è sempre la puntata iniziale $s$---sembrerebbe
una strategia perfetta, in cui si vince sempre.

La seconda domanda interessante a cui rispondere è: quale somma di denaro
ci aspettiamo di dover giocare, in media, prima di poter abbandonare il tavolo
con la nostra vincita netta $s$? O, in altre parole, quanto è il valore di
aspettazione $\expect{\mathcal{S}}$? Per rispondere dobbiamo calcolare la
probabilità che $T$ esca per la prima volta dopo esattamente $n$ lanci,
il che è banale, perché si tratta di $n$ eventi indipendenti con
probabilità di successo (e fallimento) pari ad $\nicefrac{1}{2}$
\begin{align*}
  \prob{\overbrace{CC\ldots CC}^{n - 1~{\mathrm volte}} T} = \frac{1}{2^n}.
\end{align*}
A questo punto abbiamo tutti gli ingredienti per rispondere alla nostra domanda,
e banalmente
\begin{align}
  \expect{\mathcal{S}} = \sum_{n = 1}^\infty \frac{s(2^n - 1)}{2^n} =
  \sum_{n = 1}^\infty \left( s - \frac{s}{2^n} \right),
\end{align}
che chiaramente diverge perché il primo termine della serie è costante.
Fisicamente questo significa che, benché la strategia, in astratto,
garantisca una vincita netta, la somma media necessaria per lasciare il tavolo
ed incassare la vincita diverge---la strategia funziona solo se si ha a
disposizione una quantità infinita di denaro.

Il punto della questione sta ovviamente nel fatto che la nostra strategia di
incremento della puntata è esponenziale---e se non siete convinti ripensate a
cosa abbiamo detto a proposito nella sezione~\ref{sec:esponenziali_logaritmi}.
Per fissare le idee, la probabilità che occorrano $30$ lanci per vincere
è estremamente piccola: $1/2^{30} \approx 9.3 \times 10^{-10}$---se giocassimo
senza interruzione al ritmo di un lancio al secondo vedremmo in media
una occorrenza di questo evento ogni $34$ anni circa. Ma nel momento in cui
ciò accadesse, assumendo una puntata iniziale di $s = 1$~euro, dovremmo avere
$2^{31} - 1 \approx 2.15$~miliardi (!) di euro per poter proseguire al lancio
successivo.


\section{Cenni alle variabili multi-variate: indipendenza, covarianza e correlazione}
\label{sec:covarianza}

La nostra discussione, fino a questo momento, è stata largamente incentrata
su variabili casuali \emph{singole}. Abbiamo definito l'indipendenza tra eventi,
ma non abbiamo dedicato molta attenzione alle relazioni tra variabili casuali
diverse, ed in particolare al concetto di indipendenza tra variabili casuali,
per cui completiamo il capitolo colmando questa lacuna.

Quando si ha a che vedere con un insieme di variabili casuali $x_1 \ldots x_n$,
il problema è completamente specificato dalla loro funzione di distribuzione
congiunta, vale a dire dalla probabilità (o densità di probabilità) che
ciascuna delle variabili assuma un valore specifico
\begin{align*}
  \dccases[.]{}{\prob{x_1, \ldots, x_n}}{p(x_1,\ldots, x_n)}
\end{align*}


\subsection{Densità di probabilità bi-variate: concetti di base}

Consideriamo per semplicità il caso di due variabili casuali continue
$x_1$ ed $x_2$, descritte dalla densità di probabilità congiunta $p(x_1, x_2)$.
Analogamente a quanto abbiamo fatto nel caso di una variabile, richiederemo che
la densità di probabilità sia positiva e propriamente
\begin{align*}
  p(x_1, x_2) \geq 0 \quad \text{e} \quad
  \int_{-\infty}^{\infty}\int_{-\infty}^{\infty}  p(x_1, x_2) \diff[x_1]\diff[x_2] = 1.
\end{align*}
Possiamo calcolare la probabilità che la coppia ordinata $(x_1, x_2)$
sia contenuta in una generica regione $A$ del piano come
\begin{align*}
  \prob{(x_1, x_2) \in A} = \int \int_A  p(x_1, x_2) \diff[x_1]\diff[x_2],
\end{align*}
e definire il valore di aspettazione di una generica funzione $f(x_1, x_2)$
come
\begin{align*}
  \expect{f(x_1, x_2)} = \int_{-\infty}^{\infty}\int_{-\infty}^{\infty}
  f(x_1, x_2) p(x_1, x_2) \diff[x_1]\diff[x_2].
\end{align*}


\begin{examplebox}
  \begin{example}\label{exp:variabili_indipendenti_pdf}
    La funzione $p(x_1, x_2) = 4x_1x_2$, nel quadrato unitario
    $0 \leq x_1 \leq 1$ e $0 \leq x_2 \leq 1$, è una funzione di distribuzione,
    in quanto è semi-definita positiva ed è correttamente normalizzata
    \begin{align*}
      \int_0^1 \int_0^1 4x_1x_2 \diff[x_1]\diff[x_2] =
      4 \int_0^1 x_1\diff[x_1] \int_0^1 x_2\diff[x_2] =
      4 \eval{\frac{x_1^2}{2}}{0}{1} \times
      \eval{\frac{x_2^2}{2}}{0}{1} = 1.
    \end{align*}
  \end{example}

  \begin{example}\label{exp:variabili_dipendenti_pdf}
      La funzione $p(x_1, x_2) = \frac{3}{2}(x_1^2 + x_2^2)$, nel quadrato
      unitario $0 \leq x_1 \leq 1$ e $0 \leq x_2 \leq 1$, è pure una funzione
      di distribuzione"
      \begin{align*}
        \int_0^1 \int_0^1 \frac{3}{2}(x_1^2 + x_2^2) \diff[x_1]\diff[x_2] =
        \frac{3}{2} \int_0^1 \diff[x_1] \int_0^1 (x_1^2 + x_2^2) \diff[x_2] =
        \frac{3}{2} \int_0^1 \left( \frac{1}{3} + x_1^2 \right)\diff[x_1] =
        \frac{3}{2} \left( \frac{1}{3} + \frac{1}{3} \right)= 1.
      \end{align*}
  \end{example}
\end{examplebox}

\pgffigtwo{pdf_2d_indipendenti}{pdf_2d_dipendenti}{
  Rappresentazione grafica delle funzioni di distribuzione di due variabili
  degli esempi~\ref{exp:variabili_indipendenti_pdf} e~\ref{exp:variabili_dipendenti_pdf}:
  $p(x_1, x_2) = 4x_1x_2$ (sinistra) e $p(x_1, x_2) = \frac{3}{2}(x_1^2 + x_2^2)$
  (destra). Entrambe rappresentano densità di probabilità correttamente
  normalizzate nel quadrato unitario $0 \leq x_1 \leq 1$ e $0 \leq x_2 \leq 1$
  (i.e., il volume sotto la superficie rappresentata dalla griglia nera è pari
  a $1$).
}


\subsection{Probabilità condizionata ed indipendenza statistica}

Con due o più variabili in gioco la situazione diventa immediatamente
interessante---e questo ci permette di rivisitare in una chiave più concreta
alcune delle cose che abbiamo visto nella parte iniziale del capitolo.
La cosa più ovvia che possiamo chiederci è: come faccio a capire se $x_1$
ed $x_2$ sono \emph{indipendenti}? La risposta si deve poter desumere dalla
densità di probabilità congiunta, ma come?

Sappiamo che $x_1$ ed $x_2$ sono indipendenti se la probabilità che $x_1$
assuma un valore compreso in un intervallo fissato non dipende dal valore assunto
da $x_2$ (e viceversa). Ovverosia, vogliamo che la densità di probabilità di
$x_1$, condizionata al fatto che $x_2$ assuma un valore specifico, non dipenda
da $x_2$ (e viceversa):
\begin{align*}
  p(x_1 \cond x_2~\text{fissato}) = p_1(x_1) \quad \text{e} \quad
  p(x_2 \cond x_1~\text{fissato}) = p_2(x_2).
\end{align*}
Il punto sottile della questione è capire come scriviamo questa probabilità
condizionata.

Ad un primo sguardo potremmo essere tentati di dire che la densità di probabilità
di $x_1$ condizionata ad $x_2$ si ottenga semplicemente fissando il valore
di $x_2$ nella densità di probabilità congiunta
\begin{align*}
  p(x_1 \cond x_2) = p(x_1, x_2).
\end{align*}
(\`E più difficile a dirsi che a farsi, ma questo è essenzialmente come dire
che, se $p(x_1, x_2) = 4x_1x_2$, la densità di probabilità di $x_1$ quando
$x_2 = 1$ è semplicemente $p_1(x_1) = 4x_1$.) Non serve molto, però, per capire
che questa prescrizione, in generale, non garantisce che la densità di
probabilità unidimensionale così ottenuta sia propriamente normalizzata.
Ovverosia, come illustrato in figura~\ref{fig:pdf_2d_indipendenti_1d_pdf_2d_dipendenti_1d},
\emph{la densità di probabilità di $x_1$ condizionata ad $x_2$ non
è semplicemente la fetta della densità di probabilità congiunta
calcolata al valore appropriato di $x_2$.}

\pgffigtwo{pdf_2d_indipendenti_1d}{pdf_2d_dipendenti_1d}{
  Proiezioni delle densità di probabilità di due variabili
  degli esempi~\ref{exp:variabili_indipendenti_pdf} e~\ref{exp:variabili_dipendenti_pdf}
  lungo alcune rette orizzontali di esempio, nel piano $x_1$--$x_2$, a
  $x_2$ fissato. Come è ovvio, queste funzioni non sono in generale
  propriamente normalizzate nell'intervallo $[0, 1]$, per cui non possono
  rappresentare densità di probabilità.
}

Siamo però sulla strada giusta---la nostra prima risposta era sostanzialmente
corretta a meno di una costante di normalizzazione:
\begin{align}\label{eq:prob_cond_variabili}
  p(x_1 \cond x_2) =
  \frac{p(x_1, x_2)}{\int_{-\infty}^{\infty} p(x_1, x_2) \diff[x_1]}.
\end{align}
(Non dovrebbe essere difficile dimostrare che, questa volta, la risposta
è correttamente normalizzata, ma fermatevi un attimo per assicurarvi di aver
capito.) Il denominatore, dopo l'integrazione su $x_1$, è solamente
funzione di $x_2$ (lo stesso avviene a variabili scambiate), e prende
generalmente il nome di densità di probabilità marginale
\begin{align}
  p_1(x_1) = \int_{-\infty}^{\infty} p(x_1, x_2) \diff[x_2] \quad \text{e} \quad
  p_2(x_2) = \int_{-\infty}^{\infty} p(x_1, x_2) \diff[x_1].
\end{align}
Siamo pronti per chiudere il cerchio: due variabili casuali si dicono indipendenti se
la densità di probabilità congiunta può essere fattorizzata nel prodotto
delle due densità di probabilità marginali, ovverosia
\begin{align}\label{eq:indipendenza_variabili}
  p(x_1, x_2) = p_1(x_1) p_2(x_2).
\end{align}
La definizione non è sorprendente, poiché in questo caso la probabilità
condizionata da cui siamo partiti diviene
\begin{align*}
  p(x_1 \cond x_2) = \frac{p_1(x_1) p_2(x_2)}{p_2(x_2)} = p_1(x_1)
  \quad \text{e} \quad
  p(x_2 \cond x_1) = \frac{p_1(x_1) p_2(x_2)}{p_1(x_1)} = p_2(x_2),
\end{align*}
cioè la funzione di distribuzione unidimensionale di ciascuna delle due
variabili non dipende dal particolare valore assunto dall'altra---esattamente
ciò che intendiamo per indipendenza. Le due relazioni appena viste sono
la riscrittura in questo contesto delle~\eqref{eq:indipendenza_eventi}.
Geometricamente questo equivale a dire che due variabili $x_1$ ed $x_2$ sono
indipendenti se le proiezioni della densità di probabilità congiunta
$p(x_1, x_2)$ ad $x_2$ fissato sono uguali tra loro, a meno di una costante
moltiplicativa di normalizzazione, per tutti i valori di $x_2$ (e viceversa).

\begin{examplebox}
  \begin{example}\label{exp:variabili_indipendenti_marg}
    Consideriamo di nuovo la funzione di distribuzione congiunta
    dell'esempio~\ref{exp:variabili_indipendenti_pdf}. Le due densità di
    probabilità marginali sono
    \begin{align*}
      p_1(x_1) = \int_0^1 p(x_1, x_2)\diff[x_2] =
      \int_0^1 4x_1x_2\diff[x_2] = 2 x_1 \quad \text{e analogamente} \quad
      p_2(x_2) = 2 x_2.
    \end{align*}
    Le due variabili sono dunque indipendenti in quanto vale
    la~\eqref{eq:indipendenza_variabili}. Corrispondentemente, le proiezioni di
    $p(x_1, x_2)$ ad $x_2$ fissato sono tutte uguali tra loro a meno di una
    costante moltiplicativa di normalizzazione, come illustrato nel pannello a
    sinistra della figura~\ref{fig:pdf_2d_indipendenti_1d_pdf_2d_dipendenti_1d}.
  \end{example}

  \begin{example}
    Nel caso dell'esempio~\ref{exp:variabili_dipendenti_pdf} la cosa è più
    diversa---la densità di probabilità congiunta non si può fattorizzare
    come prima nel prodotto delle due densità di probabilità marginali
    \begin{align*}
      p_1(x_1) = \int_0^1 p(x_1, x_2)\diff[x_2] =
      \frac{3}{2}\int_0^1 (x_1^2 + x_2^2)\diff[x_2] =
      \frac{3}{2}\left( x_1^2 + \frac{1}{3} \right) =
      \frac{1}{2} + \frac{3}{2}x_1^2
      \quad \text{e} \quad
      p_2(x_2) = \frac{1}{2} + \frac{3}{2}x_2^2,
    \end{align*}
    per cui le variabili non sono indipendenti. Corrispondentemente, le proiezioni di
    $p(x_1, x_2)$ ad $x_2$ fissato nel pannello a destra della
    figura~\ref{fig:pdf_2d_indipendenti_1d_pdf_2d_dipendenti_1d} hanno forme
    diverse per valori diversi di $x_2$.
  \end{example}
\end{examplebox}

Se $x_1$ ed $x_2$ sono indipendenti, allora si dimostra banalmente che il valore
di aspettazione del loro prodotto è uguale al prodotto dei valori di
aspettazione
\begin{align}\label{eq:aspettazione_indipendenza}
  \expect{x_1x_2} =
  \int_{-\infty}^{\infty}\int_{-\infty}^{\infty} x_1x_2p(x_1, x_2)\diff[x_1]\diff[x_2] =
  \int_{-\infty}^{\infty} x_1 p_1(x_1)\diff[x_1]
  \int_{-\infty}^{\infty} x_2 p_2(x_2)\diff[x_2] =
  \expect{x_1}\expect{x_2}.
\end{align}
Notiamo, per completezza, che tutto ciò che abbiamo detto in questa
sezione vale anche, modulo la sostituzione degli integrali con sommatorie,
per variabili casuali discrete, e si estende banalmente ad un numero arbitrario
di variabili.


\subsection{Covarianza e correlazione}

Date due variabili casuali $x_1$ ed $x_2$, e dette $\mu_1 = \expect{x_1}$ e
$\mu_2 = \expect{x_2}$ le rispettive medie, definiamo la loro covarianza, che
chiameremo $\cov{x_1}{x_2}$ o $\sigma_{x_1x_2}$, come
il valore di aspettazione delle rispettive fluttuazioni attorno al valor medio
\begin{align}\label{eq:covarianza}
  \cov{x_1}{x_2} = \sigma_{x_1x_2} = \expect{(x_1 - \mu_1)(x_2 - \mu_2)}.
\end{align}
Si dimostra banalmente che
\begin{align*}
  \cov{x_1}{x_2} =
  \expect{x_1x_2} - \mu_2\expect{x_1} - \mu_1\expect{x_2} + \mu_1\mu_2 =
  \expect{x_1x_2} - \mu_1\mu_2 = \expect{x_1x_2} - \expect{x_1}\expect{x_2},
\end{align*}
che ci permette di riscrivere la~\eqref{eq:aspettazione_indipendenza} nella
forma equivalente
\begin{align*}
  \expect{x_1x_2} = \expect{x_1}\expect{x_2} \iff \cov{x_1}{x_2} = 0,
\end{align*}

Il concetto di covarianza è dunque legato a quello di indipendenza, nel senso
che \emph{se due variabili sono indipendenti, la loro covarianza è nulla}.
Questo non dovrebbe stupire, poiché se $x_1$ ed $x_2$ sono indipendenti,
allora non c'è nessun motivo per cui valori di $x_1$ al di sotto (o al di
sopra) della media debbano essere associati preferenzialmente a valori di $x_2$
al di sotto (o al di sopra della media), per cui valori positivi e negativi dei
due fattori della~\eqref{eq:covarianza} tendono cancellarsi tra di loro.

\`E interessante notare come l'implicazione inversa non valga in generale,
cioè \emph{il fatto che la covarianza tra due variabili sia nulla non
  implica necessariamente che le due variabili siano indipendenti}, come
illustrato nell'esempio~\ref{exp:covarianza_x_x2}. Il motivo è che la
covarianza è sensibile solo ad un tipo specifico di mutua dipendenza, ovvero
quello lineare.

\begin{examplebox}
  \begin{example}\label{exp:covarianza_x_x2}
    Consideriamo una variabile casuale continua $x$ con funzione di
    distribuzione $p(x)$ simmetrica rispetto a $0$---il che implica che tutti
    i momenti algebrici di ordine dispari sono nulli. Si ha banalmente
    \begin{align*}
      \cov{x}{x^2} = \expect{x^3} - \expect{x}\expect{x^2} = 0
    \end{align*}
    poiché sia $\expect{x^3}$ che $\expect{x}$ sono nulli. Dunque la
    covarianza tra $x$ ed $x^2$ è nulla, eppure è ovvio che le due
    variabili non sono indipendenti, in quanto la seconda è univocamente
    determinata dalla prima.
  \end{example}
\end{examplebox}

Da un punto di vista matematico, la covarianza è una forma bilineare
simmetrica, nel senso che gode delle seguenti proprietà elementari
\begin{align}
  \cov{x_1}{x_2} & = \cov{x_2}{x_1}\\
  \cov{c_1x_1 + c_2x_2}{x_3} & = c_1\cov{x_1}{x_3} + c_2 \cov{x_2}{x_3}\\
  \cov{x_1}{c_2x_2 + c_3x_3} & = c_2\cov{x_1}{x_2} + c_3 \cov{x_1}{x_3}
\end{align}
Si dimostra anche banalmente che la covarianza di una variabile casuale con
una costante è identicamente nulla e che la covarianza di una variabile
casuale con se stessa non è altro che la varianza della variabile stessa
\begin{align}
  \cov{x}{c} & = 0 \\
  \cov{x}{x} & = \var{x}.
\end{align}

Le covarianze \cov{x_i}{x_j} di $n$ variabili casuali si possono organizzare in
modo naturale in una matrice simmetrica $n \times n$ che prende il nome di
\emph{matrice di covarianza} e che, nel caso $n = 2$, si scrive esplicitamente
come
\begin{align}
  \Sigma =
  \begin{bmatrix}
    \cov{x_1}{x_1} & \cov{x_1}{x_2}\\
    \cov{x_2}{x_1} & \cov{x_2}{x_2}
  \end{bmatrix} =
  \begin{bmatrix}
    \sigma^2_1 & \cov{x_1}{x_2}\\
    \cov{x_2}{x_1} & \sigma^2_2
  \end{bmatrix} =
  \begin{bmatrix}
    \sigma^2_1 & \cov{x_1}{x_2}\\
    \cov{x_1}{x_2} & \sigma^2_2
  \end{bmatrix}.
\end{align}
(La generalizzazione al caso generale dovrebbe essere ovvia.)

A partire dalla covarianza si definisce una sua versione \emph{riscalata}
che prende il nome di correlazione $\corr{x_1}{x_2}$ o $\varrho_{x_1x_2}$
\begin{align}
  \corr{x_1}{x_2} = \varrho_{x_1x_2} = \frac{\cov{x_1}{x_2}}{\sigma_1 \sigma_2},
\end{align}
in cui $\sigma_1$ e $\sigma_2$ sono le deviazioni standard di $x_1$ ed $x_2$,
rispettivamente. La correlazione possiede banalmente tutte le proprietà della
covarianza---in particolare è nulla se le variabili $x_1$ ed $x_2$ sono
indipendenti (ma, di nuovo, l'implicazione inversa non è vera in generale).

La correlazione è una quantità adimensionale che assume valori compresi
tra $-1$ ed $1$ (non lo dimostreremo formalmente, ma la covarianza è una
sorta di prodotto scalare, mentre la deviazione standard ha il significato di
una norma, per cui questa proprietà segue essenzialmente dalla disuguaglianza
di Cauchy-Schwarz). Se la correlazione tra due variabili è nulla, esse si
dicono \emph{scorrelate}. Se la correlazione è maggiore (minore) di $0$, esse
si dicono \emph{positivamente (negativamente) correlate}. Nel caso particolare
in cui una variabile casuale dipenda linearmente da un'altra
\begin{align}
  x_2 = mx_1 + q \quad \text{(per cui $\sigma_2 = \abs{m}\sigma_1$)}
\end{align}
allora si ha
\begin{align}
  \corr{x_1}{x_2} = \frac{\cov{x_1}{x_2}}{\sigma_1\sigma_2} =
  \frac{\cov{x_1}{mx_1 + q}}{\sigma_1\sigma_2} =
  \frac{m\cov{x_1}{x_1}}{\abs{m}\sigma_1^2} = \frac{m}{\abs{m}} = \pm 1
\end{align}
ovverosia la correlazione è $\pm 1$ a seconda che il segno di $m$ sia
positivo o negativo. Si può anche dimostrare il contrario, ovverosia la
correlazione tra due variabili è (in modulo) unitaria se e solo se esse
sono legate tra loro da una relazione lineare. Ne segue banalmente che la
correlazione di una variabile con se stessa è pari all'unità
\begin{align}
  \corr{x}{x} = 1.
\end{align}

In analogia alla matrice di covarianza si definisce la \emph{matrice di
  correlazione} come
\begin{align}
  R =
  \begin{bmatrix}
    \corr{x_1}{x_1} & \corr{x_1}{x_2}\\
    \corr{x_2}{x_1} & \corr{x_2}{x_2}
  \end{bmatrix} =
  \begin{bmatrix}
    1 & \corr{x_1}{x_2}\\
    \corr{x_2}{x_1} & 1
  \end{bmatrix} =
  \begin{bmatrix}
    1 & \corr{x_1}{x_2}\\
    \corr{x_1}{x_2} & 1
  \end{bmatrix}.
\end{align}


\summary

\begin{itemize}
\item La probabilità è una misura che associa ad ogni elemento dello spazio
  degli eventi un numero reale che soddisfa gli assiomi di Kolmogorov.
\item La probabilità condizionata $\prob{E_1 \cond E_2}$ è la probabilità
  che si verifichi $E_1$, subordinata al fatto che si sia verificato $E_2$.
  I due eventi $E_1$ ed $E_2$ si dicono indipendenti se
  $\prob{E_1 \cond E_2} = \prob{E_1}$ e $\prob{E_2 \cond E_1} = \prob{E_2}$.
  La probabilità che due eventi indipendenti si verifichino insieme è
  uguale al prodotto delle probabilità.
\item Il teorema di Bayes~\eqref{eq:teorema_bayes} lega tra di loro le
  probabilità condizionate $\prob{E_1 \cond E_2}$ e $\prob{E_2 \cond E_1}$ ed
  è di fondamentale importanza nei problemi di probabilità inversa.
\item Data una variabile casuale $x$, la funzione di distribuzione è una
  funzione che associa ad ogni valore di $x$ la sua probabilità (nel caso
  discreto) o la sua densità di probabilità (nel caso continuo).
\item A partire dal concetto di valore di aspettazione di una funzione di
  variabile casuale si definiscono tutte le proprietà rilevanti delle
  distribuzioni, tra cui
  \begin{align*}
    \mu & = \expect{x} & \text{(la media)}\\
    \sigma^2 & = \expect{(x - \mu)^2} = \expect{x^2} - \mu^2 &
    \text{(la varianza)}\\
    \sigma & = \sqrt{\sigma^2} & \text{(la deviazione standard)}\\
    \mom[x_0]{n} & = \expect{(x - x_0)^n} &
    \text{(i momenti di ordine generico)}.
  \end{align*}
\end{itemize}
