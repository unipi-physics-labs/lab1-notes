\renewenvironment{enumerate}
{\begin{list}{\theenumi\rule{0.5pt}{0pt}.}{%
\usecounter{enumi}
\setlength{\itemindent}{12pt}%
\setlength{\labelwidth}{25pt}%
\setlength{\leftmargin}{\rightmargin}%
}}%
{\end{list}}

% 
\renewenvironment{itemize}
{\begin{list}{$\blacktriangleright$}{%
\setlength{\itemindent}{12pt}%
\setlength{\labelwidth}{25pt}%
\setlength{\leftmargin}{\rightmargin}%
}}
{\end{list}}

% Exercises.
% \exercise{text}
\newtheoremstyle{_exercise_}
{9pt}
{9pt}
{\rmfamily}
{}
{\scshape}
{.}
{ }
{$\blacktriangleright$ \thmname{#1} \thmnumber{#2}\thmnote{#3}}
\theoremstyle{_exercise_}
\newtheorem{_exercise_}{Esercizio}[chapter]

\newcommand{\exercise}[1]
{\begin{_exercise_} {\rmfamily #1} \end{_exercise_}}

\newenvironment{exercises}
{\clearpage
\section{Esercizi}}
{}

\newcommand{\hint}[1]
{\noindent \emph{Suggerimento: #1}}

\newcommand{\answer}[1]
{\hfill \emph{Risposta:} $\left[\, #1 \,\right]$}

% Examples.
% This used to be a \newcommand with the \begin{} and \end{} blocks of the
% newtheorem embedded, but that had the drawback that no \verbatim blocks can be
% used into command arguments. 
%
% First we define a \newtheorem...
%
\newtheoremstyle{example}
{\baselineskip}       % Space above
{\baselineskip}       % Space below
{\normalsize}         % The body font (use \slshape for slanted body)
{}                    % Indent amount (empty = no indent)
{\scshape}            % Thm head font
{.}                   % Punctuation after thm head
{ }                   % Thm head spec (can be left empty, meaning `normal')
{$\blacktriangleright$ \thmname{#1} \thmnumber{#2}%
                       \ifx\\#3\\%
                       \else~(\thmnote{#3})\fi}
\theoremstyle{example}
\newtheorem{example}{Esempio}[chapter]
%               
% ...then a new md environment...
%
\newmdenv[linecolor=lightgray,
          skipabove=0pt,
          skipbelow=0pt,
          innertopmargin=0.5\baselineskip,
          innerbottommargin=0.4\baselineskip,
          innerleftmargin=0.5\baselineskip,
          innerrightmargin=0.5\baselineskip,     
          roundcorner=4pt,
          nobreak=false]{exampleframe}
%
% Then a new environment to make the frames float...
%
\DeclareFloatingEnvironment[name=FloatExample]{floatexamplebox}

%
% ...finally we embed everything into a \newenvironment, which is the one we
% actually use.
%
\newenvironment{examplebox}[1][!htb]
{\begin{floatexamplebox}[#1]\begin{exampleframe}}
{\end{exampleframe}\end{floatexamplebox}}
%
% Done with the examples.


% Theorems
\newtheoremstyle{_theorem_}
{9pt}
{9pt}
{\rmfamily}
{}
{\scshape}
{.}
{ }
{\thmname{#1}\thmnumber{ #2}\ifx\\#3\\\else~(\thmnote{#3})\fi}
\theoremstyle{_theorem_}
\newtheorem{theorem}{Teorema}[chapter]

\newtheorem{corollary}{Corollario}[chapter]
