\chapter*{Prefazione}

Queste dispense sono state compilate a supporto del corso di Laboratorio 1
per gli studenti del primo anno del corso di Laurea in Fisica presso
l'Università di Pisa durante l'anno accademico 2016--2017, e sono state modificate
ed integrate progressivamente negli anni successivi.
Anche se l'impostazione è in parte mutuata dalle dispense della Professoressa
Liana Martinelli (che per anni sono state utilizzate come testo di riferimento
per il corso, e sono adesso liberamente disponibili alla pagina web
\url{https://bitbucket.org/lbaldini/martinelli}) credo si possa dire
senza tema di smentita che i due manoscritti siano abbastanza diversi tra loro
da poterli considerare, a tutti gli effetti, indipendenti.

Il tema centrale delle dispense è quello della rappresentazione e dell'analisi
dei dati. Accanto agli argomenti \emph{classici} che si trovano virtualmente in
qualsiasi testo di carattere introduttivo (elementi di teoria delle probabilità
e di statistica, propagazione degli errori, metodi di \fit) si è cercato di dare
spazio ad elementi appena più avanzati ma indispensabili nel bagaglio di
conoscenze del Fisico (e.g., i metodi Monte Carlo), nella speranza che possano
suscitare interesse e rendere meno arido il materiale.

Il livello della trattazione è il più elementare possibile---compatibilmente
con la sofisticatezza di alcuni concetti che è pur necessario assimilare
in un corso di laboratorio del primo anno. Si dà per scontato che lo studente
sia familiare con il calcolo differenziale ed integrale, principalmente
per funzioni di una variabile, ma, a parte questo, le dispense hanno
l'obiettivo di essere per quanto possibile auto-contenute ed i concetti
necessari sono via via richiamati al momento opportuno.

I punti qualificanti di queste dispense, che derivano direttamente
dall'esperienza di insegnamento accumulata con gli studenti del primo anno,
sono tre:
\begin{itemize}
\item i calcoli sono svolti dall'inizio alla fine, cercando di non saltare
  nessun passaggio intermedio, ed ampiamente commentati---a costo di
  sacrificare a tratti la sintesi e la scorrevolezza del testo;
\item i concetti rilevanti sono illustrati da un gran numero di esempi
  concreti, sviluppati in modo dettagliato, a diversi livelli di
  difficoltà;
\item il testo è corredato da frammenti di codice (in \python) che illustrano
  come si possano implementare in pratica alcuni dei concetti descritti
  in astratto.
\end{itemize}

I frammenti di codice sono particolarmente importanti, perché il laboratorio
è materia che si impara sul campo---ed una comprensione solida della teoria
è utile solo nel limite in cui si riesce a tradurla ed applicarla in pratica.
Gli esempi di codice non sono pensati per essere particolarmente efficienti
o eleganti, ma sono, per la maggior parte, direttamente utilizzabili.
Lo studente ha accesso alla versione elettronica su web di ciascun frammento
di codice attraverso un \foreign{hyperlink} situato nella parte superiore della
cornice corrispondente, e l'\foreign{output} fornito dall'esecuzione è riportato
nella sua interezza in calce al frammento stesso, di modo che sia chiara
non solo l'intenzione, ma anche il risultato.

\begin{snippet}[htb!]
  \bigskip % This is ugly and should be taken care of automagically.
  \hstack[0.45]{\begin{Verbatim}[label=\makebox{\href{https://bitbucket.org/lbaldini/statnotes/src/master/snippets/hello\_world.py}{https://bitbucket.org/.../hello\_world.py}},commandchars=\\\{\}]
\PY{n+nb}{print}\PY{p}{(}\PY{l+s+s1}{\PYZsq{}}\PY{l+s+s1}{Hello world!}\PY{l+s+s1}{\PYZsq{}}\PY{p}{)}

[Output]
Hello world!
\end{Verbatim}}{
    \caption{Illustrazione del più semplice programma realizzabile in \python.
      Si notino l'\foreign{hyperlink} alla versione elettronica nella parte
      superiore della cornice e l'\foreign{output} completo dell'esecuzione
      nella parte inferiore.
    }
    \label{snip:hello_world}
  }
\end{snippet}

Nel corso degli anni il numero e l'importanza dei frammenti di codice nell'economia
delle dispense sono cresciuti progressivamente, fino ad arrivare ad un livello
di integrazione sostanziale tra testo e codice. I linguaggi di programmazione
sono notoriamente argomento che evolve in fretta, e non è banale assicurarsi che
quello che si scrive oggi rimanga rilevante nel tempo. Purtuttavia, l'ecosistema
scientifico di \python\ (\numpy, \scipy, e \matplotlib) è, in questo momento,
ampiamente utilizzato per il calcolo scientifico, da una comunità vasta e vibrante,
per cui speriamo che il materiale possa essere utile non solo didatticamente,
ma anche professionalmente.

\danger%
Di tanto in tanto troverete sezioni che differiscono dalla maggior parte del
resto del contenuto delle dispense---perché focalizzate su argomenti
specifici più avanzati o per il livello leggermente più sofisticato degli
strumenti matematici necessari. Queste sezioni sono indicate con il simbolo
di curva pericolosa e, in una prima lettura, possono essere ignorate senza
che questo pregiudichi la comprensione del resto. (Ma non fatelo
sistematicamente, perché gli argomenti sono interessanti.)

\bigskip

\noindent Che dire di più? Buona lettura!

\bigskip

{\flushright Luca Baldini\\ \flushright Pisa, 21 settembre 2021\\}
