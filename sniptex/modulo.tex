\begin{Verbatim}[label=\makebox{\href{https://github.com/unipi-physics-labs/statnotes/tree/main/snippy/modulo.py}{https://github.com/.../modulo.py}},commandchars=\\\{\}]
\PY{c+c1}{\PYZsh{} Define m and p, and initialize an empty list.}
\PY{n}{m} \PY{o}{=} \PY{l+m+mi}{11}
\PY{n}{p} \PY{o}{=} \PY{l+m+mi}{7}
\PY{n}{sequence} \PY{o}{=} \PY{p}{[}\PY{p}{]}
\PY{c+c1}{\PYZsh{} Calculate k\PYZca{}p mod m for all positive k \PYZlt{} m.}
\PY{c+c1}{\PYZsh{} Note ** is the power and \PYZpc{} the modulo operator.}
\PY{k}{for} \PY{n}{k} \PY{o+ow}{in} \PY{n+nb}{range}\PY{p}{(}\PY{n}{m}\PY{p}{)}\PY{p}{:}
    \PY{n}{sequence}\PY{o}{.}\PY{n}{append}\PY{p}{(}\PY{n}{k}\PY{o}{*}\PY{o}{*}\PY{n}{p} \PY{o}{\PYZpc{}} \PY{n}{m}\PY{p}{)}
\PY{c+c1}{\PYZsh{} Print the sequence.}
\PY{n+nb}{print}\PY{p}{(}\PY{n}{sequence}\PY{p}{)}

[Output]
[0, 1, 7, 9, 5, 3, 8, 6, 2, 4, 10]
\end{Verbatim}
